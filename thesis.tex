
% ----------------------------------------------------------------------
%                   LATEX TEMPLATE FOR PhD THESIS
% ----------------------------------------------------------------------

% based on Harish Bhanderi's PhD/MPhil template, then Uni Cambridge
% http://www-h.eng.cam.ac.uk/help/tpl/textprocessing/ThesisStyle/
% corrected and extended in 2007 by Jakob Suckale, then MPI-CBG PhD programme
% and made available through OpenWetWare.org - the free biology wiki


%: Style file for Latex
% Most style definitions are in the external file PhDthesisPSnPDF.
% In this template package, it can be found in ./Latex/Classes/
\documentclass[twoside,11pt]{Latex/Classes/PhDthesisPSnPDF}

\makeatletter
        \setlength{\@fptop}{0pt}
\makeatother


%: Macro file for Latex
% Macros help you summarise frequently repeated Latex commands.
% Here, they are placed in an external file /Latex/Macros/MacroFile1.tex
% An macro that you may use frequently is the figuremacro (see introduction.tex)
\include{Latex/Macros/MacroFile1}
\usepackage[table]{xcolor}
%% The amssymb package provides various useful mathematical symbols
\usepackage{amssymb}

%% The amsthm package provides extended theorem environments
\usepackage{amsthm}

%% amsmath for math environment
\usepackage{amsmath}

\DeclareMathOperator*{\argmin}{arg\,min}
\DeclareMathOperator*{\argmax}{arg\,max}
\DeclareMathOperator*{\sign}{sign}
\DeclareMathOperator*{\infspie}{inf}


% to break equation
%\usepackage{mathpazo}
%\usepackage{mathptmx}
%\usepackage[mathpazo]{flexisym}
%\usepackage{breqn}

%% For clever reference
%\usepackage{cleveref}

%% color package
\usepackage{color}

%% figure package
\usepackage{epsf,graphicx}
\usepackage{epstopdf}
\usepackage{subfigure}
\usepackage{transparent}

%% New environment to have some indent inside enumerate environment
\usepackage{enumitem}

%% To create acronym for proper glossary
\usepackage{acro}

%% To number the line in the article
\usepackage{lineno}

%% Environment to include table with notes
\usepackage{array}
\usepackage{threeparttable}
\usepackage{booktabs}
\usepackage{multirow}
\usepackage{siunitx}

%% In order to change size of margin
\usepackage{geometry}
\usepackage{changepage}
\usepackage{lscape}
%% Colorpackage for table
\usepackage{colortbl}
\usepackage{tabularx}
\usepackage{arydshln}

%% To use URL referencing
\usepackage{url}
%\usepackage[hidelinks]{hyperref}

%% In order to draw some graphs
\usepackage{tikz,xifthen}
\usepackage{tikz-qtree}
\usetikzlibrary{decorations.pathmorphing} % noisy shapes
\usetikzlibrary{fit}					% fitting shapes to coordinates
\usetikzlibrary{backgrounds}	% drawing the background after the foreground
\usetikzlibrary{shapes,arrows,shadows}
\usetikzlibrary{calc,decorations.pathreplacing,decorations.markings,positioning}
\usetikzlibrary{snakes,decorations.text,shapes,patterns}
%\usepackage{scalefnt,lmodern,booktabs}

%% Paxkage for cross and tick symbols
\usepackage{pifont}
\newcommand{\cmark}{\color{green!60!black!80}\ding{51}}
\newcommand{\mmark}{{\color{green!60!black!80}\ding{51}}$^{!}$}
\newcommand{\xmark}{\color{red!60!black!80}\ding{55}}
\newcommand{\cmarksmall}{\color{green!60!black!80}\ding{51}}
\newcommand{\mmarksmall}{{\color{green!60!black!80}\ding{51}}$^{!}$}
\newcommand{\xmarksmall}{\color{red!60!black!80}\ding{55}}
\newcommand{\Conv}{\mathop{\scalebox{1.5}{\raisebox{-0.2ex}{$\ast$}}}}%

\definecolor{autoGuided}{rgb}{ 0.3765    0.7294    0.9412}
\newcommand{\autoGuidedColor}{(light-Blue)}
\definecolor{fullyAuto}{rgb}{ 0.0941    0.3843    0.6627}
\newcommand{\fullyAutoColor}{(dark-blue)}
\definecolor{semiAuto}{rgb}{ 0.0784    0.5059    0.1686}
\newcommand{\semiAutoColor}{(light-green)}
\definecolor{fullyGuided}{rgb}{ 0.4275    0.6902    0.3176}
\newcommand{\fullyGuidedColor}{(dark-green)}

\DeclareSIUnit\ppm{ppm}
\DeclareSIUnit\px{px}

\usepackage{ltxtable}
\usepackage{listings}
\usepackage{color}
%\usepackage[toc]{appendix}

\definecolor{codegreen}{rgb}{0,0.6,0}
\definecolor{codegray}{rgb}{0.5,0.5,0.5}
\definecolor{codepurple}{rgb}{0.58,0,0.82}
\definecolor{backcolour}{rgb}{0.95,0.95,0.92}

\lstdefinestyle{mystyle}{
    backgroundcolor=\color{backcolour},
    commentstyle=\color{codegreen},
    keywordstyle=\color{magenta},
    numberstyle=\tiny\color{codegray},
    stringstyle=\color{codepurple},
    basicstyle=\footnotesize,
    breakatwhitespace=false,
    breaklines=true,
    captionpos=b,
    keepspaces=true,
    numbers=left,
    numbersep=5pt,
    showspaces=false,
    showstringspaces=false,
    showtabs=false,
    tabsize=2
}

\lstset{style=mystyle}
\usepackage{setspace}

%\acrodef{cap}[CaP]{prostate cancer}
\DeclareAcronym{cap}{
short = CaP,
long = prostate cancer
}
%\acrodef{cade}[CADe]{computer-aided detection}
\DeclareAcronym{cade}{
short = CADe,
long = computer-aided detection
}
%\acrodef{cadx}[CADx]{computer-aided diagnosis}
\DeclareAcronym{cadx}{
short = CADx,
long = computer-aided diagnosis
}
\DeclareAcronym{lm}{
short = LM, 
long = Leung-Malik set
}
%\acrodef{us}[US]{ultrasound}
\DeclareAcronym{us}{
short = UTS,
long = ultrasound
}
%\acrodef{ct}[CT]{computer tomography}
\DeclareAcronym{ct}{
short = CT,
long = computer tomography
}
%\acrodef{cad}[CAD]{computer-aided detection and diagnosis}
\DeclareAcronym{cad}{
short = CAD,
long = computer-aided detection and diagnosis
}
%\acrodef{mri}[MRI]{magnetic resonance imaging}
\DeclareAcronym{mri}{
short = MRI,
long = magnetic resonance imaging
}
%\acrodef{nmr}[NMR]{nuclear magnetic resonance}
\DeclareAcronym{nmr}{
short = NMR,
long = nuclear magnetic resonance
}

\DeclareAcronym{omp}{
  short = OMP,
  long =  orthogonal matching pursuit 
}
\DeclareAcronym{adb}{
  short = AdB, 
  long = AdaBoost
}
\DeclareAcronym{gb}{
  short = GB, 
  long = Gradient Boosting
}

\DeclareAcronym{mp}{
  short = MP,
  long =  Matching Pursuit 
}
%\acrodef{t2w}[T$_2$-W]{T$_2$ Weighted}
\DeclareAcronym{t2w}{
short = T$_2$-W,
long = T$_2$ Weighted
}
%\acrodef{dce}[DCE]{dynamic contrast-enhanced}
\DeclareAcronym{dce}{
short = DCE,
long = dynamic contrast-enhanced
}
%\acrodef{dw}[DW]{diffusion weighted}
\DeclareAcronym{dw}{
short = DW,
long = diffusion weighted
}
%\acrodef{mrsi}[MRSI]{magnetic resonance spectroscopy imaging}
\DeclareAcronym{mrsi}{
short = MRSI,
long = magnetic resonance spectroscopy imaging
}
%\acrodef{bph}[BPH]{benign prostatic hyperplasia}
\DeclareAcronym{bph}{
short = BPH,
long = benign prostatic hyperplasia
}
%\acrodef{pz}[PZ]{peripheral zone}
\DeclareAcronym{pz}{
short = PZ,
long = peripheral zone
}
%\acrodef{cz}[CZ]{central zone}

\DeclareAcronym{mpmri}{
short = mp-MRI,
long = multiparametric \ac{mri}
}
\DeclareAcronym{cz}{
short = CZ,
long = central zone
}
%\acrodef{tz}[TZ]{transitional zone}
\DeclareAcronym{tz}{
short = TZ,
long = transitional zone
}
%\acrodef{cg}[CG]{central gland}
\DeclareAcronym{cg}{
short = CG,
long = central gland
}
%\acrodef{psa}[PSA]{prostate-specific antigen}
\DeclareAcronym{psa}{
short = PSA,
long = prostate-specific antigen
}
%\acrodef{trus}[TRUS]{transrectal ultrasound}
\DeclareAcronym{trus}{
short = TRUS,
long = transrectal ultrasound
}
%\acrodef{tr}[TR]{repetition time}
\DeclareAcronym{tr}{
short = TR,
long = repetition time
}
%\acrodef{te}[TE]{echo time}
\DeclareAcronym{te}{
short = TE,
long = echo time
}
%\acrodef{si}[SI]{signal intensity}
\DeclareAcronym{si}{
short = SI,
long = signal intensity
}
%\acrodef{ees}[EES]{extravascular-extracellular space}
\DeclareAcronym{ees}{
short = EES,
long = extravascular-extracellular space
}
%\acrodef{t1w}[T$_1$-W]{T$_1$ Weighted}
\DeclareAcronym{t1w}{
short = T$_1$-W,
long = T$_1$ Weighted
}
%\acrodef{fse}[FSE]{Fast Spin-Echo}
\DeclareAcronym{fse}{
short = FSE,
long = Fast Spin-Echo
}
%\acrodef{adc}[ADC]{Apparent Diffusion Coeffient}
\DeclareAcronym{adc}{
short = ADC,
long = apparent diffusion coefficient
}
%\acrodef{roi}[ROI]{region of interest}
\DeclareAcronym{roi}{
short = ROI,
long = region of interest
}
%\acrodef{cse}[CSE]{chemical shift effect}
\DeclareAcronym{cse}{
short = CSE,
long = chemical shift effect
}
%\acrodef{snr}[SNR]{signal-to-noise}
\DeclareAcronym{snr}{
short = SNR,
long = signal-to-noise
}
\DeclareAcronym{se}{
short = SE, 
long = sensitivity
}
\DeclareAcronym{sp}{
short = SP, 
long = specificity
}
%\acrodef{gs}[GS]{Gleason score}
\DeclareAcronym{gs}{
short = GS,
long = Gleason score
}
%\acrodef{ersspc}[ERSSPC]{European Randomized Study of Screening for Prostate Cancer}
\DeclareAcronym{ersspc}{
short = ERSSPC,
long = European randomized study of screening for prostate cancer
}
%\acrodef{plco}[PLCO]{Prostate, Lung, Colorectal and Ovarian}
\DeclareAcronym{plco}{
short = PLCO,
long = prostate lung colorectal and ovarian
}
%\acrodef{fig}[Fig.]{figure}
\DeclareAcronym{fig}{
short = Fig.,
long = figure,
class = latex
}
\DeclareAcronym{tab}{
short = Table,
long = table,
class = latex
}
\DeclareAcronym{eq}{
short = Eq.,
long = equation,
class = latex
}
\DeclareAcronym{sec}{
short = Sect.,
long = section,
class = latex
}
\DeclareAcronym{chp}{
short = Chap.,
long = Chapter,
class = latex
}

\DeclareAcronym{fov}{
short = FOV,
long = field of view
}
\DeclareAcronym{dwt}{
short = DWT,
long = discrete wavelet transform
}
\DeclareAcronym{dwst}{
short = DWST,
long = discrete wavelet squared transform
}
\DeclareAcronym{map}{
short = MAP,
long = maximum \textit{a posteriori}
}
\DeclareAcronym{ml}{
short = ML,
long = maximum likelihood
}
\DeclareAcronym{mle}{
short = MLE,
long = maximum likelihood estimation
}
\DeclareAcronym{mrf}{
short = MRF,
long = Markov random field
}
\DeclareAcronym{itk}{
short = ITK,
long = Insight Segmentation and Registration Toolkit
}
\DeclareAcronym{es}{
short = ES,
long = Evolution Strategy
}
\DeclareAcronym{scf}{
short = SCF,
long = sparse coded features
}
\DeclareAcronym{bow}{
short = BoW,
long = bag of words
}
\DeclareAcronym{pdf}{
short = PDF,
long = probability density function
}
\DeclareAcronym{gscale}{
short = \textit{g}-scale,
long = generalized scale
}
\DeclareAcronym{aif}{
short = AIF,
long = arterial input function
}
\DeclareAcronym{svd}{
short = SVD,
long = singular value decomposition
}
\DeclareAcronym{mse}{
short = MSE,
long = mean squared error
}
\DeclareAcronym{mi}{
short = MI,
long = mutual information
}
\DeclareAcronym{mantra}{
short = MANTRA,
long = multi-attribute non-initializing texture reconstruction based active shape model
}
\DeclareAcronym{asm}{
short = ASM,
long = active shape model
}
\DeclareAcronym{pca}{
short = PCA,
long = principal components analysis
}
\DeclareAcronym{weritas}{
short = WERITAS,
long = weighted ensemble of regional image textures for active shape model segmentation
}
\DeclareAcronym{staple}{
short = STAPLE,
long = simultaneous truth and performance level estimation
}
\DeclareAcronym{lda}{
short = LDA,
long = linear discriminant analysis
}
\DeclareAcronym{lbp}{
short = LBP,
long = local binary pattern
}
\DeclareAcronym{tps}{
short = TPS,
long = thin plate spline
}
\DeclareAcronym{acm}{
short = ACM,
long = active contour model
}
\DeclareAcronym{cmi}{
short = CMI,
long = combined mutual information
}
\DeclareAcronym{svm}{
short = SVM,
long = support vector machines
}
\DeclareAcronym{rvm}{
short = RVM,
long = relevant vector machine
}
\DeclareAcronym{rbf}{
short = RBF,
long = radial basis function
}
\DeclareAcronym{knn}{
short = $k$-NN,
long = $k$-nearest neighbour
}
\DeclareAcronym{nn}{
short = NN,
long = neareast neighbour
}
\DeclareAcronym{dct}{
short = DCT,
long = discrete cosine transform
}
\DeclareAcronym{hog}{
short = HOG,
long = histogram of oriented gradient
}
\DeclareAcronym{dft}{
short = DFT,
long = discrete fourier transform
}
\DeclareAcronym{us1}{
short = US,
long = under-sampling
}
\DeclareAcronym{os}{
short = OS,
long = over-sampling
}
\DeclareAcronym{ros}{
short = ROS,
long = random-over-sampling
}
\DeclareAcronym{rus}{
short = RUS,
long = random-under-sampling
}
\DeclareAcronym{nm}{
short = NM,
long = nearmiss
}
\DeclareAcronym{nm3}{
short = NM-3,
long = nearmiss-3
}
\DeclareAcronym{nm2}{
short = NM-2,
long = nearmiss-2
}
\DeclareAcronym{nm1}{
short = NM-1,
long = nearmiss-1
}
\DeclareAcronym{iht}{
short = IHT,
long = instance-hardness-threshold
}
\DeclareAcronym{smote}{
short = SMOTE,
long = synthetic minority over-sampling techniques
}
\DeclareAcronym{smoteb1}{
short = SMOTE-b1,
long = SMOTE-borderline1
}
\DeclareAcronym{smoteb2}{
short = SMOTE-b2,
long = SMOTE-borderline2
}
\DeclareAcronym{mrmr}{
short = mRMR,
long = minimum redundancy maximum relevance
}
\DeclareAcronym{lle}{
short = LLE,
long = locally linear embedding
}
\DeclareAcronym{ica}{
short = ICA,
long = independent components analysis
}
\DeclareAcronym{qda}{
short = QDA,
long = quadratic discriminant analysis
}
\DeclareAcronym{id3}{
short = ID3,
long = iterative dichotomiser 3
}
\DeclareAcronym{cart}{
short = CART,
long = classification and regression tree
}
\DeclareAcronym{bagging}{
short = bagging,
long = bootsrap aggregating
}
\DeclareAcronym{loo}{
short = LOOCV,
long = leave-one-out cross-validation
}
\DeclareAcronym{lopo}{
short = LOPO CV,
long = leave-one-patient-out cross-validation
}

\DeclareAcronym{kcv}{
short = $k$-CV,
long = $k$-fold cross-validation
}
\DeclareAcronym{roc}{
short = ROC,
long = receiver operating characteristic
}
\DeclareAcronym{froc}{
short = FROC,
long = free-response receiver operating characteristic
}
\DeclareAcronym{auc}{
short = AUC,
long = area under the curve
}
\DeclareAcronym{rmse}{
short = RMSD,
long = root-mean-square deviation
}
\DeclareAcronym{rms}{
  short = RMS,
  long = root mean square
}
\DeclareAcronym{srsf}{
  short = SRSF,
  long =  square-root slope function
}
\DeclareAcronym{pun}{
short = PUN,
long = phenomenological universalities
}
\DeclareAcronym{etl}{
short = ETL,
long = echo train ength
}

\DeclareAcronym{rf}{
short = RF,
long = random forest
}
\DeclareAcronym{dna}{
short = DNA,
long = deoxyribonucleic acid
}

\DeclareAcronym{glcm}{
short = GLCM,
long = gray-level co-occurence matrix
}

\DeclareAcronym{iccvb}{
short = I2Cvb,
long = initiative for collaborative computer vision benchmarking
}

\DeclareAcronym{mloss}{
short = MLOSS 2015,
long = machine learning open source software 2015
}

\DeclareAcronym{ci}{
short = CI,
long = continuous integration
}

\DeclareAcronym{cern}{
short = CERN,
long = european organization for nuclear research
}

\DeclareAcronym{doi}{
short = DOI,
long = digital object identifier
}

\DeclareAcronym{pd}{
short = PD,
long = proton density
}

\DeclareAcronym{anova}{
short = ANOVA,
long = analysis of variance
}



%: ----------------------------------------------------------------------
%:                  TITLE PAGE: name, degree,..
% ----------------------------------------------------------------------
% below is to generate the title page with crest and author name

%if output to PDF then put the following in PDF header
\ifpdf  
    \pdfinfo { /Title  (Computer-Aided Diagnosis for Prostate Cancer using Multi-Parametric Magnetic Resonance Imaging)
               /Creator (TeX)
               /Producer (pdfTeX)
               /Author (Guillaume Lemaitre g.lemaitre58@gmail.com)
               /CreationDate (D:201610151200ss)  %format D:YYYYMMDDhhmmss
               /ModDate (D:201610151200ss)
               /Subject (PhD Dissertation of Guillaume Lemaitre)
               /Keywords (CAD, mp-MRI, spectroscopy, MRI) }
    \pdfcatalog { /PageMode (/UseOutlines)
                  /OpenAction (fitbh)  }
\fi


\title{Computer-Aided Diagnosis for Prostate Cancer using Multi-Parametric Magnetic Resonance Imaging}



% ----------------------------------------------------------------------
% The section below defines www links/email for author and institutions
% They will appear on the title page of the PDF and can be clicked
\ifpdf
  \author{\href{mailto:guillaume.lemaitre@udg.edu}{Guillaume Lema\^itre}}
%  \cityofbirth{born in XYZ} % uncomment this if your university requires this
%  % If city of birth is required, also uncomment 2 sections in PhDthesisPSnPDF
%  % Just search for the "city" and you'll find them.
  % The crest is a graphics file of the logo of your research institution.
  % Place it in ./0_frontmatter/figures and specify the width

%% First university
  \firstlab{\href{http://le2i.cnrs.fr/}{LE2I}}
  \firstlogolab{\includegraphics[width=2cm]{logos/logole2i.eps}}
  \firstuni{\href{http://www.u-bourgogne.fr/}{Universit\'e de Bourgogne}}
  \firstlogouni{\includegraphics[width=2cm]{logos/logo-ub-no-bg.pdf}}

  
%% Second university
  \secondlab{\href{http://vicorob.udg.es/}{ViCOROB}}
  \secondlogolab{\includegraphics[width=2cm]{logos/vicorobLogo1.png}}

  \seconduni{\href{https://www.udg.edu/}{Universitat de Girona}}
  \secondlogouni{\includegraphics[scale =0.5]{logos/UdG_dues_linies_centrat_blau.png}}

  \supervisora{Fabrice M\'eriaudeau (LE2I/CISIR - UBFC/UTP)}
  \supervisorb{Robert Mart\'i Marly (ViCOROB - UdG)}
  \supervisorc{Jordi Freixenet Bosch (ViCOROB - UdG)}
  \supervisord{Paul Michael Walker (LE2I - UBFC)}  
  
% If you are not creating a PDF then use the following. The default is PDF.
\else
  \author{Guillaume Lema\^itre}
%  \cityofbirth{born in XYZ}
%% First university
  \firstlab{\href{http://le2i.cnrs.fr/}{LE2I}}
  \firstlogolab{\includegraphics[width=2cm]{logos/logole2i.eps}}
  \firstuni{\href{http://www.u-bourgogne.fr/}{Universit\'e de Bourgogne}}
  \firstlogouni{\includegraphics[width=2cm]{logos/logoubblue.eps}}
  
%% Second university
  \secondlab{\href{http://vicorob.udg.es/}{ViCOROB}}
  \secondlogolab{\includegraphics[width=2cm]{logos/logovicorob.eps}}
  \seconduni{\href{https://www.udg.edu/}{Universitat de Girona}}
  \secondlogouni{\includegraphics[width=2cm]{logos/logoudg.eps}}
  
  
  \supervisora{Fabrice M\'eriaudeau (LE2I/CISIR - UBFC/UTP)}
  \supervisorb{Robert Mart\'i Marly (ViCOROB - UdG)}
  \supervisorc{Jordi Freixenet Bosch (ViCOROB - UdG)}
  \supervisord{Paul Michael Walker (LE2I - UBFC)}
\fi

%\renewcommand{\submittedtext}{change the default text here if needed}
\degree{Philosophi\ae Doctor (PhD)}
\degreedate{November 2016}


% ----------------------------------------------------------------------
       
% turn of those nasty overfull and underfull hboxes
\hbadness=10000
\hfuzz=50pt


%: --------------------------------------------------------------
%:                  FRONT MATTER: dedications, abstract,..
% --------------------------------------------------------------
\onehalfspacing
%\DeclareInstance{acro-title}{empty}{sectioning}{name-format =}
\begin{document}

%\language{english}

% sets line spacing
%\renewcommand\baselinestretch{1.2}
\baselineskip=18pt plus1pt


%: ----------------------- generate cover page ------------------------

\maketitle  % command to print the title page with above variables

%: ----------------------- cover page back side ------------------------
% Your research institution may require reviewer names, etc.
% This cover back side is required by Dresden Med Fac; uncomment if needed.

\newpage
{\pagestyle{plain}
\vspace{10mm}
Reviewers:

\begin{itemize}
\item[] Su Ruan, Professor at Universit\'e de Rouen - LITIS
\item[] Reyer Zwiggelaar, Professor at Aberystwyth Universtiy - Vision, Graphics, and Visualisation Group
%\item[] Soumya Ghose, Research Associate at Case Western Reserve University
\end{itemize}

\vspace{20mm}
Day of the defense: 28 November 2016

\vspace{20mm}
\hspace{70mm}Signature from head of PhD committee:



%: ----------------------- abstract ------------------------

% Your institution may have specific regulations if you need an abstract and where it is to be placed in the document. The default here is just after title.

%
% Thesis Abstract -----------------------------------------------------


%\begin{abstractslong}    %uncommenting this line, gives a different abstract heading
\begin{abstracts}        %this creates the heading for the abstract page
Prostate cancer (CaP) is the second most diagnosed cancer in men all over the world.
CaP growth is characterized by two main types of evolution: (i) the slow-growing tumours progress slowly and usually remain confined to the prostate gland; (ii) the fast-growing tumours metastasize from prostate gland to other organs, which might lead to incurable diseases.
Therefore, early diagnosis and risk assessment play major roles in patient treatment and follow-up.
In the last decades, new imaging techniques based on Magnetic Resonance Imaging (MRI) have been developed improving diagnosis.
In practise, diagnosis can be affected by multiple factors such as observer variability and visibility and complexity of the lesions.
In this regard, computer-aided detection and computer-aided diagnosis systems are being designed to help radiologists in their clinical practice.

Our research extensively analyzes the current state-of-the-art in the development of computer-aided diagnosis and detection systems for prostate cancer detection.
Currently, no computer-aided system using all available MRI modalities has been proposed and tested on a common dataset.
Therefore, we propose a new computer-aided system taking advantage of all MRI modalities (i.e., \acs*{t2w}-\acs*{mri}, \acs*{dce}-\acs*{mri}, DW-\acs*{mri}, \acs*{mrsi}).
Particular attention is paid to the normalization of the \acs*{mri} modalities prior to develop our computer-aided system.
This system has been extensively tested on a dataset which has been made publicly available.
\end{abstracts}
%\end{abstractlongs}
%-------------------------------------------------------------------------

\begin{abstractCatalan}

El c\`ancer de pr\`ostata (CaP) \'es el segon c\`ancer m\'es diagnosticat en homes a tot el m\'on.
El creixement del CaP es caracteritza per dos tipus principals d'evoluci\'o: (i) els tumors de creixement lent que progressen lentament i en general romanen confinats en la gl\`andula de la pr\`ostata; (ii) els tumors de creixement r\`apid que desenvolupen met\`astasi de la pr\`ostata a altres \`organs, el que podria conduir a malalties incurables.
Conseq\"uentment, el diagn\`ostic preco\c{c} i l'avaluaci\'o del risc exerceixen un paper important en el tractament del pacient i el seguiment.
En les \'ultimes d\`ecades s'han desenvolupat noves t\`ecniques d'imatge basades en imatge de resson\`ancia magn\`etica (RM, o MRI de l'angl\`es) per millorar el diagn\`ostic.
A la pr\`actica, el diagn\`ostic pot ser afectat per diversos factors com ara la variabilitat de l'observador i la visibilitat i la complexitat de les lesions.
En aquest sentit, s'estan desenvolupant sistemes per a l'ajuda a la detecci\'o i diagn\`ostic per ordinador per ajudar els radi\`olegs en la seva pr\`actica cl\'inica.

La nostra recerca analitza \`ampliament l'estat de l'art en el desenvolupament de sistemes per a l'ajuda a la detecci\'o i diagn\`ostic per ordinador per a la detecci\'o del c\`ancer de pr\`ostata.
En l'actualitat, no hi ha cap sistema d'ajuda al diagn\`ostic que utilitzi totes les modalitats de MRI disponibles i que hagi estat avaluat en un conjunt de dades com\'u.
Per tant, proposem un nou sistema d'ajuda al diagn\`ostic per ordinador aprofitant totes les modalitats de resson\`ancia magn\`etica (\'es a dir \acs*{t2w}-MRI, DCE-MRI, DW-MRI, MRSI).
Com a etapa pr\`evia al desenvolupament del sistema, es presta especial atenci\'o a la normalitzaci\'o de les modalitats de resson\`ancia magn\`etica.
El sistema desenvolupat ha estat avaluat extensivament en un conjunt de dades que s'han posat a disposici\'o p\'ublica.
 
\end{abstractCatalan}

% ---------------------------------------------------------------------- 
\begin{abstractSpanish}

El c\'ancer de pr\'ostata (CaP) es el segundo c\'ancer m\'as diagn\'osticado en hombres en todo el mundo.
El crecimiento del CaP se caracteriza por dos tipos principales de evoluci\'on: (i) los tumores de crecimiento lento que progresan lentamente y por lo general permanecen confinados en la gl\'andula de la pr\'ostata; (ii) los tumores de crecimiento r\'apido que desarrollan met\'astasis de la pr\'ostata a otros \'organos, lo que podr\'ia conducir a enfermedades incurables.
Consecuentemente, el diagn\'ostico precoz y la evaluaci\'on del riesgo desempe\~nan un papel importante en el tratamiento del paciente y el seguimiento. En las \'ultimas d\'ecadas se han desarrollado  nuevas t\'ecnicas de imagen basadas en imagen de resonancia magn\'etica (RM, o MRI del ingl\'es) para mejorar el diagn\'ostico. En la pr\'actica, el diagn\'ostico puede ser afectado por varios factores tales como la variabilidad del observador y la visibilidad y la complejidad de las lesiones.
En este sentido, se est\'an desarrollando sistemas para la ayuda a la detecci\'on y diagn\'ostico por ordenador para ayudar a los radi\'ologos en su pr\'actica cl\'inica.

Nuestra investigaci\'on analiza ampliamente el estado del arte en el desarrollo de sistemas para la ayuda a la detecci\'on y diagn\'ostico por ordenador para la detecci\'on del c\`ancer de pr\'ostata.
En la actualidad, no existe ning\'un sistema de ayuda al diagn\'ostico que utilice todas las modalidades de MRI disponibles y que haya sido evaluado en un conjunto de datos com\'un.
Por lo tanto, proponemos un nuevo sistema de ayuda al diagn\'ostico por ordenador aprovechando todas las modalidades de resonancia magn\'etica (es decir T2W-MRI, DCE-MRI, DW-MRI, MRSI).
Como etapa previa al desarrollo del sistema, se presta especial atenci\'on a la normalizaci\'on de las modalidades de resonancia magn\'etica.
El sistema desarrollado ha sido evaluado extensivamente en un conjunto de datos que se han puesto a disposici\'on p\'ublica.
 
\end{abstractSpanish}

%-------------------------------------------------------------------------
\begin{abstractFrench}

Le cancer de la prostate est le second type de cancer le plus diagnostiqu\'e au monde.
Il est caract\'eris\'e par deux evolutions distinctes : (i) les tumeurs \`a croissances lentes progressent lentement et restent g\'en\'eralement confin\'ees dans la glande prostatique; (ii) les tumeurs \`a croissances rapides se m\'etastasent de la prostate \`a d'autres organes p\'eriph\'eriques, pouvant causer le d\'evelopement de maladies incurables.
C'est pour cela qu'un diagnostic pr\'ecoce et une \'evaluation du risque jouent des r\^oles majeurs dans le traitement et le suivi du patient.
Durant la derni\`ere d\'ec\'enie, de nouvelles m\'ethodes d'imagerie bas\'ees sur l'Imagerie par R\'esonance Magn\'etique (IRM) ont \'et\'e d\'evelop\'ees.
En pratique, le diagnostic clinique peut \^etre affect\'e par de multiples facteurs comme la variabilit\'e entre observateurs et la complexit\'e des l\'esions lues.
Pour ce faire, des syst\`emes de d\'etection et de diagnostic assist\'e par ordinateur (DAO) ont \'et\'e d\'evelop\'es pour aider les radiologistes durant leurs t\^aches cliniques.

Notre recherche analyse extensivement l'\'etat de l'art actuel concernant le d\'evelopement des syst\`emes de DAO pour la d\'etection du cancer de la prostate.
Actuellement, il n'\'existe aucun syst\`eme de DAO utilisant toutes les modalit\'es IRM disponibles et qui plus est, test\'e sur une base de donn\'ees commune.
Par cons\'equent, nous proposons un nouveau syst\`eme de DAO tirant profit de toutes les modalit\'es IRM (i.e., T2W-MRI, DCE-MRI, DW-MRI, MRSI).
Une attention particuli\`ere est port\'ee sur la normalisation de ces donn\'ees multi-param\'etriques avant la conception du syst\`eme de DAO.
De plus, notre syst\`eme de DAO a \'et\'e test\'e sur une base de donn\'ees que nous rendons publique.

\end{abstractFrench}


% The original template provides and abstractseparate environment, if your institution requires them to be separate. I think it's easier to print the abstract from the complete thesis by restricting printing to the relevant page.
%\begin{abstractseparate}
%  
% Thesis Abstract -----------------------------------------------------


%\begin{abstractslong}    %uncommenting this line, gives a different abstract heading
\begin{abstracts}        %this creates the heading for the abstract page
Prostate cancer (CaP) is the second most diagnosed cancer in men all over the world.
CaP growth is characterized by two main types of evolution: (i) the slow-growing tumours progress slowly and usually remain confined to the prostate gland; (ii) the fast-growing tumours metastasize from prostate gland to other organs, which might lead to incurable diseases.
Therefore, early diagnosis and risk assessment play major roles in patient treatment and follow-up.
In the last decades, new imaging techniques based on Magnetic Resonance Imaging (MRI) have been developed improving diagnosis.
In practise, diagnosis can be affected by multiple factors such as observer variability and visibility and complexity of the lesions.
In this regard, computer-aided detection and computer-aided diagnosis systems are being designed to help radiologists in their clinical practice.

Our research extensively analyzes the current state-of-the-art in the development of computer-aided diagnosis and detection systems for prostate cancer detection.
Currently, no computer-aided system using all available MRI modalities has been proposed and tested on a common dataset.
Therefore, we propose a new computer-aided system taking advantage of all MRI modalities (i.e., \acs*{t2w}-\acs*{mri}, \acs*{dce}-\acs*{mri}, DW-\acs*{mri}, \acs*{mrsi}).
Particular attention is paid to the normalization of the \acs*{mri} modalities prior to develop our computer-aided system.
This system has been extensively tested on a dataset which has been made publicly available.
\end{abstracts}
%\end{abstractlongs}
%-------------------------------------------------------------------------

\begin{abstractCatalan}

El c\`ancer de pr\`ostata (CaP) \'es el segon c\`ancer m\'es diagnosticat en homes a tot el m\'on.
El creixement del CaP es caracteritza per dos tipus principals d'evoluci\'o: (i) els tumors de creixement lent que progressen lentament i en general romanen confinats en la gl\`andula de la pr\`ostata; (ii) els tumors de creixement r\`apid que desenvolupen met\`astasi de la pr\`ostata a altres \`organs, el que podria conduir a malalties incurables.
Conseq\"uentment, el diagn\`ostic preco\c{c} i l'avaluaci\'o del risc exerceixen un paper important en el tractament del pacient i el seguiment.
En les \'ultimes d\`ecades s'han desenvolupat noves t\`ecniques d'imatge basades en imatge de resson\`ancia magn\`etica (RM, o MRI de l'angl\`es) per millorar el diagn\`ostic.
A la pr\`actica, el diagn\`ostic pot ser afectat per diversos factors com ara la variabilitat de l'observador i la visibilitat i la complexitat de les lesions.
En aquest sentit, s'estan desenvolupant sistemes per a l'ajuda a la detecci\'o i diagn\`ostic per ordinador per ajudar els radi\`olegs en la seva pr\`actica cl\'inica.

La nostra recerca analitza \`ampliament l'estat de l'art en el desenvolupament de sistemes per a l'ajuda a la detecci\'o i diagn\`ostic per ordinador per a la detecci\'o del c\`ancer de pr\`ostata.
En l'actualitat, no hi ha cap sistema d'ajuda al diagn\`ostic que utilitzi totes les modalitats de MRI disponibles i que hagi estat avaluat en un conjunt de dades com\'u.
Per tant, proposem un nou sistema d'ajuda al diagn\`ostic per ordinador aprofitant totes les modalitats de resson\`ancia magn\`etica (\'es a dir \acs*{t2w}-MRI, DCE-MRI, DW-MRI, MRSI).
Com a etapa pr\`evia al desenvolupament del sistema, es presta especial atenci\'o a la normalitzaci\'o de les modalitats de resson\`ancia magn\`etica.
El sistema desenvolupat ha estat avaluat extensivament en un conjunt de dades que s'han posat a disposici\'o p\'ublica.
 
\end{abstractCatalan}

% ---------------------------------------------------------------------- 
\begin{abstractSpanish}

El c\'ancer de pr\'ostata (CaP) es el segundo c\'ancer m\'as diagn\'osticado en hombres en todo el mundo.
El crecimiento del CaP se caracteriza por dos tipos principales de evoluci\'on: (i) los tumores de crecimiento lento que progresan lentamente y por lo general permanecen confinados en la gl\'andula de la pr\'ostata; (ii) los tumores de crecimiento r\'apido que desarrollan met\'astasis de la pr\'ostata a otros \'organos, lo que podr\'ia conducir a enfermedades incurables.
Consecuentemente, el diagn\'ostico precoz y la evaluaci\'on del riesgo desempe\~nan un papel importante en el tratamiento del paciente y el seguimiento. En las \'ultimas d\'ecadas se han desarrollado  nuevas t\'ecnicas de imagen basadas en imagen de resonancia magn\'etica (RM, o MRI del ingl\'es) para mejorar el diagn\'ostico. En la pr\'actica, el diagn\'ostico puede ser afectado por varios factores tales como la variabilidad del observador y la visibilidad y la complejidad de las lesiones.
En este sentido, se est\'an desarrollando sistemas para la ayuda a la detecci\'on y diagn\'ostico por ordenador para ayudar a los radi\'ologos en su pr\'actica cl\'inica.

Nuestra investigaci\'on analiza ampliamente el estado del arte en el desarrollo de sistemas para la ayuda a la detecci\'on y diagn\'ostico por ordenador para la detecci\'on del c\`ancer de pr\'ostata.
En la actualidad, no existe ning\'un sistema de ayuda al diagn\'ostico que utilice todas las modalidades de MRI disponibles y que haya sido evaluado en un conjunto de datos com\'un.
Por lo tanto, proponemos un nuevo sistema de ayuda al diagn\'ostico por ordenador aprovechando todas las modalidades de resonancia magn\'etica (es decir T2W-MRI, DCE-MRI, DW-MRI, MRSI).
Como etapa previa al desarrollo del sistema, se presta especial atenci\'on a la normalizaci\'on de las modalidades de resonancia magn\'etica.
El sistema desarrollado ha sido evaluado extensivamente en un conjunto de datos que se han puesto a disposici\'on p\'ublica.
 
\end{abstractSpanish}

%-------------------------------------------------------------------------
\begin{abstractFrench}

Le cancer de la prostate est le second type de cancer le plus diagnostiqu\'e au monde.
Il est caract\'eris\'e par deux evolutions distinctes : (i) les tumeurs \`a croissances lentes progressent lentement et restent g\'en\'eralement confin\'ees dans la glande prostatique; (ii) les tumeurs \`a croissances rapides se m\'etastasent de la prostate \`a d'autres organes p\'eriph\'eriques, pouvant causer le d\'evelopement de maladies incurables.
C'est pour cela qu'un diagnostic pr\'ecoce et une \'evaluation du risque jouent des r\^oles majeurs dans le traitement et le suivi du patient.
Durant la derni\`ere d\'ec\'enie, de nouvelles m\'ethodes d'imagerie bas\'ees sur l'Imagerie par R\'esonance Magn\'etique (IRM) ont \'et\'e d\'evelop\'ees.
En pratique, le diagnostic clinique peut \^etre affect\'e par de multiples facteurs comme la variabilit\'e entre observateurs et la complexit\'e des l\'esions lues.
Pour ce faire, des syst\`emes de d\'etection et de diagnostic assist\'e par ordinateur (DAO) ont \'et\'e d\'evelop\'es pour aider les radiologistes durant leurs t\^aches cliniques.

Notre recherche analyse extensivement l'\'etat de l'art actuel concernant le d\'evelopement des syst\`emes de DAO pour la d\'etection du cancer de la prostate.
Actuellement, il n'\'existe aucun syst\`eme de DAO utilisant toutes les modalit\'es IRM disponibles et qui plus est, test\'e sur une base de donn\'ees commune.
Par cons\'equent, nous proposons un nouveau syst\`eme de DAO tirant profit de toutes les modalit\'es IRM (i.e., T2W-MRI, DCE-MRI, DW-MRI, MRSI).
Une attention particuli\`ere est port\'ee sur la normalisation de ces donn\'ees multi-param\'etriques avant la conception du syst\`eme de DAO.
De plus, notre syst\`eme de DAO a \'et\'e test\'e sur une base de donn\'ees que nous rendons publique.

\end{abstractFrench}

%\end{abstractseparate}


%: ----------------------- tie in front matter ------------------------

\include{0_frontmatter/dedication}
\include{0_frontmatter/publication}

%: ----------------------- contents ------------------------

\setcounter{secnumdepth}{3} % organisational level that receives a numbers
\setcounter{tocdepth}{3}    % print table of contents for level 3
% levels are: 0 - chapter, 1 - section, 2 - subsection, 3 - subsection
\clearpage

%: ----------------------- list of acronyms/figures/tables ------------------------

%\begin{abbreviations}
\addcontentsline{toc}{chapter}{List of Abbreviations}
\printacronyms[exclude-classes=latex, name=List of Abbreviations, heading=chapter*]
%\end{abbreviations}

\listoffigures	% print list of figures

\listoftables  % print list of tables


\include{0_frontmatter/acknowledgement}

\tableofcontents            % print the table of contents

% Thesis Abstract -----------------------------------------------------


%\begin{abstractslong}    %uncommenting this line, gives a different abstract heading
\begin{abstracts}        %this creates the heading for the abstract page
Prostate cancer (CaP) is the second most diagnosed cancer in men all over the world.
CaP growth is characterized by two main types of evolution: (i) the slow-growing tumours progress slowly and usually remain confined to the prostate gland; (ii) the fast-growing tumours metastasize from prostate gland to other organs, which might lead to incurable diseases.
Therefore, early diagnosis and risk assessment play major roles in patient treatment and follow-up.
In the last decades, new imaging techniques based on Magnetic Resonance Imaging (MRI) have been developed improving diagnosis.
In practise, diagnosis can be affected by multiple factors such as observer variability and visibility and complexity of the lesions.
In this regard, computer-aided detection and computer-aided diagnosis systems are being designed to help radiologists in their clinical practice.

Our research extensively analyzes the current state-of-the-art in the development of computer-aided diagnosis and detection systems for prostate cancer detection.
Currently, no computer-aided system using all available MRI modalities has been proposed and tested on a common dataset.
Therefore, we propose a new computer-aided system taking advantage of all MRI modalities (i.e., \acs*{t2w}-\acs*{mri}, \acs*{dce}-\acs*{mri}, DW-\acs*{mri}, \acs*{mrsi}).
Particular attention is paid to the normalization of the \acs*{mri} modalities prior to develop our computer-aided system.
This system has been extensively tested on a dataset which has been made publicly available.
\end{abstracts}
%\end{abstractlongs}
%-------------------------------------------------------------------------

\begin{abstractCatalan}

El c\`ancer de pr\`ostata (CaP) \'es el segon c\`ancer m\'es diagnosticat en homes a tot el m\'on.
El creixement del CaP es caracteritza per dos tipus principals d'evoluci\'o: (i) els tumors de creixement lent que progressen lentament i en general romanen confinats en la gl\`andula de la pr\`ostata; (ii) els tumors de creixement r\`apid que desenvolupen met\`astasi de la pr\`ostata a altres \`organs, el que podria conduir a malalties incurables.
Conseq\"uentment, el diagn\`ostic preco\c{c} i l'avaluaci\'o del risc exerceixen un paper important en el tractament del pacient i el seguiment.
En les \'ultimes d\`ecades s'han desenvolupat noves t\`ecniques d'imatge basades en imatge de resson\`ancia magn\`etica (RM, o MRI de l'angl\`es) per millorar el diagn\`ostic.
A la pr\`actica, el diagn\`ostic pot ser afectat per diversos factors com ara la variabilitat de l'observador i la visibilitat i la complexitat de les lesions.
En aquest sentit, s'estan desenvolupant sistemes per a l'ajuda a la detecci\'o i diagn\`ostic per ordinador per ajudar els radi\`olegs en la seva pr\`actica cl\'inica.

La nostra recerca analitza \`ampliament l'estat de l'art en el desenvolupament de sistemes per a l'ajuda a la detecci\'o i diagn\`ostic per ordinador per a la detecci\'o del c\`ancer de pr\`ostata.
En l'actualitat, no hi ha cap sistema d'ajuda al diagn\`ostic que utilitzi totes les modalitats de MRI disponibles i que hagi estat avaluat en un conjunt de dades com\'u.
Per tant, proposem un nou sistema d'ajuda al diagn\`ostic per ordinador aprofitant totes les modalitats de resson\`ancia magn\`etica (\'es a dir \acs*{t2w}-MRI, DCE-MRI, DW-MRI, MRSI).
Com a etapa pr\`evia al desenvolupament del sistema, es presta especial atenci\'o a la normalitzaci\'o de les modalitats de resson\`ancia magn\`etica.
El sistema desenvolupat ha estat avaluat extensivament en un conjunt de dades que s'han posat a disposici\'o p\'ublica.
 
\end{abstractCatalan}

% ---------------------------------------------------------------------- 
\begin{abstractSpanish}

El c\'ancer de pr\'ostata (CaP) es el segundo c\'ancer m\'as diagn\'osticado en hombres en todo el mundo.
El crecimiento del CaP se caracteriza por dos tipos principales de evoluci\'on: (i) los tumores de crecimiento lento que progresan lentamente y por lo general permanecen confinados en la gl\'andula de la pr\'ostata; (ii) los tumores de crecimiento r\'apido que desarrollan met\'astasis de la pr\'ostata a otros \'organos, lo que podr\'ia conducir a enfermedades incurables.
Consecuentemente, el diagn\'ostico precoz y la evaluaci\'on del riesgo desempe\~nan un papel importante en el tratamiento del paciente y el seguimiento. En las \'ultimas d\'ecadas se han desarrollado  nuevas t\'ecnicas de imagen basadas en imagen de resonancia magn\'etica (RM, o MRI del ingl\'es) para mejorar el diagn\'ostico. En la pr\'actica, el diagn\'ostico puede ser afectado por varios factores tales como la variabilidad del observador y la visibilidad y la complejidad de las lesiones.
En este sentido, se est\'an desarrollando sistemas para la ayuda a la detecci\'on y diagn\'ostico por ordenador para ayudar a los radi\'ologos en su pr\'actica cl\'inica.

Nuestra investigaci\'on analiza ampliamente el estado del arte en el desarrollo de sistemas para la ayuda a la detecci\'on y diagn\'ostico por ordenador para la detecci\'on del c\`ancer de pr\'ostata.
En la actualidad, no existe ning\'un sistema de ayuda al diagn\'ostico que utilice todas las modalidades de MRI disponibles y que haya sido evaluado en un conjunto de datos com\'un.
Por lo tanto, proponemos un nuevo sistema de ayuda al diagn\'ostico por ordenador aprovechando todas las modalidades de resonancia magn\'etica (es decir T2W-MRI, DCE-MRI, DW-MRI, MRSI).
Como etapa previa al desarrollo del sistema, se presta especial atenci\'on a la normalizaci\'on de las modalidades de resonancia magn\'etica.
El sistema desarrollado ha sido evaluado extensivamente en un conjunto de datos que se han puesto a disposici\'on p\'ublica.
 
\end{abstractSpanish}

%-------------------------------------------------------------------------
\begin{abstractFrench}

Le cancer de la prostate est le second type de cancer le plus diagnostiqu\'e au monde.
Il est caract\'eris\'e par deux evolutions distinctes : (i) les tumeurs \`a croissances lentes progressent lentement et restent g\'en\'eralement confin\'ees dans la glande prostatique; (ii) les tumeurs \`a croissances rapides se m\'etastasent de la prostate \`a d'autres organes p\'eriph\'eriques, pouvant causer le d\'evelopement de maladies incurables.
C'est pour cela qu'un diagnostic pr\'ecoce et une \'evaluation du risque jouent des r\^oles majeurs dans le traitement et le suivi du patient.
Durant la derni\`ere d\'ec\'enie, de nouvelles m\'ethodes d'imagerie bas\'ees sur l'Imagerie par R\'esonance Magn\'etique (IRM) ont \'et\'e d\'evelop\'ees.
En pratique, le diagnostic clinique peut \^etre affect\'e par de multiples facteurs comme la variabilit\'e entre observateurs et la complexit\'e des l\'esions lues.
Pour ce faire, des syst\`emes de d\'etection et de diagnostic assist\'e par ordinateur (DAO) ont \'et\'e d\'evelop\'es pour aider les radiologistes durant leurs t\^aches cliniques.

Notre recherche analyse extensivement l'\'etat de l'art actuel concernant le d\'evelopement des syst\`emes de DAO pour la d\'etection du cancer de la prostate.
Actuellement, il n'\'existe aucun syst\`eme de DAO utilisant toutes les modalit\'es IRM disponibles et qui plus est, test\'e sur une base de donn\'ees commune.
Par cons\'equent, nous proposons un nouveau syst\`eme de DAO tirant profit de toutes les modalit\'es IRM (i.e., T2W-MRI, DCE-MRI, DW-MRI, MRSI).
Une attention particuli\`ere est port\'ee sur la normalisation de ces donn\'ees multi-param\'etriques avant la conception du syst\`eme de DAO.
De plus, notre syst\`eme de DAO a \'et\'e test\'e sur une base de donn\'ees que nous rendons publique.

\end{abstractFrench}

}

%: --------------------------------------------------------------
%:                  MAIN DOCUMENT SECTION
% --------------------------------------------------------------

% the main text starts here with the introduction, 1st chapter,...
\mainmatter

%\renewcommand{\chaptername}{} % uncomment to print only "1" not "Chapter 1"


%: ----------------------- subdocuments ------------------------
% Parts of the thesis are included below. Rename the files as required.
% But take care that the paths match. You can also change the order of appearance by moving the include commands.
\acresetall
% this file is called up by thesis.tex
% content in this file will be fed into the main document

%: ----------------------- introduction file header -----------------------
\chapter{Introduction}\label{chap:1}

% the code below specifies where the figures are stored
\ifpdf
    \graphicspath{{1_introduction/figures/}}
\else
    \graphicspath{{1_introduction/figures/}}
\fi

% Prostate Cancer:
%%% This part contains:
%%% - Anatomy basis of the prostate
%%% - Statistics regarding prostate cancers
%\section{Prostate Cancer}\label{section:intro:prostatecancer}

%\subsection{Anatomy}\label{subsection:intro:prostatecancer:anatomy}
\section{Prostate anatomy}\label{section:intro:anatomy}

\begin{figure}
\centering
\includegraphics[height=0.25\textheight]{1_introduction/figures/anatomy/prostate2D.eps}
\caption[Sagittal anatomy of prostate.]{Sagittal anatomy scheme of the male reproductive system (copyright by~\cite{Geckomedia2011}).}
\label{fig:prostatelocation}
\end{figure}

\begin{figure}
	\centering
	\hspace*{\fill}
	\subfigure[Transverse anatomy of the prostate.]{
			\centering
			\includegraphics[height=0.15\textheight]{1_introduction/figures/anatomy/prostateTransverse.eps}
			\label{fig:anatomyProstateTransverse}}
			\hfill
	\subfigure[Sagittal anatomy of the prostate.]{
			\centering
			\includegraphics[height=0.23\textheight]{1_introduction/figures/anatomy/prostateSagital.eps}
			\label{fig:anatomyProstateSagittal}}\hspace*{\fill}
	\caption[Prostate anatomy.]{Prostate anatomy with division in different zones. \textit{AFT:} anterior fibromuscular tissue, \textit{CZ:} central zone, \textit{ED:} ejaculatory duct, \textit{NVB:} neurovascular bundle, \textit{PUT:} periurethral tissue, \textit{PZ:} peripheral zone, \textit{U:} urethra, \textit{TZ:} transitional zone, \textit{B:} base, \textit{M:} median, \textit{A:} apex (copyright by~\cite{Choi2007}).}
	\label{fig:anatomyProstateZone}
\end{figure}

The prostate is an exocrine gland of the male reproductive system having an inverted pyramidal shape, which is located below the bladder and in front of the rectum as shown in \acs{fig}\,\ref{fig:prostatelocation}.
It measures approximately \SI{3}{\cm} in height by \SI{2.5}{\cm} in depth and its weight is estimated from \SIrange{7}{16}{\gram} for an adult~\cite{Leissner1979}.
The prostate size increases at two distinct stages during physical development: initially at puberty to reach its normal size, then again after 60 years of age leading to \ac{bph}~\cite{Parfait2010}.

A zonal classification of the prostate has been suggested by \citeauthor{McNeal1981}~\cite{McNeal1981}, as depicted in \acs{fig}\,\ref{fig:anatomyProstateZone}.
Subsequently, this categorization has been widely accepted in the literature~\cite{Hricak1987,Villers1991,Coakley2000,Parfait2010} and is used during all medical examinations (e.g., biopsy, \ac{mri} screening).
The classification is based on dividing the gland into 3 distinct regions: (i) the \ac{cz} accounting for \SIrange{20}{25}{\percent} of the whole prostate gland, (ii) the \ac{tz} standing for \SI{5}{\percent}, and (iii) the \ac{pz} representing the \SI{70}{\percent}.
In \ac{mri} images, tissues of \ac{cz} and \ac{tz} are difficult to distinguish and are usually merged into a common region, denominated \ac{cg}.
As part of this classification, the prostate is divided into 3 longitudinal portions depicted in \acs{fig}\,\ref{fig:anatomyProstateSagittal}: (i) base, (ii) median gland, and (iii) apex.

%% \begin{figure}
%% 	\centering
%% 	\includegraphics[width=0.65\textwidth]{anatomy/prostate2D.eps}
%% 	\caption{Sagittal anatomy scheme of the male reproductive system \cite{Geckomedia2011}.}
%% 	\label{fig:intro:prostatecancer:anatomy:anatomyProstate2D}
%% \end{figure}


%% \begin{figure}
%% 	\centering
%% 	\includegraphics[width=0.50\textwidth]{anatomy/prostate2D2.eps}
%% 	\caption{Representation of the prostate. 1: Vas deferens, 2: Ampulla, 3: Seminal vesicle, 4: Excretory duct of seminal vesicle, 5: Prostate contour, 6: Ejaculatory duct, 7: Prostatic urticle, 8: Glandular tissue, 9: Urethral sphincter, 10: Urethra, 11: Seminal colliculus, 12: Urethral crest \cite{Wikipedia2011}.}
%% 	\label{fig:intro:prostatecancer:anatomy:anatomyProstate2D2}
%% \end{figure}


%% \begin{figure}
%% 	\centering
%% 	\subfigure[Transverse anatomy of the prostate.]{
%% 			\centering
%% 			\includegraphics[width=0.4\textwidth]{anatomy/prostateTransverse.eps}
%% 			\label{fig:anatomyProstateTransverse}}
%% 	~~~
%% 	\subfigure[Sagital anatomy of the prostate.]{
%% 			\centering
%% 			\includegraphics[width=0.4\textwidth]{anatomy/prostateSagital.eps}
%% 			\label{fig:anatomyProstateSagital}}
%% 	\caption{Presentation of the different zones of the prostate. \textit{AFT:} anterior fibromuscular tissue, \textit{CZ:} central zone, \textit{ED:} ejaculatory duct, \textit{NVB:} neurovascular bundle, \textit{PUT:} periurethral tissue, \textit{PZ:} peripherical zone, \textit{U:} urethra, \textit{TZ:} transitional zone \cite{Choi2007}.}
%% 	\label{fig:intro:prostatecancer:anatomy:anatomyProstateZone}
%% \end{figure}

\section{Prostate carcinoma}
Prostate cancer \ac{cap} has been reported on a worldwide scale to be the second most frequently diagnosed cancer of men accounting for \SI{13.6}{\percent}~\cite{Ferlay2010}.
Statistically, in 2008, the number of new diagnosed cases has been estimated to be $899,000$ with no less than $258,100$ deaths~\cite{Ferlay2010}.
In United States, aside from skin cancer, \ac{cap} is declared to be the most commonly diagnosed cancer among men, implying that approximately 1 in 6 men will be diagnosed with \ac{cap} during their lifetime and 1 in 36 will die from this disease, causing \ac{cap} to be the second most common cause of cancer death among men~\cite{Siegel2013,Society2013}.

Despite active research to determine the causes of \ac{cap}, a fuzzy list of risk factors has arisen~\cite{Society2010}.
The etiology has been linked to the following factors~\cite{Society2010}: (i) family history~\cite{Giovannucci2007,Steinberg1990}, (ii) genetic factors~\cite{Freedman2006,Amundadottir2006,Agalliu2009}, (iii) race-ethnicity~\cite{Giovannucci2007,Hoffman2001}, (iv) diet~\cite{Giovannucci2007,Ma2009,Alexander2010}, and (v) obesity~\cite{Giovannucci2007,Rodriguez2007}.
This list of risk factors alone cannot be used to diagnose \ac{cap} and in this way, screening enables early detection and treatment.

\ac{cap} growth is characterized by two main types of evolution~\cite{Strum2005}: slow-growing tumours, accounting for up to \SI{85}{\percent} of all \acp{cap}~\cite{Lu-Yao2009}, progress slowly and usually stay confined to the prostate gland.
For such cases, treatment can be substituted with active surveillance.
In contrast, the second variant of \acp{cap} develops rapidly and metastasises from prostate gland to other organs, primarily the bones~\cite{Oster2013}.
Bone metastases, being an incurable disease, significantly affects the morbidity and mortality rate~\cite{Ye2007}.
Hence, the results of the surveillance have to be trustworthy in order to distinguish aggressive from slow-growing \ac{cap}.

\ac{cap} is more likely to come into being in specific regions of the prostate.
In that respect, around \SIrange{70}{80}{\percent} of \acp{cap} originate in \ac{pz} whereas \SIrange{10}{20}{\percent} in \ac{tz}~\cite{Carrol1987,McNeal1988,Stamey1998}.
Only about \SI{5}{\percent} of \acp{cap} occur in \ac{cz}~\cite{McNeal1988,Cohen2008}.
However, those cancers appear to be more aggressive and more likely to invade other organs due to their locations~\cite{Cohen2008}.




%%%%%%%%%%%%%%%%%%%%%%%%%% From previous draft %%%%%%%%%%%%%%%%%%%%%%%%%%%%%%%%%%%%%%%%%%%%%%%%

%% \subsection{Statistics}\label{subsection:intro:prostatecancer:statistics}

%% \subsubsection{Overview}\label{subsubsection:intro:prostatecancer:statistics:overview}

%% The World Health Organization (WHO) published in 2008 that PCa was the second most frequently diagnosed cancer of men and the fifth most common cancer overall \cite{Ferlay2010}. No less than 899,000 new cases where detected worldwide in 2008 \cite{Ferlay2010}. As presented on Fig. ~\ref{fig:intro:prostatecancer:statistics:overview:repartitionCancer}, PCa accounts for approximately 7.1\% (Fig. ~\ref{fig:intro:prostatecancer:statistics:overview:repartitionCancerIncidence}) of all cancers diagnosed in 2008 and 3.4\% (Fig. ~\ref{fig:intro:prostatecancer:statistics:overview:repartitionCancerDeaths}) of all cancers deaths in 2008 \cite{Ferlay2010}.

%% \begin{figure}
%% 	\centering
%% 	\subfigure[Estimated number cancers cases for both sexes and all ages.]{
%% 			\centering
%% 			\includegraphics[width=0.65\textwidth]{statistics/repartitionCancerIncidence.eps}
%% 			\label{fig:intro:prostatecancer:statistics:overview:repartitionCancerIncidence}}
%% 	~
%% 	\subfigure[Estimated number cancers deaths for both sexes and all ages.]{
%% 			\centering
%% 			\includegraphics[width=0.65\textwidth]{statistics/repartitionCancerDeaths.eps}
%% 			\label{fig:intro:prostatecancer:statistics:overview:repartitionCancerDeaths}}
%% 	\caption{Cancer estimations in 2008 by the World Health Organization (WHO) \cite{Ferlay2010}.}
%% 	\label{fig:intro:prostatecancer:statistics:overview:repartitionCancer}
%% \end{figure}

%% \subsubsection{Risk Factors}\label{subsubsection:intro:prostatecancer:statistics:riskfactors}

%% The risk factors can be categorized in three different classes: 

%% \begin{itemize}
%% 	\item Age: age is the most important risk factor for PCa. The diagnosis of PCa for men over 50 years old. PCa rate increases upto about 70 and declines thereafter \cite{AmericanCancerSociety2010}.
%% 	\item Genetic factors: it has been shown that the probability to have a cancer is higher when a member of the family has been already diagnosed \cite{AmericanCancerSociety2010}.
%% 	\item Race: in the United States, the Africo Americans have a higher probability of developing a PCa than European American and Hispanic men \cite{AmericanCancerSociety2010}.
%% \end{itemize}

%% \subsection{Diagnosis and Medical Exams}\label{subsubsection:intro:prostatecancer:diagnosis}

%% The presence of PCa may be suggested in several ways: digital rectal examination, Prostate Specific Antigen (PSA\g) test, biopsy using transrectal ultrasound (TRUS\g) and magnetic resonance imaging (MRI\g-MRSI\g).

%% \subsubsection{Digital Rectal Examination}\label{subsubsection:intro:prostatecancer:diagnosis:rectalexamination}
%% Both benign prostatic hyperplasia and cancer may lead to an increasing size of the prostate. A rectal examination may allow detection of harder nodules within the softer prostatic tissue. The advantages are that this method is very fast and does not need any special equipment.
%% \subsubsection{PSA test}\label{subsubsection:intro:prostatecancer:diagnosis:psa}
%% The PSA is a protein secreted by the prostate. A higher-than-normal level of PSA can indicate an abnormality of the prostate: a benign prostatic hyperplasia or a cancer. However, other factors can lead to an increasing level of PSA such as prostate infections, irritations, a recent ejaculation or a recent rectal examination, etc.
%% The PSA can be found in the blood in two different forms: free PSA (about 10\%) and linked to another protein (about 90\%).
%% A level of PSA higher than 10 $ng.mL^{-1}$ is considered as pathologic \cite{Parfait2010}. If the PSA level is between 10 $ng.mL^{-1}$ and 4 $ng.mL^{-1}$, the patient is considered as suspicious \cite{Parfait2010}. In that case, the ratio free PSA over total PSA is computed. If the ratio is higher than 15\%, the case is considered as pathologic.
%% \subsubsection{TRUS}\label{subsubsection:intro:prostatecancer:diagnosis:trus}
%% As described in Sect. ~\ref{subsection:intro:prostatecancer:anatomy}, the prostate is localized in front of the rectum. Hence, its position allows one to carry out a biopsy using transrectal ultrasound (TRUS) in order to localise more precisely an eventual cancer (Fig. ~\ref{fig:intro:prostatecancer:diagnosis:trus}).
%% \textbf{\textit{\textsc{Add example of images of TRUS PCa and not}}}
%% \begin{figure}
%% 	\centering
%% 	\includegraphics[width=0.45\textwidth]{diagnosis/trus/trus.eps}
%% 	\caption{Biopsy of the prostate using TRUS}
%% 	\label{fig:intro:prostatecancer:diagnosis:trus}
%% \end{figure}
%% \textbf{\textit{\textsc{Add information about protocol: manipulation of the patients, which equipment (see Jhimli thesis)}}}
%% The biopsy is usually prescribed when the PSA level is higher-than-normal or abnormalities were detected during a rectal examination. At least six different samples are taken from the right and left parts of the three different zones: apex, median and base. The samples are analysed in order to determine the presence of a cancer.
%% \textbf{\textit{\textsc{Add more information on the specificities and accuracy of the techniques. Add also what are the advantages (real-time)}}}

\section{\acs*{cap} screening and imaging techniques}\label{sec:intro:screening}
%\subsection{Current \ac{cap} screening}\label{subsec:intro:current-screening}

Current \ac{cap} screening consists of three different stages.
First, \ac{psa} control is performed to distinguish between low- and high-risk \ac{cap}.
To assert such diagnosis, samples are taken during prostate biopsy and finally analyzed to evaluate the prognosis and the stage of \ac{cap}.
In this section, we present a detailed description of the current screening as well as its drawbacks.

Since its introduction in mid-1980s, \ac{psa} is widely used for \ac{cap} screening~\cite{Etzioni2002}.
A higher-than-normal level of \ac{psa} can indicate an abnormality of the prostate either as a \ac{bph} or a cancer~\cite{Hoeks2011}.
However, other factors can lead to an increased \ac{psa} level such as prostate infections, irritations, a recent ejaculation, or a recent rectal examination~\cite{Parfait2010}.
\ac{psa} is found in the bloodstream in two different forms: free \ac{psa} accounting for about \SI{10}{\percent} and linked to another protein for the remaining \SI{90}{\percent}.
A level of \ac{psa} higher than \SI{10}{\nano\gram\per\milli\liter} is considered to be at risk~\cite{Parfait2010}.
If the \ac{psa} level is ranging from \SIrange{4}{10}{\nano\gram\per\milli\liter}, the patient is considered as suspicious~\cite{Barentsz2012}.
In that case, the ratio of free \ac{psa} to total \ac{psa} is computed; if the ratio is higher than \SI{15}{\percent}, the case is considered as pathological~\cite{Parfait2010}.

\Iac{trus} biopsy is carried out for cases which are considered pathological.
At least 6 different samples are taken randomly from the right and left parts of the 3 different prostate zones: apex, median, and base.
These samples are further evaluated using the Gleason grading system~\cite{Gleason1977}.
The scoring scheme to characterize the biopsy sample is composed of 5 different patterns which correspond to grades ranging from 1 to 5.
A higher grade is associated with a poorer prognosis~\cite{Epstein2005}.
Then, in the Gleason system, 2 scores are assigned corresponding to (i) the grade of the most present tumour pattern, and (ii) the grade of the second most present tumour pattern~\cite{Epstein2005}.
A higher \ac{gs} indicates a more aggressive tumour~\cite{Epstein2005}.
Also, it should be noted that biopsy is an invasive procedure which can result in serious infection or urine retention~\cite{Hara2005,Chou2011}.

Although \ac{psa} screening has been shown to improve early detection of \ac{cap}~\cite{Chou2011}, its lack of reliability motivates further investigations using \ac{mri}-based \ac{cad}.
Two reliable studies --- carried out in the United States~\cite{Andriole2009} and in Europe~\cite{Schroeder2012, Hugosson2010} --- have attempted to assess the impact of early detection of \ac{cap}, with diverging outcomes~\cite{Chou2011,Heidenreich2013}.
The study carried out in Europe\footnote{The \ac{ersspc} started in the 1990s in order to evaluate the effect of \ac{psa} screening on mortality rate.} concluded that \ac{psa} screening reduces CaP-related mortality by \SIrange{21}{44}{\percent}~\cite{Schroeder2012, Hugosson2010}, while the American\footnote{The \ac{plco} cancer screening trial is carried out in the United States and intends to ascertain the effects of screening on mortality rate.} trial found no such effect~\cite{Andriole2009}.
However, both studies agree that \ac{psa} screening suffers from low specificity, with an estimated rate of \SI{36}{\percent}~\cite{Schroder2008}.
Both studies also agree that over-treatment is an issue: decision making regarding treatment is further complicated by difficulties in evaluating the aggressiveness and progression of \ac{cap}~\cite{Delpierre2013}. 

Hence, new screening methods should be developed with improved specificity of detection as well as more accurate risk assessment (i.e., aggressiveness and progression).
Current research is focused on identifying new biological markers to replace \ac{psa}-based screening~\cite{Bourdoumis2010,Morgan2011,Brenner2013}.
Until such research comes to fruition, these needs can be met through active-surveillance strategy using \ac{mpmri} techniques~\cite{Hoeks2011,Moore2013}.
An \ac{mri}-\acs{cad} system, which is an area of active research and forms the focus of this thesis, can be incorporated into this screening strategy allowing a more systematic and rigorous follow-up.

Another weakness of the current screening strategy lies in the fact that \ac{trus} biopsy does not provide trustworthy results.
Due to its ``blind'' nature, there is a chance of missing aggressive tumours or detecting microfocal ``cancers'', which influences the aggressiveness-assessment procedure~\cite{Noguchi2001}.
As a consequence, over-diagnosis is estimated at up to \SI{30}{\percent}~\cite{Haas2007}, while missing clinically significant \ac{cap} is estimated at up \SI{35}{\percent}~\cite{Taira2010}.
In an effort to solve both issues, alternative biopsy approaches have been explored.
\ac{mri}/\ac{us}-guided biopsy has been shown to outperform standard \ac{trus} biopsy~\cite{Delongchamps2013}.
There, \ac{mpmri} images are fused with \ac{us} images in order to improve localization and aggressiveness assessment to carry out biopsies.
Human interaction plays a major role in biopsy sampling which can lead to low repeatability; by reducing potential human errors at this stage, the \acs{cad} framework can be used to improve repeatability of examination.
\ac{cap} detection and diagnosis can benefit from the use of \acs{cad} and \ac{mri} techniques.

In an effort to improve the current stage of \ac{cap} diagnosis and detection, this thesis is intended to develop the principles of a \ac{mpmri}-\acs{cad} system. 
A description of the different \ac{mri} modalities is presented in \acs{chp}\,\ref{chap:2}. 
%In the following sections, these techniques will be presented in addition to an overview of \acs{cad} for \ac{cap}.

\section{\acs*{cad} systems for \acs*{cap}}\label{sec:intro:cad} 
During the last century, physicists have focused on constantly innovating in terms of imaging techniques assisting radiologists to improve cancer detection and diagnosis.
However, human diagnosis still suffers from low repeatability, synonymous with erroneous detection or interpretations of abnormalities throughout clinical decisions~\cite{Giger2008,Hambrock2013}.
These errors are driven by two majors causes~\cite{Giger2008}: observer limitations (e.g., constrained human visual perception, fatigue or distraction) and the complexity of the clinical cases themselves, for instance due to imbalanced data --- the number of healthy cases is more abundant than malignant cases --- or overlapping structures.
%% On the one hand, observer limitations (e.g., constrained human visual perception, fatigue or distraction) are the principal human issues.
%% On the other hand, the second reason is linked to the clinical cases themselves, for instance due to unbalanced data (number of healthy cases more abundant than malignant cases) or overlapping structures resulting from limitations of imaging techniques.

Computer vision has given rise to many promising solutions, but, instead of focusing on fully automatic computerized systems, researchers have aimed at providing computer image analysis techniques to aid radiologists in their clinical decisions~\cite{Giger2008}.
In fact, these investigations brought about both concepts of \ac{cade} and \ac{cadx} grouped under the acronym \ac{cad}.
Since those first steps, evidence has shown that \ac{cad} systems enhance the diagnosis performance of radiologists.
\citeauthor{Chan1999} reported a significant \SI{4}{\percent} improvement in breast cancer detection~\cite{Chan1999}, which has been confirmed in later studies~\cite{Dean2006}.
Similar conclusions have been drawn in the case of lung nodule detection~\cite{Li2004}, colon cancer~\cite{Petrick2008}, or \ac{cap} as well~\cite{Hambrock2013}.
\citeauthor{Chan1999} also hypothesized that \acs{cad} systems will be even more efficient assisting inexperienced radiologists than senior radiologists~\cite{Chan1999}.
That hypothesis has been tested by \citeauthor{Hambrock2013} and confirmed in case of \ac{cap} detection~\cite{Hambrock2013}.
In this particular study, inexperienced radiologists obtained equivalent performance to senior radiologists, both using \acs{cad} whereas the accuracy of their diagnosis was significantly poorer without \ac{cad}'s help.

In contradiction with the aforementioned statement, \ac{cad} for \ac{cap} is a young technology due to the fact that is based on a still young imaging technology: \ac{mri}~\cite{Hegde2013}.
Indeed, four distinct \ac{mri} modalities are employed in \ac{cap} diagnosis which have been mainly developed after the mid-1990s: (i) \ac{t2w}-\ac{mri}~\cite{Hricak1983}, (ii) \ac{dce}-\ac{mri}~\cite{HuchBoni1995}, (iii) \ac{mrsi}~\cite{Kurhanewicz1996}, and (iv) \ac{dw}-\ac{mri}~\cite{Scheidler1999}.
In addition, the increase of magnetic field strength in clinical settings, from \SIrange{1.5}{3}{\tesla}, and the development of endorectal coils, both improved image spatial resolution~\cite{Swanson2001} needed to perform more accurate diagnosis.
It is for this matter that the development of \ac{cad} for \ac{cap} is still lagging behind the fields stated above.

The further chapters aim at first, to provide an overview of the current state-of-the-art of \ac{cad} for \ac{cap} and later, according to the drawn conclusions, to propose a \ac{cad} which takes advantages of \ac{mpmri} modalities. 
A review of the current proposed \ac{cad} for \ac{cap} is presented in \acs{chp}\,\ref{chap:3}.

 
%% It can be noted that these techniques came into existence relatively recently mainly due to technological progress. In addition, the increase of magnetic field strength and the development of endorectal coil, both improved image spatial resolution \cite{Swanson2001}) needed to perform more accurate diagnosis. It is for this matter that development of \acs{cad} for \ac{cap} is lagging behind the other fields stated above.

%% In the late eighties, the first \acs{cad} systems were developed to detect anomalies on chest radiographies and mammograms \cite{Doi1987,Chan1987,Giger1988}).
%% In the past twenty years, extensive investigations were conducted in the advancement of \acs{cad} systems, migrating from intensive time consuming algorithms performed on reduced number of cases to ``fast'' processing on a large medical dataset. These works were focused on diverse organ cancer diagnosis making use of numerous imaging modalities: micro-calcification detection in breast mammography \cite{Rangayyan2007,Elter2009}) and \ac{us} imaging \cite{Cheng2010}), lung nodules detection based on \ac{ct} \cite{Chan2008,Suzuki2012}), colon tumours detection \cite{Suzuki2012}) and melanoma detection using dermoscopy imaging \cite{Korotkov2012}). Noting the abundance of diverse \acs{cad} systems, these fields achieved a certain maturity which can be explained by the imaging techniques employed. Indeed, x-rays, \ac{us} as well as \ac{ct} are medical imaging techniques developed all before the 1970s and were subject to intensive research.


%% The first study using \ac{mri} as inputs of \acs{cad} system was published ten years ago by \cite{Chan2003}. Despite this, no less than fifty studies have been reviewed for this survey since that seminal work. To the best of our knowledge, there is no review in the literature regarding the advancement of \acs{cad} systems devoted specifically to \ac{cap} detection and diagnosis. Thus, our aim with this survey is threefold: (i) provide an overview of developed \acs{cad} systems for \ac{cap} detection and diagnosis based on \ac{mri} modalities (ii) assess the different work and (iii) pointing out avenues for future work.

%% As discussed further in Sect.~\ref{subsubsec:CAD}, \acs{cad} systems share a common framework. Stages involved in \acs{cad} work-flow can be categorized into six distinctive processes: (i) pre-processing, (ii) segmentation, (iii) registration, (iv) feature detection, (v) feature selection and extraction and (vi) classification. The first three stages are used to enhance data as well as to extract regions of interest and, in the case of multi-modal sources, to merge information of those heterogeneous sources in a joint reference system. The last three categories deal with pattern recognition, machine learning and data mining notions and more precisely with the data classification problem. First, information is detected from the different data sources and a subset of relevant features is selected and/or extracted. Then, this meaningful data will then be classified in order to provide the probability of malignancy of the area of interest and will assist radiologists in their diagnosis decisions (see Fig.~\ref{fig:wkfcad}).

%% %% This paper is organized as follows: Sect.~\ref{sec:background} deals with general information about human prostate and background about \ac{cap}. Methods regarding \ac{cap} screening and imaging techniques used are also presented as well as an introduction on the \acs{cad} framework. Sections~\ref{sec:imaprocfra} -~\ref{sec:dataclassfra} review techniques used in different steps involved in a \acs{cad} work-flow which will be our main contribution. Image regularization framework including pre-processing (Sect.~\ref{subsec:preprocessing}), segmentation (Sect.~\ref{subsec:segmentation}) and registration (Sect.~\ref{subsec:registration}) will be covered as well as the image classification framework comprising of feature detection (Sect.\ref{subsec:featuredetection}), feature selection and extraction (see Sect.~\ref{subsec:featureselectionextraction}) and feature classification (Sect.~\ref{subsec:classification}). Results and discussion are reported in Sect.~\ref{sec:discussion} followed by a concluding section.



\section{Research objectives}\label{sec:intro:motivation}

%As argue later in \acs{sec}\,\ref{subsec:chp3:dis:gen-dis},

From the previous section, it is obvious that \ac{cap}, as any type of cancers, is one of the major societal challenges in health care.
Until the causes of \ac{cap} are remaining unknown, screening with the use of \ac{cad} systems is the only solution.
Therefore, the main motivation of this thesis is to design and investigate a \ac{cad} system for the detection of \ac{cap}.
Furthermore, \ac{mri} has been shown to offer promising modalities for prostate screening.
Subsequently, the developed \ac{cad} system has to be based on \ac{mpmri} modalities.

To achieve this main objective, we propose to review the current state-of-the-art of the mono- and multi-parametric \ac{cad} systems.
From this review, the current scientific barriers will be identified and our proposed \ac{mpmri} \ac{cad} has addressed these drawbacks.
\section{Thesis outline} \label{sec:intro:outline}

This thesis describes the research work which resulted in the development and validation of several crucial pre-processing steps and ultimately a \ac{mpmri}-\ac{cad} system for the detection of \ac{cap}.
The different \ac{mpmri} modalities are presented in \textbf{\acs{chp}\,\ref{chap:2}}.
Latter towards the main objective of this thesis, an extensive study of the state-of-the-art on mono- and multi-parametric \ac{mri} \ac{cad} systems for \ac{cap} detection is presented in \textbf{\acs{chp}\,\ref{chap:3}}.
\textbf{\acl{chp}}~\ref{chap:4} presents the materials used and produced along by this thesis. 
\textbf{\acl{chp}}~\ref{chap:5} contains our first technical contribution regarding a crucial step of normalization for \ac{t2w}-\ac{mri} and \ac{dce}-\ac{mri} modalities.
\textbf{\acl{chp}}~\ref{chap:6} proposes a \ac{mpmri} \ac{cad} system along with extensive experiments and evaluations.
Finally, \textbf{\acs{chp}}\,\ref{chap:7} concludes the thesis and presents avenues for future research.


	% background information
\acresetall
\chapter{MRI Imaging Techniques}\label{chap:2}

% the code below specifies where the figures are stored
\ifpdf
    \graphicspath{{2_modality/figures/}}
\else
    \graphicspath{{2_modality/figures/}}
\fi

% Prostate Cancer:
%%% This part contains:
%%% - Anatomy basis of the prostate
%%% - Statistics regarding prostate cancers

%\section{MRI principles}\label{sec:chp2:principles}

%\section{\acs*{mri} imaging techniques}\label{sec:chp2:imaging}

\Ac{mri} provides promising imaging techniques to overcome the drawbacks of current clinical screening techniques mentioned in \acs{sec}\,\ref{chap:1}.
Unlike \ac{trus} biopsy, \ac{mri} examination is a non-invasive protocol and has been shown to be the most accurate and harmless technique currently available~\cite{Turkbey2012}.
In this section, we review different \ac{mri} imaging techniques developed for \ac{cap} detection and diagnosis.
Features strengthening each modality will receive particular attention together with their drawbacks.
Commonly, these features form the basis for developing analytic tools and automatic algorithms.
However, we refer the reader to \acs{sec}\,\ref{subsec:chp3:img-clas:CADX-fea-dec} for more details on automatic feature detection methods since they are part and parcel of the \acs{cad} framework.

%% % We are using enumerate with a small margin and some indent to organize our thoughts by paragraphs.
%% \setenumerate{listparindent=\parindent,itemsep=10px}
%% \setlist{noitemsep}

\begin{figure}
\centering
	\hspace*{\fill}
	\subfigure[\acs*{t2w}-\acs*{mri} slice of a healthy prostate acquire with a \SI{1.5}{\tesla} \acs*{mri} with an endorectal coil. The blue contour represents the \acs*{cg} while the \acs*{pz} corresponds to the green contour.]{\label{subfig:t2whealthy}\includegraphics[width=0.3\linewidth]{2_modality/figures/t2w/t2w_healthy.eps}} \hfill
	\subfigure[\acs*{t2w}-\acs*{mri} slice of a prostate with a \acs*{cap} highlighted in the \acs*{pz} using a \SI{3}{\tesla} \acs*{mri} scanner without an endorectal coil.]{\label{subfig:t2wcancerpz}\includegraphics[width=0.3\linewidth]{2_modality/figures/t2w/t2w_cancer_pz.eps}} \hfill
	\subfigure[\acs*{t2w}-\acs*{mri} slice of a prostate with a \acs*{cap} highlighted in the \acs*{cg} using a \SI{1.5}{\tesla} \acs*{mri} scanner with an endorectal coil.]{\label{subfig:t2wcancercg}\includegraphics[width=0.3\linewidth]{2_modality/figures/t2w/t2w_cancer_cg.eps}}
	\hspace*{\fill}
	\caption[Rendering of \acs*{t2w}-\acs*{mri} prostate images.]{Rendering of \acs*{t2w}-\acs*{mri} prostate image with both \SI{1.5}{\tesla} and \SI{3}{\tesla} \acs*{mri} scanner.}
	\label{fig:t2w}
\end{figure}

%T2W \ac{mri}
\section{\acs*{t2w}-\acs*{mri}}\label{subsec:chp2:imaging:t2w} 
\ac{t2w}-\ac{mri} has been the first \ac{mri}-modality used to perform \ac{cap} diagnosis using \ac{mri}~\cite{Hricak1983}.
Nowadays, radiologists make use of it for \ac{cap} detection, localization, and staging purposes.
This imaging technique is well suited to render zonal anatomy of the prostate~\cite{Barentsz2012}. 

This modality relies on a sequence based on setting a long \ac{tr}, reducing the T$_{1}$ effect in \ac{nmr} signal measured, and fixing the \ac{te} to sufficiently large values in order to enhance the T$_{2}$ effect of tissues.
Thus, \ac{pz} and \ac{cg} tissues are well perceptible in these images.
The former is characterized by an intermediate/high-\ac{si} while the latter is depicted by a low-\ac{si}~\cite{Hricak1987}.
An example of a healthy prostate is shown in \acs{fig}\,\ref{subfig:t2whealthy}.

In \ac{pz}, round or ill-defined low-SI masses are synonymous with \acp{cap}~\cite{Hricak1983} as shown in \acs{fig}\,\ref{subfig:t2wcancerpz}.
Detecting \ac{cap} in \ac{cg} is more challenging.
In fact both normal \ac{cg} tissue and malignant tissue, have a low-\ac{si} in \ac{t2w}-\ac{mri}, reinforcing difficulties to distinguish one among them.
However, \acp{cap} in \ac{cg} appear often as homogeneous mass possessing ill-defined edges with lenticular or ``water-drop'' shapes~\cite{Akin2006,Barentsz2012} as depicted in \acs{fig}\,\ref{subfig:t2wcancercg}. 

\ac{cap} aggressiveness has been shown to be inversely correlated with \ac{si}.
Indeed, \acp{cap} assessed with a \ac{gs} of 4-5 implied lower \ac{si} than the one with a \ac{gs} of 2-3~\cite{Wang2008}.

In spite of the availability of these useful and encouraging features, the \ac{t2w} modality lacks reliability~\cite{Kirkham2006,Hoeks2011}.
Sensitivity is affected by the difficulties in detecting cancers in \ac{cg}~\cite{Kirkham2006} while specificity rate is highly affected by outliers~\cite{Barentsz2012}.
In fact, various conditions emulate patterns of \ac{cap} such as \ac{bph}, post-biopsy hemorrhage, atrophy, scars, and post-treatment~\cite{Hricak1987,Quint1991,Scheidler1999,Cruz2002,Barentsz2012}.
These issues are partly addressed using more innovative and advanced modalities.

%T2 Map
\section{T$_2$ map} \label{subsec:chp2:imaging:t2}
As previously mentioned, \ac{t2w}-\ac{mri} modality shows low sensitivity.
Moreover, \ac{t2w}-\ac{mri} images are a composite of multiple effects~\cite{Hegde2013}.
However, T$_2$ values alone have been shown to be more discriminative~\cite{Liu2011} and highly correlated with citrate concentration, a biological marker in \ac{cap}~\cite{Liney1996,Liney1997}.

T$_2$ values are computed using the characteristics of transverse relaxation which is formalized as in \acs{eq}\,\eqref{eq:tramag}.

\begin{equation}
	M_{xy}(t) = M_{xy}(0) \exp \left( - \frac{t}{\text{T}_2} \right) \ ,
	\label{eq:tramag}
\end{equation}

\noindent where $M_{xy}(0)$ is the initial value of $M_{xy}(t)$ and T$_2$ is the relaxation time.

By rearranging \acs{eq}\,\eqref{eq:tramag}, T$_2$ map is computed by performing a linear fitting on the model presented in \acs{eq}\,\eqref{eq:t2map} using several TE, $t=\{ \text{TE}_1,\text{TE}_2, \dotsc ,\text{TE}_m \}$.

\begin{equation}
	\ln \left[ \frac{M_{xy}(t)}{M_{xy}(0)} \right] = - \frac{t}{\text{T}_2} \ .
	\label{eq:t2map}
\end{equation}

The \Ac{fse} sequence has been shown to be particularly well suited in order to build a T$_2$ map and obtain accurate T$_2$ values~\cite{Liney1996a}.
Similar to \ac{t2w}-\ac{mri}, T$_2$ values associated with \ac{cap} are significantly lower than those of healthy tissues~\cite{Liney1996,Gibbs2001}.

\begin{figure}
\centering
	\hspace*{\fill}
	\subfigure[\acs*{t1w}-\acs*{mri} image where the cancer is delimited by the red contour. The green area was still not invaded by the \acs*{cap}]{\label{subfig:t1w}\includegraphics[width=0.4\linewidth]{2_modality/figures/dce/slice.eps}} \hfill
	\subfigure[Enhancement curve computed during the \acs*{dce}-\acs*{mri} analysis. The red curve is typical from \acs*{cap} cancer while the green curve is characteristic of healthy tissue.]{\label{subfig:dce}\includegraphics[width=0.45\linewidth]{2_modality/figures/dce/dce_cancer_healthy.eps}}
	\hspace*{\fill}
	\caption[Enhancement of \acs*{dce}-\acs*{mri} signal.]{Illustration of typical enhancement signal observed in \acs*{dce}-\acs*{mri} analysis collected with a \SI{3}{\tesla} \acs*{mri} scanner.}
	\label{fig:dceana}
\end{figure}

%DCE \ac{mri}
\section{\acs*{dce}-\acs*{mri}}\label{subsec:chp2:imaging:dce}
\ac{dce}-\ac{mri} is an imaging technique which exploits the vascularity characteristic of tissues.
Contrast media, usually gadolinium-based, is injected intravenously into the patient.
The media extravasates from vessels to \ac{ees} and is released back into the vasculature before being eliminated by the kidneys~\cite{Gribbestad2005}.
Furthermore, the diffusion speed of the contrast agent may vary due to several parameters: (i) the permeability of the micro-vessels, (ii) their surface area, and (iii) the blood flow~\cite{Padhani2002}.

Healthy \ac{pz} is mainly made up of glandular tissue, around \SI{70}{\percent}~\cite{Choi2007}, which implies a reduced interstitial space restricting exchanges between vessels and \ac{ees}~\cite{Buckley2004,Niekerk2009}.
Normal \ac{cg} has a more disorganized structure, composed of mainly fibrous tissue~\cite{Choi2007,Hoeks2011}, which facilitates the arrival of the contrast agent in \ac{ees}~\cite{Niekerk2013}.
To understand the difference between contrast media kinetic in malignant tumours and the two previous behaviours mentioned, one has to focus on the process known as angiogenesis~\cite{Carmeliet2000}.
In order to ensure growth, malignant tumours produce and release angiogenic promoter substances~\cite{Carmeliet2000}.
These molecules stimulate the creation of new vessels towards the tumour~\cite{Carmeliet2000}.
However, the new vessel networks in tumours differ from those present in healthy tissue~\cite{Gribbestad2005}.
They are more porous due to the fact that their capillary walls have a large number of ``openings''~\cite{Gribbestad2005,Choi2007}.
In contrast to healthy cases, this increased vascular permeability results in increased contrast agent exchanges between vessels and \ac{ees}~\cite{Verma2012}.

By making use of the previous aspects, \ac{dce}-\ac{mri} is based on an acquisition of a set of \ac{t1w}-\ac{mri} images over time.
The gadolinium-based contrast agent shortens T$_1$ relaxation time enhancing contrast in \ac{t1w}-\ac{mri} images.
The aim is to post-analyze the pharmacokinetic behaviour of the contrast media concentration in prostate tissues~\cite{Verma2012}.
The image analysis is carried out in two dimensions: (i) in the spatial domain on a pixel-by-pixel basis and (ii) in the time domain corresponding to the consecutive images acquired with the \ac{mri}.
Thus, for each spatial location, a signal linked to contrast media concentration is measured as shown in \acs{fig}\,\ref{subfig:dce}~\cite{Tofts2010}. 

By taking the above remarks into account, \acp{cap} is characterized by a signal having an earlier and faster enhancement and an earlier wash-out --- i.e, the rate of the contrast agent flowing out of the tissue --- as shown in \acs{fig}\,\ref{subfig:dce}~\cite{Verma2012}.
Three different approaches exist to analyze these signals with the aim of labelling them as corresponding to either normal or malignant tissues.

Qualitative analysis is based on a qualitative assessment of the signal shape~\cite{Hoeks2011}.
Quantitative approaches consist of inferring pharmocokinetic parameter values~\cite{Tofts2010}.
Those parameters are part of mathematical-pharmacokinetic models which are directly based on physiological exchanges between vessels and \ac{ees}.
Several pharmacokinetic models have been proposed such as the Kety model~\cite{Kety1951}, the Tofts model~\cite{Tofts1997}, and mixed models~\cite{Larsson1996,StLawrence1998}.
The last family of methods mixed both approaches and are grouped together under the heading of semi-quantitative methods.
They rely on shape characterization using mathematical modelling to extract a set of parameters such as wash-in gradient, wash-out, integral under the curve, maximum signal intensity, time-to-peak enhancement, and start of enhancement~\cite{Hoeks2011,Verma2012}.
These parameters are depicted in \acs{fig}\,\ref{fig:dceparam}.
It has been shown that semi-quantitative and quantitative methods improve localization of \ac{cap} when compared with qualitative methods~\cite{Rosenkrantz2013}.
\Ac{sec}~\ref{subsubsec:chp3:img-clas:CADX-fea-dec:DCE-fea} provides a full description of quantitative and semi-quantitative approaches.

\ac{dce}-\ac{mri} combined with \ac{t2w}-\ac{mri} has shown to enhance sensitivity compared to \ac{t2w}-\ac{mri} alone~\cite{Jager1997,Kim2005,Schlemmer2004,Zelhof2009}.
Despite this fact, \ac{dce}-\ac{mri} possesses some drawbacks.
Due to its ``dynamic'' nature, patient motions during the image acquisition lead to spatial mis-registration of the image set~\cite{Verma2012}.
Furthermore, it has been suggested that malignant tumours are difficult to distinguish from prostatitis located in \ac{pz} and \ac{bph} located in \ac{cg}~\cite{Hoeks2011,Verma2012}.
These pairs of tissues tend to have similar appearances.
Later studies have shown that \acp{cap} in \ac{cg} do not always manifest in homogeneous fashion.
Indeed, tumours in this zone can present both hypo-vascularization and hyper-vascularization which illustrates the challenge of \ac{cap} detection in \ac{cg}~\cite{Niekerk2013}.

\begin{figure}
\centering
	\hspace*{\fill}
	\subfigure[\acs*{dw}-\acs*{mri} image acquired with a \SI{1.5}{\tesla} \acs*{mri} scanner. The cancer corresponds to the high \acs*{si} region highlighted in red.]{\label{subfig:dwi}\includegraphics[width=0.25\linewidth]{2_modality/figures/dwi/dwi_cancer.eps}} \hfill
	\subfigure[\acs*{adc} map computer after acquisition of \acs*{dw}-\acs*{mri} images with \SI{1.5}{\tesla} \acs*{mri} scanner. The cancer corresponds to the low \acs*{si} region highlighted in red.]{\label{subfig:adc}\includegraphics[width=0.25\linewidth]{2_modality/figures/dwi/adc_cancer.eps}}
	\hspace*{\fill}
	\caption[Example of \acs*{dw}-\acs*{mri} and \acs*{dce} map.]{Illustration of of \acs*{dw}-\acs*{mri} and \acs*{adc} map. The signal intensity corresponding to cancer are inversely correlated on these modalities.}
	\label{fig:dwi}
\end{figure}

%DWI \ac{mri}
\section{\acs*{dw}-\acs*{mri}}\label{subsec:chp2:imaging:dw}
As previously mentioned in the introduction, \ac{dw}-\ac{mri} is the most recent \ac{mri} imaging technique aiming at \ac{cap} detection and diagnosis~\cite{Scheidler1999}.
This modality exploits the variations in the motion of water molecules in different tissues~\cite{LeBihan1988,Koh2007}.

The distinction between healthy and \ac{cap} in \ac{dw}-\ac{mri} is based on the following physiological bases.
On the one hand, \ac{pz}, as previously mentioned, is mainly a glandular and tubular structure allowing water molecules to move freely~\cite{Choi2007,Hoeks2011}.
On the other hand, \ac{cg} is made up of muscular or fibrous tissue causing the motion of the water molecules to be more constrained and heterogeneous than in \ac{pz}~\cite{Hoeks2011}.
Then, \ac{cap} growth leads to the destruction of normal glandular structure and is associated with an increase in cellular density~\cite{Hoeks2011,Koh2007,Somford2008}.
Furthermore, these factors both have been shown to be inversely correlated with water diffusion~\cite{Koh2007,Somford2008}: higher cellular density implies a restricted water diffusion.
Thus, water diffusion in \ac{cap} will be more restricted than both healthy \ac{pz} and \ac{cg}~\cite{Koh2007,Hoeks2011}.

From the \ac{nmr} principle side, \ac{dw}-\ac{mri} sequence produces contrasted images due to variation of water molecules motion.
The method is based on the fact that the signal in \ac{dw}-\ac{mri} images is inversely correlated to the degree of random motion of water molecules~\cite{Huisman2003}.
In fact, gradients are used in \ac{dw}~\ac{mri} modality to encode spatial location of nuclei temporarily.
Simplifying the problem in only one direction, a gradient is applied in that direction, dephasing the spins of water nuclei.
Hence, the spin phases vary along the gradient direction depending of the gradient intensity at those locations.
Then, a second gradient is applied aiming at cancelling the spin dephasing.
Thus, the immobile water molecules will be subject to the same gradient intensity as the initial one while moving water molecules will be subject to a different gradient intensity.
Thus, spins of moving water molecules will stay dephased whereas spins of immobile water molecules will come back in phase.
As a consequence, a higher degree of random motion results in a more significant signal loss whereas a lower degree of random motion is synonymous with lower signal loss~\cite{Huisman2003}.
Under these conditions, the \ac{mri} signal is measured as:

\begin{align}
  M_{x,y}\left(t,b\right) & = M_{x,y}(0) \exp \left( - \frac{t}{\text{T}_2} \right) S_{\text{ADC}}(b) \ , \label{eq:t2dif} \\
  S_{\text{ADC}}(b) & = \exp \left( -b \times \text{ADC} \right) \ , \label{eq:dif}
\end{align}

\noindent where $S_{\text{ADC}}$ refers to signal drop due to diffusion effect, $\text{ADC}$ is the \acl{adc}, and $b$ is the attenuation coefficient depending only on the gradient pulses parameters: (i) gradient intensity and (ii) gradient duration~\cite{LeBihan1986}.

By using this formulation, image acquisition with a parameter $b$ equal to \SI{0}{\second\per\milli\metre\squared} corresponds to a \ac{t2w}-\ac{mri} acquisition.
Then, increasing the attenuation coefficient $b$ --- i.e., increase gradient intensity and duration --- enhances the contrast in \ac{dw}-\ac{mri} images.

To summarize, in \ac{dw}-\ac{mri} images, \acp{cap} are characterized by high-\ac{si} compared to normal tissues in \ac{pz} and \ac{cg} as shown in \acs{fig}\,\ref{subfig:dwi}~\cite{Barentsz2012}.
However, some tissues in \ac{cg} can look similar to \ac{cap} with higher \ac{si}~\cite{Barentsz2012}.

Diagnosis using \ac{dw}-\ac{mri} combined with \ac{t2w}-\ac{mri} has shown a significant improvement compared with \ac{t2w}-\ac{mri} alone and provides highly contrasted images~\cite{Shimofusa2005,Padhani2011,Choi2007}.
As drawbacks, this modality suffers from poor spatial resolution and specificity due to false positive detection~\cite{Choi2007}.
With a view to eliminate these drawbacks, radiologists use quantitative maps extracted from \ac{dw}-\ac{mri}, which is presented in the next section.

%ADC map
\section{\acs*{adc} map}\label{subsec:chp2:imaging:adc} 
The \ac{nmr} signal measured for \ac{dw}-\ac{mri} images is not only affected by diffusion as shown in \acs{eq}\,\eqref{eq:t2dif}.
However, the signal drop --- \acs{eq}\,\eqref{eq:dif} --- is formulated such that the only variable is the acquisition parameter $b$~\cite{LeBihan1986}.
The \ac{adc} is considered as a ``pure'' diffusion coefficient and is extracted to build a quantitative map known as the \acs{adc} map.
From \acs{eq}\,\eqref{eq:t2dif}, it is clear that performing multiple acquisitions only varying $b$ will not have any effect on the term  $M_{x,y}(0) \exp \left( - \frac{t}{\text{T}_2} \right)$.
Thus, \acs{eq}\,\eqref{eq:t2dif} can be rewritten as:
\begin{equation}
	S(b) = S_0 \exp \left( -b \times \text{ADC} \right) \ .
	\label{eq:t2adcrew}
\end{equation}

To compute the \ac{adc} map, a minimum of two acquisitions are necessary: (i) for $b$ equal to \SI{0}{\second\per\milli\metre\squared} where the measured signal is equal to $S_0$, and (ii) $b_1$ greater than \SI{0}{\second\per\milli\metre\squared}, typically \SI{1000}{\second\per\milli\metre\squared}.
Then, the \ac{adc} map can be computed as:

\begin{equation}
	\text{ADC} = - \frac{\ln \left( \cfrac{S(b_1)}{S_0} \right) }{b_1} \ .
	\label{eq:adcres1}
\end{equation}

More accurate \ac{adc} maps are computed by acquiring a set of images with different values for the parameter $b$ and fitting linearly a semi-logarithm function using the model presented in \acs{eq}\,\eqref{eq:t2adcrew}.

Regarding the appearance of the \ac{adc} maps, it has been previously stated that by increasing the value of $b$, the signal of \ac{cap} tissue increases significantly.
Considering \acs{eq}\,\eqref{eq:adcres1}, the tissue appearance in the \ac{adc} map is the inverse of \ac{dw}-\ac{mri} images.
Then, \ac{cap} tissue is associated with low-\ac{si} whereas healthy tissue appears brighter as depicted in \acs{fig}\,\ref{subfig:adc}~\cite{Barentsz2012}.

Similar to the gain achieved by \ac{dw}-\ac{mri}, diagnosis using \ac{adc} map combined with \ac{t2w}-\ac{mri} significantly outperforms \ac{t2w}-\ac{mri} alone~\cite{Doo2012,Choi2007}.
Moreover, it has been shown that \ac{adc} coefficient is correlated with \ac{gs}~\cite{Hambrock2011,Itou2011,Peng2013}.

However, some tissues of the \ac{cg} mimic \ac{cap} with low-\ac{si}~\cite{Kirkham2006} and image distortion can arise due to hemorrhage~\cite{Choi2007}.
It has also been noted that a high variability of the \ac{adc} occurs between different patients making it difficult to define a static threshold to distinguish \ac{cap} from non-malignant tumours~\cite{Choi2007}. 

\begin{figure}
	\centering
	\hspace*{\fill}
	\subfigure[Illustration of an \acs*{mrsi} spectrum of a healthy voxel acquired with a \SI{3}{\tesla} \acs*{mri}.]{\label{subfig:mrsihea}\includegraphics[width=0.45\linewidth]{2_modality/figures/mrsi/mrsi_healthy.eps}} \hfill
	\subfigure[Illustration of an \acs*{mrsi} spectrum of a cancerous voxel acquired with a \SI{3}{\tesla} \acs*{mri}.]{\label{subfig:mrsican}\includegraphics[width=0.45\linewidth]{2_modality/figures/mrsi/mrsi_cancer.eps}}
	\hspace*{\fill}
	\caption[Illustration of healthy and cancerous \acs*{mrsi} spectrum.]{Illustration of an \acs*{mrsi} spectrum for both healthy and cancerous voxels with a \SI{3}{\tesla} \acs*{mri}. The highlighted areas correspond to the related concentration of the metabolites which is computed by integrating the area under each peak. Acronyms: choline (Cho), spermine (Spe), creatine (Cr) and citrate (Cit).}
	\label{fig:mrsi}
\end{figure}


%MRSI
\section{\acs*{mrsi}}\label{subsec:chp2:imaging:mrsi}
\ac{cap} induces metabolic changes in the prostate compared with healthy tissue.
Thus, \ac{cap} detection can be carried out by tracking changes of metabolite concentration in prostate tissue.
\ac{mrsi} is an \ac{nmr}-based technique which generates spectra of relative metabolite concentration in \iac{roi}.

In order to track changes of metabolite concentration, it is important to know which metabolites are associated with \ac{cap}.
To address this question, clinical studies identified three biological markers: (i) citrate, (ii) choline, and (iii) polyamines composed mainly of spermine, and in less abundance of spermidine and putrescine~\cite{Awwad2012,Costello2006,Giskeodegard2013}. 

Citrate is involved in the production and secretion of the prostatic fluid, and the glandular prostate cells are associated with a high production of citrate enabled by zinc accumulation by these same cells~\cite{Costello2006}.
However, the metabolism allowing the accumulation of citrate requires a large amount of energy~\cite{Costello2006}.
In contrast, malignant cells do not have high zinc levels leading to lower citrate levels due to citrate oxidization~\cite{Costello2006}.
Furthermore, this change results in a more energy-efficient metabolism enabling malignant cells to grow and spread~\cite{Costello2006}.

An increased concentration of choline is related to \ac{cap}~\cite{Awwad2012}.
Malignant cell development requires epigenetic mechanisms resulting in metabolic changes and relies on two mechanisms: \ac{dna} methylation and phospholid metabolism which both result in choline uptake, explaining its increased level in \ac{cap} tissue~\cite{Awwad2012}.
Spermine is also considered as a biological marker in \ac{cap}~\cite{Graaf2000,Giskeodegard2013}.
In \ac{cap}, reduction of the ductal volume due to shifts in polyamine homeostasis might lead to a reduced spermine concentration~\cite{Graaf2000}.

To determine the concentration of these biological markers, one has to focus on the \ac{mrsi} modality.
In theory, in presence of a homogeneous magnetic field, identical nuclei precesses at the same operating frequency known as the Lamor frequency~\cite{Haacke1999}.
However, \ac{mrsi} is based on the fact that identical nuclei will slightly precess at different frequencies depending on the chemical environment in which they are immersed~\cite{Haacke1999}, a phenomenon known as the \ac{cse}~\cite{Parfait2010}.
Given this property, metabolites are identified and their concentrations are determined.
In this regard, the Fourier transform is used to obtain the frequency spectrum of the \ac{nmr} signal~\cite{Haacke1999,Parfait2010}.
In this spectrum, each peak is associated with a particular metabolite and the area under each peak corresponds to the relative concentration of this metabolite, as illustrated in \acs{fig}\,\ref{fig:mrsi}~\cite{Parfait2010}.

Two different quantitative approaches are used to decide whether or not the spectra of \iac{roi} is associated with \ac{cap}: (i) relative quantification or (ii) absolute quantification~\cite{Lemaitre2011}.
In relative quantification, the ratio of choline-polyamines-creatine to citrate is computed.
The integral of the signal is computed from choline to creatine --- i.e., from \SIrange{3.21}{3.02}{\ppm} --- because the peaks in this region can be merged at clinical magnetic field strengths~\cite{Hoeks2011,Graaf2000}, as depicted in \acs{fig}\,\ref{fig:mrsi}).
Considering the previous assumptions that choline concentration rises and citrate concentration decreases in the presence of \ac{cap}, the ratio computed should be higher in malignant tissue than in healthy tissue. 

In contrast with relative quantification, absolute quantification measures molar concentrations by normalizing relative concentrations using water as reference~\cite{Lemaitre2011}.
In this case, ``true'' concentrations are directly used to differentiate malignant from healthy tissue.
However, this method is not commonly used as it requires an additional step of acquiring water signals, inducing time and cost acquisition constraints.

\ac{mrsi} allows examination with high specificity and sensitivity compared to other \ac{mri} modalities~\cite{Choi2007}.
Furthermore, it has been shown that combining \ac{mrsi} with \ac{mri} improves detection and diagnosis performance~\cite{Scheidler1999a,Kaji1998,Vilanova2009}.
Citrate and spermine concentrations are inversely correlated with the \ac{gs} allowing us to distinguish low- from high- grade \acp{cap}~\cite{Giskeodegard2013}.
However, choline concentration does not provide the same properties~\cite{Giskeodegard2013}.

Unfortunately, \ac{mrsi} also presents several drawbacks.
First, \ac{mrsi} acquisition is time consuming which prevents this modality from being used in daily clinical practise~\cite{Barentsz2012}.
In addition, \ac{mrsi} suffers from low spatial resolution due to the fact that \ac{snr} is linked to the voxel size.
However, this issue is addressed by developing new scanners with higher magnetic field strengths such as \SI{7.5}{\tesla}~\cite{Giskeodegard2013}.
Finally, a high variability of the relative concentrations between patients has been observed~\cite{Choi2007}.
The same observation has been made depending on the zones studied (ie., \ac{pz}, \ac{cg}, base, mid-gland, apex)~\cite{Walker2010,Lemaitre2011}.
Due to this variability, it is difficult to use a fixed threshold in order to differentiate \ac{cap} from healthy tissue.

\section{Summary and conclusions}

\acs{tab}~\ref{tab:modmri} provides an overview of the different modalities presented in the previous section.
Indeed, each \ac{mri} modality alone provides a different discriminative level to distinguish \ac{cap} from healthy tissue.
A recurrent statement in the literature is, however, the ability to combine these \ac{mri} modalities would lead to the best diagnosis performance.
In this regard, we will present in the next chapter automatic tools which have been developed to design \ac{mpmri} \ac{cad} systems for the detection of \ac{cap}.

\begin{landscape}
\begin{table}
  \scriptsize
    \caption[Overview of the features associated with each \acs*{mri} modality used for medical diagnosis by radiologists.]{Overview of the features associated with each \acs*{mri} modality used for medical diagnosis by radiologists. Acronyms: \acf{cap} - \acf{si} - \acf{gs}.}\label{tab:modmri}
    \begin{threeparttable}
      \centering
      \noindent
      \begin{tabularx}{\linewidth}{@{} l X X X l @{}}
        \toprule
        \textbf{Modality} & \textbf{Significant features} & \textbf{\acs*{cap}} & \textbf{Healthy tissue} & \textbf{\acs*{gs} correlation} \\
        \midrule
        \acs*{t2w}-\acs*{mri} & \acs*{si} & low-\acs*{si} in \acs*{pz}~\cite{Hricak1987} & intermediate to high-\acs*{si} in \acs*{pz}~\cite{Hricak1987} & +~\cite{Wang2008} \\ 
        & Shape & round or ill-defined mass in \acs*{pz}~\cite{Hricak1983} &  & 0 \\
        & \acs*{si} & low-\acs*{si} in \acs*{cg}~\cite{Akin2006,Barentsz2012} & low-\acs*{si} in \acs*{cg}~\cite{Akin2006,Barentsz2012} & 0 \\
        & Shape & homogeneous mass with ill-defined edges in \acs*{cg}~\cite{Akin2006, Barentsz2012} &  & 0 \\ \\
        T$_2$ map & \acs*{si} & low-\acs*{si}~\cite{Liney1996,Gibbs2001} & intermediate to high-\acs*{si}~\cite{Liney1996,Gibbs2001} & +~\cite{Liu2011,Liney1996,Liney1997}  \\ \\
        \acs*{dce} \acs*{mri} & Semi-quantitative features~\cite{Verma2012}: & & & \\
        & $\bullet$ wash-in & faster & slower & 0 \\
        & $\bullet$ wash-out & faster & slower & 0 \\
        & $\bullet$ integral under the curve & higher & lower & 0 \\
        & $\bullet$ maximum signal intensity & higher & lower & 0 \\
        & $\bullet$ time-to-peak enhancement & faster & slower & 0 \\ \\
        & Quantitative features (Tofts' parameters~\cite{Tofts2010}): & & & \\
        & $\bullet$ $\text{k}_{\text{ep}}$ & higher & lower & 0 \\
        & $\bullet$ $\text{K}^{\text{trans}}$ & higher & lower & 0 \\ \\
        \acs*{dw} \acs*{mri} & \acs*{si} & higher-\acs*{si}~\cite{Huisman2003,Barentsz2012} & lower-\acs*{si}~\cite{Huisman2003,Barentsz2012} & + \\ \\ 
        \acs*{adc} map & \acs*{si} & low-\acs*{si}~\cite{Barentsz2012} & high-\acs*{si}~\cite{Barentsz2012} & +~\cite{Hambrock2011, Itou2011, Peng2013} \\ \\
        \acs*{mrsi}& Metabolites: & & & \\
        & $\bullet$ citrate (2.64 ppm)~\cite{Verma2010} & lower concentration~\cite{Awwad2012,Costello2006,Graaf2000} & higher concentration~\cite{Awwad2012,Costello2006,Graaf2000} & +~\cite{Giskeodegard2013} \\
        & $\bullet$ choline (3.21 ppm)~\cite{Verma2010} & higher concentration~\cite{Awwad2012,Costello2006,Graaf2000} & lower concentration~\cite{Awwad2012,Costello2006,Graaf2000} & 0~\cite{Giskeodegard2013} \\
        & $\bullet$ spermine (3.11 ppm)~\cite{Verma2010} & lower concentration~\cite{Awwad2012,Costello2006,Graaf2000} & higher concentration~\cite{Awwad2012,Costello2006,Graaf2000} & +~\cite{Giskeodegard2013} \\
        \bottomrule
      \end{tabularx}
      \begin{tablenotes}
      \item Notes:
      \item + = significantly correlated;
      \item 0 = no correlation.
      \end{tablenotes}
    \end{threeparttable}
  \label{tab:modmri}
\end{table}
\end{landscape}

						% aims of the project

\acresetall
\chapter{Review of CAD sytems for CaP}\label{chap:3}
% the code below specifies where the figures are stored
\ifpdf
    \graphicspath{{3_review/figures/}}
\else
    \graphicspath{{3_review/figures/}}
\fi

\begin{figure*}
  \centering

  % Define block styles used later

  \tikzstyle{module}=[draw, draw=blue!80, text width=10em, 
  text centered, minimum height=5em, minimum width = 15em, drop shadow, rounded corners,
  fill=blue!30]
  
  \tikzstyle{vecArrow} = [thick, decoration={markings,mark=at position
    1 with {\arrow[semithick]{open triangle 60}}},
  double distance=1.4pt, shorten >= 5.5pt,
  preaction = {decorate},
  postaction = {draw,line width=1.4pt, white,shorten >= 4.5pt}]

  % Define distances for bordering
  \def\blockdist{1.5}
  \def\edgedist{2.5}

  \begin{tikzpicture}[node distance=3cm,thick,scale=0.6, every node/.style={scale=0.6},path image/.style={
      path picture={
        \node at (path picture bounding box.center) {
          \includegraphics[width=1cm]{#1}
        };}}]
    \tikzstyle{conefill} = [path image=,fill opacity=0.8]
    \node[module=above:pre] (pre) at (4.5,-2.6) {\Large Pre-processing};
    \node[module,below of=pre] (seg) {\Large Segmentation};
    \node[module,below of=seg] (reg) {\Large Registration};

    \path[->,dashed] (seg.west) edge [bend right=70] node {} (reg.west);
    \path[->,dashed] (reg.east) edge [bend right=70] node {} (seg.east);

    \draw[->] (pre)--(seg);
    \draw[->] (seg)--(reg);

    \begin{pgfonlayer}{background}
      \path (pre.west |- pre.north)+(-0.9,1.0+\blockdist) node (a) {};
      \path (reg.east |- reg.south)+(+0.9,-0.5) node (b) {};
      
      \path[fill=blue!10,rounded corners, draw=blue!20, dashed] (a) rectangle (b);
    \end{pgfonlayer}
    
    \path (pre.north) +(0,+\blockdist) node (bgreg) {\Large Image regularization};

    \begin{scope}[node distance=10cm]
      \node[module] (det) [below right=0cm and 3cm of pre] {\Large Features detection};
    \end{scope}
    \begin{scope}[node distance=3.5cm]
      \node[module,above of=det] (roi) {\Large ROIs\\detection/selection};
    \end{scope}
    \node[module,below of=det] (sel) {\Large Features\\selection/extraction};
    \node[module,below of=sel] (cla) {\Large Features\\classification/fusion};

    \draw[->] (roi)--(det);
    \draw[->] (det)--(sel);
    \draw[->] (sel)--(cla);

    \begin{pgfonlayer}{background}
      \path (roi.west |- roi.north)+(-0.25,0.8) node (c) {};
      \path (roi.east |- roi.south)+(+0.25,-0.25) node (d) {};
      
      \path[fill=blue!20,rounded corners, draw=blue!25, dashed] (c) rectangle (d);
    \end{pgfonlayer}

    \path (roi.west |- roi.north) +(.75,0.4) node (bgfea) {\Large \textbf{CADe}};

    \begin{pgfonlayer}{background}
      \path (det.west |- det.north)+(-0.25,0.8) node (c) {};
      \path (cla.east |- cla.south)+(+0.25,-0.25) node (d) {};
      
      \path[fill=blue!20,rounded corners, draw=blue!25, dashed] (c) rectangle (d);
    \end{pgfonlayer}

    \path (roi.west |- det.north) +(.75,0.4) node (bgfea) {\Large \textbf{CADx}};     

    % Define the place where the arrow should start anf finish
    \path (seg.east |- seg.north)+(+1.15,0) node (e) {};
    \path (sel.west |- seg.north)+(-1.0,0) node (f) {};

    \draw[double distance =3pt,preaction={-triangle 90,thin,draw,shorten >=-1mm}] (e) -- (f) node[midway,above] {\large Regularized data};

    \begin{scope}[yshift=34,xshift=-86]
      \transparent{0.6}\draw[path image=2_modality/figures/tikzimage/t2.eps] (0,0) rectangle (1.0,1.0);
    \end{scope}

    \begin{scope}[yshift=31,xshift=-83]
      \transparent{0.6}\draw[path image=2_modality/figures/tikzimage/t2.eps] (0,0) rectangle (1.0,1.0);
    \end{scope}

    \begin{scope}[yshift=28,xshift=-80]
      \transparent{0.8}\draw[path image=2_modality/figures/tikzimage/t2.eps] (0,0) rectangle (1.0,1.0);
      \path (0,0)+(-1.65,0.3) node {\Large T$_2$-W MRI};
    \end{scope}

    \begin{scope}[yshift=-33,xshift=-86]
      \transparent{0.6}\draw[path image=2_modality/figures/tikzimage/t2.eps] (0,0) rectangle (1.0,1.0);
    \end{scope}

    \begin{scope}[yshift=-36,xshift=-83]
      \transparent{0.6}\draw[path image=2_modality/figures/tikzimage/t2.eps] (0,0) rectangle (1.0,1.0);
    \end{scope}

    \begin{scope}[yshift=-39,xshift=-80]
      \transparent{0.8}\draw[path image=2_modality/figures/tikzimage/t2.eps] (0,0) rectangle (1.0,1.0);
      \path (0,0)+(-1.35,0.3) node {\Large T$_2$ map};
    \end{scope}

    \begin{scope}[yshift=-100,xshift=-86]
      \transparent{0.6}\draw[path image=2_modality/figures/tikzimage/dce.eps] (0,0) rectangle (1.0,1.0);
    \end{scope}

    \begin{scope}[yshift=-103,xshift=-83]
      \transparent{0.6}\draw[path image=2_modality/figures/tikzimage/dce.eps] (0,0) rectangle (1.0,1.0);
    \end{scope}

    \begin{scope}[yshift=-106,xshift=-80]
      \transparent{0.8}\draw[path image=2_modality/figures/tikzimage/dce.eps] (0,0) rectangle (1.0,1.0);
      \path (0,0)+(-1.65,0.3) node {\Large DCE MRI};
    \end{scope}

    \begin{scope}[yshift=-167,xshift=-86]
      \transparent{0.6}\draw[path image=2_modality/figures/tikzimage/dwi1.eps] (0,0) rectangle (1.0,1.0);
    \end{scope}

    \begin{scope}[yshift=-170,xshift=-83]
      \transparent{0.6}\draw[path image=2_modality/figures/tikzimage/dwi1.eps] (0,0) rectangle (1.0,1.0);
    \end{scope}

    \begin{scope}[yshift=-173,xshift=-80]
      \transparent{0.8}\draw[path image=2_modality/figures/tikzimage/dwi1.eps] (0,0) rectangle (1.0,1.0);
      \path (0,0)+(-1.65,0.3) node {\Large DW MRI};
    \end{scope}

    \begin{scope}[yshift=-234,xshift=-86]
      \transparent{0.6}\draw[path image=2_modality/figures/tikzimage/adc.eps] (0,0) rectangle (1.0,1.0);
    \end{scope}

    \begin{scope}[yshift=-237,xshift=-83]
      \transparent{0.6}\draw[path image=2_modality/figures/tikzimage/adc.eps] (0,0) rectangle (1.0,1.0);
    \end{scope}

    \begin{scope}[yshift=-240,xshift=-80]
      \transparent{0.8}\draw[path image=2_modality/figures/tikzimage/adc.eps] (0,0) rectangle (1.0,1.0);
      \path (0,0)+(-1.5,0.3) node {\Large ADC};
    \end{scope}

    \begin{scope}[yshift=-301,xshift=-86]
      \transparent{0.6}\draw[path image=2_modality/figures/tikzimage/mrsi.eps] (0,0) rectangle (1.0,1.0);
    \end{scope}

    \begin{scope}[yshift=-304,xshift=-83]
      \transparent{0.6}\draw[path image=2_modality/figures/tikzimage/mrsi.eps] (0,0) rectangle (1.0,1.0);
    \end{scope}

    \begin{scope}[yshift=-307,xshift=-80]
      \transparent{0.8}\draw[path image=2_modality/figures/tikzimage/mrsi.eps] (0,0) rectangle (1.0,1.0);
      \path (0,0)+(-1.1,0.3) node {\Large MRSI};
    \end{scope}

    \path (pre.west |- roi.north)+(-3.5,1.0+\blockdist) node (g) {};
    \path (reg.west |- cla.south)+(-3.5,-0.5) node (h) {};

    \draw[decorate,decoration={brace,raise=6pt,amplitude=10pt}, thick]
    (g)--(h) ;
    
    \path (seg.west |- seg.north)+(-2.5,0) node (i) {};
    \path (seg.west |- seg.north)+(-0.9,0) node (j) {};
    
    \draw[double distance =3pt,preaction={-triangle 90,thin,draw,shorten >=-1mm}] (i) -- (j);   

    \path (sel.east |- seg.north)+(2,0) node (k) {};
    \path (sel.east |- seg.north)+(0.5,0) node (l) {};
    
  \end{tikzpicture}
  \caption{Common \acs*{cad} framework based on \acs*{mri} images used to detect \acs*{cap}.}
  \label{fig:wkfcad}
\end{figure*}


As previously mentioned \acs{sec}\,\ref{sec:intro:cad}, \acp{cad} are developed to advise and backup radiologists in their tasks of \ac{cap} detection and diagnosis, but not to provide fully automatic decisions~\cite{Giger2008}.
\acp{cad} can be divided into two different sub-groups: either as \ac{cade}, with the purpose to highlight probable lesions in \ac{mri} images, or \ac{cadx}, which focuses on differentiating malignant from non-malignant tumours~\cite{Giger2008}.
Moreover, an intuitive approach, motivated by developing a framework combining detection-diagnosis, is to mix both \ac{cade} and \ac{cadx} by using the output of the former mentioned as a input of the latter named.
Although the outcomes of these two systems should differ, the framework of both \ac{cad} systems is similar.
A general \ac{cad} work-flow is presented in \acs{fig}\,\ref{fig:wkfcad}.
%The \ac{cad} work-flow is presented in \acs{fig}~\ref{fig:wkfcad}.

\ac{mri} modalities mentioned in \acs{chp}\,\ref{chap:2} are used as inputs of \ac{cad} for \ac{cap}.
These images acquired from the different modalities show a large variability between patients: the prostate organ can be located at different positions in images --- due to patient motion, variation of acquisition plan --- and the \ac{si} can be corrupted with noise or artifacts during the acquisition process caused by the magnetic field non-homogeneity or the use of endorectal coil.
%It can be noted that \ac{adc} map is not considered as an input since it is a feature derived from the \ac{dw} \ac{mri} images.
To address these issues, the first stage of \ac{cad} is to pre-process \ac{mpmri} images to reduce noise, remove artifacts, and standardize the \ac{si}.
Subsequently, most of the later processes are only focusing on the prostate organ; therefore it is necessary to segment the prostate in each \ac{mri} modality to define it as a \ac{roi}.
However, data may suffer from misalignment due to patient motions or different acquisition parameters.
Therefore, a registration step is usually performed so that all the previously segmented \ac{mri} images are in the same reference frame.
Registration and segmentation can be swapped depending on the strategy chosen.

Some studies do not fully apply the methodology depicted in \acs{fig}\,\ref{fig:wkfcad}.
Details about those can be found in \acs{tab}~\ref{tab:sumpap}.
Some studies bypass the pre-processing stages to proof the robustness of their approaches to noise or other artifacts, by using directly the raw data as inputs of their \ac{cad} systems.
In some cases, prostate segmentation is performed manually as well as registration.
Sometimes, it is also assumed that no patient motions occur during the acquisition procedure, removing the need of registering the \ac{mpmri} images.

Once the data are regularized, it becomes possible to extract features and classify the data to obtain either the location of possible lesions (i.e., \ac{cade}) or/and the malignancy nature of these lesions (i.e., \ac{cadx}).

In \iac{cade} framework, \textit{possible lesions are segmented automatically} and further used as input of \iac{cadx}.
Nevertheless, some works also used a fusion of \ac{cade}-\ac{cadx} framework in which a voxel-based features are directly used, in which the location of the malignant lesions are obtained as results.
On the other hand, manual lesions segmentation is not considered to be part of \ac{cade}.
%The output of the \ac{cade} is used as input of the \ac{cadx}.

\Ac{cadx} is composed of the processes allowing to \textit{distinguish malignant from non-malignant tumours}.
Here, \ac{cap} malignancy is defined using the grade of the \ac{gs} determined after post biopsy or prostatectomy.
As presented in \ac{fig}\,\ref{fig:wkfcad}, \ac{cadx} is usually composed of the three common steps used in a classification framework: (i) features detection, (ii) feature extraction/selection, and (iii) feature classification.


%% We divided \ac{cadx} into three different stages. First, salient features are extracted, in an pixel-based or region-based manner, from \ac{mri} images to characterize the lesion. Of course, more discriminative features will be associated with a robust and accurate likelihood cancer map. Frequently, the number of features extracted can be large resulting in redundant or insufficient discriminative features which will negatively affect the performances of the further classification. Therefore, a step consists of selecting the best features or/and reducing the number of dimensions is commonly used. Then, this modified feature vector is finally classified using different pattern recognition approaches.

%% As pointed out in the introduction, performance of \ac{cap} detection and diagnosis are affected by observer interpretation and limitations \cite{Giger2008,Hambrock2013}. \ac{cad} offers a possible solution in order to reduce this variability. As mentioned in the introduction, the effects of \ac{cad} on the observer performance has been studied \cite{Hambrock2013}, with results showing that \acp{cad} benefit to less-experienced radiologist to perform similarly as experienced radiologist in their tasks \cite{Hambrock2013}. 


%\section{Literature classification}\label{sec:chp3:Literature-classification}

This chapter is organized using the methodology presented in \acs{fig}\,\ref{fig:wkfcad}.
Methods embedded in the image regularization framework are presented initially to subsequently focus on the image classification framework, being divided into \ac{cade} and \ac{cadx}.
Finally, we present a summary of the results reported in the state-of-the-art as well as a discussion that follows.
%before to focus on the image classification framework, the later being divided into \ac{cade} and \ac{cadx}. 
\Acl{tab}~\ref{tab:sumpap} summarizes the 56 different \ac{cad} studies reviewed in this section.
The first set of information reported is linked to the data acquisition such as the number of patients included in the study, the modalities acquired as well as the strength of the field of the scanner used.
Subsequently, information about the prostate zones considered in the \ac{cad} analysis --- i.e. \ac{pz} or \ac{cg} --- are reported since that detecting \ac{cap} in the \ac{cg} is a more challenging problem and has received particular attention only in the recent publications.

The papers have been selected by investigating referenced international peer-reviewed journals as well as international peer-reviewed conferences.
Additionally, a breadth-search first (or snowball sampling) have been used to refine missing publications.
Only studies proposing \ac{cad} systems specifically for \ac{cap} have been reviewed.

%% Characteristics related to \ac{mri} acquisition as well as \ac{cad} strategies are reported.
%% Only methods used in \ac{cad} system are discussed.

\input{3_review/Table-survey.tex}
\section{Image regularization framework}\label{sec:chp3:img-reg}

This section provides a review of the methods used in \acp{cad} for \ac{cap} in order to \emph{regularize} the \ac{mpmri} images.
At first, we present the pre-processing methods in \acs{sec}\,\ref{subsec:chp3img-reg:prepro}, focusing mainly on the denoising and artefacts removal methods as well as standardization of \ac{si}.
\Acl{sec}~\ref{subsec:chp3:img-reg:seg} and \acs{sec}\,\ref{subsec:chp3:img-reg:reg} summarize the segmentation and registration methods, which are processes allowing the \ac{cad} to only operate on the prostate organ and ensuring that the \ac{mpmri} images are aligned in the same reference frame.

%We start with pre-processing methods, focusing mainly on the reduction of noise level and artefacts as well as standardization of \ac{si}, following by a segmentation and registration sections.

\subsection{Pre-processing}\label{subsec:chp3img-reg:prepro}
Three different groups of pre-processing methods are commonly applied to images as initial stage in \acp{cad} for \ac{cap}.
These methods are explained for both \ac{mri} and \ac{mrsi} modalities.%, while a summary of the applied methods in \ac{cad} is presented in \ac{tab}~\ref{tab:summary-preproc}.

%% \setenumerate{listparindent=\parindent,itemsep=10px}
%% \setlist{noitemsep}
%% \begin{enumerate}[leftmargin=*]

\subsubsection{\acs*{mri} modalities}\label{subsubsec:ch3:mriprepro}

\begin{figure}
\centering
	\includegraphics[width=0.7\linewidth]{3_review/figures/processing/pre-processing/noise/noisedistr.eps}
	\caption[Illustration of a Gaussian and Rayleigh distributions.]{Illustration of a Gaussian and Rayleigh distribution. Although the mode of these distributions are identical, it can be noted that the Rayleigh distribution ($\mu=1.253$) is suffering of a bias term when compared with the Gaussian distribution ($\mu=1$).}
	\label{fig:noisedistr}
\end{figure}

%Noise filtering
%\item[$-$] \textbf{\textit{Noise filtering:}} 
\paragraph{Noise filtering} The \ac{nmr} signal, measured and acquired in the k-space, is affected by noise.
This noise obeys a complex Gaussian white noise mainly due to thermal noises in the patient~\cite{Nowak1999}.
Furthermore, \ac{mri} images visualized by radiologists are in fact the magnitude images resulting from the complex Fourier transform of the k-space data.
The complex Fourier transform does not affect the Gaussian noise characteristics since this is a linear and orthogonal transform~\cite{Nowak1999}.
However, the calculation of the magnitude is a non-linear transform --- i.e., the square root of the sum of squares of real and the imaginary parts --- implying that the noise distribution is no longer Gaussian; it indeed follows a Rician distribution making the denoising task more challenging.
Briefly, a Rician distribution is characterized as follows: in low-\ac{si} region (low-\ac{snr}), it can be approximated with a Rayleigh distribution while in high-\ac{si} region (high-\ac{snr}), it is similar to a Gaussian distribution~\cite{Manjon2008}.
Refer to \acs{fig}\,\ref{fig:noisedistr} to observe the difference between a Gaussian and a Rayleigh distribution.
Comprehensive reviews regarding denoising methods can be found in~\cite{Buades2005,Mohan2014}.

Median filtering is the simplest approach used to address the denoising issue in \ac{mri} images~\cite{Ozer2009,Ozer2010}.
In both studies, \citeauthor{Ozer2010}~\cite{Ozer2009,Ozer2010} used a square-shaped kernel of size \SI[product-units=repeat]{5x5}{\px}.% with the image resolutions ranging from \SI[product-units=repeat]{320x256}{\px} to \SI[product-units=repeat]{256x128}{\px} for the \ac{t2w}-\ac{mri} and the other remaining modalities used, respectively.

More recently, \citeauthor{rampun2016quantitative} used a combination of median and anisotropic diffusion filter~\cite{rampun2015classifying,rampun2016computer,rampun2016computerb,rampun2016quantitative}, proposed in~\cite{ling2002smoothing}.
In low-\ac{snr} images, the gradient generated by an edge and noise can be similar, making the denoising by diffusion more challenging.
In this condition, the threshold allowing to locally differentiate a noise gradient from an edge gradient needs to be increased, at the cost of blurring edges after filtering.
Therefore, \citeauthor{ling2002smoothing}~\cite{ling2002smoothing} proposed to apply a standard anisotropic diffusion filter with a low threshold followed by a median filtering to remove large noise spikes.

\citeauthor{samarasinghe2016semi} filtered \ac{dce}-\ac{mri} images with a sliding 3D Gaussian filter~\cite{samarasinghe2016semi}.
However, from a theoretical point of view, this simple filtering method is not well formalized to address the noise distribution in \ac{mri} images.
That is why more complex approaches have been proposed to overcome this problem.
Another common method used to denoise \ac{mri} images is based on wavelet decomposition and shrinkage.
This filtering exploits the sparsity property of the wavelet decomposition.
The projection of a noisy signal from the spatial-domain to the wavelet-domain implies that only few wavelet coefficients contribute to the ``signal-free noise'' while all wavelet coefficients contribute to the noise~\cite{Donoho1994}.
Therefore, insignificant wavelet coefficients are thresholded/attenuated to enforce the sparsity in the wavelet-domain, which results to a denoising process in the spatial domain.
Investigations focus on the strategies to perform the most adequate coefficient shrinkage (e.g., thresholding, singularity property, or Bayesian framework)~\cite{Pizurica2002}.
\citeauthor{Ampeliotis2008} denoised the magnitude \ac{mri} images~\cite{Ampeliotis2007,Ampeliotis2008} --- i.e., \ac{t2w}-\ac{mri} and \ac{dce}-\ac{mri} --- by wavelet shrinkage, using thresholding techniques~\cite{Mallat2008}.
However, since the wavelet transform is an orthogonal transform, the Rician distribution of the noise is preserved in the wavelet-domain.
Hence, for low-\ac{snr}, the wavelet and scaling coefficients still suffer from a bias due to this specific noise distribution~\cite{Nowak1999}.
That is why, \citeauthor{Lopes2011} filtered \ac{t2w}-\ac{mri} images~\cite{Lopes2011}, using the method proposed in~\cite{Pizurica2003} based on joint detection and estimation theory.
In this approach, the wavelet coefficients ``free-of-noise'' are estimated from the noisy wavelet coefficients using a \ac{map} estimate.
Furthermore, the designed estimator takes spatial context into account by including both local and global information in the prior probabilities.
The different probabilities needed by the \ac{map} are empirically estimated by using mask images, representing the locations of the significant wavelet coefficients.
These mask images are computed by thresholding the detail images obtained from the wavelet decomposition.
To remove the bias from the wavelet and scaling coefficients, the squared magnitude \ac{mri} image is computed instead of the magnitude \ac{mri} image as proposed in~\cite{Nowak1999}.
This involves changing the Rician distribution to a scaled non-central Chi-squared distribution.
It implies that the wavelet coefficients are also unbiased estimators and the scaling coefficients are unbiased estimators but up to a constant $C$ as defined in \acs{eq}\,\eqref{eq:nowakC} which needs to be subtracted from each scaling coefficient such as:

\begin{equation}
	C=2^{(J+1)}\hat{\sigma}^2 \ ,
	\label{eq:nowakC}
\end{equation}

\noindent where $J$ is the number of levels of the wavelet decomposition and $\hat{\sigma}$ is an estimate of the noise standard deviation.

\begin{figure}
\centering
\includegraphics[width=0.3\linewidth]{3_review/figures/processing/pre-processing/bias/t2w_bias_antenna.eps}
\caption[Non-homogeneity artifacts due to perturbation of the endorectal coil.]{Example of artifacts with high \acs*{si} due to perturbation from the endorectal coil which create non-homogeneity.}
\label{fig:bias}
\end{figure}


% Artifacts filtering
%\item[$-$] \textbf{\textit{Bias correction:}} 
\paragraph{Bias correction} Besides being corrupted by noise, \ac{mri} images are also affected by the inhomogeneity of the \ac{mri} field commonly referred to as bias field~\cite{Styner2000}.
This bias field results in a smooth variation of the \ac{si} through the image.
When an endorectal coil is used, a resulting artifact of an hyper-intense signal is observed around the coil as depicted in \acs{fig}\,\ref{fig:bias}.
As a consequence, the \ac{si} of identical tissues varies depending on their spatial location in the image making further processes such as segmentation, registration, or classification more challenging~\cite{Jungke1987,Vovk2007}.
A comprehensive review of bias correction methods is proposed in~\cite{Vovk2007}.

The model of image formation is usually formalized as:

\begin{equation}
	s(\mathbf{x}) = o(\mathbf{x})b(\mathbf{x}) + \eta(\mathbf{x}) \ ,
	\label{eq:biasmodel}
\end{equation}

\noindent where $s(\mathbf{x})$ is the corrupted \ac{si} at the pixel for the image coordinates $\mathbf{x} = \{x,y\}$, $o(\mathbf{x})$ is the ``noise-free signal'' , $b(\mathbf{x})$ is the bias field function and $\eta(\mathbf{x})$ is an additive white Gaussian noise.
%
%By using property of logarithm, the model of Eq. \eqref{eq:biasmodel} becomes additive such that:
%
%\begin{eqnarray}
%	\log s(\mathbf{x}) - \log b(\mathbf{x}) & = & \log \left( o(\mathbf{x}) + \frac{\eta(\mathbf{x})}{b(\mathbf{x})} \right) \ , \\ \nonumber
%	& = & \log \hat{o}(x) \ .
%\end{eqnarray}
%
%\noindent where $\hat{o}(\mathbf{x})$ is the signal only degraded by noise \cite{Styner2000}.

Hence, the task of bias correction involves estimating the bias function $b(\mathbf{x})$ in order to infer the ``signal-free bias'' $o(\mathbf{x})$.
% and subtract to the logarithm of the initial signal in order to obtain an estimated ``signal-free bias''.


\citeauthor{Viswanath2009} corrected this artifact on \ac{t2w}-\ac{mri} images~\cite{Viswanath2009}, using the model proposed in~\cite{Styner2000}, in which \citeauthor{Styner2000} model the bias field function by using a linear combination of Legendre polynomials $f_i$ as:

\begin{align}
	\hat{b}(\mathbf{x},\mathbf{p}) & = \sum_{i=0}^{m-1} p_i f_i(\mathbf{x}) \\ \nonumber
                                       & =  \sum_{i=0}^{l} \sum_{j=0}^{l-i} p_{ij} P_i(x) P_j(y) \ ,
	\label{eq:biascorr}
\end{align}

\noindent where $\hat{b}(\cdot)$ is the bias estimation with the image coordinates $\mathbf{x} = \{x,y\}$ and the $m$ coefficients of the linear combination $\mathbf{p} = {p_{11},\dotsc,p_{ij}}$; $m$ can be defined as $m=(l+1)\frac{(l+2)}{2}$ where $l$ is the degree of Legendre polynomials chosen and $P_i(\cdot)$ denotes a Legendre polynomial of degree $i$.

This family of functions offers to model the bias function as a smooth inhomogeneous function across the image.
To estimate the set of parameters $\mathbf{p}$, a cost function is defined which relies on the following assumptions: (i) an image is composed of $k$ regions with a mean $\mu_k$ and a variance $\sigma^{2}_{k}$ for each particular class, and (ii) each noisy pixel belongs to one of the $k$ regions with its \ac{si} value close to the class mean $\mu_k$.
Hence, the cost function is defined as:

\begin{equation}
	C(\mathbf{p}) = \sum_{\mathbf{x}} \prod_{k} \rho_k(s(\mathbf{x}) - \hat{b}(\mathbf{x},\mathbf{p}) - \mu_k) \ ,
	\label{eq:costbias}
\end{equation}

\begin{equation}
	\rho_k(x) = \frac{x^2}{x^2 + 3 \sigma_k^2} \ ,
	\label{eq:mestbias}
\end{equation}

\noindent where $\rho_k(\cdot)$ is a M-estimator allowing estimations to be less sensitive to outliers than the usual squared distance~\cite{Li1996}.

Finally, the parameters $\mathbf{p}$ are estimated by finding the minimum of the cost function $C(\mathbf{p})$, which was optimized using the non-linear $(1+1)$ \ac{es} optimizer~\cite{Styner1997}.

In a later publication, \citeauthor{Viswanath2012}~\cite{Viswanath2012} as well as \citeauthor{giannini2015fully}~\cite{giannini2015fully} corrected \ac{t2w}-\ac{mri} using the well known N3 algorithm~\cite{Sled1998} in which \citeauthor{Sled1998} infer the bias function using the \acp{pdf} of the signal and bias.
Taking advantage of the logarithm property, the model in \acs{eq}\,\eqref{eq:biasmodel} becomes additive as expressed in \acs{eq}\,\eqref{eq:logbias}.

\begin{eqnarray}
	\log s(\mathbf{x}) & = & \log b(\mathbf{x}) + \log \left( o(\mathbf{x}) + \frac{\eta(\mathbf{x})}{b(\mathbf{x})} \right) \ , \nonumber \\
	& \approx & \log b(\mathbf{x}) + \log \hat{o}(\mathbf{x}) \ , \label{eq:logbias}
\end{eqnarray}

\noindent where $\hat{o}(\mathbf{x})$ is the signal only degraded by noise. \citeauthor{Sled1998} show that \acs{eq}\,\eqref{eq:logbias} is related to \acp{pdf} such that:

\begin{equation}
	S(s) = B(s) * O(s) \ ,
	\label{eq:distrbias} 
\end{equation}

\noindent where $S(\cdot)$, $B(\cdot)$, and $O(\cdot)$ are the \acp{pdf} of $s(\cdot)$, $b(\cdot)$, and $o(\cdot)$, respectively.

The corrupted signal $s$ is restored by finding the multiplicative field $b$ which maximizes the frequency content of the distribution $O$.
\citeauthor{Sled1998}~\cite{Sled1998} argued that a brute-force search through all possible fields $b$ and selecting the one which maximizes the high frequency content of $O$ is possible but far too complex.
By assimilating the bias field distribution to be a near Gaussian distribution as \textit{a priori}, it is then possible to infer the distribution $O$ using the Wiener deconvolution given $B$ and $S$ and later estimate the corresponding smooth field $b$.

\citeauthor{Lv2009} corrected the non-homogeneity in \ac{t2w}-\ac{mri} images~\cite{Lv2009} by using the method proposed in~\cite{Madabhushi2006}.
\citeauthor{Madabhushi2006}~\cite{Madabhushi2006} proposed to correct the \ac{mri} images by detecting the image foreground via \ac{gscale} in an iterative manner and estimating a bias field function based on a 2\textsuperscript{nd} order polynomial model.
First, the background of the \ac{mri} image is eliminated by thresholding, in which the threshold value is commonly equal to the mean \ac{si} of the considered image.
Then, a seeded region growing algorithm is applied in the image foreground, considering every thresholded pixel as a potential seed.
However, pixels already assigned to a region are not considered any more as a potential seed.
As in seeded region growing algorithm~\cite{Shapiro2001}, two criteria are taken into account to expand a region.
First, the region grows using a connected-neighbourhood, initially defined by the user.
Then, the homogeneity of \ac{si} is based on a fuzzy membership function taking into account the absolute difference of two pixel \ac{si}.
Depending on the membership value --- corresponding to a threshold which needs to be defined --- the pixel considered is merged or not to the region.
Once this segmentation is performed, the largest region $R$ is used as a mask to select pixels of the original image and the mean \ac{si}, $\mu_{R}$, is computed. 
The background variation $b(\mathbf{x})$ is estimated as:

\begin{equation}
	b(\mathbf{x}) = \frac{s(\mathbf{x})}{\mu_{R}}, \ \forall \mathbf{x} \in R \ ,
	\label{eq:backest}
\end{equation}

\noindent where $s(\mathbf{x})$ is the original \ac{mri} image.

Finally, a 2\textsuperscript{nd} order polynomial $\hat{b}_{\Theta}(\mathbf{x})$ is fitted in a least-squares sense as in \acs{eq}\,\eqref{eq:lsolv},

\begin{equation}
	\hat{\Theta} = \argmin_{\Theta} | b(\mathbf{x}) - \hat{b}_{\Theta}(\mathbf{x}) |^{2}, \ \forall \mathbf{x} \in R \ .
	\label{eq:lsolv}
\end{equation}

Finally, the whole original \ac{mri} image is corrected by dividing it by the estimated bias field function $\hat{b}_{\Theta}(\mathbf{x})$.
The convergence is reached when the number of pixels in the largest region $R$ does not change significantly between two iterations.

%SI normalization
%\item[$-$] \textbf{\textit{\Ac{si} normalization/standardization:}}
\paragraph{\Ac{si} normalization/standardization} As discussed in the later section, segmentation or classification tasks are usually composed of a learning stage using a set of training patients.
Hence, one can emphasize the desire to perform automatic diagnosis with a high repeatability or in other words, one would ensure to obtain consistent \ac{si} of tissues across patients of the same group --- i.e., healthy patients \textit{vs.} patients with \ac{cap} --- for each \ac{mri} modality.
However, it is a known fact that variability between patients occurs during the \ac{mri} examinations even using the same scanner, protocol or sequence parameters~\cite{Nyul1999}.
Hence, the aim of normalization or standardization of the \ac{mri} data is to remove the variability between patients and enforce the repeatability of the \ac{mri} examinations.
These standardization methods are categorized either as statistical-based standardization or organ \ac{si}-based standardization

\citeauthor{Artan2010}~\cite{Artan2009,Artan2010}, \citeauthor{Ozer2010}~\cite{Ozer2009,Ozer2010}, and \citeauthor{rampun2016quantitative}~\cite{,rampun2015classifying,rampun2015computer,rampun2016computer,rampun2016computerb,rampun2016quantitative} standardized \ac{t2w}-\ac{mri}, \ac{dce}-\ac{mri}, and \ac{dw}-\ac{mri} images by computing the \textit{standard score} (also called \textit{z-score}) of the pixels of the \ac{pz} as:

\begin{equation}
	I_s(\mathbf{x}) = \frac{ I_r(\mathbf{x}) - \mu_{pz}}{\sigma_{pz}}, \ \forall \mathbf{x} \in \text{PZ} \ ,
	\label{eq:meansta}
\end{equation}

\noindent where $I_s(\mathbf{x})$ is the standardized \ac{si} with the image coordinates $\mathbf{x} = \{x,y\}$, $I_r(\mathbf{x})$ is the raw \ac{si}, $\mu_{pz}$ is the mean \ac{si} of the \ac{pz} and $\sigma_{pz}$ is the \ac{si} standard deviation in the \ac{pz}.
This transformation enforces the image \ac{pdf} to have a zero mean and a unit standard deviation.
In a similar way, \citeauthor{Liu2013} normalized \ac{t2w}-\ac{mri} by making use of the median and inter-quartile range for all the pixels~\cite{Liu2013}.

\citeauthor{Lv2009} scaled the \ac{si} of \ac{t2w}-\ac{mri} images using the method proposed in~\cite{Nyul2000} based on \ac{pdf} matching~\cite{Lv2009}.
This approach is based on the assumption that \ac{mri} images from the same sequence should share the same \ac{pdf} appearance.
Hence, one can approach this issue by transforming and matching the \acp{pdf} using some statistical landmarks such as quantiles.
Using a training set, these statistical landmarks --- such as minimum, 25\textsuperscript{th} percentile, median, 75\textsuperscript{th} percentile, and maximum --- are extracted for $N$ training images:

\begin{eqnarray}	
	\Phi_{0} & = & \{ \phi_{0}^{1}, \phi_{0}^{2}, \cdots, \phi_{0}^{N} \} \ , \nonumber \\
	\Phi_{25} & = & \{ \phi_{25}^{1}, \phi_{25}^{2}, \cdots, \phi_{25}^{N} \} \ , \nonumber \\
	\Phi_{50} & = & \{ \phi_{50}^{1}, \phi_{50}^{2}, \cdots, \phi_{50}^{N} \} \ ,  \label{eq:quantileStd} \\
	\Phi_{75} & = & \{ \phi_{75}^{1}, \phi_{75}^{2}, \cdots, \phi_{75}^{N} \} \ , \nonumber \\
	\Phi_{100} & = & \{ \phi_{100}^{1}, \phi_{100}^{2}, \cdots, \phi_{100}^{N} \} \ , \nonumber
\end{eqnarray}

\noindent where $\phi_{n^\text{th}}^{i^{\text{th}}}$ is the $n^{\text{th}}$ percentile of the $i^{\text{th}}$ training image.

\begin{figure}
	\centering
	\includegraphics[width=0.7\linewidth]{3_review/figures/processing/pre-processing/normalization/linear_transform_parts.eps}
	\caption{Example of piecewise linear normalization as proposed in~\cite{Nyul2000}.}
	\label{fig:imnorm}
\end{figure}

Then, the mean of each statistical landmarks $\{ \bar{\Phi}_{0}, \bar{\Phi}_{25}, \bar{\Phi}_{50}, \bar{\Phi}_{75}, \bar{\Phi}_{100} \}$ is also calculated.
Once this training stage is performed, a piecewise linear transformation $\mathcal{T}(\cdot)$ is computed as in \acs{eq}\,\eqref{eq:linearMap}.
For each test image $t$, this transformation maps each statistical landmark $\varphi_{(\cdot)}̂^{t}$ of the image $t$ to the pre-learned statistical landmarks $\bar{\Phi}_{(\cdot)}$.
An example of such piecewise linear function is depicted in \acs{fig}\,\ref{fig:imnorm}.

\begin{equation}
\small
\mathcal{T}(s(\mathbf{x})) =
  \begin{cases}
    \lceil \bar{\Phi}_{0}+( s(\mathbf{x}) - \varphi_{0}^{t} ) \left( \frac{\bar{\Phi}_{25} - \bar{\Phi}_{0}}{\varphi_{25}^{t} - \varphi_{0}^{t}} \right) \rceil \ , & \text{if $\varphi_{0}^{t} \leq s(\mathbf{x})<\varphi_{25}^{t})$} \ , \\
    \lceil \bar{\Phi}_{25}+( s(\mathbf{x}) - \varphi_{25}^{t} ) \left( \frac{\bar{\Phi}_{50} - \bar{\Phi}_{25}}{\varphi_{50}^{t} - \varphi_{25}^{t}} \right) \rceil \ , & \text{if $\varphi_{25}^{t} \leq s(\mathbf{x})<\varphi_{50}^{t})$} \ , \\
    \lceil \bar{\Phi}_{50}+( s(\mathbf{x}) - \varphi_{50}^{t} ) \left( \frac{\bar{\Phi}_{75} - \bar{\Phi}_{50}}{\varphi_{75}^{t} - \varphi_{50}^{t}} \right) \rceil \ , & \text{if $\varphi_{50}^{t} \leq s(\mathbf{x})<\varphi_{75}^{t})$} \ , \\
    \lceil \bar{\Phi}_{75}+( s(\mathbf{x}) - \varphi_{75}^{t} ) \left( \frac{\bar{\Phi}_{100} - \bar{\Phi}_{75}}{\varphi_{100}^{t} - \varphi_{75}^{t}} \right) \rceil \ , & \text{if $\varphi_{75}^{t} \leq s(\mathbf{x})\leq \varphi_{100}^{t})$} \ ,
  \end{cases}
  \label{eq:linearMap}
\end{equation}

\citeauthor{Viswanath2012} used a variant of the piecewise linear normalization presented in~\cite{Madabhushi2006a}, to standardize \ac{t2w}-\ac{mri} images~\cite{Viswanath2009,Viswanath2011,Viswanath2012}.
Instead of computing the \ac{pdf} of an entire image, a pre-segmentation of the foreground is carried out via \ac{gscale} which has been discussed in the bias correction section.
Once the foreground is detected, the largest region is extracted, and the regular piecewise linear normalization is applied.

\begin{figure}
\centering
	\hspace*{\fill}
	\subfigure[Illustration and location of the bladder on a \ac{t2w}-\ac{mri} image acquired with a \SI{3}{\tesla} \ac{mri} scanner]{\label{subfig:bladder} \includegraphics[width=0.3\linewidth]{3_review/figures/processing/pre-processing/niaf/t2w_bladder.eps}} \hfill
	\subfigure[Illustration and location of the femoral arteries on a \ac{t1w}-\ac{mri} image acquired with a \SI{3}{\tesla} \ac{mri} scanner]{\label{subfig:arteries} \includegraphics[width=0.3\linewidth]{3_review/figures/processing/pre-processing/niaf/t1w_arteries.eps}}
	\hspace*{\fill}
	\caption{Illustration of the two organs used in~\cite{Niaf2011,Niaf2012} to normalize \acs*{t2w}-\acs*{mri} and \acs*{t1w}-\acs*{mri} images.}
	\label{fig:niaf}
\end{figure}

The standardization problem can be tackled by normalizing the MRI images using the \ac{si} of some known organs present in these images. 
\citeauthor{Niaf2012} and \citeauthor{lehaire2014computer} normalized \ac{t2w}-\ac{mri} images by dividing the original \ac{si} of the images by the mean \ac{si} of the bladder~\cite{Niaf2011,Niaf2012,lehaire2014computer}, which is depicted in \acs{fig}\,\ref{subfig:bladder}.
\citeauthor{giannini2015fully} also normalized the same modality but using the signal intensity of the obturator muscle~\cite{giannini2015fully}.
Likewise, \citeauthor{Niaf2011} standardized the \ac{t1w}-\ac{mri} images using the \ac{aif}~\cite{Niaf2011}.
They computed the \ac{aif} by taking the mean of the \ac{si} in the most enhanced part of the common femoral arteries --- refer to \acs{fig}\,\ref{subfig:arteries} --- as proposed in~\cite{Wiart2007}.
Along the same line, \citeauthor{samarasinghe2016semi} normalized the \ac{si} of lesion regions in \ac{t1w}-\ac{mri} using the mean intensity of the prostate gland in the same modality~\cite{samarasinghe2016semi}.

%\end{enumerate}

\subsubsection{\acs*{mrsi} modality}

As presented in \acs{sec}\,\ref{subsec:chp2:imaging:mrsi}, \ac{mrsi} is a modality related to a one dimensional signal.
Hence, specific pre-processing steps for this type of signals have been applied instead of standard signal processing methods.

%% \setenumerate{listparindent=\parindent,itemsep=10px}
%% \setlist{noitemsep}
%% \begin{enumerate}[leftmargin=*]

\begin{figure}
  \centering
  \includegraphics[width=0.7\linewidth]{3_review/figures/processing/pre-processing/phase/phase.eps}
  \caption[Illustration of phase malignant in an \acs*{mrsi} spectra.]{Illustration of phase misalignment in an \acs*{mrsi} spectra acquired with a \SI{3}{\tesla} \acs*{mrsi} scanner. Note the distortion of the signal specially visible for the water and citrate peaks visible at \SI{5}{\ppm} and \SI{3}{\ppm}, respectively}
  \label{fig:phase}
\end{figure}

%\item[$-$] \textbf{\textit{Phase correction:}}
\paragraph{Phase correction} Acquired \ac{mrsi} spectra suffer from zero-order and first-order phase misalignment~\cite{Chen2002,Osorio-Garcia2012} as depicted in \acs{fig}\,\ref{fig:phase}. 
\citeauthor{Parfait2012} and \citeauthor{trigui2017automatic} used a method proposed by \citeauthor{Chen2002} where the phase of \ac{mrsi} signal is corrected based on entropy minimization in the frequency domain~\cite{Parfait2012,trigui2016classification,trigui2017automatic}.
The corrected \ac{mrsi} signal $o(\xi)$ can be expressed as:

\begin{eqnarray}
	\Re(o(\xi)) & = & \Re(s(\xi))\cos(\Phi(\xi)) - \Im(\xi)\sin(\Phi(\xi)) \ , \nonumber  \\
	\Im(o(\xi)) & = & \Im(s(\xi))\cos(\Phi(\xi)) + \Re(\xi)\sin(\Phi(\xi)) \ , \nonumber \\
	\Phi(\xi) & = & \phi_0 + \phi_1 \frac{\xi}{N} \ , \label{eq:mrsiphcorr}
\end{eqnarray}

\noindent where $\Re(\cdot)$ and $\Im(\cdot)$ are the real and imaginary part of the complex signal, respectively, $s(\xi)$ is the corrupted \ac{mrsi} signal, $\phi_0$ and $\phi_1$ are the zero-order and first-order phase correction terms respectively and $N$ is the total number of samples of the \ac{mrsi} signal.

\citeauthor{Chen2002} tackled this problem as an optimization in which $\phi_0$ and $\phi_1$ have to be inferred.
Hence, the simplex Nelder-Mead optimizer~\cite{Nelder1965} is used to minimize the following cost function based on the \textit{Shannon entropy} formulation:

\begin{equation}
	\hat{\Phi} = \argmin_{\Phi} \left[ - \sum \Re(s'(\xi)) \ln \Re(s'(\xi)) + \lambda \|\Re(s(\xi))\|_2 \right] \ ,
	\label{eq:phcost}
\end{equation}

\noindent where $s'(\xi)$ is the first derivative of the corrupted signal $s(\xi)$ and $\lambda$ is a regularization parameter.
Once the best parameter $\Phi$ vector is obtained, the \ac{mrsi} signal is corrected using \acs{eq}\,\eqref{eq:mrsiphcorr}.

\begin{figure}
\centering
\includegraphics[width=0.7\linewidth]{3_review/figures/processing/pre-processing/water/water_fat.eps}
\caption[Illustration of water and fat residues in \acs*{mrsi} signal after suppression during acquisition.]{Illustration of the residues of water and fat even after their suppression during the acquisition protocol. The acquisition has been carried out with a \SI{3}{\tesla} \acs*{mri}.}
\label{fig:waterfat}
\end{figure}

%\item[$-$] \textbf{\textit{Water and lipid residuals filtering:}} 
\paragraph{Water and lipid residuals filtering} The water and lipid metabolites occur in much higher concentrations than the metabolites of interest, namely choline, creatine, and citrate~\cite{Zhu2010,Osorio-Garcia2012}.
Fortunately, specific \ac{mrsi} sequences have been developed in order to suppress water and lipid metabolites using pre-saturation techniques~\cite{Zhu2010}.
However, these techniques do not perfectly remove water and lipids peaks and some residuals are still present in the \ac{mrsi} spectra as illustrated in \ac{fig}\,\ref{fig:waterfat}.
Therefore, different post-processing methods have been proposed to enhance the quality of the \ac{mrsi} spectra by removing these residuals.
For instance, \citeauthor{Kelm2007}~\cite{Kelm2007} used the HSVD algorithm proposed by \citeauthor{Pijnappel1992}~\cite{Pijnappel1992} which models the \ac{mrsi} signal by a sum of exponentially damped sine waves in the time domain as \acs{eq}\,\eqref{eq:fidsig}.
%In the time domain, a \ac{mrsi} signal $s(t)$ is modelled by a sum of $K$ exponentially damped sinusoids such that:

\begin{equation}
	s(t) = \sum_{k=1}^{K} a_{k}\exp(i \phi_k) \exp( -d_{k} + i 2 \pi f_{k} ) t + \eta(t) \ ,
	\label{eq:fidsig}
\end{equation}

\noindent where $a_k$ is the amplitude proportional to the metabolite concentration with a resonance frequency $f_{k}$, $d_k$ represents the damping factor of the exponential, $\phi_k$ is the first-order phase, and $\eta(t)$ is a complex white noise.

The ``noise-free signal'' can be found using the \ac{svd} decomposition~\cite{Pijnappel1992}.
Therefore, the noisy signal is reorganized inside a Hankel matrix $H$.
It can be shown that the signal is considered as a ``noise-free signal'' if the rank of $H$ is equal to rank $K$.
However, due to the presence of noise, $H$ is in fact a full rank matrix.
Thus, to recover the ``noise-free signal'', the rank of $H$ is truncated to $K$ using its \ac{svd} decomposition.
Hence, knowing the cut off frequencies of water --- i.e., \SI{4.65}{\ppm} --- and lipid --- i.e., \SI{2.2}{\ppm} --- metabolites, their corresponding peaks are reconstructed and subtracted from the original signal~\cite{Laudadio2002}.
	
%\item[$-$] \textbf{\textit{Baseline correction:}} 
\paragraph{Baseline correction} Sometimes, the problem discussed in the above section regarding the lipid molecules is not addressed simultaneously with water residuals suppression.
Lipids and macro-molecules are known to affect the baseline of the \ac{mrsi} spectra, causing errors while quantifying metabolites, especially the citrate metabolite.

\citeauthor{Parfait2012} made the comparison of two different methods to detect the baseline and correct the \ac{mrsi} spectra~\cite{Parfait2012} which are based on~\cite{Lieber2003,Devos2004}.
\citeauthor{Lieber2003} corrected the baseline in the frequency domain by fitting a low degree polynomial $p(x)$ --- e.g., 2\textsuperscript{nd} or 3\textsuperscript{rd} degree --- to the \ac{mrsi} signal $s(x)$ in a least-squares sense~\cite{Lieber2003}.
Then, the values of the fitted polynomial are re-assigned as:

\begin{equation}
	p_f(x) = 
	\begin{cases}
		p(x) \ , & \text{if $p(x) \leq s(x)$} \ , \\
		s(x) \ , & \text{if $p(x) > s(x)$} \ . \\
	\end{cases}
	\label{eq:lieber}
\end{equation}

Finally, this procedure of fitting and re-assignment is repeated on $p_f(x)$ until a stopping criterion is reached.
The final polynomial function is subtracted from the original signal $s(x)$ to correct it.
\citeauthor{Parfait2012}~\cite{Parfait2012} modified this algorithm by convolving a Gaussian kernel to smooth the \ac{mrsi} signal instead of fitting a polynomial function, keeping the rest of the algorithm identical. 
Unlike~\citeauthor{Lieber2003}~\cite{Lieber2003}, \citeauthor{Devos2004}~\cite{Devos2004} corrected the baseline in the time domain by multiplying the \ac{mrsi} signal by a decreasing exponential function as:
\begin{equation}
	c(t) = \exp (- \beta t) \ ,
	\label{eq:devos}
\end{equation}

\noindent with a typical $\beta$ value of $0.15$.
However, \citeauthor{Parfait2012} concluded that the method proposed in~\cite{Lieber2003} outperformed the one in~\cite{Devos2004}.
The later study of \citeauthor{trigui2017automatic} used this conclusion and adopted the same method~\cite{trigui2016classification,trigui2017automatic}.

In the contemporary work of \citeauthor{Tiwari2012}~\cite{Tiwari2012}, the authors detected the baseline using a local non-linear fitting method avoiding regions with significant peaks, which have been detected using an experimentally parametric signal-to-noise ratio set to a value larger than \SI{5}{\decibel}.


\begin{figure}
\centering
\includegraphics[width=0.7\linewidth]{3_review/figures/processing/pre-processing/frequency/frequency.eps}
\caption[Illustration of frequency misalignment in a \acs*{mrsi} spectra.]{Illustration of frequency misalignment in a \acs*{mrsi} spectra acquired with a \SI{3}{\tesla} \acs*{mrsi} scanner. The water peak is known to be aligned at \SI{4.65}{\ppm}. However, it can be seen that the peak on this spectra is aligned at around \SI{5.1}{\ppm}.}
\label{fig:frequency}
\end{figure}

%\item[$-$] \textbf{\textit{Frequency alignment:}} 
\paragraph{Frequency alignment}
Due to variations of the experimental conditions, a frequency shift is commonly observed in the \ac{mrsi} spectra~\cite{Chen2002,Osorio-Garcia2012} as depicted in \acs{fig}\,\ref{fig:frequency}.
\citeauthor{Tiwari2012}~\cite{Tiwari2012} corrected this frequency shift by first detecting known metabolite peaks such as choline, creatine, or citrate and minimizing the frequency error between the experimental and theoretical values for each of these peaks~\cite{Tiwari2012}.

%\item[$-$] \textbf{\textit{Normalization:}} 
\paragraph{Normalization}
The \ac{nmr} spectra is subject to variations due to intra-patient variations and non homogeneity of the magnetic field.
\citeauthor{Parfait2012} as in~\cite{Devos2004} compared two methods to normalize \ac{mrsi} signal~\cite{Parfait2012}.
In each method, the original \ac{mrsi} spectra is divided by a normalization factor, similar to the intensity normalization described earlier.
The first approach consists in estimating the water concentration from an additional \ac{mrsi} sequence where the water has not been suppressed.
The estimation is performed using the previously HSVD algorithm.
The second approach does not require any additional acquisition and is based on the L$_2$ norm of the \ac{mrsi} spectra $\|s(\xi)\|_2$. 
It should be noted that both \citeauthor{Parfait2012} and \citeauthor{Devos2004} concluded that the L$_2$ normalization is the most efficient method~\cite{Parfait2012}.
Lately, \citeauthor{trigui2017automatic} used the L$_2$ normalization in their framework~\cite{trigui2016classification,trigui2017automatic}.
 
%\end{enumerate}

\subsubsection{Summary}

The different pre-processing methods are summarized in \acs{tab}~\ref{tab:summary-preproc}.

\begin{table}
  \caption{Overview of the pre-processing methods used in \acs*{cad} systems.}
  \scriptsize
  \centering
  \begin{tabular}{l r}
    \toprule
    \textbf{Pre-processing operations} & \textbf{References} \\
    \midrule
    \textbf{\ac{mri} pre-processing:} & \\
    \quad Noise filtering: &  \\
    \quad \quad $\bullet$ Anisotropic median-diffusion filtering & \cite{rampun2015classifying,rampun2015computer,rampun2016computer,rampun2016computerb,rampun2016quantitative}  \\
    \quad \quad $\bullet$ Gaussian filtering & \cite{samarasinghe2016semi}  \\
    \quad \quad $\bullet$ Median filtering & \cite{Ozer2009,Ozer2010}  \\
    \quad \quad $\bullet$ Wavelet-based filtering & \cite{Ampeliotis2007,Ampeliotis2008,Lopes2011} \\ \\ [-1.5ex]
    \quad Bias correction: & \\
    \quad \quad $\bullet$ Parametric methods & \cite{Lv2009,Viswanath2009,giannini2015fully} \\
    \quad \quad $\bullet$ Non-parametric methods & \cite{Viswanath2011} \\ \\ [-1.5ex]
    \quad Standardization: & \\
    \quad \quad $\bullet$ Statistical-based normalization: & \cite{Artan2009,Artan2010,Lv2009,Ozer2009,Ozer2010,rampun2015classifying,rampun2015computer,rampun2016computer,rampun2016computerb,rampun2016quantitative,Viswanath2009,Viswanath2011,Viswanath2012} \\
    \quad \quad $\bullet$ Organ \ac{si}-based normalization & \cite{Niaf2011,Niaf2012,lehaire2014computer,samarasinghe2016semi} \\ \\ [-1.5ex]
    \textbf{\ac{mrsi} pre-processing:} & \\ \\ [-1.5ex]
    \quad Phase correction & \cite{Parfait2012,trigui2016classification,trigui2017automatic} \\
    \quad Water and lipid residuals filtering & \cite{Kelm2007} \\
    \quad Baseline correction & \cite{Parfait2012,Tiwari2012,trigui2016classification,trigui2017automatic} \\
    \quad Frequency alignment & \cite{Tiwari2012,trigui2016classification,trigui2017automatic} \\
    \quad Normalization & \cite{Parfait2012,trigui2016classification,trigui2017automatic} \\
    \bottomrule
  \end{tabular}
\label{tab:summary-preproc}
\end{table}

\input{3_review/segmentation.tex}
\input{3_review/registration.tex}

\section{Image classification framework}\label{sec:chp3:img-clas}


\subsection{\acs*{cade}: \acsp*{roi} detection/selection} \label{subsec:chp3:img-clas:roiSel}

\begin{table}
  \caption{Overview of the \acs*{cade} strategies employed in \acs*{cad} systems.}
  \scriptsize
  \centering
  \begin{tabularx}{\textwidth}{l >{\raggedleft\arraybackslash}X}
    \toprule
    \textbf{\ac{cade}: \acp{roi} selection strategy} & \textbf{References} \\
    \midrule
    All voxels-based approach & \cite{Artan2009,Artan2010,Giannini2013,Kelm2007,Liu2009,Lopes2011,Matulewicz2013,Mazzetti2011,Ozer2009,Ozer2010,Parfait2012,Sung2011,Tiwari2007,Tiwari2008,Tiwari2009,Tiwari2009a,Tiwari2010,Tiwari2012,Tiwari2013,Viswanath2008,Viswanath2008a,Viswanath2009,Viswanath2011,Viswanath2012,trigui2016classification,trigui2017automatic,lehaire2014computer,khalvati2015automated,rampun2015classifying,rampun2015computer,rampun2016computer,rampun2016computerb,rampun2016quantitative} \\
    Lesions candidate detection & \cite{Litjens2011,Litjens2012,Litjens2014,Vos2012,cameron2014multiparametric,cameron2016maps} \\
    \bottomrule
  \end{tabularx}
\label{tab:cade}
\end{table}

As discussed in the introduction and shown in \acs{fig}\,\ref{fig:wkfcad}, the image classification framework is often composed of a \ac{cade} and a \ac{cadx}.
In this section, we focus on studies which embed a \ac{cade} in their framework.
Two approaches are considered to define a \ac{cade}: (i) voxel-based delineation and (ii) lesion segmentation.
These methods are summarized in \acs{tab}~\ref{tab:cade}.
The first strategy is in fact linked to the nature of the classification framework and concerns the majority of the studies reviewed~\cite{Artan2009,Artan2010,Giannini2013,Kelm2007,Liu2009,Lopes2011,Matulewicz2013,Mazzetti2011,Ozer2009,Ozer2010,Parfait2012,Sung2011,Tiwari2007,Tiwari2008,Tiwari2009,Tiwari2009a,Tiwari2010,Tiwari2012,Tiwari2013,Viswanath2008,Viswanath2008a,Viswanath2009,Viswanath2011,Viswanath2012,trigui2016classification,trigui2017automatic,lehaire2014computer,khalvati2015automated,rampun2015classifying,rampun2015computer,rampun2016computer,rampun2016computerb,rampun2016quantitative}.
Each voxel is a possible candidate and will be classified as cancer or healthy.
The second group of methods is composed of method implementing a lesion segmentation algorithm to delineate potential candidates to further obtain a diagnosis through the \ac{cadx}.
This approach is borrowed from other application areas such as breast cancer.
These methods are in fact very similar to the classification framework used in \ac{cadx} later.

Regarding lesion candidate detection, \citeauthor{Vos2012} highlighted lesion candidates by detecting blobs in the \ac{adc} map~\cite{Vos2012}.
These candidates are filtered using some \textit{a priori} criteria such as \ac{si} or diameter.
As mentioned in \acs{sec}\,\ref{subsec:chp2:imaging:mrsi} and \acs{tab}~\ref{tab:modmri}, low \ac{si} in \ac{adc} map can be linked to potential \ac{cap}.
Hence, blob detectors are suitable to highlight these regions. 
Blobs are detected in a multi-resolution scheme, by computing the three main eigenvalues $\{ \lambda_{\sigma,1},\lambda_{\sigma,2},\lambda_{\sigma,3} \}$ of the Hessian matrix, for each voxel location of the \ac{adc} map at a specific scale $\sigma$~\cite{Li2003}.
The probability $p$ of a voxel $\mathbf{x}$ being a part of a blob at the scale $\sigma$ is given by:

\begin{equation}
P(\mathbf{x},\sigma) = \begin{cases}
	\frac{\| \lambda_{\sigma,3}(\mathbf{x}) \|^{2}}{\| \lambda_{\sigma,1} (\mathbf{x}) \|} \ , & \text{if } \lambda_{\sigma,k}(\mathbf{x}) > 0 \text{ with } k = \{1,2,3\} \  , \\
	0 \ , & \text{otherwise} \ .
\end{cases}
\label{eq:blobdet}
\end{equation}

\noindent The fusion of the different scales is computed as:

\begin{equation}
	L(\mathbf{x}) = \max P(\mathbf{x},\sigma) , \forall \sigma \ .
	\label{eq:fusionBlob}
\end{equation}

The candidate blobs detected are then filtered depending on their appearances --- i.e., maximum of the likelihood of the region, diameter of the lesion --- and their \ac{si} in \ac{adc} and \ac{t2w}-\ac{mri} images.
The detected regions are then used as inputs for the \ac{cadx}.
\citeauthor{cameron2016maps} used a similar approach by automatically selecting low \ac{si} connected regions in the \ac{adc} map with a size larger than \SI{1}{\milli\metre\squared}~\cite{cameron2014multiparametric,cameron2016maps}.

\citeauthor{Litjens2011} used a pattern recognition approach in order to delineate the \acp{roi}~\cite{Litjens2011}.
A blobness map is computed in the same manner as in~\cite{Vos2010} using the multi-resolution Hessian blob detector on the \ac{adc} map, \ac{t2w}, and pharmacokinetic parameters maps (see \acs{sec}\,\ref{subsec:chp3:img-clas:CADX-fea-dec} for details about those parameters).
Additionally, the position of the voxel $\mathbf{x}=\{x,y,z\}$ is used as a feature as well as the Euclidean distance of the voxel to the prostate center.
Hence, each feature vector is composed of 8 features and a \ac{svm} classifier is trained using a \ac{rbf} kernel (see \acs{sec}\,\ref{subsec:chp3:img-clas:CADX-clas} for more details).

Subsequently, \citeauthor{Litjens2012} modified this approach by including only features related to the blob detection on the different maps as well as the original \acp{si} of the parametric images~\cite{Litjens2012}.
Two new maps are introduced based on texture and a \ac{knn} classifier is used instead of a \ac{svm} classifier.
The candidate regions are then extracted by performing a local maxima detection followed by post-processing region-growing and morphological operations. 

\subsection{\acs*{cadx}: Feature detection} \label{subsec:chp3:img-clas:CADX-fea-dec}

Discriminative features which help to recognize \ac{cap} from healthy tissue need to be first detected.
This processing is known in computer vision as feature extraction. 
However, feature extraction also refers to the name given in pattern recognition to some types of dimension reduction methods which are later presented.
In order to avoid confusion between these two aspects, in this survey, the procedure ``detecting'' or ``extracting'' features from images and signals is defined as feature detection.
This section summarizes the different features used in \ac{cad} for \ac{cap}.

\subsubsection{Image-based features}\label{subsubsec:chp3:img-clas:CADX-fea-dec:Img-fea}

This section focuses on image-based features which can be categorized into two categories: (i) voxel-wise detection and (ii) region-wise detection.

\paragraph{Voxel-wise detection}
This strategy refers to the fact that a feature is extracted at each voxel location.
As discussed in \acs{chp}\,\ref{chap:2}, \ac{cap} has an influence on the \ac{si} in \ac{mpmri} images.
Therefore, intensity-based feature is the most commonly used feature~\cite{Ampeliotis2007,Ampeliotis2008,Vos2008,rampun2016computerb,rampun2015classifying,Giannini2013,Artan2009,Artan2010,Chan2003,Langer2009,Litjens2011,Litjens2012,Litjens2014,Liu2009,Ozer2009,Ozer2010,trigui2016classification,trigui2017automatic,cameron2014multiparametric,cameron2016maps,khalvati2015automated,chung2015prostate,giannini2015fully,Niaf2011,Niaf2012,lehaire2014computer}.
This feature consists in the extraction of the intensity of the \ac{mri} modality of interest.

Edge-based features have also been used to detect \ac{si} changes but bring additional information regarding the \ac{si} transition.
Each feature is computed by convolving the original image with an edge operator.
Three operators are commonly used: (i) Prewitt operator~\cite{Prewitt1970}, (ii) Sobel operator~\cite{Sobel1970}, and (iii) Kirsch operator~\cite{Kirsch1971}.
These operators differ due to the kernel used which attenuates more or less the noise.
Multiple studies used the resulting magnitude and orientation of the edges computed in their classification frameworks~\cite{Niaf2011,Niaf2012,Tiwari2009a,Tiwari2010,Tiwari2013,Viswanath2008,Viswanath2011,rampun2016quantitative,rampun2015computer,rampun2016computer,lehaire2014computer,khalvati2015automated,chung2015prostate}.

\begin{figure}
	\hspace*{\fill}
		\subfigure[$\theta=0^{\circ}$.]{\label{subfig:gab1} \includegraphics[width=0.2\linewidth]{3_review/figures/feature-detection/gabor/gabor_1.eps}} \hfill
		\subfigure[$\theta=60^{\circ}$.]{\label{subfig:gab2} \includegraphics[width=0.2\linewidth]{3_review/figures/feature-detection/gabor/gabor_2.eps}} \hfill
		\subfigure[$\theta=120^{\circ}$.]{\label{subfig:gab3} \includegraphics[width=0.2\linewidth]{3_review/figures/feature-detection/gabor/gabor_3.eps}} \hfill
		\subfigure[$\theta=180^{\circ}$.]{\label{subfig:gab4} \includegraphics[width=0.2\linewidth]{3_review/figures/feature-detection/gabor/gabor_4.eps}}
	\hspace*{\fill}
	\caption[Illustration of 4 different Gabor filters.]{Illustration of 4 different Gabor filters varying their orientations $\theta$.}
	\label{fig:gabor}
\end{figure}

Gabor filters~\cite{Gabor1946,Daugman1985} offer an alternative to the usual edge detector, with the possibility to tune the direction and the frequency of the filter to encode a specific pattern. 
A Gabor filter is defined by the modulation of a Gaussian function with a sine wave which can be further rotated and is formalized as in \acs{eq}\,\ref{eq:gabor}.
\begin{equation}
	g(x,y;\theta,\psi,\sigma,\gamma) = \exp \left( - \frac{x'^{2}+ \gamma^{2}y'^{2}}{2 \sigma^{2}} \right) \cos \left( 2 \pi \frac{x'}{\lambda} + \psi \right) \ ,
        \label{eq:gabor}
\end{equation}

\noindent with 

\begin{eqnarray}
	x' & = & s\left( x \cos \theta + y \sin \theta \right) \ , \nonumber \\
	y' & = & s \left( - x \sin \theta + y \cos \theta \right) \ , \nonumber
\end{eqnarray}

\noindent where $\lambda$ is the wavelength of the sinusoidal factor, $\theta$ represents the orientation of the Gabor filter, $\psi$ is the phase offset, $\sigma$ is the standard deviation of the Gaussian envelope, $\gamma$ is the spatial aspect ratio, and $s$ is the scale factor.
In an effort to characterize pattern and texture, a bank of Gabor filters is usually created with different angles, scale, and frequency --- refer to \acs{fig}\,\ref{fig:gabor} --- and then convolved with the image.
\citeauthor{Viswanath2012}~\cite{Viswanath2012}, \citeauthor{Tiwari2012}~\cite{Tiwari2012} and more recently \citeauthor{khalvati2015automated}~\cite{khalvati2015automated} and \citeauthor{chung2015prostate}~\cite{chung2015prostate} have designed a bank of Gabor filters to characterized texture and edge information in \ac{t2w}-\ac{mri} and \ac{dw}-\ac{mri} modalities.

Texture-based features provide other characteristics discerning \ac{cap} from healthy tissue.
The most common texture analysis for image classification is based on the \ac{glcm} with their related statistics which have been proposed by \citeauthor{Haralick1973} in~\cite{Haralick1973}.
In a neighborhood around a central voxel, a \ac{glcm} is build considering each voxel pair defined by a specific distance and angle.
Then, using the \ac{glcm}, a set of statistical features is computed as defined in \acs{tab}~\ref{tab:glcm} and assigned to the location of the central voxel.
Therefore, $N$ --- up to 14 --- statistical maps are derived from the \ac{glcm} analysis, one per statistics presented in \acs{tab}~\ref{tab:glcm}.
\ac{glcm} is commonly used in \ac{cad} systems, on the different \ac{mri} modalities, namely \ac{t2w}-\ac{mri}, \ac{dce}-\ac{mri}, or \ac{dw}-\ac{mri}~\cite{Antic2013,Niaf2011,Niaf2012,Tiwari2009a,Tiwari2010,Tiwari2013,Viswanath2008,Viswanath2009,Viswanath2011,Viswanath2012,trigui2016classification,rampun2015computer,rampun2016computer,rampun2016quantitative,cameron2014multiparametric,cameron2016maps,khalvati2015automated,chung2015prostate,lehaire2014computer}.
However, the statistics extracted from the \ac{glcm} across studies vary.
Along the same line, \citeauthor{rampun2016computer} extracted from \ac{t2w}-\ac{mri}~\cite{rampun2016computer,rampun2015computer} Tamura features~\cite{tamura1978textural} composed of three features to characterize texture: (i) coarseness, (ii) contrast, and (iii) directionality.

\begin{table}
  \caption[The 14 statistical features used in conjunction with \acs*{glcm} analysis.]{The 14 statistical features for texture analysis commonly computed from the \acs*{glcm} $p$ as presented by~\cite{Haralick1973}.}
  \scriptsize
  \renewcommand{\arraystretch}{1.5}
  \centering
  \begin{tabular}{ll}
    \toprule
    \textbf{Statistical features} & \textbf{Formula} \\
    \midrule
    Angular second moment & $\sum_i \sum_j p(i,j)^2 $  \\
    Contrast & $\sum_{n=0}^{N_g - 1} n^2 \left[ \sum_{i=1}^{N_g - 1} \sum_{j=1}^{N_g - 1} p(i,j) \right] \ , | i-j |=n  $ \\
    Correlation & $\frac{\sum_i \sum_j (ij) p(i,j) - \mu_x \mu_y}{\sigma_x \sigma_y}  $ \\
    Variance & $\sum_i \sum_j (i - \mu)^2 p(i,j)  $ \\
    Inverse difference moment & $\sum_i \sum_j \frac{1}{1+(i - \mu)^2} p(i,j)  $ \\
    Sum average & $\sum_{i=2}^{2N_g} i p_{x+y}(i)  $ \\
    Sum variance & $\sum_{i=2}^{2N_g} (i-f_s)^2 p_{x+y}(i)  $ \\
    Sum entropy & $ - \sum_{i=2}^{2N_g} p_{x+y}(i) \log p_{x+y}(i)  $ \\
    Entropy & $ - \sum_i \sum_j p(i,j) \log p(i,j) $ \\
    Difference variance & $\sum_{i=0}^{N_g-1} i^2 p_{x-y}(i)  $ \\
    Difference entropy & $ - \sum_{i=0}^{N_g-1} p_{x-y}(i) \log p_{x-y}(i)  $ \\
    Info. measure of corr. 1 & $\frac{S(X;Y)-S_1(X;Y)}{\max(S(X),S(Y))}  $ \\
    Info. measure of corr. 2 & $\sqrt{\left( 1 - \exp \left[ -2( H_2(X;Y) - H(X;Y) ) \right] \right)}  $ \\
    Max. corr. coeff. & $ \sqrt{\lambda_2} \ , \text{of } Q(i,j) = \sum_k \frac{p(i,k)p(j,k)}{p_x(i)p_y(k)}  $ \\
    \bottomrule
  \end{tabular}
  \label{tab:glcm}
\end{table}

\citeauthor{Lopes2011} used fractal analysis  and more precisely a local estimation of the fractal dimension~\cite{Benassi1998}, to describe the texture roughness at a specific location.
The fractal dimension is estimated through a wavelet-based method in multi-resolution analysis.
They showed that cancerous tissues have a higher fractal dimension than healthy tissue.

\citeauthor{Chan2003} described texture using the frequency signature via the \acf{dct}\cite{Ahmed1974} defining a neighbourhood of \SI[product-units=repeat]{7x7}{\px} for modalities used, namely \ac{t2w}-\ac{mri} and \ac{dw}-\ac{mri}.
The \ac{dct} allows to decompose a portion of an image into a coefficient space, where few of these coefficients encode the significant information.
The \ac{dct} coefficients are computed such as:

\begin{equation}
	C_{k_1,k_2} = \sum_{m=0}^{M-1} \sum_{n=0}^{N-1} p_{m,n} \cos \left[ \frac{\pi}{M} \left( m + \frac{1}{2} \right) k_1 \right] \cos \left[ \frac{\pi}{N} \left( n + \frac{1}{2} \right) k_2 \right] \ ,
\end{equation}

\noindent where $C_{k_1,k_2}$ is the \ac{dct} coefficient at the position $k_1,k_2$, $M$ and $N$ are the dimension of the neighbourhood and $p_{m,n}$ is the pixel \ac{si} at the position $\{m,n\}$.

Regarding other features, \citeauthor{Viswanath2012} projected \ac{t2w}-\ac{mri} images into the wavelet space, using the Haar wavelet, and used the resulting coefficients as features~\cite{Viswanath2012}.
%The wavelet family used for the decomposition was the Haar wavelet.
\citeauthor{Litjens2011} computed the texture map based on \ac{t2w}-\ac{mri} images using a Gaussian filter bank~\cite{Litjens2011}.
Likewise, \citeauthor{rampun2016computer} employed a rotation invariant filter bank proposed in~\cite{leung2001representing}.
The bank is composed of 48 filters including Gaussian filters, first and second derivatives of Gaussian filters as well as Laplacian of Gaussian.

%% Euclidean distance from each voxel to the prostate center as well as the individual distance in the three directions $x$, $y$ and $z$. \cite{Chan2003} embedded the same information but this time using cylindrical coordinate $r$, $\theta$ and $z$ corresponding to the radius, azimuth and elevation respectively.

\paragraph{Region-wise detection}

Unlike the previous section, another strategy is to study a region instead of each pixel independently.
Usually, the feature maps are computed using the method presented in voxel-based approach followed by a step in which features are computed in some specific delineated regions to characterize them.

The most common feature type is based on statistics and more specifically the statistic-moments such as mean, standard deviation, kurtosis, and skewness~\cite{Ampeliotis2007,Ampeliotis2008,Tiwari2009a,Tiwari2010,Tiwari2013,Viswanath2008,Viswanath2009,Viswanath2012,rampun2016quantitative,rampun2015computer,rampun2016computer,Antic2013,Viswanath2011,Peng2013,cameron2014multiparametric,cameron2016maps,khalvati2015automated,chung2015prostate,Litjens2011,Litjens2012,Litjens2014,Niaf2011,Niaf2012,lehaire2014computer}.
Additionally, some studies extract additional statistical landmarks based on percentiles~\cite{Vos2008a,Antic2013,Peng2013,Vos2010,Litjens2011,Litjens2012,Litjens2014,Niaf2011,Niaf2012,Vos2012,lehaire2014computer}
The percentiles to use are manually determined by observing the \ac{pdf} of the features and checking which values allow the best to differentiate malignant from healthy tissue.

Further statistics are computed through the use of histogram-based features.
\citeauthor{Liu2013} introduced 4 different types of histogram-based features to characterize hand-delineated lesions~\cite{Liu2013}.
The first type corresponds to the histogram of the \ac{si} of the image.
The second type is the \ac{hog}~\cite{Dalal2005} which encodes the local shape of the object of interest by using the distribution of the gradient directions.
This descriptor is extracted mainly in three steps.
First, the gradient image and its corresponding magnitude and direction are computed.
Then, the \ac{roi} is divided into cells and an oriented-based histogram is generated for each cell.
At each pixel location, the orientation of the gradient votes for a bin of the histogram and this vote is weighted by the magnitude of the same gradient.
Finally, the cells are grouped into blocks and each block is normalized.
The third histogram-based type used in~\cite{Liu2013} is the shape context introduced in~\cite{Belongie2002}.
The shape context is also a way to describe the shape of an object of interest.
First, a set of points defining edges have to be detected and for each point of each edge, a log-polar-based histogram is computed using the relative points distribution.
The last set of histogram-based feature extracted is based on the framework described in~\cite{Zhao2012} which is using the Fourier transform of the histogram created via \acf{lbp}~\cite{Ojala1996}.
\Ac{lbp} is generated by comparing the value of the central pixel with its neighbours, defined through a radius and the number of connected neighbours.
Then, in the \ac{roi}, the histogram of the \ac{lbp} distribution is computed.
The \acf{dft} of the \ac{lbp} histogram is used to make the feature invariant to rotation.

Another subset of features are anatomical-based features and have been used in~\cite{Litjens2012,Litjens2014,Matulewicz2013,cameron2014multiparametric,cameron2016maps}.
\citeauthor{Litjens2012} computed the volume, compactness, and sphericity related to the given region~\cite{Litjens2012, Litjens2014}.
Additionally, \citeauthor{Litjens2014} also introduced a feature based on symmetry in which they compute the mean of a candidate lesion as well as its mirrored counter-part and compute the quotient as feature~\cite{Litjens2014}.
\citeauthor{Matulewicz2013} introduced 4 features corresponding to the percentage of tissue belonging to the regions \ac{pz}, \ac{cg}, periurethral region, or outside the prostate region for the considered \ac{roi}~\cite{Matulewicz2013}.
Finally, \citeauthor{cameron2016maps} defined 4 features based on morphology and asymmetry:
(i) the difference of morphological closing and opening of the \ac{roi}, (ii) the difference of the initial perimeter and the one after removing the high-frequency components, (iii) the difference between the initial \ac{roi} and the one after removing the high-frequency components, and (iv) the asymmetry by computing the difference of the two areas splitting the \ac{roi} by its major axes~\cite{cameron2014multiparametric,cameron2016maps}.

The last group of region-based feature is based on fractal analysis.
This group of features is based on estimating the fractal dimension which is a statistical index representing the complexity of the analyzed texture.
\citeauthor{Lv2009} proposed two features based on fractal dimension: (i) texture fractal dimension and (ii) histogram fractal dimension~\cite{Lv2009}.
The first feature is based on estimating the fractal dimension on the \ac{si} of each image and thus this feature is a statistical characteristic of the image roughness.
The second fractal dimension is estimated using the \ac{pdf} of each image and characterizes the complexity of the \ac{pdf}.
\citeauthor{Lopes2011} proposed a 3D version to estimate the fractal dimension of a volume using a wavelet decomposition~\cite{Lopes2011}.

\subsubsection{\acs*{dce}-based features}\label{subsubsec:chp3:img-clas:CADX-fea-dec:DCE-fea}

\ac{dce}-\ac{mri} is more commonly based on a \ac{si} analysis over time as presented in \acs{sec}\,\ref{subsec:chp2:imaging:dce}.
In this section, the specific features extracted for \ac{dce}-\ac{mri} analysis are presented.

\begin{table}
  \caption{Parameters used as features for a \acs*{dce} semi-quantitative analysis in \acs*{cad} systems.}
  \scriptsize
  \centering
  \begin{tabularx}{\textwidth}{l X}
    \toprule
    \textbf{Semi-quantitative features} & \textbf{Explanations} \\
    \midrule
    \textbf{Amplitude features:} & \\ \\ [-1.5ex]
    \quad $S_0$ & Amplitude at the onset of the enhancement \\
    \quad $S_{\max}$ & Amplitude corresponding to $95\%$ of the maximum amplitude \\
    \quad $S_{p}$ & Amplitude corresponding to the maximum amplitude \\
    \quad $S_f$ & Amplitude at the final time point \\ \\ [-1.5ex]
    \textbf{Time features:} & \\ \\ [-1.5ex]
    \quad $t_0$ & Time at the onset of the enhancement \\
    \quad $t_{\max}$ & Time corresponding to $95\%$ of the maximum amplitude \\
    \quad $t_{p}$ & Time corresponding to the maximum amplitude \\
    \quad $t_{f}$ & Final time \\
    \quad $t_{tp}$ & Time to peak which is the time from $t_0$ to $t_p$ \\ \\ [-1.5ex]
    \textbf{Derivatives and integral features:} & \\ \\ [-1.5ex]
    \quad $WI$ & Wash-in rate corresponding to the signal slope from $t_0$ to $t_m$ or $t_p$ \\
    \quad $WO$ & Wash-out rate corresponding to the signal slope from $t_m$ or $t_p$ to $t_p$ \\
    \quad $IAUC$ & Initial area under the curve which is the area between $t_0$ to $t_{f}$ \\
    \bottomrule
  \end{tabularx}
\label{tab:semiqua}
\end{table}

\paragraph{Whole-spectra approach}
Some studies are using the whole \ac{dce} time series as feature vector~\cite{Ampeliotis2007,Ampeliotis2008,Tiwari2012,Viswanath2008a,Viswanath2008}.
In some cases, the high-dimensional feature space is reduced using dimension reduction methods as it will be presented in the \acs{sec}\,\ref{subsec:chp3:img-clas:CADX-fea-ext}.

\begin{figure}
  \centering
  \includegraphics[width=.8\linewidth]{3_review/figures/feature-detection/dce/dce_cancer_parameters.eps}
  \caption[Semi-quantitative features used for \acs*{dce}-\acs*{mri}.]{Graphical representation of the different semi-quantitative features used for \acs*{dce}-\acs*{mri} analysis.}
  \label{fig:dceparam}
\end{figure}

\paragraph{Semi-quantitative approach}
Semi-quantitative approaches are based on mathematically modelling the \ac{dce} time series.
The parameters modelling the signal are commonly used, mainly due to the simplicity of their computation~\cite{Puech2009,Mazzetti2011,Niaf2011,Niaf2012,Sung2011,trigui2016classification,trigui2017automatic,lehaire2014computer,samarasinghe2016semi,giannini2015fully}.
Parameters included in semi-quantitative analysis are summarized in \acs{tab}~\ref{tab:semiqua} and also graphically depicted in \acs{fig}\,\ref{fig:dceparam}.
A set of time features corresponding to specific amplitude level (start, maximum, and end) are extracted.
Then, derivative and integral features are also considered as discriminative and are commonly computed.

\paragraph{Quantitative approach}
As presented in \acs{chp}\,\ref{chap:2}, quantitative approaches correspond to mathematical-pharmacokinetic models based on physiological exchanges.
Four different models have been used in \ac{cad} for \ac{cap} systems.
The most common model reviewed is the \textit{Brix model}~\cite{Artan2009,Artan2010,Sung2011,Liu2009,Ozer2009,Ozer2010}.
This model is formalized such as:
\begin{equation}
	\frac{S(t)}{S(0)} = 1 + A k_{ep} \left( \frac{\exp( -k_{ep} t ) - \exp( -k_{el} t )}{k_{el} - k_{ep}} \right) \ ,
	\label{eq:brixmod}
\end{equation}

\noindent where $S(\cdot)$ is the \ac{dce} signal, $A$ is the parameter simulating the tissue properties, $k_{el}$ is the parameter related to the first-order elimination from the plasma compartment, and $k_{ep}$ is the parameter of the transvascular permeability.
The parameters $k_{ep}$, $k_{el}$, and $A$ are computed from the \ac{mri} data and used as features.

Another model is Tofts model~\cite{Tofts1997} which has been used in~\cite{Langer2009,Giannini2013,Niaf2011,Niaf2012,Mazzetti2011,lehaire2014computer,giannini2015fully}.
In this model, the \ac{dce} signal relative to the concentration is presented as:
\begin{equation}
	C_t(t) = v_p C_p(t) + K_{trans} \int_{0}^{t} C_p(\tau) \exp( -k_{ep}(t-\tau) ) \ d\tau \ ,
	\label{eq:tofts} 
\end{equation}

\noindent where $C_t(\cdot)$ is the concentration of the medium, $C_p(\cdot)$ is the \ac{aif} which has to be estimated independently, $K_{trans}$ is the parameter related to the diffuse transport of media across the capillary endothelium, $k_{ep}$ is the parameter related to the exchanges back into the vascular space, and $v_e$ is the extravascular-extracellular space fraction defined such that $v_e = 1 - v_p$.
In this model, parameters $K_{trans}$, $k_{ep}$, and $v_e$ are computed and used as features.

\citeauthor{Mazzetti2011} and \citeauthor{giannini2015fully} used the Weibull function~\cite{Mazzetti2011,Giannini2013,giannini2015fully} which is formalized as:

\begin{equation}
	S(t) = A t \exp( -t^{B} ) \ ,
	\label{eq:weibull}
\end{equation}

\noindent where $A$ and $B$ are the two parameters which have to be inferred.

They also used another empirical model which is based on the West-like function and named the \ac{pun}~\cite{Castorina2006}, formalized as:
\begin{equation}
	S(t) = \exp \left[ r t + \frac{1}{\beta} a_0 - r \left( \exp( \beta t ) - 1 \right) \right] \ ,
	\label{eq:pun}
\end{equation}
\noindent where the parameters $\beta$, $a_0$ and $r$ are inferred.
For all these models, the parameters are inferred using an optimization curve fitting approach.

\subsubsection{\acs*{mrsi}-based features}\label{subsubsec:chp3:img-clas:CADX-fea-dec:MRSI-fea}

\paragraph{Whole spectra approach}
As in the case of \ac{dce} analysis, one common approach is to incorporate the whole \ac{mrsi} spectra in the feature vector for classification~\cite{Kelm2007,Parfait2012,Tiwari2007,Tiwari2009,Tiwari2013,Tiwari2009a,Tiwari2010,Viswanath2008a,Matulewicz2013,trigui2016classification,trigui2017automatic}.
Sometimes post-processing involving dimension reduction methods is performed to reduce the complexity during the classification as it will be presented in \acs{sec}\,\ref{subsec:chp3:img-clas:CADX-fea-ext}.

\paragraph{Quantification approach}
We can reiterate that in \ac{mrsi} only few biological markers --- i.e., choline, creatine, and citrate metabolites --- are known to be useful to discriminate \ac{cap} and healthy tissue.
Therefore, only the concentrations of these metabolites are considered as a feature prior to classification.
In order to perform this quantification, 4 different approaches have been used.
\citeauthor{Kelm2007} used the following models~\cite{Kelm2007}: QUEST~\cite{Ratiney2005}, AMARES~\cite{Vanhamme1997}, and VARPRO~\cite{Coleman1993}.
They are all time-domain quantification methods varying by the type of pre-knowledge embedded and the optimization approaches used to solve the quantification problem.
Unlike the time-domain quantification approaches, \citeauthor{Parfait2012} used the LcModel approach proposed in~\cite{Provencher1993} which solves the optimization problem in the frequency domain.
Although \citeauthor{Parfait2012} used each metabolite relative concentration individually~\cite{Parfait2012}, other authors such as \citeauthor{Kelm2007} proposed to compute relative concentrations as the ratios of metabolites as shown in \acs{eq}\,\ref{eq:ratio1} and \acs{eq}\,\ref{eq:ratio2}.

\begin{eqnarray}
	R_1 & = & \frac{ [ \text{Cho} ] + [ \text{Cr} ]}{[ \text{Cit} ]} \ . \label{eq:ratio1} \\
	R_2 & = & \frac{[ \text{Cit} ]}{[\text{Cho}]+[\text{Cr}]+[\text{Cit}]} \ , \label{eq:ratio2}
\end{eqnarray}
\noindent where $\text{Cit}$, $\text{Cho}$ and $\text{Cr}$ are the relative concentration of citrate, choline, and creatine, respectively.

Recently \citeauthor{trigui2017automatic} used an absolute quantification approach from which water sequences are acquired to compute the absolute concentration of the metabolites~\cite{trigui2016classification,trigui2017automatic}.
Absolute quantification using water as reference is based on the fact that the fully relaxed signal from water or metabolites is proportional to the number of moles of the molecules in the voxel~\cite{gasparovic2006use}.

\paragraph{Wavelet decomposition approach} 
\citeauthor{Tiwari2012} performed a wavelet packet decomposition~\cite{Coifman1992} of the spectra using the Haar wavelet basis function and use its coefficients as features.

\subsubsection{Summary}

The feature detection methods used in \ac{cad} are summarized in \acs{tab}~\ref{tab:feat}.  

\input{3_review/Table-CADX-feature-detection}

\subsection{\acs*{cadx}: Feature selection and feature extraction} \label{subsec:chp3:img-clas:CADX-fea-ext}
As presented in the previous section, it is a common practise to extract a wide variety of features.
While dealing with \ac{mpmri}, the feature space created is a high-dimensional space which might mislead or corrupt the classifier during the training phase.
Therefore, it is of interest to reduce the number of dimensions before proceeding to the classification task.
The strategies used can be grouped as: (i) feature selection and (ii) feature extraction.
In this section only the methods used in \ac{cad} for \ac{cap} systems are presented.

\subsubsection{Feature selection}\label{subsubsec:chp3:img-clas:CADX:fea-ext:sel}
The feature selection strategy is based on selecting the most discriminative feature dimensions of the high-dimensional space.
Thus, the low-dimensional space is then composed of a subset of the original features detected.
In this section, methods employed in \ac{cad} for \ac{cap} detection are presented.
A more extensive review specific to feature selection is available in~\cite{Saeys2007}.

\citeauthor{Niaf2012} make use of the p-value by using the independent two-sample t-test with equal mean for each feature dimension~\cite{Niaf2011,Niaf2012}.
In this statistical test, there are 2 classes: \ac{cap} and healthy tissue.
Hence, for each particular feature, the distribution of each class is characterized by their means $\bar{X}_1$ and $\bar{X}_2$ and standard deviation $s_{X_1}$ and $s_{X_2}$.
Therefore, the null hypothesis test is based on the fact that these both distribution means are equal.
The t-statistic used to verify the null hypothesis is formalized such that:

\begin{eqnarray}
t & = & \frac{\bar {X}_1 - \bar{X}_2}{s_{X_1X_2} \cdot \sqrt{\frac{1}{n_1}+\frac{1}{n_2}}} \ , \label{eq:tstat} \\
s_{X_1X_2} & = & \sqrt{\frac{(n_1-1)s_{X_1}^2+(n_2-1)s_{X_2}^2}{n_1+n_2-2}} \ , \nonumber
\end{eqnarray}

\noindent where $n_1$ and $n_2$ are the number of samples in each class.
From \acs{eq}\,\eqref{eq:tstat}, more the means of the class distribution diverge, the larger the $t$-statistic $t$ will be, implying that this particular feature is more relevant and able to make the distinction between the two classes. 

The $p$-value statistic is deduced from the $t$-test and corresponds to the probability of obtaining such an extreme test assuming that the null hypothesis is true~\cite{Goodman1999}.
%Hence, smaller the $p$-value, the more likely we are to reject the null hypothesis and more relevant the feature is likely to be.
Hence, smaller the $p$-value, the more likely the null hypothesis to be rejected and more relevant the feature is likely to be.
Finally, the features are ranked and the most significant features are selected.
However, this technique suffers from a main drawback since it assumes that each feature is independent, which is unlikely to happen and introduces a high degree of redundancy in the features selected.

\citeauthor{Vos2012} in~\cite{Vos2012} employed a similar feature ranking approach but make use of the Fisher discriminant ratio to compute the relevance of each feature dimension.
Taking the aforementioned formulation, the Fisher discriminant ratio is formalized as the ratio of the interclass variance to the intraclass variance as:

\begin{equation}
F_r = \frac{(\bar{X}_1 - \bar{X}_2)^2}{s^{2}_{X_1}+s^{2}_{X_2}} \ .
\label{eq:fisherratio}
\end{equation}

Therefore, a relevant feature dimension is selected when the interclass variance is maximum and the intraclass variance in minimum.
Once the features are ordered, the authors select the feature dimensions with the largest Fisher discriminant ratio.

\Ac{mi} is a possible metric to use for selecting a subset of feature dimensions.
This method has previously been presented in \acs{sec}\,\ref{subsec:chp3:img-reg:reg} and expressed in \acs{eq}\,\eqref{eq:midef}.
%%%% As previously presented in Sect.~\ref{subsec:chp3:img-reg:reg} (see Eq.~\eqref{eq:midef}), the computation of the entropies involves the estimation of some \acp{pdf} and the data being usually continous variables, it is then necessary to estimate the \acp{pdf} using a method such as Parzen windows.
%%%% Definition of the \ac{mi} was presented in Sect.~\ref{subsec:chp3img-reg:reg} and formalized in Eq. \eqref{eq:midef} and as previously mentioned, the computation of the entropies involves the estimation of some \acp{pdf} and the data being usually continuous variables, it is then necessary to estimate the \acp{pdf} using a method such as Parzen windows.
\citeauthor{Peng2005}~\cite{Peng2005} introduced two main criteria to select the feature dimensions based on \ac{mi}: (i) maximal relevance and (ii) minimum redundancy.
Maximal relevance criterion is based on the paradigm that the classes and the feature dimension which has to be selected have to share a maximal \ac{mi} and is formalized as:
\begin{equation}
  \argmax Rel(\mathbf{x},c) = \frac{1}{|\mathbf{x}|} \sum_{x_i \in \mathbf{x}} MI(x_i,c)  \ , 
  \label{eq:mRel}
\end{equation}
\noindent where $\mathbf{x} = \{x_i; i=1,\cdots,d\}$ is a feature vector of $d$ dimensions and $c$ is the class considered.
As in the previous method, using maximal relevance criterion alone imply an independence between each feature dimension.
The minimal redundancy criterion enforce the selection of a new feature dimension which shares as little as possible \ac{mi} with the previously selected feature dimensions such that:
\begin{equation}
  \argmin Red(\mathbf{x}) = \frac{1}{|\mathbf{x}|^2} \sum_{x_i,x_j \in \mathbf{x}} MI(x_i,x_j)  \ . 
  \label{eq:mRed}
\end{equation}
Combination of these two criteria is known as the \ac{mrmr} algorithm~\cite{Peng2005}.
Two combinations are usually used: (i) the difference or (ii) the quotient.
This method has been used at several occasions for the selecting a subset of features prior to classification~\cite{Niaf2011,Niaf2012,lehaire2014computer,Viswanath2012,khalvati2015automated,chung2015prostate}.

\subsubsection{Feature extraction}\label{subsubsec:chp3:img-clas:CADX:fea-ext:ext}
The feature extraction strategy is related to dimension reduction methods but not selecting discriminative features.
Instead, these methods aim at mapping the data from the high-dimensional space into a low-dimensional space to maximize the separability between the classes.
As in the previous sections, only methods employed in \ac{cad} system are reviewed in this section.
We refer the reader to~\cite{Fodor2002} for a full review of feature extraction techniques.

\ac{pca} is the most commonly used linear mapping method in \ac{cad} systems.
\ac{pca} is based on finding the orthogonal linear transform mapping the original data into a low-dimensional space.
The space is defined such that the linear combinations of the original data with the $k^{th}$ greatest variances lie on the $k^{th}$ principal components~\cite{Jolliffe2002}.
The principal components are computed by using the eigenvectors-eigenvalues decomposition of the covariance matrix.
Let $\mathbf{x}$ denote the data matrix.
Then, the covariance matrix and eigenvectors-eigenvalues decomposition are defined as in \acs{eq}\,\eqref{eq:covmat}, and \acs{eq}\,\eqref{eq:eigpca}, respectively. 
The eigenvectors-eigenvalues decomposition can be formalized as:
\begin{equation}
  \Sigma = \mathbf{x}^{\text{T}} \mathbf{x} \ .
  \label{eq:covmat}
\end{equation}

\begin{equation}
  \mathbf{v}^{-1} \Sigma \mathbf{v} = \Lambda \ ,
  \label{eq:eigpca}
\end{equation}
\noindent where $\mathbf{v}$ are the eigenvectors matrix and $\Lambda$ is a diagonal matrix containing the eigenvalues. 

It is then possible to find the new low-dimensional space by sorting the eigenvectors using the eigenvalues and finally select the eigenvectors corresponding to the largest eigenvalues.
The total variation that is the sum of the principal eigenvalues of the covariance matrix~\cite{Fodor2002}, usually corresponds to the \SIrange{95}{98}{\percent} of the cumulative sum of the eigenvalues.
\citeauthor{Tiwari2012} used \ac{pca} in order to reduce the complexity of feature space~\cite{Tiwari2008,Tiwari2009,Tiwari2012}.

Non-linear mapping has been also used for dimension reduction and is mainly based on Laplacian eigenmaps and \acf{lle} methods.
Laplacian eigenmaps also referred as spectral clustering in computer vision, aim to find a low-dimensional space in which the proximity of the data should be preserved from the high-dimensional space~\cite{Shi2000,Belkin2001}.
Therefore, two adjacent data points in the high-dimensional space should also be close in the low-dimensional space.
Similarly, two distant data points in the high-dimensional space should also be distant in the low-dimensional space.
To compute this projection, an adjacency matrix is defined as:
\begin{equation}
	W(i,j) = \exp \| \mathbf{x}_i - \mathbf{x}_j \|_2 \ ,
	\label{eq:gew}
\end{equation}

\noindent where $\mathbf{x}_i$ and $\mathbf{x}_j$ are the two samples considered.
Then, the low-dimensional space is found by solving the generalized eigenvectors-eigenvalues problem:

\begin{equation}
	(D-W)\mathbf{y} = \lambda D \mathbf{y} \ ,
	\label{eq:geeig}
\end{equation}

\noindent where $D$ is a diagonal matrix such that $D(i,i) = \sum_j W(j,i)$.
Finally the low-dimensional space is defined by the $k$ eigenvectors of the $k$ smallest eigenvalues~\cite{Belkin2001}.
\citeauthor{Tiwari2009a}~\cite{Tiwari2007,Tiwari2009,Tiwari2009a} and \citeauthor{Viswanath2008}~\cite{Viswanath2008} used this spectral clustering to project their feature vector into a low-dimensional space.
The feature space in these studies is usually composed of features extracted from a single or multiple modalities and then concatenated before applying the Laplacian eigenmaps dimension reduction technique.

\citeauthor{Tiwari2013} used a slightly different approach by combining the Laplacian eigenmaps techniques with a prior multi-kernel learning strategy~\cite{Tiwari2009,Tiwari2013}.
First, multiple features are extracted from multiple modalities.
The features of a single modality are then mapped to a higher-dimensional space via the Kernel trick~\cite{Aizerman1964}, namely a Gaussian kernel.
Then, each kernel is linearly combined to obtain a combined kernel $K$ and the adjacency matrix $W$ is computed.
Finally, the same scheme as in the Laplacian eigenmaps is applied.
However, in order to use the combined kernel, \acs{eq}\,\eqref{eq:geeig} is rewritten as:

\begin{equation}
  K (D-W) K^{\text{T}} \mathbf{y} = \lambda K D K^{\text{T}} \mathbf{y} \ ,
  \label{eq:sesmik}
\end{equation}
\noindent which is solved as a generalized eigenvectors-eigenvalues problem as previously.
\citeauthor{Viswanath2011} used Laplacian eigenmaps inside a bagging framework in which multiple embeddings are generated by successively selecting feature dimensions~\cite{Viswanath2011}.

\Ac{lle} is another common non-linear dimension reduction technique widely used, first proposed in~\cite{Roweis2000}.
\ac{lle} is based on the fact that a data point in the feature space is characterized by its neighbourhood.
Thus, each data point in the high-dimensional space is transformed to represent a linear combination of its $k$-nearest neighbours.
This can be expressed as:
\begin{equation}
	\hat{\mathbf{x}}_i = \sum_j W(i,j) \mathbf{x}_j \ ,
	\label{eq:lincomlle}
\end{equation}

\noindent where $\hat{\mathbf{x}}_i$ are the data points estimated using its neighbouring data points $\mathbf{x}_j$, and $W$ is the weight matrix.
The weight matrix $W$ is estimated using a least square optimization as in \acs{eq}\,\eqref{eq:lslle}.
%Hence, this problem which has to be solved at this stage is to estimate the weight matrix $W$. This problem can be tackled using a least square optimization scheme by optimizing the following objective function:
\begin{eqnarray}
	\hat{W} & = & \argmin_{W} \sum_i | \mathbf{x}_i - \sum_j W(i,j)\mathbf{x}_j |^{2} \ , \label{eq:lslle} \\
	&& \text{subject to } \sum_j W(i,j) = 1 \ , \nonumber
\end{eqnarray}

Then, the essence of \ac{lle} is to project the data into a low-dimensional space, while retaining the data spatial organization.
Therefore, the projection into the low-dimensional space is tackled as an optimization problem as:

\begin{equation}
	\hat{\mathbf{y}} = \argmin_{\mathbf{y}} \sum_i | \mathbf{y}_i - \sum_j W(i,j)\mathbf{y}_j |^{2} \ .
	\label{eq:lowprojlle}
\end{equation}

This optimization is solved as an eigenvectors-eigenvalues problem by finding the $k^{\text{th}}$ eigenvectors corresponding to the $k^{\text{th}}$ smallest eigenvalues of the sparse matrix $(I-W)^{\text{T}}(I-W)$.

\citeauthor{Tiwari2008} used a modified version of the \ac{lle} algorithm in which they applied \ac{lle} in a bagging approach with multiple neighbourhood sizes~\cite{Tiwari2008}.
The different embeddings obtained are then fused using the \ac{ml} estimation.

Another way of reducing the complexity of high-dimensional feature space is to use the family of so-called dictionary-based methods.
\Ac{scf} representation has become very popular in other computer vision application and has been used by \citeauthor{lehaire2014computer} in~\cite{lehaire2014computer}.
The main goal of sparse modeling is to efficiently represent the images as a linear combination of a few typical patterns, called atoms, selected from a dictionary.
Sparse coding consists of three main steps: sparse approximation, dictionary learning, and low-level features projection~\cite{rubinstein2008efficient}.

\emph{Sparse approximation -} Given a dictionary $\mathbf{D} \in \mathbb{R}^{n \times K}$ composed of $K$ atoms and an original signal $\mathbf{y} \in \mathbb{R}^{n}$ --- i.e., one feature vector ---, the sparse approximation corresponds to find the sparest vector $\mathbf{x} \in \mathbb{R}^{K}$ such that:

\begin{equation}
  \argmin_{\mathbf{x}}\|\mathbf{y - Dx} \|_{2} \qquad  \text{s.t.} \  \|\mathbf{x}\|_{0} \leq \lambda \, \label{eq:sprapp} \ ,
\end{equation}
\noindent where $\lambda$ is a specified sparsity level.

Solving the above optimization problem is an NP-hard problem~\cite{elad2010sparse}.
However, approximate solutions are obtained using greedy algorithms such as \ac{mp}~\cite{mallat1993matching} or \ac{omp}~\cite{pati1993orthogonal,davis1997adaptive}.

\emph{Dictionary learning -} As stated previously, the sparse approximation is computed given a specific dictionary $\mathbf{D}$, which involves a learning stage from a set of training data.
This dictionary is learned using $K$-\acs*{svd} which is a generalized version of $K$-means clustering and uses \ac{svd}. 
The dictionary is built, in an iterative manner by solving the optimization problem of \acs{eq}\,\eqref{eq:dct}, by alternatively computing the sparse approximation of $\mathbf{X}$ and the dictionary $\mathbf{D}$.
\begin{equation}
  \argmin_{\mathbf{D,X}} \|\mathbf{Y} - \mathbf{D}\mathbf{X}\|_{2} \qquad  \text{s.t.} \  \|\mathbf{x}_{i}\|_{1} \leq \lambda \,\label{eq:dct} \ ,
\end{equation}
\noindent where $\mathbf{Y}$ is a training set of low-level descriptors, $\mathbf{X}$ is the associated sparse coded matrix --- i.e., set of high-level descriptors --- with a sparsity level $\lambda$, and $\mathbf{D}$ is the dictionary with $K$ atoms.
Given $\mathbf{D}$, $\mathbf{X}$ is computed using the batch-\ac{omp} algorithm, while given $\mathbf{X}$, $\mathbf{D}$ is sequentially updated, one atom at a time using \ac{svd}. 

\emph{Low-level features projection -} Once the dictionary is learned, each set of low-level features $\mathbf{F}_{I}$ previously extracted is encoded using the dictionary $\mathbf{D}$, solving the optimization problem presented in \acs{eq}\,\eqref{eq:sprapp} such that $\mathbf{F}_{I} \simeq \mathbf{DX}_{I}$.

The \ac{bow} approach offers an alternative method~\cite{Sivic2003} for feature extraction.
\Ac{bow} was used by \citeauthor{rampun2016computerb} in~\cite{rampun2015classifying,rampun2016computerb}.
This model represents the features by creating a codebook or visual dictionary, from the set of low-level features.
The set of low-level features are clustered using \textit{k}-means to create the dictionary with \textit{k} clusters known as visual words.
Once the codebook is created from the training set, the low-level descriptors are replaced by their closest word within the codebook.
The final descriptor is a histogram of size \textit{k} which represents the codebook occurrences for a given mapping.

% THIS IS FROM MY THESIS, YOU CAN USE THIS, If you like

%% \acf{scf}, or sparse signal representation, has become very popular over the past few decades and has led to state-of-the-art results in various applications such as face recognition~\cite{wright2009robust}, image denoising, image inpainting~\cite{elad2006image}, and image classification~\cite{sidibe2015discrimination}. 
%% The main goal of sparse modeling is to efficiently represent the samples/images as a linear combination of a few typical patterns, called atoms, selected from the dictionary.
%% Sparse coding consists of three main steps: (i) dictionary learning, (ii) low-level feature projection, and (iii) feature pooling~\cite{rubinstein2008efficient}. 
%% Considering our dictionary $D \in R^{n \times K}$ with $K$ atoms, where each column of $D$ represents one atom, the sparse coding problem of a signal $y \in R^{n}$ is defined as finding the sparsest vector $x$ so that $y \approx Dx$. 
%% This is an optimization problem that can be formulated as:
%% \begin{equation}
%% \min_{x} \|y-Dx\|_{2} \qquad \text{s.t.}\ \|x\|_{0} \leq \lambda \,,
%% \end{equation}  
%% \noindent where $\lambda$ is the sparsity level and $l^{0}$-norm accounts for the minimum number of non-zero elements in the sparse vector $x$. 
%% This optimization problem is NP hard~\cite{Elad2010}, subsequently approximation solutions are proposed either by using greedy algorithms such as \ac{mp}~\cite{mallat1993matching} and \ac{omp}~\cite{davis1994adaptive}, or by replacing the $l^{0}$-norm with $l^{1}$-norm such as in the \ac{bp} algorithm~\cite{chen1998atomic}.
%% %\noindent where $\lambda$ is the sparsity level and $l^{0}$-norm accounts for the minimum number of non-zero elements in sparse vector $x$. 
%% %This optimization problem is NP hard~\cite{Elad2010}.
%% %Subsequently approximation solutions are proposed either by using greedy algorithms such as \ac{mp}~\cite{mallat1993matching} and \ac{omp}~\cite{davis1994adaptive} or by replacing the $l^{0}$-norm with $l^{1}$-norm such as \ac{bp}~\cite{chen1998atomic}.
%% The dictionary is learned using K-SVD, a generalized version of \textit{k}-means clustering that uses \ac{svd}~\cite{aharon2006img}. 
%% The dictionary is built such that:
%% \begin{equation}
%%   \min_{Dx} \|y - Dx\|_{2} \qquad  \text{s.t.} \ \forall i \ \|x_{i}\|_{1} \leq \lambda \,,
%% \end{equation}

%% \noindent where $y$ is a low-level descriptor, $x$ is the sparse coded descriptor (i.e., high-level descriptor) with a sparsity level $\lambda$, and $D$ is the dictionary with $K$ atoms.
%% The K-SVD algorithm solves the optimization problem iteratively by alternating between $x$ and $D$. 
%% With $D$, the sparse code matrix $x$ is computed by any of the pursuit algorithms, and with $x$, $D$ is updated one atom at a time using \ac{svd}. 

%% Once the dictionary is learned, each $y_{i} \in R^{n}$ signal can be projected using $D$ to form a set of sparse codes $x_{i} \in R^{K}$. 
%% In the case of image samples, the sparse representation can be generated for patches in the image.
%% In this case, the final mapping is based on a combination of sparse codes, for instance by taking the maximum code from all the patches: 
%% \begin{equation}
%% f_{i} = \max_{j}(\vert X_{l}(i,j)\vert) \qquad  \forall  i = 1, 2, .., K \,,
%% \end{equation} 
%% \noindent where $X_{l} \in R^{K \times P}$ is the sparse code matrix~\cite{sidibe2015discrimination}. 
%% \end{description}

\subsubsection{Summary}

The feature selection and extraction used in \ac{cad} systems are summarized in \acs{tab}~\ref{tab:featext}.

\begin{table}
  \caption{Overview of the feature selection and extraction methods used in \acs*{cad} systems.}
  \scriptsize
  \centering
  \begin{tabular}{l r}
    \toprule
    \textbf{Dimension reduction methods} & \textbf{References} \\
    \midrule
    \textbf{Feature selection:} & \\ \\ [-1.5ex]
    \quad Statistical test & \cite{Niaf2011,Niaf2012,Vos2012} \\
    \quad \ac{mi}-based methods & \cite{Niaf2011,Niaf2012,Vos2008,lehaire2014computer,khalvati2015automated,chung2015prostate} \\
    \quad Correlation-based methods & \cite{rampun2016computer,rampun2015computer} \\ \\ [-1.5ex]
    \textbf{Feature extraction:} & \\ \\ [-1.5ex]
    \quad Linear mapping & \\
    \quad \quad \acs*{pca} & \cite{Tiwari2008,Tiwari2009} \\
    \quad Non-linear mapping & \\
    \quad \quad Laplacian eigenmaps & \cite{Tiwari2007,Tiwari2009a,Tiwari2009,Tiwari2010,Viswanath2008,Viswanath2011} \\
    \quad \quad \acs*{lle} and \acs*{lle}-based & \cite{Tiwari2008,Tiwari2009,Viswanath2008a,Viswanath2008} \\
    \quad Dictionary-based learning & \\
    \quad \quad Sparse coding & \cite{lehaire2014computer} \\
    \quad \quad \acs*{bow} & \cite{rampun2016computerb,rampun2015classifying} \\
    \bottomrule
  \end{tabular}
\label{tab:featext}
\end{table} 

\subsection{\acs*{cadx}: Classification} \label{subsec:chp3:img-clas:CADX-clas}

%\subsubsection{Classifier} \label{subsubsec:chp3:img-clas:CADX-clas:clas}

Once the feature vector has been extracted and eventually the complexity reduced, it is possible to make a decision and classify this feature vector to belong to \ac{cap} or healthy tissue.
A full review of classification methods used in pattern recognition is available in~\cite{Bishop2006}.

\paragraph{Rule-based method}
\citeauthor{Lv2009} make use of a decision stump classifier to distinguish \ac{cap} and healthy classes~\cite{Lv2009}. 
\citeauthor{Puech2009} detect \ac{cap} by implementing a given set of rules and scores based on a medical support approach~\cite{Puech2009}.
During the testing, the feature vector goes through these different rules, and a final score is computed resulting to a final decision.

\paragraph{Clustering methods}
\acf{knn} is one of the simplest supervised machine learning classification methods.
In this method, a new unlabelled vector is assigned to the most represented class from its $k$ nearest-neighbours in the feature space.
The parameter $k$ is usually an odd number in order to avoid any tie case.
\ac{knn} has been one of the methods used in~\cite{Niaf2011,Niaf2012,rampun2016computerb} mainly to make a comparison with different machine learning techniques.
\citeauthor{Litjens2012} used this method to roughly detect potential \ac{cap} voxels before performing a region-based classification~\cite{Litjens2012}.

The $k$-means algorithm is an unsupervised clustering method in which the data is partitioned into $k$ clusters in an iterative manner.
First, $k$ random centroids are defined in the feature space and each data point is assigned to the nearest centroid.
Then, the centroid position for each cluster is updated by computing the mean of all the samples belonging to this particular cluster.
Both assignment and updating are repeated until the centroids are stable.
The number of clusters $k$ is usually defined as the number of classes.
This algorithm can also be used for ``on-line'' learning.
In case that new data has to be incorporated, the initial centroid positions correspond to the results of a previous $k$-means training and is followed by the assignment-updating stage previously explained.
\citeauthor{Tiwari2009} used $k$-means in an iterative procedure~\cite{Tiwari2007,Tiwari2009}.
Three clusters were defined corresponding to \ac{cap}, healthy, and non-prostate.
$k$-means is repeatedly applied and at each iteration, the voxels corresponding to the largest cluster are excluded under the assumption that it is assigned to ``non-prostate'' cluster.
The algorithm stopped when the number of voxels in all remaining clusters were smaller than a given threshold.
\citeauthor{Tiwari2008} and \citeauthor{Viswanath2008a} used $k$-means in a repetitive manner in order to be less sensitive to the centroids initialization~\cite{Viswanath2008,Viswanath2008a,Tiwari2008}.
Thus, $k$ clusters are generated $T$ times and the final assignment is performed by majority voting using a co-association matrix as proposed in~\cite{Fred2005}.

\paragraph{Linear model classifiers}
\Acf{lda} is used as a classification method in which the optimal linear separation between 2 classes is found by maximizing the inter-class variance and minimizing the intra-class variance~\cite{Friedman1989}.
The linear discriminant function is defined as:
\begin{equation}
	\delta_{k}(\mathbf{x}_i) = \mathbf{x}_i^{\text{T}} \Sigma^{-1} \mu_k - \frac{1}{2} \mu_{k}^{\text{T}} \Sigma^{-1} \mu_k + \log (\pi_k) \ ,
	\label{eq:ldafun}
\end{equation}

\noindent where $\mathbf{x}_i = \{x_1, x_2, \dots , x_n\}$ is an unlabelled feature vector of $n$ features, $\Sigma$ is the covariance matrix of the training data, $\mu_k$ is the mean vector of the class $k$, and $\pi_k$ is the prior probability of class $k$.
To perform the classification, a sample $\mathbf{x}_i$ is assigned to the class which maximizes the discriminant function as in \acs{eq}\,\eqref{eq:ldaclass}.
\begin{equation}
	C(\mathbf{x}_i) = \argmax_k \delta_k(\mathbf{x}_i) \ .
	\label{eq:ldaclass}
\end{equation}
\Ac{lda} has been used in~\cite{Antic2013,Chan2003,Niaf2011,Niaf2012,Vos2012}.

%covariance matrix $\Sigma_k$ specific at each class is computed.
%used \ac{lda} to classify their feature vectors defining two classes \ac{cap} \textit{versus} healthy

Logistic regression is also used to perform binary classification and provides the probability of an observation to belong to a class.
The posterior probability of one of the classes, $c_1$ is written as:
\begin{equation}
	p(c_1|\mathbf{x}_i) = \frac{1}{1+\exp(-\mathbf{w}^{\text{T}}\mathbf{x}_i)} \ ,
	\label{eq:postprlr}
\end{equation}

\noindent with $p(c_2|\mathbf{x}_i) = 1 - p(c_1|\mathbf{x}_i)$ and where $\mathbf{w}$ is the vector of the regression parameters allowing to obtain a linear combination of the input feature vector $\mathbf{x}_i$.
Thus, an unlabelled observation $\mathbf{x}_i$ is assigned to the class which maximizes the posterior probability as shown in \acs{eq}\,\eqref{eq:posprobreg}.

\begin{equation}
	C(\mathbf{x}_i) = \argmax_k p(C=k|\mathbf{x}_i) \ .
	\label{eq:posprobreg}
\end{equation}

From \acs{eq}\,\eqref{eq:postprlr}, one can see that the key to classification using logistic regression model is to infer the set of parameters $\mathbf{w}$ through a learning stage using a training set.
This vector of parameters $\mathbf{w}$ is inferred by estimating the maximum likelihood.
This step is performed through an optimization scheme, using a quasi-Newton method~\cite{Byrd1995}, which seeks in an iterative manner for the local minimum in the derivative of \acs{eq}\,\eqref{eq:postprlr}.
This method has been used to create a linear probabilistic model in~\cite{Kelm2007,Puech2009,lehaire2014computer,rampun2015computer}.

\paragraph{Non-linear model classifier}
\citeauthor{Viswanath2012} used \acf{qda} instead of \ac{lda}~\cite{Viswanath2012}.
Unlike in \ac{lda} in which one assumes that the class covariance matrix $\Sigma$ is identical for all classes, a covariance matrix $\Sigma_k$ specific to each class is computed.
Thus, \acs{eq}\,\eqref{eq:ldafun} becomes:
\begin{equation}
	\delta_{k}(\mathbf{x}_i) = \mathbf{x}_i^{\text{T}} \Sigma_{k}^{-1} \mu_k - \frac{1}{2} \mu_{k}^{\text{T}} \Sigma_{k}^{-1} \mu_k + \log (\pi_k) \ ,
	\label{eq:qdafun}
\end{equation}

\noindent where $\mathbf{x}_i$ has additional terms corresponding to the pairwise products of individual features such as $\{x_1, x_2, \dots , x_n, x_1^2, x_1x_2, \dots x_n^2\}$.
The classification scheme in the case of the \ac{qda} is identical to \acs{eq}\,\eqref{eq:ldaclass}.

\paragraph{Probabilistic classifiers}
The most commonly used classifier is the naive Bayes classifier which is a probabilistic classifier assuming independence between each feature dimension~\cite{Rish2001}.
This classifier is based on Bayes' theorem:

\begin{equation}
	p(C=k|\mathbf{x}) = \frac{p(C)p(\mathbf{x}|C)}{p(\mathbf{x})} \ ,
	\label{eq:bayth}
\end{equation}

\noindent where $p(C=k|\mathbf{x})$ is the posterior probability, $p(C)$ is the prior probability, $p(\mathbf{x}|C)$ is the likelihood, and $p(\mathbf{x})$ is the evidence. 
However, the evidence term is usually discarded since it is not class dependent and plays the role of a normalization term.
Hence, in a classification scheme, an unlabelled observation is classified to the class which maximizes the posterior probability as:

\begin{eqnarray}
	C(\mathbf{x}_i) & = & \argmax_k p(C=k|\mathbf{x}_i) \ , \label{eq:maxbay} \\
	p(C=k|\mathbf{x}_i) & = & p(C=k) \prod_{j=1}^{n} p(x_{ij},|C=k) \ , \label{eq:postbay}
\end{eqnarray}

\noindent where $d$ is the number of dimensions of the feature vector $\mathbf{x}_i = \{x_{i1},\cdots,x_{id}\}$.
Usually, a model includes both the prior and likelihood probabilities and it is common to use an equal prior probability for each class or eventually a value based on the relative frequency derived from the training set.
Regarding the likelihood probability, it is common to choose a Gaussian distribution to characterize each class.
Thus, each class is characterized by two parameters: (i) the mean and (ii) the standard deviation.
These parameters are inferred from the training set by using the \ac{mle} approach.
The naive Bayes classifier has been used in~\cite{Giannini2013,Mazzetti2011,Niaf2011,Niaf2012,Niaf2012,cameron2014multiparametric,cameron2016maps,rampun2015classifying,rampun2016computerb,rampun2015computer,rampun2016computer}.

\paragraph{Ensemble learning classifiers}
\Ac{adb} is an adaptive method based on an ensemble learning method and initially proposed in~\cite{Freund1997}. 
\Ac{adb} linearly combines several weak learners resulting into a final strong classifier.
A weak learner is defined as a classification method performing slightly better than a random classifier.
Popular choices regarding the weak learner classifiers are: decision stump or decision tree learners such as \ac{id3}~\cite{Quinlan1986}, C4.5~\cite{Quinlan1993}, and \ac{cart}~\cite{Breiman1984}.

\Ac{adb} is considered as an adaptive method in the way that the weak learners are selected.
The selection is performed in an iterative manner.
At each iteration $t$, the weak learner selected $h_t$ corresponds to the one minimizing the classification error on a distribution of weights $D_t$, that is associated with the training samples.
Each weak learner is assigned a weight $\alpha_t$ as:

\begin{equation}
	\alpha_t = \frac{1}{2} \ln \frac{1 - \epsilon_t}{\epsilon_t} \ ,
	\label{eq:wclssada}
\end{equation}

\noindent where $\epsilon_t$ corresponds to the classification error rate of the weak learner on the distribution of weight $D_t$.

Before performing a new iteration, the distribution of weights $D_t$ is updated such that the weights associated with the misclassified samples by $h_t$ increase and the weights of well classified samples decrease as shown in \acs{eq}\,\eqref{eq:rewada}.

\begin{equation}
	D_{t+1}(i) = \frac{ D_t(i) \exp \left( -\alpha_t y_i h_{t}(\mathbf{x}_{i} ) \right) }{ Z_t  } \ ,
	\label{eq:rewada} 
\end{equation}

\noindent where $\mathbf{x}_i$ is the $i^{\text{th}}$ sample corresponding to class $y_i$ and $Z_t$ is a normalization factor forcing $D_{t+1}$ to be a probability distribution. 
This procedure allows to select a weak learner at the next iteration $t+1$ which will classify in priority the previous misclassified samples. 
Thus, after $T$ iterations, the final strong classifier corresponds to the linear combination of the weak learners selected and the classification is performed such that:

\begin{equation}
	C(\mathbf{x}_i) = \sign \left( \sum_{t=1}^{T} \alpha_t h_t(\mathbf{x}_i) \right) \ .
	\label{eq:strclaada}
\end{equation}

\citeauthor{Lopes2011} make use of the \ac{adb} classifier to perform their classification~\cite{Lopes2011} while \citeauthor{Litjens2014} used the GentleBoost variant~\cite{Friedman1998} which provides a modification of the function affecting the weight at each weak classifier~\cite{Litjens2014}.

\Ac{rf} is a classification method which is based on creating an ensemble of decision trees and was introduced in~\cite{Breiman2001}.
In the learning stage, multiple decision tree learners~\cite{Breiman1984} are trained.
However, each decision tree is trained using a different dataset.
Each of these datasets corresponds to a bootstrap sample generated by randomly choosing $n$ samples with replacement from the initially $N$ samples available~\cite{Efron1979}.
Then, randomization is also part of the decision tree growth.
At each node of the decision tree, from the bootstrap sample of $D$ dimensions, a number of $d \ll D$ dimensions will be randomly selected.
Finally, the $d^{\text{th}}$ dimension in which the classification error is minimum is used.
This best ``split'' classifier is often evaluated using \ac{mi} or Gini index.
Finally, each tree is grown as much as possible without using any pruning procedure.
In the prediction stage, the unlabelled sample is introduced in each tree and each of them assign a class.
Finally, it is common to use a majority voting approach to choose the final class label.
The \ac{rf} classifier has been used in~\cite{Kelm2007,Litjens2014,Tiwari2012,Tiwari2013,Viswanath2009,trigui2017automatic,trigui2016classification,samarasinghe2016semi,rampun2015classifying,rampun2016computerb,rampun2015computer,rampun2016computer}.

\begin{figure}
\centering
	\begin{tikzpicture}
    [level distance=1.75cm,sibling distance=1.5cm,scale=.95,every node/.style={scale=0.95}, 
   edge from parent path={(\tikzparentnode) -- (\tikzchildnode)}]
	\Tree [.\node (foo) {\includegraphics[width=2cm]{3_review/figures/classification/pbt-simulation/pbt_tree_1.eps}}; 
    \edge node[auto=right] {};
    [.\node{\includegraphics[width=2cm]{3_review/figures/classification/pbt-simulation/pbt_tree_2_1.eps}};
      \edge node[auto=right] {};  
      [.\node{\includegraphics[width=2cm]{3_review/figures/classification/pbt-simulation/pbt_tree_3_2.eps}}; ]
      \edge node[auto=left] {};  
      [.\node{\includegraphics[width=2cm]{3_review/figures/classification/pbt-simulation/pbt_tree_3_1.eps}}; ]
    ]
    \edge node[auto=left] {};
    [.\node{\includegraphics[width=2cm]{3_review/figures/classification/pbt-simulation/pbt_tree_2_2.eps}};
    ]
    ];
	\end{tikzpicture}
\caption[Representation of the probabilistic boosting-tree.]{Representation of the capabilities of the probabilistic boosting-tree algorithm to split at each node of the tree the positive and negative samples.}
\label{fig:pbtsim}
\end{figure}

Probabilistic boosting-tree is another ensemble learning classifier which shares principles with \ac{adb} but using them inside a decision tree~\cite{Tu2005}. 
In the training stage, the probabilistic boosting-tree method grows a decision tree and at each node, a strong classifier is learned in an almost comparable scheme to \ac{adb}.
Once the strong learner is trained, the training set is split into two subsets which are used to train the next strong classifiers in the next descending nodes.
Thus, three cases are conceivable to decide in which branch to propagate each sample training $\mathbf{x}_i$:

\begin{itemize}
	\item if $q(+1, \mathbf{x}_i) - \frac{1}{2} > \epsilon$ then $\mathbf{x}_i$ is propagated to the right branch set and a weight $w_i=1$ is assigned. 
	\item if $q(-1, \mathbf{x}_i) - \frac{1}{2} > \epsilon$ then $\mathbf{x}_i$ is propagated to the left branch set and a weight $w_i=1$ is assigned.
	\item else $\mathbf{x}_i$ will be propagated in both branches with $w_i=q(+1, \mathbf{x}_i)$ in the right branch and $w_i=q(-1, \mathbf{x}_i)$ in the left branch.
\end{itemize}

\noindent with $\mathbf{w} = w_i, i=\{1,\cdots,N\}$ corresponding to distribution of weights, $N$ the number of samples as in \ac{adb} and $q(\cdot)$ is defined as:

\begin{eqnarray}
	q(+1, \mathbf{x}_i) & = & \frac{\exp(2H(\mathbf{x}_i))}{1+\exp(2H(\mathbf{x}_i))} \ , \label{eq:regada1} \\
	q(-1, \mathbf{x}_i) & = & \frac{\exp(-2H(\mathbf{x}_i))}{1+\exp(-2H(\mathbf{x}_i))} \ . \label{eq:regada2}
\end{eqnarray}

Employing such a scheme tends to divide the data in such a way that positive and negative samples are naturally split as shown in \acs{eq}\,\ref{fig:pbtsim}.
In the classification stage, the unlabelled sample $\mathbf{x}$ is propagated through the tree, where at each node, it is classified by each strong classifier previously learned and where an estimation of the posterior distribution is computed.
The posterior distribution corresponds to the sum of the posterior distribution at each node of the decision tree.
The probabilistic boosting-tree classifier has been used in~\cite{Tiwari2009a,Tiwari2012,Tiwari2010,Viswanath2011}.

\paragraph{Kernel method}
A Gaussian process for classification is a kernel method in which it is assumed that the data can be represented by a single sample from a multivariate Gaussian distribution~\cite{Rasmussen2005}.
In the case of linear logistic regression for classification, the posterior probability is expressed as:
\begin{eqnarray}
	p(y_i|\mathbf{x}_i,\mathbf{w}) & = & \sigma(y_i f(\mathbf{x}_i)) \ , \label{eq:gp1} \\
	f(\mathbf{x}_i) & = & \mathbf{x}_i^{\text{T}} \mathbf{w} \ , \nonumber
\end{eqnarray}

\noindent where $\sigma(\cdot)$ is the logistic function and $\mathbf{w}$ are the parameters vector of the model.
Thus, the classification using Gaussian processes is based on assigning a Gaussian process prior over the function $f(\mathbf{x})$ which is characterized by a mean function $\bar{f}$ and covariance function $K$.
Therefore, in the training stage, the best mean and covariance functions have to be inferred in regard to our training data using a Newton optimization and a Laplacian approximation.
The prediction stage is performed in two stages.
First, for a new observation $\mathbf{x}_*$, the corresponding probability $p(f(\mathbf{x}_*)|f(\mathbf{x}))$ is computed such that:
\begin{eqnarray}
	p(f(\mathbf{x}_*)|f(\mathbf{x})) & = & \mathcal{N}( K_*K^{-1}\bar{f}, K_{**}-K_*(K')^{-1}K_*^{\text{T}} ) \ , \nonumber \\
	K' & = & K + W^{-1} \ , \label{eq:gp2} \\
	W & = & \nabla \nabla \log p(\mathbf{y}|f(\mathbf{x})) \ , \nonumber
\end{eqnarray}

\noindent where $K_{**}$ is the covariance function $k(\mathbf{x}_*, \mathbf{x}_*)$ the testing sample $\mathbf{x}_*$, $K_{*}$ is the covariance function $k(\mathbf{x}, \mathbf{x}_*)$ of training-testing samples $\mathbf{x}$ and $\mathbf{x}_*$.
Then, the function $f(\mathbf{x}_*)$ is squashed using the sigmoid function and the probability of the class membership is defined such that:

\begin{equation}
	C(\mathbf{x}_*) = \sigma\left( \frac{\bar{f}(\mathbf{x_*})}{\sqrt{1+var(f(\mathbf{x}_*))}} \right) \ .
	\label{eq:gp3}
\end{equation}

Only \citeauthor{Kelm2007} used Gaussian process for classification of \ac{mrsi} data~\cite{Kelm2007}.

\paragraph{Sparse kernel methods}
In a classification scheme using Gaussian processes, when a prediction is performed, the whole training data are used to assign a label to the new observations.
That is why this method is also called kernel method.
Sparse kernel category is composed of methods which rely only on a few labelled observations of the training set to assign the label of new observations~\cite{Bishop2006}.

\Acf{svm} is a sparse kernel method aiming at finding the best linear hyper-plane --- non-linear separation is discussed further --- which separates 2 classes such that the margin between the two classes is maximized~\cite{Vapnik1963}.
The margin is in fact the region defined by 2 hyper-planes splitting the 2 classes, such that there is no points lying in between.
The distance between these 2 hyper-planes is equal to $\frac{2}{\|\mathbf{w}\|}$ where $\mathbf{w}$ is the normal vector of the hyper-plane splitting the classes.
Thus, maximizing the margin is equivalent to minimizing the norm $\|\mathbf{w}\|$.
Hence, this problem is solved by an optimization approach and formalized as:

\begin{equation}
\begin{aligned}
& \argmin_{\mathbf{w}}
& & \frac{1}{2} \| \mathbf{w}^2\| \ , \\
& \text{subject to}
& & y_i(\mathbf{w}.\mathbf{x}_i - b) \geq 1, \; i = \{ 1, \ldots, N \} \ ,
\end{aligned}
\label{eq:svm1}
\end{equation}

\noindent where $\mathbf{x}_i$ is a training sample with is corresponding class label $y_i$.
From \acs{eq}\,\eqref{eq:svm1}, it is important to notice that only few points from the set of $N$ points are selected which later define the hyper-plane.
This constraint is imposed in the optimization problem using Lagrange multipliers $\boldsymbol{\alpha}$.
All points which are not lying on the margin are assigned a corresponding $\alpha_i = 0$, which is formalized as \acs{eq}\,\eqref{eq:svm2}.

\begin{equation}
	\arg\min_{\mathbf{w},b } \max_{\boldsymbol{\alpha}\geq 0 } \left\{ \frac{1}{2}\|\mathbf{w}\|^2 - \sum_{i=1}^{n}{\alpha_i[y_i(\mathbf{w}\cdot \mathbf{x_i} - b)-1]} \right\} \ .
	\label{eq:svm2}
\end{equation}

The different parameters are inferred using quadratic programming.
This version of \ac{svm} is known as hard-margin since no points can lie in the margin area.
However, it is highly probable not to find any hyper-plane splitting the classes such as specified previously.
Thus, a soft-margin optimization approach has been proposed~\cite{Cortes1995}, where points have the possibility to lie on the margin but at the cost of a penalty $\xi_i$ which is minimized in the optimization process such that:

\begin{equation}
\small
\arg\min_{\mathbf{w},\mathbf{\xi}, b } \max_{\boldsymbol{\alpha},\boldsymbol{\beta} } \left\{ \frac{1}{2}\|\mathbf{w}\|^2+C \sum_{i=1}^n \xi_i - \sum_{i=1}^{n}{\alpha_i[y_i(\mathbf{w}\cdot \mathbf{x_i} - b) -1 + \xi_i]} - \sum_{i=1}^{n} \beta_i \xi_i \right\} \ .
\end{equation}

The decision to assign the label to a new observation $\mathbf{x}_i$ is taken such that:

\begin{equation}
	C(\mathbf{x}_i) = \sign \left( \sum_{n=1}^{N} \alpha_n (\mathbf{x}_n . \mathbf{x}_i) + b_0 \right) \ ,
	\label{eq:svmdec} 
\end{equation}

\noindent where $\mathbf{x}_n|n=\{1,\cdots,S\}$, $S$ being the support vectors.

\ac{svm} can also be used as a non-linear classifier by performing a Kernel trick~\cite{Boser1992}.
The original data $\mathbf{x}$ is projected to a high-dimensional space in which it is assumed that a linear hyper-plane splits the 2 classes.
Different kernels are popular such as the \ac{rbf} kernel, polynomial kernels, or sigmoid kernels.
In \ac{cad} for \ac{cap} systems, \ac{svm} is the most popular classification method and has been used in a multitude of research works~\cite{Artan2009,Artan2010,Chan2003,Litjens2011,Litjens2012,Liu2013,Lopes2011,Niaf2011,Niaf2012,Ozer2009,Ozer2010,Parfait2012,Peng2013,Sung2011,Tiwari2012,Vos2008,Vos2008a,Vos2010,Vos2012,giannini2015fully,trigui2017automatic,lehaire2014computer,khalvati2015automated,chung2015prostate}.

\Acf{rvm} is a sparse version of Gaussian process previously presented, proposed in~\cite{Tipping2001}.
\ac{rvm} is identical to a Gaussian process with the following covariance function~\cite{Quinonero-Candela2002}:

\begin{equation}
	K_{RVM}(\mathbf{x}_p,\mathbf{x}_q) = \sum_{j=1}^{M} \frac{1}{\alpha_j} \Phi_j ( \mathbf{x}_p ) \Phi_j ( \mathbf{x}_q ) \ ,
 	\label{eq:rvm}
\end{equation}

\noindent where $\phi(\cdot)$ is a Gaussian basis function, $\mathbf{x}_i|i=\{1,\cdots,N\}$ are the $N$ training points, and $\boldsymbol{\alpha}$ are the weights vector.
As mentioned in~\cite{Quinonero-Candela2002}, the sparsity regarding the relevance vector arises for some $j$, the weight $\alpha_j^{-1} = 0$.
The set of weights $\boldsymbol{\alpha}$ is inferred using the expectation maximization algorithm.
\citeauthor{Ozer2010} used of \ac{rvm} and make a comparison with \ac{svm} for the task of \ac{cap} detection~\cite{Ozer2009,Ozer2010}.

\paragraph{Neural network} 
Multilayer perceptron is a feed-forward neural network considered as the most successful model of this kind in pattern recognition~\cite{Bishop2006}.
The most well known model used is based on 2 layers where a prediction of an observation is computed as:

\begin{equation}
	C(\mathbf{x}_n,w_{ij}^{(1)},w_{kj}^{(2)}) = \sigma \left[ \sum_{j=0}^{M} w_{kj}^{(2)} \  h \left( \sum_{i=0}^{D} w_{ij}^{(1)} x_{in} \right) \right] \ ,
	\label{eq:annmlp}
\end{equation}

\noindent where $h(\cdot)$ and $\sigma(\cdot)$ are 2 activation functions usually non-linear, $w_{ij}^{(1)}$ and $ w_{kj}^{(2)}$ are the weights associated with the linear combination with the input feature $\mathbf{x}_n$ and the hidden unit.

\input{3_review/fig-NN-1.tex}

A graphical representation of this network is presented in \acs{eq}\,\ref{fig:mlp}.
Relating \acs{fig}\,\ref{fig:mlp} with \acs{eq}\,\eqref{eq:annmlp}, it can be noted that this network is composed of some successive non-linear mapping of the input data.
First, a linear combination of the input vector $\mathbf{x}_n$ is mapped into some hidden units through a set of weights $w_{ij}^{(1)}$.
This combination becomes non-linear by the use of the activation function $h(\cdot)$ which is usually chosen to be a sigmoid function.
Then, the output of the networks consists of a linear combination of the hidden units and the set of weights $w_{kj}^{(2)}$.
This combination is also mapped non-linearly using an activation function $\sigma(\cdot)$ which is usually a logistic function.
Thus, the training of such a network resides in finding the best weights $w_{ij}^{(1)}$ and $ w_{kj}^{(2)}$ which model the best the training data.
The error of this model is computed as:

\begin{equation}
	E(w_{ij}^{(1)},w_{kj}^{(2)}) = \frac{1}{2} \sum_{n=1}^{N} \left( C(\mathbf{x}_n,w_{ij}^{(1)},w_{kj}^{(2)}) - y(\mathbf{x}_n) \right) ^{2} \ ,
	\label{eq:mlpcost}
\end{equation}

\noindent where $\mathbf{x}_n|n=\{1,\cdots,N\}$ are the $N$ training vectors with their corresponding class label $y(\mathbf{x}_n)$.

Therefore, the best set of weights is inferred in an optimization framework where the error $E(\cdot)$ needs to be minimized.
This optimization is performed using a gradient descent method where the derivative of \acs{eq}\,\eqref{eq:mlpcost} is computed using the back-propagation algorithm proposed by~\cite{Rumelhart1988}.
This type of network has been used multiple times~\cite{Matulewicz2013,Parfait2012,trigui2017automatic,trigui2016classification,rampun2016computer}.

\input{3_review/fig-NN-2.tex}

Probabilistic neural networks are another type of feed-forward networks which is derived from the multilayer perceptron case and has been proposed by~\cite{Specht1988}.
This classifier is modelled by changing the activation function $h(\cdot)$ in \acs{eq}\,\eqref{eq:annmlp} to an exponential function such that:

\begin{equation}
	h(\mathbf{x}_n) = \exp \left( - \frac{ (\mathbf{w}_j - \mathbf{x})^{\text{T}}(\mathbf{w}_j - \mathbf{x}) }{2\sigma^2} \right) \ ,
	\label{eq:pnn1}
\end{equation}

\noindent where $\sigma$ is a free parameter set by the user.

The other difference of the probabilistic neural networks compared with the multilayer perceptron networks resides in the architecture as shown in \acs{fig}\,\ref{fig:pnn}.
This network is formed by 2 hidden layers.
The first hidden layer consists of the pattern layer, in which the mapping is done using \acs{eq}\,\eqref{eq:pnn1}.
This pattern layer is sub-divided into a number of groups corresponding to the number of classes.
The second hidden layer corresponds to the summation layer which simply sums the output of each sub-group of the pattern layer.
This method is used in~\cite{Ampeliotis2007,Ampeliotis2008,Viswanath2011}.

\paragraph{Graphical model classifiers}
\Ac{mrf} is used as a lesion segmentation method to detect \ac{cap}.
First, let define $s$ as a pixel which belongs to a certain class denoted by $\omega_s$.
The labelling process is defined as $\omega = \{\omega_s, s \in I\}$ where $I$ is the set of all the pixels inside the image.
The observations corresponding to \ac{si} in the image are noted $\mathcal{F} = \{ f_s | s \in I \}$.
Thus, the image process $\mathcal{F}$ represents the deviation from the labelling process $\omega$~\cite{Kato2001}.
Hence, lesion segmentation is equivalent to estimating the best $\hat{\omega}$ which maximizes the posterior probability $p(\omega|\mathcal{F})$.
Thus, using a Bayesian approach, this problem is formulated such that:

\begin{equation}
	p(\omega|\mathcal{F}) = \argmax_{\omega} \prod_{s \in I} p(f_s | \omega_s) p(\omega) \ .
	\label{eq:mrf1}
\end{equation}

It is generally assumed that $p(f_s | \omega_s)$ follows a Gaussian distribution and that the pixels classes $\lambda = \{1,2\}$ for a binary classification are characterized by their respective mean $\mu_{\lambda}$ and standard deviation $\sigma_{\lambda}$.
Then, $\omega$ is a Markov random field, thus:

\begin{equation}
	p(\omega) =  \frac{1}{Z} \exp\left( -U(\omega) \right)  \ ,
	\label{eq:mrf2}
\end{equation}

\noindent where $Z$ is a normalization factor to obtain a probability value, $U(\cdot)$ is the energy function.

Thus, the segmentation problem is solved as an optimization problem where the energy function $U(\cdot)$ has to be minimized.
There are different possibilities to define the energy function $U(\cdot)$.
However, it is common to define the energy function such that it combines two types of potential function: (i) a local term relative to the pixel itself and (ii) a smoothing prior which embeds neighbourhood information which penalizes the energy function affecting the region homogeneity.
This optimization of such a function can be performed using an algorithm such as iterated conditional modes~\cite{Kato2001}.
\citeauthor{Liu2009} and \citeauthor{Ozer2010} used \ac{mrf} as an unsupervised method to segment lesions in \ac{mpmri} images~\cite{Liu2009,Ozer2010}.
\citeauthor{Artan2010} and \citeauthor{chung2015prostate} used conditional random fields instead of \ac{mrf} for \ac{mri} segmentation~\cite{Artan2009,Artan2010,chung2015prostate}.
The difference between these 2 methods resides in the fact that conditional probabilities are defined such as:

\begin{equation}
	p(\omega|\mathcal{F}) =  \frac{1}{Z} \exp \left[ - \sum_{s \in I} V_{C1}(\omega_s|\mathcal{F}) - \sum_{\{s,r\} \in C } V_{C2} (\omega_s,\omega_r|\mathcal{F})  \right] \ .
\label{eq:crf}
\end{equation}

\noindent $V_{C1}(\cdot)$ is the state (or partition) feature function and $V_{C2}(\cdot)$ is the transition (or edge) feature function~\cite{Kato2012}.

\subsubsection{Summary}

Classification methods used to distinguish \ac{cap} from healthy tissue in in \ac{cad} systems are summarized in \acs{tab}~\ref{tab:class}.

\begin{table}
  \caption{Overview of the classifiers used in \acs*{cad} systems.}
  \scriptsize
  \begin{tabularx}{\textwidth}{l >{\raggedleft\arraybackslash}X@{}}
    \toprule
    \textbf{Classifier} & \textbf{References} \\
    \midrule
    \textbf{Rule-based method:} & \cite{Lv2009,Puech2009} \\ \\ [-1.5ex]
    \textbf{Clustering methods:} & \\
    \quad $k$-means clustering & \cite{Tiwari2007,Tiwari2008,Tiwari2009} \\
    \quad \acs{knn} & \cite{Litjens2012,Niaf2011,Niaf2012,rampun2016computerb} \\ \\ [-1.5ex]
    \textbf{Linear model classifiers:} & \\
    \quad \acs{lda} & \cite{Antic2013,Chan2003,Litjens2014,Niaf2011,Niaf2012,Vos2012} \\
    \quad Logistic regression & \cite{Kelm2007,Langer2009,lehaire2014computer,rampun2015computer} \\ \\ [-1.5ex]
    \textbf{Non-linear classifier:} & \\
    \quad \acs{qda} & \cite{Viswanath2012} \\ \\ [-1.5ex]
    \textbf{Probabilistic classifier:} & \\
    \quad Naive Bayes & \cite{Giannini2013,Mazzetti2011,Niaf2011,Niaf2012,cameron2014multiparametric,cameron2016maps,rampun2015classifying,rampun2016computerb,rampun2015computer,rampun2016computer} \\ \\ [-1.5ex]
    \textbf{Ensemble learning classifiers:} & \\
    \quad \acs*{adb} & \cite{Litjens2014,Lopes2011} \\
    \quad \acs*{rf} & \cite{Kelm2007,Litjens2014,Tiwari2012,Tiwari2013,Viswanath2009,trigui2017automatic,trigui2016classification,samarasinghe2016semi,rampun2015classifying,rampun2016computerb,rampun2015computer,rampun2016computer} \\
    \quad Probabilistic boosting tree & \cite{Tiwari2009,Tiwari2010,Tiwari2012} \\ \\ [-1.5ex]
    \textbf{Kernel method:} & \\
    \quad Gaussian processes & \cite{Kelm2007} \\ \\ [-1.5ex]
    \textbf{Sparse kernel methods:} & \\
    \quad \acs{svm} & \cite{Artan2009,Artan2010,Chan2003,Litjens2011,Litjens2012,Liu2013,Lopes2011,Niaf2011,Niaf2012,Ozer2009,Ozer2010,Parfait2012,Peng2013,Sung2011,Tiwari2012,Vos2008,Vos2008a,Vos2010,Vos2012,giannini2015fully,trigui2017automatic,lehaire2014computer,khalvati2015automated,chung2015prostate} \\
    \quad \acs{rvm} & \cite{Ozer2009,Ozer2010} \\ \\ [-1.5ex]
    \textbf{Neural network:} & \\ 
    \quad Multiple layer perceptron & \cite{Matulewicz2013,Parfait2012,trigui2017automatic,trigui2016classification,rampun2016computer} \\
    \quad Probabilistic neural network & \cite{Ampeliotis2007,Ampeliotis2008,Viswanath2011} \\ \\ [-1.5ex]
    \textbf{Graphical model classifiers:} & \\
    \quad Markov random field & \cite{Liu2009,Ozer2010} \\
    \quad Conditional random field & \cite{Artan2009,Artan2010,chung2015prostate} \\
    \bottomrule
  \end{tabularx}
\label{tab:class}
\end{table}

\subsection{Model validation} \label{subsec:chp3:img-clas:CADX-val}

\begin{table}
  \caption{Overview of the model validation techniques used in \acs*{cad} systems.}
  \centering
  \scriptsize
  % \renewcommand{\arraystretch}{1.5}
  \begin{tabularx}{\textwidth}{@{}l >{\raggedleft\arraybackslash}X@{}}
    \toprule
    \textbf{Model validation techniques} & \textbf{References} \\ \\ [-1.5ex]
    \midrule
    \quad \acs*{loo} & \cite{Ampeliotis2007,Ampeliotis2008,Antic2013,Artan2009,Artan2010,Chan2003,Giannini2013,Kelm2007,Litjens2012,Litjens2014,Mazzetti2011,Niaf2011,Niaf2012,Ozer2009,Ozer2010,Peng2013,Puech2009,Tiwari2013,Viswanath2011,Vos2008,Vos2008,Vos2010,cameron2016maps,cameron2014multiparametric,lehaire2014computer,khalvati2015automated,chung2015prostate} \\ \\ [-1.5ex]
    \quad \acs*{kcv} & \cite{Litjens2011,Parfait2012,Tiwari2009,Tiwari2009a,Tiwari2010,Tiwari2012,Viswanath2012,Viswanath2009,Vos2012,trigui2016classification,trigui2017automatic,rampun2015classifying,rampun2015computer,rampun2016computer,rampun2016computerb,rampun2016quantitative} \\ \\ [-1.5ex]
    \bottomrule
  \end{tabularx}
\label{tab:valmod}
\end{table}

In pattern recognition, the use of model validation techniques to assess the performance of a classifier plays an important role for reporting results.
Two techniques are broadly used in the development of \ac{cad} systems and are summarized in \acs{tab}~\ref{tab:valmod}.
The most popular technique used in \ac{cad} systems is the \ac{loo} technique.
From the whole data, one patient is kept for validation and the other cases are used for training.
This manipulation is repeated until each patient has been used for validation.
This technique is popular when working with a limited number of patients, allowing to train on representative number of cases even with a small dataset.
However, \ac{loo} cross-validation suffers from a large variance and is considered as an unreliable estimate~\cite{Efron1983}.

The other technique is the \ac{kcv} technique which is based on splitting the dataset into $k$ subsets where the samples are randomly selected.
Then, one fold is kept for testing and the remaining subsets are used for training.
The classification is then repeated as in the \ac{loo} technique.
In fact \ac{loo} is a particular case of \ac{kcv} when $k$ equals the number of patients.
In the reviewed papers, the typical values used for $k$ has been set to three and five.
\Ac{kcv} is regarded as more appropriate than \ac{loo}, but the number of patients in the dataset needs to be large enough for the results to be meaningful.

\subsection{Evaluation measures} \label{subsec:chp3:img-clas:eval-mea}

\begin{table}
  \caption{Overview of the evaluation metrics used in \acs*{cad} systems.}
  \scriptsize
  \begin{tabularx}{\textwidth}{@{}l >{\raggedleft\arraybackslash}X@{}}
    \toprule
    \textbf{Evaluation metrics} & \textbf{References} \\
    \midrule
    \quad Accuracy & \cite{Artan2009,Artan2010,Liu2009,Sung2011,Tiwari2012} \\
    \quad Sensitivity - Specificity & \cite{Artan2009,Artan2010,Giannini2013,Liu2009,Lopes2011,Mazzetti2011,Ozer2009,Ozer2010,Parfait2012,Peng2013,Tiwari2008,Tiwari2009,Viswanath2008,Viswanath2008a,trigui2016classification,trigui2017automatic,samarasinghe2016semi,cameron2014multiparametric,cameron2016maps,khalvati2015automated} \\
    \quad \acs*{roc} - \acs*{auc} & \cite{Ampeliotis2008,Antic2013,Chan2003,Giannini2013,Kelm2007,Langer2009,Liu2013,Lopes2011,Lv2009,Matulewicz2013,Mazzetti2011,Niaf2011,Niaf2012,Peng2013,Tiwari2009a,Tiwari2010,Tiwari2012,Tiwari2013,Viswanath2009,Viswanath2011,Viswanath2012,Vos2008,Vos2008a,Vos2010,giannini2015fully,lehaire2014computer,rampun2015classifying,rampun2015computer,rampun2016computer,rampun2016computerb,rampun2016quantitative} \\
    \quad \acs*{froc} & \cite{Litjens2011,Litjens2012,Vos2012} \\
    \quad Dice's coefficient & \cite{Artan2009,Artan2010,Liu2009,Ozer2009} \\
    \bottomrule
  \end{tabularx}
\label{tab:evatec}
\end{table}

Several metrics are used in order to assess the performance of a classifier and are summarized in \acs{tab}~\ref{tab:evatec}.
Voxels in the \ac{mri} image are classified into healthy or malign tissue and compared with a ground-truth.
This allows to compute a confusion matrix by counting true positive (TP), true negative (TN), false positive (FP), and false negative (FN) samples.
From this analysis, different statistics are extracted. 

The first statistic used is the accuracy which is computed as the ratio of true detection to the number of samples.
However, depending on the strategy employed in the \ac{cad} work-flow, this statistic is highly biased by a high number of true negative samples which boost the accuracy score overestimating the actual performance of the classifier.
That is why, the most common statistics computed are sensitivity and specificity defined in \acs{eq}\,\eqref{eq:sens} and \acs{eq}\,\eqref{eq:spec}, respectively.
The metrics give a full overview of the performance of the classifier.

\begin{equation}
  \text{SE} = \frac{\text{TP}}{\text{TP} + \text{FN}} \ ,
  \label{eq:sens}
\end{equation}

\begin{equation}
  \text{SP} = \frac{\text{TN}}{\text{TN} + \text{FP}} \ .
  \label{eq:spec}
\end{equation}

These statistics are also used to compute the \ac{roc} curves~\cite{Metz2006}, which give information about voxel-wise classification.
This analysis represents graphically the sensitivity as a function of $(1 - \text{specificity})$, which is in fact the false positive rate, by varying the discriminative threshold of the classifier.
By varying this threshold, more true negative samples are found but often at the cost of detecting more false negatives.
However, this fact is interesting in \ac{cad} since it is possible to obtain a high sensitivity and to ensure that no cancers are missed even if more false alarms have to be investigated or the opposite.
A statistic derived from \ac{roc} analysis is the \acf{auc} which corresponds to the area under the \ac{roc} and is a measure used to make comparisons between models.

The \acf{froc} extends the \ac{roc} analysis but to a lesion-based level.
The same confusion matrix is computed where the sample are not pixels but lesions.
However, it is important to define what is a true positive sample in that case.
Usually, a lesion is considered as a true positive sample if the region detected by the classifier overlaps ``sufficiently'' the one delineated in the ground-truth.
However, ``sufficiently'' is a subjective measure defined by each researcher and can correspond to one pixel only.
However, an overlap of \SIrange{30}{50}{\percent} is usually adopted.
Finally, in addition to the overlap measure, the Dice's coefficient is often computed to evaluate the accuracy of the lesion localization.
This coefficient consists of the ratio between twice the number of pixels in common and the sum of the pixels of the lesions in the ground-truth $\text{GT}$ and the output of the classifier $\text{S}$, defined as shown in \acs{eq}\,\eqref{eq:dice}.

\begin{equation}
  Q_D = \frac{2 | \text{GT} \cap \text{S} |}{| \text{GT} | + | \text{S} |} \ .
  \label{eq:dice}
\end{equation}


\section{Conclusion}

In this paper, we presented one of the the first \ac{cad} system using all
the \ac{mpmri} modalities for prostate cancer detection.
Indeed, \ac{mrsi} has nearly never been used together with the other
modalities.
With an extensive validation approach to select the best
features, the best balancing strategy as well as the best classifier,
we obtained results on a rather complicated dataset of 17 patients
with an average \ac{auc} of $0.836 \pm 0.083$ which put our system in the
state-of-the-art, even so different \ac{cad}s were tested on different
datasets.

			
\acresetall
\graphicspath{{4_materials/figures/}}
\chapter{Materials}\label{chap:4}

\begin{quote}
  \textbf{Replication}\quad The repetition of a scientific
  experiment or trial to obtain a consistent result.
\end{quote}

\noindent Such definition of \emph{replication} reveals the importance of reproducible research, since the confirmation of results obtained from independent studies is considered the scientific gold standard to build our body of knowledge.

\citeauthor{peng2011reproducible} states the excitement and wanders that computational science brings to the scientific landscape, but he also exposes the limitations in the scientific community to evaluate its published findings due to the lack of reproducibility~\cite{peng2011reproducible}.
In order to overcome such limitations, \citeauthor{peng2011reproducible}~\cite{peng2011reproducible} proposes reproducibility spectrum to be covered to move from non-reproducible publication to fully-reproducible research (see \acs{fig}\,\ref{fig:reproducibility_spectrum}).
%% sik:
%% Specifically, \citeauthor{peng2011reproducible} argues that the original data, the code of methods developed should both be made available along with the executable that lead to the published results so that the three are fully coupled\,\cite{peng2011reproducible}.
Specifically, \citeauthor{peng2011reproducible}~\cite{peng2011reproducible} argues that the original data and executable code which lead to the published results should be all coupled and available. 

\begin{figure}
\centering
\includegraphics[width=.7\textwidth]{reproducibility_spectrum}
\caption[The spectrum of reproducibility.]{The spectrum of reproducibility (copyright by~\cite{peng2011reproducible}).}
\label{fig:reproducibility_spectrum}
\end{figure}

Furthermore, \citeauthor{varoquaux2015Software}~\cite{varoquaux2015Software} in his article %\citetitle{varoquaux2015Software}
\emph{Of Software and Science. Reproducible science: what, why, and how} summarizes a discussion which took place in \emph{\acs{mloss}} workshop regarding the issue of reproducible science.
%% sik:
%% The take away is that the reproducibility spectrum proposed by \citeauthor{peng2011reproducible} falls short because it focuses on providing material to backup publications but oversights the importanc of sound reusable material and methods, which are the foundation of future scientific developments despite being cast out of the success formula in academia where only impact factor seems to matter.
He concluded that the reproducibility spectrum proposed by \citeauthor{peng2011reproducible}~\cite{peng2011reproducible} falls short because it focuses on providing materials to backup publications but oversights the importance of sound reusable materials and methods, which are the foundation of future scientific developments despite being cast out of the success formula in academia where only impact factor seems to matter.

%% sik:
%% With all this\footnote{better connector needed. Maybe a sentece}, the structure of this chapter is going to appear odd to some readers since our intention with this \nameref{chap:4} chapter is two fold:
%% (i) use this chapter to position ourselves with respect of reproductible research; and, (ii) describe all the resources or outcomes from this thesis that make for reproductible research.
With respect to aforementioned discussion our intention with this chapter is two fold: (i) position ourselves with respect to reproducible research and (ii) describe all the resources and outcomes from this thesis that allow our experiments be reproducible.
%% sik:
%% The former part is more of philosophical discussion or proclamation of the working systems and protocols emerging from this thesis work.
The former part is a philosophical discussion of the working framework used during the thesis. 
%% While the later is a more concrete description of the resources to reproduce the work of our thesis.
The latter is a concrete description of the resources to reproduce the work of this thesis.
Therefore, in the remainder of this section we first present the structure of our framework towards reproducible research, before to present our dataset which is publicly available through our website.
We conclude by presenting the open source toolboxes developed during this thesis.
%% Underlying design choices are out of the scope of this chapter.
%% Those regarding the methodologies developed for this thesis work can be found in other chapters.

\section{Our efforts towards reproducible research}
To conduct our research we have developed working strategies based on existing and in-house platforms.
This section describes the current and upcoming states of the framework, which has been refined through the thesis.
%% sik:
%% This section describes its current status with upcoming fixtures resulting from writing and reviewing this thesis, since these working strategies follow an iterative incremental procedure.

\subsection{\acs{iccvb}}
We have created the \acs{iccvb} website and its associated GitHub community, which stands on the following core pillars stated on the website\footnote{\url{http://i2cvb.github.io/}}.

\paragraph{Why? Vision: Ease the access to research}
The first need in modern research, regardless its application domain, is related to the access to reliable data for its subsequent study.
However, data gathering is an entrance barrier for most of the researchers mainly due to factors as diverse costs, infrastructure, availability, etc.
Moreover, isolated endeavours to gather these data without granting public access lead to the creation of muda (``waste''): waste of resources and inability to compare results and validate conclusions.

Despite being highlighted by numerous research works, the lack of usable, public, reliable, and accessible data remains disregarded in many fields.
The \ac{iccvb} is a wake-up call for addressing and breaking the entrance barriers in research due to data and/or isolation by applying collaborative strategies.

\paragraph{What? Mission: Provide open data; evaluation methods; comparison framework; reporting platform}
The lack of common data combined with non-aggregated assessing strategies result in non-existent or misleading comparisons which make difficult to acknowledge relevant novel methodologies.
A common duty to the research and development communities is to overcome these limitations, which can be successfully addressed by co-creation and collaborative work.

\Ac{iccvb} aims at serving as foundation for collecting and sharing data as well as providing common evaluation methodologies.
Furthermore, the use of common data and evaluation is the only way to achieve fair comparison.

\paragraph{Who? Protagonists: Research groups and individuals from all walks of life to shape a transparent community}
\Ac{iccvb} creates for everyone the opportunity to pursue common goals through sharing, collaborating and team-working, to empower the individuals by taking advantages of personal skills and resources.
As a consequence, young researchers will find an eco-system for self-improvement in which work will be rewarded through benchmarking compilation.

\paragraph{How? Strategy: Transfer successful practises from free software and quality management}
\Ac{iccvb} community challenges the impossible as well as the current status quo in research.
Therefore, we strive to settle a multi-skilled community pursuing common goals to achieve excellence through collaborative continuous improvement.

At \ac{iccvb}, we believe that Free software and quality management have already reshaped the world and that it is time to apply some of the successful practices learned in such domains to expand the boundaries of research in computer vision and specially for the medical imaging case.

\subsection{Software Quality}

All the different implemented source codes for this thesis have been released to support future development and the possibility to build a consistent benchmark.
All available code is primarily developed in Python with a concern of:
(i) \emph{Quality assurance} by developing unit tests, automatic code quality checking, and code consistency checking using \texttt{PEP8} standards;
(ii) \emph{Continuous integration} is achieved through tools as Travis CI to easily integrate new contributions and ensure back-compatibility;
(iii) \emph{Community-based development} by using collaborative tools --- git, GitHub, and gitter --- to ease collaborative programming, issue tracking, code integration, and idea discussions;
(iv) \emph{Documentation} through a description of the developed API.

\subsection{Working strategy}
As aforesaid, we developed a website and a GitHub community.
%% sik:
%% The website is used mainly as facade to our project but its core is the \acs{iccvb} GitHub community and this is how we strive for repeatable research.
The website is used mainly as front-hand of our project while the \ac{iccvb} GitHub community is the main core and this is how we strive for repeatable research.
Our research is based on collaborating with specialists to collect data, coding experiments, interpret, and communicate the observations.
%% Back to the data, re-code the experiments, re-interpret the observations.
%% In essence research is an iterative incremental process that require to ensure the quality at each step and its final deliver of each iteration is a manuscript that pin-point a position in the history of this research.
In essence, research is an iterative and incremental process that needs to ensure the quality of each of its step. 
%% The final deliverable of each iteration is a manuscript which shows the thinking process of this thesis.
%% For some publication it might refer to a small portion of our research or quick proof of concept while for other works it refers to a larger portion.

%% Consolidating an analysis pipeline, a standard visualization, or any computational aspect of work into a software library is a sure way to make the publications more reproducible.
%% Maintaining the library will ensure that results are still reproducible on new hardware, or with evolution of the general software stack.
%% Documentation and curated examples will lower the bar to reuse and facilitate replication of the original scientific results.
%% To avoid feature technical debt, libraries call for focused efforts on selecting the most important operations and become projects on their own.\,\cite{varoquaux2015Software}.

%\paragraph{Publication releases}
%% The rest of this section summarizes our working strategy based on the experience acquired during this thesis that lead to the aforesaid.
%% Publications, experiments, etc. become projects.

The outcome of any research is highlighted through publications.
In our work strategy publications of each project are sub-projects of the main projects which are both hosted in GitHub.
%% These projects have their code and documentation hosted in GitHub while its associated publications are subprojects with their source also hosted in GitHub.
This allows us to review projects and publications in the same manner taking advantage of issue tracking and \acs{ci}.
The data of the project are hosted at \acs{cern}, provided with a \acs{doi} using \emph{Zenodo}, and disseminated through \acs{iccvb} website.
At the time of publication, the code is released to freeze its state. 
\emph{Zenodo} provides a \acs{doi} to reference the release.
Releases are incorporated to the \acs{ci} systems to detect back-compatibility breaks and fix them by release reviews.
Evolution of our tools and libraries also captured by the \acs{ci}.

\section{Prostate data}
This section describes the datasets used in this thesis which are also available through the \acs{iccvb} website.

\subsection{\SI{1.5}{\tesla} General Electric scanner}

The \ac{mpmri} data are acquired from a cohort of patients with higher-than-normal level of \ac{psa}.
The acquisition is performed using a \SI{1.5}{\tesla} whole body GE Signa \ac{mri} scanner (General Electric, Milwaukee, WI, USA) with an endorectal coil (Medrad, Pittsburgh, PA, USA), using sequences to obtain \ac{t2w}-\ac{mri}, \ac{dce}-\ac{mri}, \ac{dw}-\ac{mri}, and \ac{mrsi}.
Aside of the \ac{mri} examination, these patients also have undergone a guided-biopsy.
% The dataset is composed of a total of 20 patients of which 18 patients have biopsy proven \ac{cap} and 2 patients are ``healthy'' with negative biopsies.
% Therefore, 13 patients have a \ac{cap} in the \ac{pz}, 3 patients have \ac{cap} in the \ac{cg}, 2 patients have invasive \ac{cap} in both \ac{pz} and \ac{cg}, and finally 2 patients are considered as ``healthy''.
% An experienced radiologist has segmented the prostate organ as well as the prostate zones, and \ac{cap} on the \ac{t2w}-\ac{mri}.

Three-dimensional \ac{t2w} fast spin-echo (\ac{tr}:\SI{3480}{\ms}, \ac{te}:\SI{113.6}{\ms}, \ac{etl}: 16, slice thickness: \SI{3}{\mm}) images are then acquired in an oblique axial plane with a  $320 \times 224$ acquisition matrix and a pixel spacing of \SI{0.27}{\milli\metre}.

%The nominal matrix and \ac{fov} of the 3D \ac{t2w} fast spin-echo images are \SI[product-units=repeat]{320x256}{\milli\metre\squared} and \SI[product-units=repeat]{280x240}{\milli\metre\squared}, respectively, thereby affording sub-millimetric pixel resolution within the imaging plane.

\ac{dce}-\ac{mri} is performed using a fat suppressed 3D fast spoiled gradient echo (\ac{tr}/\ac{te}/Flip angle: \SI{4.42}{\ms}/\SI{2.10}{\ms}/\SI{12}{\degree}; Matrix: $320 \times 192$; slab of 40 partitions of \SI{3.5}{\mm} thickness; temporal resolution: \SI{10}{\s}/slab over approximately \SI{5}{\minute}).
A power injector (Medrad, Indianola, USA) is used to provide a bolus injection of Gd-DTPA (Dotarem, Guerbet, Roissy, France) at a dose of \SI{0.2}{\ml} Gd-DTPA/kg of body weight.

\ac{dw}-\ac{mri} images have been acquired using the single-shot spin-echo echo-planar imaging (EPI) technique.
The diffusion-encoding gradients have been applied using a pulsed gradient spin-echo technique resulting in diffusion images acquired at 2 b-values --- i.e., \SI{100}{\second\per\milli\meter\squared} and \SI{1400}{\second\per\milli\meter\squared} --- and in the 3 orthogonal directions.
Sequential sampling of the k-space has been used with a \ac{te} of \SI{100.1}{\ms}, a \ac{tr} of \SI{10825}{\ms}, a bandwidth of \SI{1953}{\hertz\per\px}, and an acquisition matrix size of $128 \times 128$.

\ac{mrsi} is performed using a water and lipid suppressed double-spin-echo point-resolved spectroscopic (PRESS) sequence optimized for quantification detection of choline and citrate metabolites.
Water and lipid have been suppressed using a dual-band spectral spatial pulse technique.
%In order to eliminate signals from adjacent tissues, especially periprostatic lipids and the rectal wall up to 8 outer voxel saturation pulses have been used.
Datasets have been acquired as $16 \times 8 \times 8$ phase-encoded spectral arrays, a \ac{te} of \SI{130}{\ms}, a \ac{tr} of \SI{1000}{\ms}.%, and \SI{13}{\minute} of acquisition time.
%A spectral bandwidth of \SI{1250}{\hertz} has been used with 512 data points.
%A combination of an elliptic weighted averaged k-space acquisition scheme 3D filtering of the signal in k-space have been used, the latter in order to reduce intervoxel signal combination.
%Shimming has been carried out using the Siemenbens 3D Mapshim routine on a voxel adapted to the volume of the entire prostate gland.

\subsection{\SI{3}{\tesla} Siemens scanner}\label{sec:data3t}

The \ac{mpmri} data are acquired from a cohort of patients with higher-than-normal level of \ac{psa}.
The acquisition is performed using a \SI{3}{\tesla} whole body \ac{mri} scanner (Siemens Magnetom Trio TIM, Erlangen, Germany) using sequences to obtain \ac{t2w}-\ac{mri}, \ac{dce}-\ac{mri}, \ac{dw}-\ac{mri}, and \ac{mrsi}.
Aside of the \ac{mri} examination, these patients also have undergone a guided-biopsy.
The dataset is composed of a total of 19 patients of which 17 patients have biopsy proven \ac{cap} and 2 patients are ``healthy'' with negative biopsies.
From those 17, 12 patients have a \ac{cap} in the \ac{pz}, 3 patients have \ac{cap} in the \ac{cg}, 2 patients have invasive \ac{cap} in both \ac{pz} and \ac{cg}.
An experienced radiologist has segmented the prostate organ --- on \ac{t2w}-\ac{mri}, \ac{dce}-\ac{mri}, and \ac{adc}-\ac{mri} --- as well as the prostate zones --- i.e., \ac{pz} and \ac{cg} ---, and \ac{cap} on the \ac{t2w}-\ac{mri}.

A \SI{3}{\mm} slice fat-suppressed \ac{t2w} fast spin-echo sequence (\ac{tr}: \SI{3400}{\ms}, \ac{te}: \SI{85}{\ms}, \ac{etl}:13) is used to acquire images in sagittal and oblique coronal planes, the latter planes being orientated perpendicular or parallel to the prostate \ac{pz} – rectal wall axis.
Three-dimensional \ac{t2w} fast spin-echo (\ac{tr}: \SI{3600}{\ms}, \ac{te}: \SI{143}{\ms}, \ac{etl}: 109, slice thickness: \SI{1.25}{\mm}) images are then acquired in an oblique axial plane.
The nominal matrix and \ac{fov} of the 3D \ac{t2w} fast spin-echo images are \SI[product-units=repeat]{320x256}{\milli\metre\squared} and \SI[product-units=repeat]{280x240}{\milli\metre\squared}, respectively, thereby affording sub-millimetric pixel resolution within the imaging plane.

\ac{dce}-\ac{mri} is performed using a fat suppressed 3D T$_1$ VIBE sequence (\ac{tr}: \SI{3.25}{\ms},\ac{te}: \SI{1.12}{\ms}, Flip angle:\SI{10}{\degree}; Matrix: $256 \times 192$; \ac{fov}: $280 \times 210$ (with \SI{75}{\percent} rectangular \ac{fov}); slab of 16 partitions of \SI{3.5}{\mm} thickness; temporal resolution: \SI{6}{\s}/slab over approximately \SI{5}{\minute}).
A power injector (Medrad, Indianola, USA) is used to provide a bolus injection of Gd-DTPA (Dotarem, Guerbet, Roissy, France) at a dose of \SI{0.2}{\ml} Gd-DTPA/kg of body weight.

\ac{dw}-\ac{mri} images have been acquired using the single-shot spin-echo echo-planar imaging (EPI) technique.
As proposed by \citeauthor{stejskal1965spin}~\cite{stejskal1965spin}, the diffusion-encoding gradients have been applied using a pulsed gradient spin-echo technique resulting in diffusion images acquired at 2 b-values --- i.e., \SI{100}{\second\per\milli\meter\squared} and \SI{800}{\second\per\milli\meter\squared} --- and in the 3 orthogonal directions.
Sequential sampling of the k-space has been used with a \ac{te} of \SI{101}{\ms}, a \ac{tr} of \SI{4200}{\ms}, and a bandwidth of \SI{1180}{\hertz\per\px}.
Other parameters included a \ac{fov} of \SI{240}{\milli\metre}, an acquisition matrix size of $128 \times 128$ and a slice thickness of \SI{3.5}{\milli\metre}.
The \ac{adc} map has been directly generated by the Siemens workstation from the raw data on a pixel-by-pixel basis.

\ac{mrsi} is performed using a water and lipid suppressed double-spin-echo point-resolved spectroscopic (PRESS) sequence optimized for quantification detection of choline and citrate metabolites.
Water and lipid have been suppressed using a dual-band spectral spatial pulse technique.
In order to eliminate signals from adjacent tissues, especially periprostatic lipids and the rectal wall up to 8 outer voxel saturation pulses have been used.
Datasets have been acquired as $16 \times 12 \times 16$ --- interpolated to $16 \times 16 \times 16$ phase-encoded spectral arrays, a \ac{te} of \SI{140}{\ms}, a \ac{tr} of \SI{720}{\ms} and \SI{13}{\minute} of acquisition time.
A spectral bandwidth of \SI{1250}{\hertz} has been used with 512 data points.
A combination of an elliptic weighted averaged k-space acquisition scheme 3D filtering of the signal in k-space have been used, the latter in order to reduce intervoxel signal combination.
Shimming has been carried out using the Siemenbens 3D Mapshim routine on a voxel adapted to the volume of the entire prostate gland.
Additional unsuppressed water acquisitions at \ac{te} of \SI{30}{\ms}, \SI{80}{\ms}, and \SI{140}{\ms} of \SI{1.5}{\minute} have also been performed in order to allow quantification with respect to prostate water.
Systematic verification of the global shim --- i.e., over the complete 3D PRESS-selected volume --- revealed line widths at half-height of the water peak of the order of \SIrange{20}{30}{\hertz}, routinely.
The line widths for individual voxels are of the order of \SIrange{8}{12}{\hertz}.
The total examination time, including the time spent positioning the patient, is approximately 45 minutes.


\section{\texttt{imbalanced-learn} toolbox}\label{chp4:sec:imblearn}

The \texttt{imbalanced-learn} toolbox\footnote{G. Lema\^itre is one of the core contributors} is an open-source python toolbox aiming at providing a wide range of methods to cope with the problem of imbalanced dataset frequently encountered in machine learning and pattern recognition.
The implemented state-of-the-art methods can be categorized into 4 groups: (i) under-sampling, (ii) over-sampling, (iii) combination of over- and under-sampling, and (iv) ensemble learning methods.
The proposed toolbox only depends on \texttt{numpy}, \texttt{scipy}, and \texttt{scikit-learn} and is distributed under MIT license.
Furthermore, it is fully compatible with \texttt{scikit-learn} and is part of the \texttt{scikit-learn-contrib} supported project.
Documentation, unit tests as well as integration tests are provided to ease usage and contribution.
The toolbox is publicly available in GitHub\footnote{\url{https://github.com/scikit-learn-contrib/imbalanced-learn}}.

To illustrate the developed API and the compatibility with \texttt{scikit-learn}, an example of a pipeline using a \ac{pca} decomposition, a \ac{smote} over-sampler, and a \ac{knn} classifier is presented below:

\begin{lstlisting}[language=Python, caption=Code snippet to over-sample a dataset using \acs*{smote} in conjunction with \ac{pca} and a \ac{knn} classifier.]
from sklearn.datasets import make_classification
from sklearn.cross_validation import train_test_split as tts
from sklearn.decomposition import PCA
from sklearn.neighbors import KNeighborsClassifier as KNN
from sklearn.metrics import classification_report
from imblearn.over_sampling import SMOTE
from imblearn.pipeline import Pipeline
X, y = make_classification(n_classes=2, class_sep=2,
                           n_informative=3, n_redundant=1, flip_y=0,
                           n_features=20, n_clusters_per_class=1,
                           n_samples=1000, weights=[0.1, 0.9])
pca = PCA()
smt = SMOTE()
knn = KNN()
pipeline = Pipeline([('smt', smt), ('pca', pca), ('knn', knn)])
X_train, X_test, y_train, y_test = tts(X, y, random_state=42)
pipeline.fit(X_train, y_train)
y_hat = pipeline.predict(X_test)
\end{lstlisting}

\section{\texttt{protoclass} toolbox}\label{chp4:sec:protoclass}

The \texttt{protoclass} toolbox\footnote{G. Lema\^itre is the lead contributor} is an open-source python toolbox providing tools for fast prototyping of machine learning pipeline in medical imaging.
It implements most of the state-of-the-art feature detection techniques presented in \acs{chp}\,\ref{chap:3}.
To illustrate the API, an example is given in which a \ac{t2w}-\ac{mri} volume is normalized and the voxels corresponding to the prostate are extracted and can be used easily with \texttt{scikit-learn}.
This toolbox is publicly available on GitHub\footnote{\url{https://github.com/glemaitre/protoclass}}.

\begin{lstlisting}[language=Python, caption=Code snippet to normalize a volume and extract some voxels.]
import os
from protoclass.data_management import T2WModality
from protoclass.data_management import GTModality
from protoclass.preprocessing import GaussianNormalization
from protoclass.extraction import IntensitySignalExtraction

# Define the path the different data path
path_t2w = '/data/T2W'
path_gt = ['/data/GT/prostate']
label_gt = ['prostate']

# Read the T2W
t2w_mod = T2WModality()
t2w_mod.read_data_from_path(path_t2w)

# Read the ground-truth
gt_mod = GTModality()
gt_mod.read_data_from_path(label_gt, path_gt)

# Normalize the T2W modality
t2w_norm = GaussianNormalization(T2WModality())
t2w_norm.fit(t2w_mod, ground_truth=gt_mod, cat=label_gt[0])
t2w_mod = t2w_norm.normalize(t2w_mod, ground_truth=gt_mod,
                             cat=label_gt[0])

# Extract the voxel from the prostate
ise = IntensitySignalExtraction(t2w_mod)
data = ise.transform(t2w_mod, ground_truth=gt_mod, cat=label_gt[0])
\end{lstlisting}

	
\acresetall
\chapter{Normalization/Standardization of T2W-MRI and DCE-MRI Images} \label{chap:5}
\Ac{cad} systems are usually designed as a sequential process consisting of four stages: pre-processing, segmentation, registration, and classification.
We presented in \acs{sec}\,\ref{subsubsec:ch3:mriprepro} the state-of-the-art techniques for normalization/standardization of \ac{mri} modality among other pre-processing steps.
As a conclusion, we can stress that only little attention has been dedicated to this topic.
Data normalization is, however, a crucial and important step of the chain to design a robust classifier and overcome the inter-patient intensity variations.

In this chapter, we focus on the normalization of \ac{t2w}-\ac{mri} and \ac{dce}-\ac{mri} modalities.
On the one hand, we investigate two novel \ac{t2w}-\ac{mri} normalization methods based on (i) Rician \emph{a priori} and (ii) \ac{srsf} representation and compare them with the state-of-the-art methods.
On the other hand, we propose and investigate a fully automated framework for \ac{dce}-\ac{mri} normalization, the first of its kind.

\section{Normalization of \ac{t2w}-\ac{mri} images} \label{sec:chp5:T2-norm}

This section focuses on \ac{t2w}-\ac{mri} normalization.
First, the related work is presented in \ac{sec}\,\ref{subsec:chp5:relwork1} before focusing on two new normalization methods which are presented and investigated in \acs{sec}\,\ref{subsec:chp5:T2-norm:meth} and \acs{sec}\,\ref{subsec:chp5:T2-norm:Exp-res}

\subsection{Related work}\label{subsec:chp5:relwork1}

We briefly recall the state-of-the-art methods which have been proposed for the normalization of \ac{t2w}-\ac{mri} prostate images.

\citeauthor{Artan2010}~\cite{Artan2009,Artan2010}, \citeauthor{Ozer2010}~\cite{Ozer2009,Ozer2010}, and \citeauthor{rampun2016computerb}~\cite{rampun2015classifying,rampun2015computer,rampun2016computer,rampun2016computerb} used a parametric method to normalize \ac{t2w}\ac{mri} images.
This parametric method is based on computing the standard score --- also known as \emph{z-score} --- of the \ac{pz} voxels such as: 
\begin{equation}
  I_{s}(x) = \frac{I_{r}(x) - \mu_{PZ}}{\sigma_{PZ}}, \forall x\in PZ ,
  \label{eq:zscore}
\end{equation}
\noindent where, $I_{s}(x)$ and $I_{r}(x)$ are the standardized and the raw signal intensity, respectively, and $\mu_{PZ}$ and $\sigma_{PZ}$ are the mean and standard deviation of the \ac{pz} signal intensity, respectively. 
This transformation enforces the image \ac{pdf} to have a zero mean and a unit standard deviation.
However, this normalization is not appropriate if the \ac{pdf} does not follow a Gaussian distribution as illustrated in Fig.\,\ref{fig:fitting}

\citeauthor{Lv2009}~\cite{Lv2009} used the non-parametric method which is a piecewise-linear normalization, proposed by \citeauthor{Nyul2000} in~\cite{Nyul2000}.
For a given patient, a warping function is inferred by matching some specific landmarks --- i.e., different percentiles --- of the current \ac{pdf} to the same landmarks learned during a training phase from several patients. 
The mapping between each landmark is performed using a linear mapping.
\citeauthor{Viswanath2012} used a variant of the previous method by segmenting first the image using region growing with a pre-defined homogeneity criterion and keeping only the largest region to build the \ac{pdf}~\cite{Viswanath2012}.
Nevertheless, the warping functions inferred by these methods suffer from abrupt changes --- refer to \acs{fig}\,\ref{fig:maplinear} --- around the landmarks position, leading to a disrupt \ac{pdf} in the normalized image.

In this section, we evaluate and compare different normalization approaches in the context of \ac{t2w}-\ac{mri} prostate image normalization.
Our contribution is threefold: (i) a parametric normalization approach based on a Rician \textit{a priori}; (ii) a non-parametric normalization approach based on a method used in registration of functional data; and (iii) a novel evaluation metric to asses quantitatively the alignment of the \acp{pdf} independently of the assumed distribution. 
These methods are compared qualitatively and quantitatively, with both \textit{z-score} normalization and piecewise-linear normalization.

\subsection{Methodology}\label{subsec:chp5:T2-norm:meth}

\subsubsection{Parametric normalization using Rician \textit{a priori}}\label{subsubsec:chp5:T2-norm:meth:rician}
As previously stated, proper normalization of the \ac{mri} data during pre-processing is a key problem that has been addressed using parametric and non-parametric strategies.
We believe that normalizing \ac{mri} data using a parametric model based on a Rician distribution would improve the results.
Expecting this improvement by changing the data model from the widely used Gaussian distribution to Rician distribution is reasonable.
Indeed, \citeauthor{Bernstein1989} state that \ac{mri} data theoretically follow a Rayleigh distribution for a low-\ac{snr} scenarios while it appears closer to a Gaussian distribution when the \ac{snr} increases~\cite{Bernstein1989}.
Figure~\ref{fig:fitting} shows the intensity spectrum for some \ac{mri} prostate data as well as the fitted Gaussian and Rician distributions for 2 different patients.
In this figure the solid-black line represents the Rician fitting while the dotted-black shows the fitted Gaussian.
A qualitative assessment of the underlying distribution is performed by overlying the fitted distribution, while quantitative results of the fitting are given in terms of \ac{rms}.
It can be highlighted that the Rician model better fits the data than the Gaussian model.

\begin{figure}
  \centering
  \subfigure[\acs*{rms}: Rice \num{4.13e-5} - Normal \num{2.59e-3}]{
    \label{fig:p1}\includegraphics[width=0.48\textwidth]{5_normalization/figures/T2-normalization/03}}\hfill
  \subfigure[\acs*{rms}: Rice \num{2.25e-4} - Normal \num{9.57e-4}]{
    \label{fig:p2}\includegraphics[width=0.48\textwidth]{5_normalization/figures/T2-normalization/06}}\hfill
%  \subfloat[][]{
%    \label{fig:p3}\includegraphics[width=0.3\textwidth]{14}}
  \caption[Visual evaluation of the goodness of fitting using Rician and Gaussian distribution for 2 different \acs*{mri} prostate data.]{Visual evaluation of the goodness of fitting using Rician and Gaussian distribution for 2 different \acs*{mri} prostate data. For each data the solid black line represents the Rician fitting while the dotted represents the Gaussian distribution.}
  \label{fig:fitting}
\end{figure}

The normalization is carried out through the following 3 steps: 
(i) fit a Rician model --- \acs{eq}\,\eqref{eq:rice} --- to each prostate \ac{pdf} using non-linear least squares minimization, namely Levenberg-Marquardt; 
(ii) compute the mean --- \acs{eq}\,\eqref{eq:meanr} --- and variance --- \acs{eq}\,\eqref{eq:var} --- of the Rician model;
(iii) normalize the entire data using the \textit{z-score} similarly as in~\acs{eq}\,\eqref{eq:zscore}.

\begin{equation}
  f(x| \nu, \sigma) = \frac{x}{\sigma^2}\exp\left( \frac{- (x^2 + \nu^2)}{2\sigma^2} \right) I_0 \left( \frac{x \nu}{\sigma^2} \right) \ ,
  \label{eq:rice}
\end{equation}

\begin{equation}
  \mu_{r} = \sigma  \sqrt{\frac{\pi}{2}}\,\,L_{1/2}(-\frac{\nu^2}{2\sigma^2})  \ ,
  \label{eq:meanr}
\end{equation}

\begin{equation}
  \sigma_{r} = 2\sigma^2+\nu^2-\frac{\pi\sigma^2}{2}L_{1/2}^2\left(\frac{-\nu^2}{2\sigma^2}\right)  \ ,
  \label{eq:var}
\end{equation}

\noindent where $\nu$ and $\sigma$ are the distance between the reference point and the center of the bi-variate distribution and the scale, respectively; $L_{1/2}$ denotes a Laguerre polynomial; $I_0$ is the modified Bessel function of the first kind with order zero.

\subsubsection{Non-parametric normalization based on \acs*{srsf}}\label{subsubsec:chp5:T2-norm:gen-model}

\citeauthor{Srivastava2011} proposed a generic method to register functional data, without any assumption regarding the models of the different functions~\cite{Srivastava2011}. 
In a nutshell, this framework relies on the \ac{srsf} representation which transforms the Fisher-Rao metric into the conventional $\mathbb{L}^2$ metric, and thus allows to define a cost function corresponding to an Euclidean distance between 2 functions in this new representation.

\paragraph{\Ac{srsf} representation}

In the proposed registration framework of functional data, 2 functions $f_1$ and $f_2$ are registered by composing $f_2$ with a warping function $\gamma$ such that:

\begin{equation}
  \argmin_{\gamma \in \Gamma} D_{FR}(f_1, (f_2 \circ \gamma)) \ ,
  \label{eq:regfun}
\end{equation}

\noindent where $D_{FR}$ is the Fisher-Rao distance and $\Gamma$ is the set of all the functions $\gamma$.

The \ac{srsf} representation is used to transform the functions and register them into this space.
The \ac{srsf} of a function $f$ is defined as:

\begin{equation}
  q(t) = \sign(\dot{f}(t))\sqrt{|\dot{f}(t)|} \ ,
  \label{eq:srsf}
\end{equation}

\noindent where $\dot{f}(t)$ corresponds to the derivative of $f$.

The major property of the \ac{srsf} representation used in the registration framework is the following: the composition of a function $f$ with a warping function $\gamma$ --- i.e., $f \circ \gamma$ --- is equivalent to \acs{eq}\,\eqref{eq:warp}, using the \ac{srsf} representation.

\begin{equation}
  \tilde{q}(t) = (q(t) \circ \gamma) \sqrt{\dot{\gamma}} \ ,
  \label{eq:warp}
\end{equation}

\noindent where $\dot{\gamma}$ is the derivative of $\gamma$.

Using this property, a cost function --- so called amplitude or $y$-distance --- is defined to measure the similarity between the 2 functions $f_1$ and $f_2$, expressed as in \acs{eq}\,\eqref{eq:cf}:

\begin{equation}
  D_y(f_1, f_2) = \underset{\gamma \in \Gamma}{\infspie} \| q_1 - (q_2 \circ \gamma) \sqrt{\dot{\gamma}} \| \ .
  \label{eq:cf}
\end{equation}

\paragraph{Registration framework}\label{par:chp5:T2-norm:regfra}

The registration framework consists of 2 steps.
First, an initialization in which the Karcher mean $\mu_f$ is computed as in \acs{eq}\,\eqref{eq:mean}

\begin{equation}
  \mu_f = \argmin_{f \in \mathcal{F}} \sum_{i = 1}^{n} D_y(f, f_i)^2 \ .
  \label{eq:mean}
\end{equation}

Then, for each function $f_i$: 
(i) compute $\gamma_{i}^{*}$ as in \ac{eq}\,\eqref{eq:warpi}; 
(ii) compute $\tilde{q}_i$ as in \ac{eq}\,\eqref{eq:warp};
(iii) update $\mu_f$ as in \ac{eq}\,\eqref{eq:mean} by replacing $f_i$ by $\tilde{f_i}$, using $\tilde{q}_i$.

\begin{equation}
  \gamma_{i}^{*} = \argmin_{\gamma \in \Gamma} \sum_{i = 1}^{n} D_y(\mu_f, f_i)^2 \ ,
  \label{eq:warpi}
\end{equation}

\noindent where $n$ is the total number of functions to be aligned.

This step is performed in an iterative manner based on the gradient of the cost function given in \acs{eq}\,\eqref{eq:mean}. 
We refer the reader to the work of \citeauthor{Srivastava2011} for more detailed discussion~\cite{Srivastava2011}.

\subsubsection{Evaluation metric}

In their work, \citeauthor{Nyul2000} evaluated the normalization methods by computing the variation of the mean of a specific tissue.
However, this measure can be biased since that the mean can also be used as a landmark with the piecewise-linear method.
Furthermore, considering a single statistic does not allow to evaluate the overall performance of a normalization.
Indeed, this statistic corresponds to evaluate a single point of the mapping function and thus a large portion of the mapping functions are disregarded. 

That is why, to evaluate the performance of the different metrics, we propose to use a spectral evaluation by decomposing the set of normalized \ac{pdf}s using \ac{pca} under the assumption that they are linearly dependent. 
Intuitively, the eigenvalues of the \ac{pca} decomposition are correlated with the alignment of the different \acp{pdf}.
Thus, in the case of a perfect alignment of the \ac{pdf}s, the first eigenvalue is much greater than the remaining since that the first eigenvector encodes all the information.
In the contrary, in the case of a misalignment of the \ac{pdf}s, more eigenvectors are needed to encode the information synonymous with larger eigenvalues.
Therefore, the cumulative sum of the normalized eigenvalues as well as the \ac{auc} are used, as depicted in \acs{fig}\,\ref{fig:qt}.

\subsection{Materials}\label{subsec:chp5:T2-norm:Exp-res}

The experiments are conducted on a subset of the public \ac{mpmri} prostate presented in \acs{sec}\,\ref{sec:data3t}.
We used the \SI{3}{\tesla} dataset which is composed of a total of 19 patients of which 17 patients had biopsy proven \ac{cap} and 2 patients are ``healthy'' with negative biopsies. 
In this study, our subset consists of 17 patients with \ac{cap}.

The different normalization methods are implemented in Python and are part of the \texttt{protoclass} toolbox presented in \acs{sec}\,\ref{chp4:sec:protoclass}.
The normalization based on \ac{srsf} uses the implementation\footnote{\url{https://github.com/glemaitre/fdasrsf}} of \citeauthor{Tucker2013}~\cite{Tucker2013}.
The piecewise-linear normalization is performed using the following set of percentiles $s \in \{0, 5, 25, 50, 75, 95, 100 \}$ as landmarks.
In the \ac{srsf}-based normalization, the \acp{pdf} are smoothed using spline-based denoising method.

\subsection{Results and discussion}

\paragraph{Qualitative results}

% \begin{figure}
%   \centering
%   \includegraphics[width=1.\textwidth]{5_normalization/figures/T2-normalization/qualitative.png}
%   \caption{Qualitative evaluation by visual inspection of the alignment of the \ac{pdf}s for the full prostate and the \ac{cap}.}
%   \label{fig:qu}
% \end{figure}

\begin{figure}
  \hspace*{\fill}
  \subfigure[Piecewise-linear mapping function.]{
    \label{fig:maplinear}\includegraphics[width=0.4\textwidth]{5_normalization/figures/T2-normalization/piecewise-linear.png}}\hfill
  \subfigure[\acs*{srsf} mapping function.]{
    \label{fig:mapsrsf}\includegraphics[width=0.4\textwidth]{5_normalization/figures/T2-normalization/srsf.png}}
  \hspace*{\fill}
  \caption[Comparison of the mapping functions found with the piecewise-linear and \acs*{srsf}-based normalization.]{Comparison of the mapping functions found with the piecewise-linear and \acs*{srsf}-based normalization. Each curve corresponds to a mapping function for a single patient.}
  \label{fig:mapping}
\end{figure}

\Acl{fig}~\ref{fig:qu} depicts the alignment of the different \acp{pdf} using the different methods implemented. 
All the methods seem to address the problem of the \ac{pdf} alignment of the full prostate data.
However, the Rician normalization outperforms the other methods when focusing solely on the \ac{cap} data.
The \ac{pdf} computed in this specific area is more skewed from its original shape in the case of the piecewise-linear normalization than with the 3 other normalization strategies.
The \ac{srsf} normalization gets unstable due to the warping function $\gamma$ found which is in practise non-smooth as shown in \acs{fig}\,\ref{fig:mapsrsf}.
Additionally, the warping function found with the piecewise-linear normalization suffer from abrupt transition around the landmarks as depicted in \acs{fig}\,\ref{fig:mapsrsf}.

\paragraph{Quantitative results}

In overall, all normalization methods improve the alignment of the \acp{pdf}.
The parametric methods outperform the non-parametric while evaluating the \ac{pdf} alignment considering the full prostate organ.
Furthermore, the Rician normalization is more appropriate than the Gaussian normalization.
The \ac{srsf}-based normalization is shown to perform poorly which might be due to the instability of the mapping function inferred.
However, by focusing on the solely on the \ac{cap} region, the \ac{srsf} outperforms the other methods followed by the Rician normalization.
Therefore, the Rician normalization outperforms the other methods with an \ac{auc} of $99.74$ and $98.25$ considering the full prostate and \ac{cap}, respectively.

\begin{landscape}

\begin{figure}
  \hspace*{\fill}
  \subfigure[Raw prostate - \acs*{auc}: 99.08.]{
    \label{subfig:raw}\includegraphics[width=.23\linewidth]{5_normalization/figures/T2-normalization/raw.pdf}}
  \hfill
  \subfigure[Raw \acs*{cap} - \acs*{auc}: 98.19.]{
    \label{subfig:raw_cap}\includegraphics[width=.23\linewidth]{5_normalization/figures/T2-normalization/raw_cap.pdf}}
  \hspace*{\fill}
  \\
  \hspace*{\fill}
  \subfigure[Gaussian prostate - \acs*{auc}: 99.58.]{
    \label{subfig:gaussian}\includegraphics[width=.23\linewidth]{5_normalization/figures/T2-normalization/gaussian.pdf}}
  \hfill
  \subfigure[Rician prostate - \acs*{auc}: 99.74.]{
    \label{subfig:rician}\includegraphics[width=.23\linewidth]{5_normalization/figures/T2-normalization/rician.pdf}}
  \hfill
  \subfigure[Linear prostate - \acs*{auc}: 99.45.]{
    \label{subfig:piecewise}\includegraphics[width=.23\linewidth]{5_normalization/figures/T2-normalization/piecewise.pdf}}
  \hfill
  \subfigure[\acs*{srsf} prostate - \acs*{auc}: 99.12.]{
    \label{subfig:srsf}\includegraphics[width=.23\linewidth]{5_normalization/figures/T2-normalization/srsf.pdf}}
  \hspace*{\fill}\\
  \hspace*{\fill}
  \subfigure[Gaussian \acs*{cap} - \acs*{auc}: 98.22.]{
    \label{subfig:gaussian_cap}\includegraphics[width=.23\linewidth]{5_normalization/figures/T2-normalization/gaussian_cap.pdf}}
  \hfill
  \subfigure[Rician \acs*{cap} - \acs*{auc}: 98.25.]{
    \label{subfig:rician_cap}\includegraphics[width=.23\linewidth]{5_normalization/figures/T2-normalization/rician_cap.pdf}}
  \hfill
  \subfigure[Linear \acs*{cap} - \acs*{auc}: 98.09.]{
    \label{subfig:piecewise_cap}\includegraphics[width=.23\linewidth]{5_normalization/figures/T2-normalization/piecewise_cap.pdf}}
  \hfill
  \subfigure[\acs*{srsf} \acs*{cap} - \acs*{auc}: 98.30.]{
    \label{subfig:srsf_cap}\includegraphics[width=.23\linewidth]{5_normalization/figures/T2-normalization/srsf_cap.pdf}}
  \hspace*{\fill}
  \caption[Qualitative evaluation for \acs*{t2w}-\acs*{mri}]{Qualitative evaluation by visual inspection of the alignment of the \acs*{pdf}s for the full prostate and the \acs*{cap} in \acs*{t2w}-\acs*{mri}. The first row corresponds to the original \acs*{pdf}}
  \label{fig:qu}
\end{figure}

\end{landscape}

\begin{figure}
  \centering
  \subfigure[]{
    \label{fig:qtfull}\includegraphics[width=0.8\textwidth]{5_normalization/figures/T2-normalization/quantitative_1.pdf}}\\
  \subfigure[]{
    \label{fig:qtcap}\includegraphics[width=0.8\textwidth]{5_normalization/figures/T2-normalization/quantitative_2.pdf}}
  \caption{Spectral evaluation using \acs*{pca} decomposition: \protect\subref{fig:qtfull} evaluation considering the full prostate, \protect\subref{fig:qtcap} evaluation considering only the \acs*{cap}.}
  \label{fig:qt}
\end{figure}

\subsection{Conclusion}\label{subsec:chp5:T2-norm:dis-con}
In this section, we propose to normalize the \ac{t2w}-\ac{mri} prostate images using two new strategies: (i) based on a Rician \textit{a priori} and (ii) based on a \ac{srsf} representation.
An extensive comparison has been conducted showing that the Rician normalization outperforms the Gaussian, \ac{srsf}-based, and piecewise-linear normalization for \ac{t2w}-\ac{mri} prostate images normalization.
As avenues for future research, the contribution of the Rician normalization must be evaluated in a classification framework.
Additionally, the \ac{srsf}-based normalization is unstable due to a non-smooth mapping function which might be solved by post-processing this function.
Although our proposed evaluation metric seems more appropriate than the previous method, we think that complementary metric should be proposed.
Furthermore, normalized \ac{t2w}-\ac{mri} can be included with other modalities in order to perform classification using \ac{mpmri} data.

\section{Normalization of \acs*{dce}-\acs*{mri} images}\label{sec:chp5:DCE-norm}

This section focuses on \ac{dce}-\ac{mri} normalization.
We recall that in \ac{dce}-\ac{mri}, a contrast media is injected intravenously and a set of images is acquired over time.
Consequently, each voxel in an image corresponds to a dynamic signal which is related to both contrast agent concentration and the vascular properties of the tissue.
Therefore, changes of the enhanced signal allows to discriminate healthy from \ac{cap} tissues.
In fact, these properties are automatically extracted using quantitative or semi-quantitative approaches~\cite{Lemaitre2015}.

\emph{Quantitative} approaches uses pharmacokinetic modelling based on a bicompartment model, namely Brix~\cite{brix1991pharmacokinetic} and Tofts~\cite{tofts1995quantitative} models.
The parameters of the Brix model are inferred assuming a linear relationship between the media concentration and the \ac{mri} signal intensity.
This assumption has shown, however, to lead to inaccurately estimate the pharmacokinetic parameters~\cite{heilmann2006determination}.
Instead, the Tofts model requires a conversion from the \ac{mri} signal intensity to concentration, which becomes a non-linear relationship using the specific equations of the \ac{mri} sequences (e.g., FLASH sequence).
Tofts modelling suffers, however, from a higher complexity~\cite{gliozzi2011phenomenological}.
Indeed, the conversion using the non-linear approach requires to acquire a T$_1$ map which is not always possible during clinical examination.
Additionally, the parameter calculation requires the \ac{aif} which is challenging to measure and can also lead to an inaccurate estimation.

\emph{Semi-quantitative} approaches are rather mathematical than pharmacokinetic modelling since no pharmacokinetic assumption regarding the relation between the \ac{mri} signal and the contrast agent are made~\cite{huisman2001accurate,gliozzi2011phenomenological}.
These methods offer the advantages to not require any knowledge about the \ac{mri} sequence nor any conversion from signal intensity to concentration.
However, they present some limitations: the heuristic approach proposed by \citeauthor{huisman2001accurate}~\cite{huisman2001accurate} requires an initial estimate of the noise standard deviation of the signal as well as some manual tuning.

Nevertheless, all presented methods suffer from 2 major drawbacks:
(i) inter-patient variability and (ii) loss of information.
The inter-patient variability is mainly due to the acquisition process and consequently leads to generalization issue while applying a machine learning algorithm.
All previous methods extract few discriminative parameters to describe the \ac{dce}-\ac{mri} signal which might lead to a loss of information.
%(i) the inter-patient variability of the data lead to a variation of the parameters estimated and subsequently to poor classification performance while designing \ac{cad} systems, and
%(ii) only few parameters are used to characterize the dynamic signal implying that some information are discarded.

In this section, we propose a fully automatic normalization method for \ac{dce}-\ac{mri} that reduces the inter-patient variability of the data.
The benefit and simplicity of our approach will be shown by classifying the whole normalized \ac{dce}-\ac{mri} signal and comparing with the state-of-the-art quantitative and semi-quantitative methods.
Additionally, we will show that using this normalization approach in conjunction with the quantitative methods improves the classification performance of most of the models.
We also propose a new clustering-based method to segment enhanced signals from the arteries, later used to estimate an \ac{aif} as well as an alternative approach to estimate the parameters of the semi-quantitative model proposed by~\cite{huisman2001accurate}.

% The benefit of our approach will be shown while using quantitative and semi-quantitative approaches.
% Additionally, we show that using the whole normalized \ac{dce}-\ac{mri} signal is preferable to quantitative and semi-quantitative methods, leading to the best classification performance.

This section is organized as follows:
First, \acs{sec}\,\ref{subsubsec:chp5:DCE-norm:norm} details our normalization strategy for \ac{dce}-\ac{mri} data.
Quantitative and semi-quantitative methods are summarized in \acs{sec}\,\ref{subsubsec:chp5:DCE-norm:stateart} with insights about their implementations.
Finally experiments and results to answer the previous stated challenges are reported in \acs{sec}\,\ref{subsec:chp5:DCE-norm:exp-res} while discussed in \acs{sec}\,\ref{subsec:chp5:DCE-norm:dis-con}, followed by a concluding section.
%Section~\ref{sec:methods} outlines our normalization strategy (Section~\ref{sec:norm}) as well as specificity regarding the state-of-the-art methods used for comparison (Section~\ref{sec:stateart}).
%The dataset, experiments, and results are reported in Section~\ref{sec:experiments} while discussed in Section~\ref{sec:discussions} followed by a concluding section.


\subsection{Methodology} \label{subsec:chp5:DCE-norm:meth}

\subsubsection{Normalization of \ac{dce}-\ac{mri} images}\label{subsubsec:chp5:DCE-norm:norm}

\begin{figure}
  \centering
  \includegraphics[width=0.7\linewidth]{5_normalization/figures/DCE-normalization/t2wImage.pdf}
  \caption{Illustration of the inter-patient variations in 17 different patients, using the \acs*{pdf} representation.}
  \label{fig:t2}
\end{figure}

In this section, we propose a method to normalize \ac{dce}-\ac{mri} prostate data to reduce inter-patient variations, although it can be applied to any \ac{dce}-\ac{mri} sequences.
As presented in the previous section, in \ac{t2w}-\ac{mri}, these variations are characterized by a shift and a scaling of the intensities as illustrated by the intensity \ac{pdf} in \acs{fig}\,\ref{fig:t2}.
Therefore, these variations can be corrected using a $z$-score approach --- i.e., normalizing the data by subtracting the mean and dividing by the standard deviation --- assuming that the data follow a specific distribution~\cite{lemaitre2016normalization}.

\begin{figure}
  \centering
  \hspace*{\fill}
  \subfigure[]{\label{subfig:pathhist}\includegraphics[width=1\textwidth]{5_normalization/figures/DCE-normalization/heatmaprep.pdf}} \hfill
  \hspace*{\fill}
  \\
  \hspace*{\fill}
  \subfigure[]{\label{subfig:pat1}\includegraphics[width=.49\textwidth]{5_normalization/figures/DCE-normalization/pat1_annotated.pdf}} \hfill
  \subfigure[]{\label{subfig:pat2}\includegraphics[width=.49\textwidth]{5_normalization/figures/DCE-normalization/pat2_annotated.pdf}} \hfill
  \hspace*{\fill}
  \caption[Illustration of the heatmap in \acs*{dce}-\acs*{mri} images.]{\ac{dce} normalization: \subref{subfig:pathhist} Illustration of the heatmap representation: all \acs*{pdf}s of the prostate gland are concatenated together to build an heatmap; \subref{subfig:pat1}-\subref{subfig:pat2} Illustration of inter-patient variations (i.e., $\Delta_i$, $\Delta_t$, and $\alpha_i$) \acs*{pdf} over time of two patients in a \acs*{dce}-\acs*{mri}.}
  \label{fig:heatmap}
\end{figure}

In \ac{dce}-\ac{mri}, the intensity \ac{pdf} of prostate gland does not follow a unique type of distribution such as Rician or Gaussian distribution, as shown in \acs{fig}\,\ref{subfig:pathhist}.
Indeed, the inter-patient variations are more complex due to the temporal acquisition.
A better representation to observe these variations is to represent the intensity \ac{pdf} of the prostate gland over time --- requiring to segment the prostate --- using a heatmap representation as shown in \acs{fig}\,\ref{subfig:pathhist}.
Analyzing this heatmap representation across patients (see Fig.\,\ref{subfig:pat2}), the following variations are highlighted:
(i) intensity offsets $\Delta_i$ of the \ac{pdf} peak,
(ii) a time offset $\Delta_t$ depending of the contrast agent arrival, and
(iii) a change of scale $\alpha_i$ related to the signal enhancement.
Therefore, our normalization method should attenuate all these variations and be performed globally across the different time sequences rather than for each independent sequence.

\paragraph{Graph-based intensity offsets correction}\label{par:chp5:DCE-norm:graph}

\begin{figure}
  \centering
  \includegraphics[width=0.7\linewidth]{5_normalization/figures/DCE-normalization/estimator.pdf}
  \caption{Illustration of the estimator found using the shortest-path through the graph.}
  \label{fig:estimator}
\end{figure}

Before to standardize each sequence, the first step of the normalization is to cancel the intensity specific at each patient, occurring due to the media injection.
As previously mentioned, the intensity \ac{pdf} does not always follow either a Rician or a Gaussian distribution over time, in \ac{dce}-\ac{mri}.
Therefore, the mean of these distributions cannot be used as a potential estimate for these offsets.
Additionally, these offsets should be characterized by a smooth transition between series over time.
Thus, this problem is solved using the graph-theory: considering the intensity \ac{pdf} over time as shown in \acs{fig}\,\ref{subfig:pathhist}, the offsets correspond to the boundary splitting the heatmap in two partitions such that they are as close as possible to the peak of the intensity \ac{pdf}, as depicted in \acs{fig}\,\ref{fig:estimator}.
Given the heatmap, a directed weighted graph $\mathcal{G}=(\mathcal{V}, \mathcal{E})$ is built by taking each bar --- i.e., the probability for a given time and pixel intensity --- of the heatmap as a node and connecting each pair of bars by an edge.
The edge weight $w_{ij}$ between 2 nodes $i$ and $j$ corresponding to 2 pixels at position $(x_i, y_i)$ and $(x_j, y_j)$, respectively, is defined as in \acs{eq}\,\eqref{eq:weight}:

\begin{equation}
  w_{ij} = \begin{cases}
    \alpha \exp(1 - \frac{H(i)}{\max(H)})       & \text{if } x_j = x_i + 1 \text{ and } y_j = y_i, \\
    (1 - \alpha) \exp(1 - \frac{H(i)}{\max(H)}) & \text{if } x_j = x_i \text{ and } y_j = y_i + 1, \\
    0                                           & \text{otherwise},
  \end{cases}
  \label{eq:weight}
\end{equation}

\noindent where $H$ is the heatmap, $\alpha$ is a smoothing parameter controlling the partitioning.

Therefore, these offsets related to $\Delta_i$ are estimated by finding the shortest-path to cross the graph using Dijkstra's algorithm.
The entry and exiting nodes are set to be the bin with the maximum probability for the first \ac{dce}-\ac{mri} serie and the bin corresponding to the median value for the last \ac{dce}-\ac{mri} serie, respectively.
To ensure a robust estimation of these offsets, the process of finding the shortest-path is repeated in an iterative manner by shifting the data and updating the heatmap as well as the graph $\mathcal{G}$.
The procedure is stopped once the offset found does not change.
In general, this process is not repeated more than 3 iterations.
The parameter $\alpha$ is set to $0.9$, empirically.
Figure~\ref{fig:estimator} illustrates the final estimation of the offsets $\Delta_i$ (i.e., red landmark) found for each \ac{dce}-\ac{mri} serie.
Therefore, each intensity offset is subtracted for each \ac{dce}-\ac{mri}.

\paragraph{Time offset and data dispersion correction}\label{par:chp5:DCE-norm:time-off}

\begin{figure}
  \centering
  \hspace*{\fill}
  \subfigure[\acs*{rmse} computed for each patient of our dataset.]{\label{fig:rmse}\includegraphics[width=.49\textwidth]{5_normalization/figures/DCE-normalization/rmse.pdf}} \hfill
  \subfigure[\acs*{rmse} after alignment using the curve parametric model.]{\label{fig:rmseal}\includegraphics[width=.49\textwidth]{5_normalization/figures/DCE-normalization/rmse_aligned.pdf}}
  \hspace*{\fill}
  \caption{Illustration of the correction of the time offset and the data dispersion.}
  \label{fig:curveal}
\end{figure}

The next variations to correct are the time offset $\Delta_t$ and the data dispersion $\sigma_i$.
By computing the \ac{rmse} of the intensities for each \ac{dce}-\ac{mri} serie, one can observe these two variations as shown in \acs{fig}\,\ref{fig:rmse}.
Therefore, to correct these variations, we propose to register each patient \ac{rmse} to a mean model which corresponds to the mean of all patients \ac{rmse}.
The parametric model to perform the registration is formulated as in \acs{eq}\,\eqref{eq:model}:

\begin{equation}
  T(\alpha, \tau, f(t)) = \alpha f(t - \tau) ,
  \label{eq:model}
\end{equation}

\noindent where $\tau$ and $\alpha$ are the two parameters handling the time offset $\Delta_i$ and global scale $\sigma_i$, respectively, $f(\cdot)$ is the \ac{rmse} function defined as:

\begin{equation}
  f(t) = \sqrt{ \left( \frac{\sum_{n=1}^{N} x(t)_{n}^2}{N}  \right) },
  \label{eq:rmsd}
\end{equation}

\noindent where $x(t)_n$ is the shifted intensity of a sample from a specific \ac{dce}-\ac{mri} serie at time $t$ from a total number of $N$ samples.

Therefore the registration problem is equivalent to:

\begin{equation}
  \argmin_{\alpha, \tau} = \sum_{t=1}^{N} \left[ T\left(\alpha, \tau, f(t)\right) - \mu(t) \right]^{2} ,
  \label{eq:cost}
\end{equation}

\noindent where $\mu(\cdot)$ is the mean model, $N$ is the number of \ac{dce}-\ac{mri} series.

Illustration of the correction applied to each \ac{rmse} patient is shown in \acs{fig}\,\ref{fig:rmseal}.
Once all these parameters have been inferred, the data are shifted as well as scaled.

The resulting normalized data can be used into 2 fashions: (i) each normalized signal can be used as a whole to determine whether the corresponding voxel is healthy or cancerous or (ii) the normalized data can be fitted using a quantitative method, as presented in the next section.
%However, for the second strategy, this is necessary to apply common intensity offsets such that the data follow a shape as expected by the different quantitative models.
%The set of offsets applied is in fact corresponding to the maximum offsets found in Sect.\,\ref{sec:intoffsets}.

\subsubsection{Quantification of \acs*{dce}-\acs*{mri}}\label{subsubsec:chp5:DCE-norm:stateart}

The quantitative approaches for detection of \ac{dce}-based features have been briefly discussed in \acs{sec}\,\ref{subsubsec:chp3:img-clas:CADX-fea-dec:DCE-fea}.
In this section, we present in details the different methods which have been used for the quantification of \ac{dce}-\ac{mri} for \ac{cap} detection~\cite{Lemaitre2015} and which will be used for comparison in this work.
Furthermore, we would like to emphasize the following additional contributions for this section: (i) a novel automatic \ac{aif} estimation algorithm based on clustering and (ii) a simplified semi-quantitative method using constrained optimization.

\paragraph{Brix and Hoffmann models}\label{par:chp5:DCE-norm:brixhoffmann}

In the Brix model~\cite{brix1991pharmacokinetic}, the \ac{mri} signal intensity is assumed to be proportional to the media concentration.
Therefore, the model is expressed as in \acs{eq}\,\eqref{eq:brix} (see also \acs{eq}\,\eqref{eq:brixmod}):

\begin{equation}
  s_n(t) = 1 + A \left[ \frac{\exp(k_{el} t') - 1}{k_{ep}(k_{ep} - k_{el})} \exp(- k_{el} t) - \frac{\exp(k_{ep} t') - 1}{k_{el}(k_{ep} - k_{el})} \exp(- k_{ep} t) \right],
  \label{eq:brix}
\end{equation}

\noindent with

\begin{equation}
  s_n(t) = \frac{s(t)}{S_0},
  \label{eq:enh}
\end{equation}

\noindent where $s(t)$ and $S_0$ are the \ac{mri} signal intensity at time $t$ and the average pre-contrast \ac{mri} signal intensity, respectively; $A$, $k_{el}$, and $k_{ep}$ are the constant proportional to the transfer constant, the diffusion rate constant, and the rate constant, respectively.
Additionally, $t'$ is set such that $0 \leq t \leq \tau$, $t' = t$ and afterwards while $t > \tau$, $t' = \tau$.

\citeauthor{hoffmann1995pharmacokinetic}~\cite{hoffmann1995pharmacokinetic} proposed a similar model as expressed in \acs{eq}\,\eqref{eq:hoffmann}, which derive from the Brix model:

\begin{equation}
  \small
  s_n(t) = 1 + \frac{A}{\tau} \left[ \frac{k_{ep} \left( \exp(k_{el} t') - 1 \right)}{k_{el}(k_{ep} - k_{el})} \exp(- k_{el} t) - \frac{\exp(k_{ep} t') - 1}{(k_{ep} - k_{el})} \exp(- k_{ep} t) \right] ,
  \label{eq:hoffmann}
\end{equation}

\noindent in which the constant $A$ is redefined by isolating the parameter $\tau$.

The parameters $A$, $k_{el}$, and $k_{ep}$ are estimated by fitting the model using non-linear least-squares optimization solved with Levenberg-Marquardt.

\paragraph{Tofts model}\label{par:chp5:DCE-norm:tofts}

The extended Tofts model is formulated as in \acs{eq}\,\eqref{eq:exttofts} (see also \acs{eq}\,\eqref{eq:tofts}):

\begin{equation}
  C_t(t) = K_{trans} C_p(t) \Conv \exp(-k_{ep}t) + v_p C_p(t),
  \label{eq:exttofts}
\end{equation}

\noindent where $\Conv$ is the convolution operator; $C_t(t)$ and $C_p(t)$ are the concentrations of contrast agent in the tissue and in the plasma, respectively; $K_{trans}$, $k_{ep}$, and $v_p$ are the volume transfer constant, the diffusion rate constant, and the plasma volume fraction, respectively.

Therefore, Tofts model requires to:
(i) detect candidate voxels from the femoral or iliac arteries and estimate a patient-based \ac{aif} signal,
(ii) convert the \ac{mri} signal intensity (i.e., \ac{aif} and dynamic signal) to a concentration, and
(iii) in the case of a population-based \ac{aif}, estimate an \ac{aif} signal.

\begin{figure}
  \centering
  \hspace*{\fill}
  \subfigure[Original image.]{\label{fig:org}\includegraphics[width=.3\textwidth]{5_normalization/figures/DCE-normalization/original.pdf}} \hfill
  \subfigure[Candidates region after clustering.]{\label{fig:cand}\includegraphics[width=.3\textwidth]{5_normalization/figures/DCE-normalization/candidate.pdf}} \hfill
  \subfigure[Regions selected after applying the different criteria.]{\label{fig:final}\includegraphics[width=.3\textwidth]{5_normalization/figures/DCE-normalization/aif.pdf}}
  \hspace*{\fill}
  \caption{Illustration of the segmentation of the area used to determine the \acs*{aif}.}
  \label{fig:aif}
\end{figure}

\begin{description}
  \item[Segmentation of artery voxels and patient-based \ac{aif} estimation] The \ac{aif} signal from \ac{dce}-\ac{mri} can be manually estimated by selecting the most-enhanced voxels from the femoral or iliac arteries~\cite{meng2010comparison}.
    Few methods have been proposed to address the automated extraction of \ac{aif} signal.
    \citeauthor{Chen2008} filtered successively the possible candidates to be considered as \ac{aif} such that~\cite{Chen2008}:
    (i) dynamic signals with small peak and voxels with a small wash-in are rejected by thresholding,
    (ii) a blob detector is used and large enough regions are kept, and
    (iii) circular and cylindricality criteria are used to reject the false positives.
    \citeauthor{zhu2011automated} proposed an iterative method selecting voxels which best fit a gamma variate function~\cite{zhu2011automated}.
    However, it requires to compute first and second derivatives as well as maximum curvature points.
    \citeauthor{shanbhag2012generalized} proposed a 4-steps algorithm~\cite{shanbhag2012generalized,fennessy2015quantitative}:
    (i) remove slices with artifacts and find the best slices based on intrinsic anatomic landmarks and enhancement characteristics,
    (ii) find the voxel candidates using the maximum enhanced voxels and a multi-label maximum entropy based thresholding algorithm,
    (iii) exclude region next to the endorectal coil, and
    (iv) select the best 5 candidates which meet enhancement characteristics and that are correlated.

    All the above methods are rather complex compromising robustness and generalization.
    Thus, we propose a simpler method which is based on the following reasonable assumptions:
    (i) all possible \ac{aif} signal candidates should have a similar shape,
    (ii) a high enhancement, and
    (iii) the arteries should be almost round and within a size range.
    Therefore, each slice is clustered into regions using K-means clustering with $k=6$.
    The cluster made of the most enhanced signals is selected since it contains the artery signals.
    In this regard, the selection criteria corresponds to the 90\textsuperscript{th} percentile of the maximum \ac{dce}-\ac{mri} signal.
    Finally, regions with an eccentricity smaller than $0.5$ and an area in the range of $[100, 400]$ voxels are kept.
    Additionally, to remove voxels contaminated by partial volume effect, only the \SI{10}{\percent} most enhanced voxels of the possible candidates are kept as proposed by~\cite{schabel2008uncertainty} and the average signal is computed.
    A summary of the different segmentation steps is presented in \acs{fig}\,\ref{fig:aif}.
    \item[Conversion of \ac{mri} signal intensity to concentration] To estimate the free parameters of the Tofts model (see \acs{eq}\,\eqref{eq:exttofts}), the concentration $C_t(t)$ and $C_p(t)$ need to be computed from the \ac{mri} signal intensity and the \ac{aif} signal, respectively.
      This conversion is based on the equation of the FLASH sequence --- see~\ref{app:signaltoconc} for details --- and is formulated as in Eq.\,\eqref{eq:conv}:
      \begin{equation}
        c(t) = \frac{1}{TR \cdot r_1} \ln\left( \frac{1 - \cos \alpha \cdot S^{*}\frac{s(t)}{S_0}}{1 - S^{*}\frac{s(t)}{S_0}} \right) - \frac{R_{10}}{r_1} ,
        \label{eq:conv}
      \end{equation}
      \noindent with,
      \begin{equation}
        S^{*} = \frac{1 - \exp(- TR \cdot R_{10})}{1 - \cos \alpha \cdot \exp(- TR \cdot R_{10})} ,
        \label{eq:sstarconv}
      \end{equation}
      \noindent where $s(t)$ is the \ac{mri} signal, $S_0$ is the \ac{mri} signal prior to the injection of the contrast media, $\alpha$ is the flip angle, $TR$ is the \acf{tr}, $R_{10}$ is the pre-contrast tissue relaxation time also equal to $\frac{1}{T_{10}}$, and $r_1$ is the relaxitivity coefficient of the contrast agent.

      $T_{10}$ can be estimated from the acquisition of a T$_1$ map.
      However, this modality is not part of the clinical trial in this research and the value of $T_{10}$ is fixed to \SI{1600}{\ms} for both blood and prostate, in accordance with the values found in the literature~\cite{fennessy2015quantitative,de2004mr,carr2011magnetic}.
      \item[Estimation of population-based \ac{aif}] While estimating the pharmacokinetic parameters from Tofts model, the \ac{aif} concentration $C_p(t)$ can be computed either from the patient or a population.
        We presented in the two previous sections the algorithms which allows to estimate the patient-based \ac{aif} concentration.
        To compare with the previous approach, we also computed a population-based \ac{aif} which will be also used later to compare the performance of both approaches.
        In that regard, the population-based \ac{aif} was estimated as in~\cite{meng2010comparison} by fitting the average patient-based \ac{aif}s to the model of~\cite{parker2006experimentally} which is formulated as in \ac{eq}\,\eqref{eq:parker}:
        \begin{equation}
          C_p(t) = \sum_{n=1}^{2} \frac{A_n}{\sigma_n \sqrt{2 \pi}} \exp\left(\frac{- (t- T_n)^2}{2\sigma_{n}^{2}}\right) + \frac{\alpha \exp(-\beta t)}{1 + \exp{-s (t - \tau)}} ,
          \label{eq:parker}
        \end{equation}
        \noindent where $A_n$, $T_n$, and $\sigma_n$ are the scaling constants, centers, and widths of the n\textsuperscript{th} Gaussian, $\alpha$ and $\beta$ are the amplitude and decay constant of the exponential; and $s$ and $\tau$ are the width and center of the sigmoid function, respectively.
\end{description}

The parameters are estimated by fitting the model using a constrained non-linear least-squares optimization, solved with the Trust Region Reflective algorithm~\cite{sorensen1982newton} and bounding the parameters to be positive.

\paragraph{\acs*{pun} model}\label{par:chp5:DCE-norm:pun}

\citeauthor{gliozzi2011phenomenological} showed that \ac{pun} approach can be used for \ac{dce}-\ac{mri} analysis~\cite{gliozzi2011phenomenological}.
The model has been successfully used in a \ac{cad} system proposed by~\citeauthor{giannini2015fully}~\cite{giannini2015fully}.
This model can be expressed as in \ac{eq}\,\eqref{eq:pun2} (see also \ac{eq}\,\eqref{eq:pun}):

\begin{equation}
  s_n(t) = \exp\left[rt + \frac{1}{\beta} \left( a_0 - r \right) \left( \exp(\beta t) - 1 \right) \right],
  \label{eq:pun2}
\end{equation}

\noindent with

\begin{equation}
  s_n(t) = \frac{s(t) - S_0}{S_0},
  \label{eq:enh}
\end{equation}

\noindent where $s(t)$ and $S_0$ are the \ac{mri} signal intensity at time $t$ and the average pre-contrast \ac{mri} signal intensity, respectively; $r$, $a_0$, and $\beta$ are the free parameters of the model.

The parameters are estimated by fitting the model using non-linear least-squares optimization solved with Levenberg-Marquardt.

\paragraph{Semi-quantitative analysis}\label{par:chp5:DCE-norm:semi}

The semi-quantitative analysis of the \ac{dce}-\ac{mri} is equivalent to extracting curve characteristics directly from the signal without a strict theoretical pharmacokinetic meaning (see \acs{tab}~\ref{tab:semiqua}).
In this work, we use the model presented by~\citeauthor{huisman2001accurate}~\cite{huisman2001accurate} which formulated the \ac{mri} signal as in \acs{eq}\,\eqref{eq:huisman}:

\begin{equation}
  s(t) = \begin{cases}
    S_0 & 0 \leq t \leq t_0 \\
    S_M - (S_M - S_0) \exp\left( \frac{-(t - t_0)}{\tau} \right) & t_0 < t \leq t_0 + 2 \tau \\
    S_M - (S_M - S_0) \exp\left( \frac{-(t - t_0)}{\tau} \right) + w(t - t_0 + 2 \tau) & t > t_0 + 2 \tau
  \end{cases}
  \label{eq:huisman}
\end{equation}

\noindent where $s(t)$ is the \ac{mri} signal intensity, $S_0$ is the pre-contrast signal intensity, $t_0$ is the time corresponding to the start of enhancement, $S_M$ and $\tau$ is the maximum of the signal and the exponential time constant, and $w$ is the slope of the linear part.

\citeauthor{huisman2001accurate}~\cite{huisman2001accurate} argue that curve fitting via least-squares minimization using Nelder-Mead algorithm leads to inaccurate estimation of the free parameters: mainly the issue comes from an incorrect estimation of the start of enhancement $t_0$ leading to incorrect estimation of the other parameters.
Therefore, they propose to:
(i) estimate robustly $t_0$,
(ii) estimate $S_0$ by averaging the samples between $0$ and $t_0$
(ii) estimate $w$ depending if the slope is significant or not,
(iii) estimate $S_M$ which should be the point at the intersection of the most probable slope line and the plateau.

Instead of these successive estimations, we propose a unified optimization in which $t_0$ is fixed since that this is a key parameter.
Therefore, $t_0$ is robustly estimated from the \ac{aif} signal since that this is the most enhanced signal in which the start of enhancement is easily identifiable.
The \ac{aif} signal is computed as presented previously.
$t_0$ is estimated by finding the maximum of the first derivative of the \ac{aif} signal, always occurring at the beginning of the signal.
Then, the function in \acs{eq}\,\eqref{eq:huisman} is fitted using non-linear least squares with the Trust Region Reflective algorithm~\cite{sorensen1982newton}.
Furthermore, the parameters $\tau$ and $S_M$ are bounded during the optimization to ensure robust estimations.
$\tau$ is bounded between $t_0$ and $t_f$ which is the time of the last sample while $S_M$ is bounded between $S_0$ and $\max(s(t))$.


From \acs{eq}\,\eqref{eq:huisman}, the following features are extracted:
(i) the wash-in corresponding to the slope between $t_0$ and $t_0 + 2 \tau$,
(ii) the wash-out corresponding to the parameter $w$,
(iii) the area under the curve between $t_0$ and the end of the signal,
(iv) the exponential time constant $\tau$, and
(v) the relative enhancement $S_M - S_0$.


\subsection{Experiments and results}\label{subsec:chp5:DCE-norm:exp-res}

%{\color{red} \textbf{Data, check with the material chapter}}

The experiments are conducted on a subset of the public \ac{mpmri} prostate presented in \acs{sec}\,\ref{sec:data3t}.
We used the \SI{3}{\tesla} dataset which is composed of a total of 19 patients of which 17 patients had biopsy proven \ac{cap} and 2 patients are ``healthy'' with negative biopsies. 
In this study, our subset consists of 17 patients with \ac{cap}.

The \ac{dce}-\ac{mri} sequences are resampled using the spatial information of the \ac{t2w}-\ac{mri} and missing data are interpolated using a linear interpolation.
The volumes of the \ac{dce}-\ac{mri} dynamic are rigidly registered, to remove any patient motion during the acquisition.
Furthermore, a non-rigid registration is performed between the \ac{t2w}-\ac{mri} and \ac{dce}-\ac{mri} in order to propagate the prostate zones and \ac{cap} ground-truths.
The resampling is implemented in C++ using the Insight Segmentation and Registration Toolkit~\cite{ibanez2005itk}.

The implementation of the registration (C++), normalization (Python), and classification pipeline (Python) are publicly available on GitHub\footnote{\url{https://github.com/I2Cvb/lemaitre-2016-nov/tree/master}}~\cite{lemaitre2016github}.
The data used for this work are also publicly available\footnote{\url{https://zenodo.org/record/61163}}~\cite{lemaitre2016dce}.

\subsubsection{Goodness of model fitting}\label{subsubsec:chp5:DCE-norm:Good}

%{\color{red} In case that we have issue with $R^2$, we need to provide the AIC since that the model are usually non-linear.}

\begin{table}
  \caption{Coefficient of determination $R^{2}$ (i.e., $\mu \ (\pm \sigma)$), while fitting data with the different quantification models.}
  \centering
  \scriptsize
  %\resizebox{\columnwidth}{!}{
  \begin{tabularx}{\textwidth}{lXXXXXX}
    \toprule
    \textbf{Data type} & \textbf{Brix} & \textbf{Hoffmann} & \textbf{Tofts pop. \acs*{aif}} & \textbf{Tofts pat. \acs*{aif}} & \textbf{\acs*{pun}} & \textbf{Semi-quantitative} \\
    \midrule
    Un-normalized & $0.85 \ (\pm 0.11)$ & $0.81 \ (\pm 0.17)$ & $0.84 \ (\pm 0.14)$ & $0.88 \ (\pm 0.12)$ & $0.27 \ (\pm 0.18)$ & $0.64 \ (\pm 0.24)$  \\
    Normalized    & $0.92 \ (\pm 0.05)$ & $0.72 \ (\pm 0.32)$ & $0.92 \ (\pm 0.06)$ & $0.90 \ (\pm 0.10)$ & $0.28 \ (\pm 0.20)$ & $0.75 \ (\pm 0.20)$  \\
    \bottomrule
  \end{tabularx}
  %}
  \label{tab:r2}
\end{table}

Parameter estimation of the quantification methods are related to fit a specific model to the \ac{dce}-\ac{mri} data.
Therefore, this section reports the goodness of fitting by computing the coefficient of determination $R^2$ such as in \acs{eq}\,\eqref{eq:r2}

\begin{equation}
  R^2 = 1 - \frac{\sum_{t = 1}^{T} (s_t - \hat{s}_t)^2}{\sum_{t = 1}^{T} (s_t - \bar{s})^2} ,
  \label{eq:r2}
\end{equation}

\noindent where $s_t$ and $\hat{s}_t$ are the original and fitted signals at time $t$, respectively; $\bar{s}$ is the average signal to be fitted.

Mean and standard-deviation of the coefficient of determination $R^{2}$ is reported in \acs{tab}~\ref{tab:r2} for each quantification model.
Brix, Hoffmann, and Tofts models are fitted with a coefficient $R^{2}$ superior to 0.80.
Additionally, the proposed \ac{pun} model does not seem to fit well the data.
Data normalization improves the coefficient $R^2$ for all the methods apart of the Hoffmann model.
The large standard deviation for this model might imply that there are some cases where the fitting fails.
Furthermore, the use of a bi-exponential model --- i.e., Brix, Hoffman, and Tofts models --- instead of a mix of mono-exponential and linear functions --- i.e., semi-quantitative analysis --- allow a better fitting.

\subsubsection{Detection of \acs*{cap} using pharmacokinetic parameters}\label{subsubsec:chp5:DCE-norm:phar}

\begin{table}
  \caption{\acs*{auc} (i.e., $\mu \ (\pm \sigma)$) for each individual pharmacokinetic parameter using a \acs*{rf} classifier.}
  \centering
  \scriptsize
  %\resizebox{\columnwidth}{!}{
  \begin{tabular}{lcc}
    \toprule
    \textbf{Features} & \textbf{Un-normalized data} & \textbf{Normalized data} \\
    \midrule
    \textbf{Brix model} & & \\
    \quad $A$         & $0.540\ (\pm 0.069)$ & $0.555\ (\pm 0.080)$ \\
    \quad $k_{el}$    & $0.549\ (\pm 0.062)$ & $0.577\ (\pm 0.093)$ \\
    \quad $k_{ep}$    & $0.506\ (\pm 0.032)$ & $0.497\ (\pm 0.019)$ \\
    \textbf{Hoffmann model} & & \\
    \quad $A$         & $0.516\ (\pm 0.020)$ & $0.508\ (\pm 0.031)$ \\
    \quad $k_{el}$    & $0.545\ (\pm 0.066)$ & $0.529\ (\pm 0.065)$ \\
    \quad $k_{ep}$    & $0.550\ (\pm 0.063)$ & $0.545\ (\pm 0.060)$ \\
    \textbf{Tofts model with population \acs*{aif}} & & \\
    \quad $K_{trans}$ & $0.556\ (\pm 0.086)$ & $0.565\ (\pm 0.097)$ \\
    \quad $k_{ep}$    & $0.506\ (\pm 0.026)$ & $0.528\ (\pm 0.038)$ \\
    \quad $v_{p}$     & $0.533\ (\pm 0.064)$ & $0.548\ (\pm 0.082)$ \\
    \textbf{Tofts model with patient \acs*{aif}} & & \\
    \quad $K_{trans}$ & $0.563\ (\pm 0.077)$ & $0.548\ (\pm 0.060)$ \\
    \quad $k_{ep}$    & $0.492\ (\pm 0.025)$ & $0.491\ (\pm 0.020)$ \\
    \quad $v_{p}$     & $0.530\ (\pm 0.069)$ & $0.495\ (\pm 0.033)$ \\
    \textbf{\acs*{pun} model} & & \\
    \quad $a_0$       & $0.521\ (\pm 0.040)$ & $0.530\ (\pm 0.045)$ \\
    \quad $r$         & $0.550\ (\pm 0.085)$ & $0.573\ (\pm 0.097)$ \\
    \quad $\beta$     & $0.531\ (\pm 0.051)$ & $0.549\ (\pm 0.068)$ \\
    \textbf{Semi-quantitative analysis} & & \\
    \quad wash-in     & $0.587\ (\pm 0.107)$ & $0.533\ (\pm 0.032)$ \\
    \quad wash-out    & $0.516\ (\pm 0.037)$ & $0.486\ (\pm 0.035)$ \\
    \quad IAUC        & $0.506\ (\pm 0.048)$ & $0.513\ (\pm 0.032)$ \\
    \quad $\tau$      & $0.565\ (\pm 0.104)$ & $0.537\ (\pm 0.089)$ \\
    \quad $S_M - S_0$ & $0.560\ (\pm 0.083)$ & $0.532\ (\pm 0.029)$ \\
    \bottomrule
  \end{tabular}
  %}
  \label{tab:resfeats}
\end{table}

\begin{figure}
  \centering
  \subfigure[Without normalization.]{\label{fig:rfpharmaunorm}\includegraphics[width=.7\textwidth]{5_normalization/figures/DCE-normalization/unormalized_methods_0.pdf}} \\
  \subfigure[With normalization.]{\label{fig:rfpharmanorm}\includegraphics[width=.7\textwidth]{5_normalization/figures/DCE-normalization/normalized_methods_0.pdf}}
  \caption{\acs*{roc} analysis using a \acs*{rf} classifier (a) with and (b) without normalization of \acs*{dce}-\acs*{mri} data for different pharmacokinetic models.}
  \label{fig:normpharmarf}
\end{figure}

To study the potential benefit of our normalization, \ac{cap} are detected at a voxel level using pharmacokinetic parameters estimated from un-normalized and normalized \ac{dce}-\ac{mri} data.
Each individual pharmacokinetic parameter is classified to evaluate their individual discriminative power to detect \ac{cap}.
Therefore, a \ac{rf} classifier is used in conjunction with a \ac{lopo}.
The use of \ac{rf} is motivated since that it leads to the best performance in the state-of-the-art methods~\cite{Litjens2014,Lemaitre2015}.
Results are summarized in \acs{tab}~\ref{tab:resfeats} in terms of \ac{auc}: in general, the discriminative power of each individual parameter is rather low and $K_{trans}$, $k_{el}$, and wash-in parameters lead to the best classification performance.
Furthermore, the obtained \ac{auc}s are in line with results reported in previous studies~\cite{giannini2015fully} since \ac{cap}s are occurring in both prostate zones --- i.e., \ac{pz} and \ac{cg} --- in our dataset.
Additionally, normalization can improve the detection of \ac{cap}; however, the benefit of normalization is more obvious by combining together the pharmacokinetic features of a given model --- e.g., $A$, $k_ep$, and $k_el$ for Brix model ---, as previously done in traditional \ac{cad} system~\cite{Lemaitre2015}.
For the latter configuration, results are summarized by performing a \ac{roc} analysis and computing the \ac{auc}, as reported in \acs{fig}\,\ref{fig:normpharmarf}.
Quantification using normalized data outperforms quantification using un-normalized data in terms of classification performance apart of Hoffmann and Tofts population-based \ac{aif} models.
The reasons behind the decrease of the \ac{auc} might be related to: (i) a poor fitting as discussed in \acs{sec}\,\ref{subsubsec:chp5:DCE-norm:Good} (cf., Hoffmann model) and (ii) a small number of patients while estimating some parameters (cf., Tofts model).
The best classification performance are obtained using the semi-quantitative approach with an \ac{auc} of 0.655.

\subsubsection{Classification of the entire enhanced \acs*{dce}-\acs*{mri} signal} \label{subsubsec:chp5:DCE-norm:class}

\begin{figure}
  \centering
  \includegraphics[width=0.7\linewidth]{5_normalization/figures/DCE-normalization/full_signal_0.pdf}
  \caption{\acs*{roc} analysis using the entire \acs*{dce}-\acs*{mri} signal with and without normalization in conjunction with a \acs*{rf} classifier.}
  \label{fig:rfnormdcesignal}
\end{figure}

As stated in the introduction, the quantification methods are extracting a set of parameters characterizing the enhancement \ac{dce}-\ac{mri} signal.
However, this extraction might lead to a loss of information.
This experiment is performed to assess if making use of the whole \ac{dce}-\ac{mri} signal instead of the just the pharmacokinetic parameters can improve the classification performance.
Therefore, each enhanced \ac{dce}-\ac{mri} signal, normalized and un-normalized, is classified using a \ac{rf} classifier in a \ac{lopo} fashion.
The \ac{roc} analysis and \ac{auc} are reported in \acs{fig}\,\ref{fig:rfnormdcesignal}.
Classification without normalization lead to the worst performance, with an \ac{auc} of 0.568.
However, data normalization in conjunction with the use of the whole \ac{dce}-\ac{mri} signal is the strategy which outperforms all others, with an \ac{auc} of 0.666.


\subsection{Discussion and conclusion}\label{subsec:chp5:DCE-norm:dis-con}

The experiments conducted in the previous section can give rise to several discussions.
In Tofts quantification, two different approaches have been used to infer the pharmacokinetic parameters: using a population-based or a patient-based \ac{aif}.
The patient-based \ac{aif} approach leads to better classification performance.
However, there are two shortcomings to take into account while advancing this fact:
(i) T$_{10}$ parameter has been fixed and not computed from a T$_1$ map and
(ii) the population-based \ac{aif} has been estimated from a cohort of only 17 patients.
These two limitations have to be considered while advancing that population-based \ac{aif} modelling is outperforming patient-based \ac{aif} modelling.

The best classification performance is reached by normalizing the \ac{dce}-\ac{mri} data and use the whole enhanced signal as feature, emphasizing the fact that a loss of information while extracting quantitative parameters.
Furthermore, this normalization is a less complex process than all quantification methods.
However, this strategy suffers from one drawback: the training time of the \ac{rf} classifier increases since that from 3 to 5 features, the feature space becomes a 40 dimensions space.

Nevertheless, this study is performed on a small cohort of patients using a single \ac{mri} machine.
Generalizing the results of this study on a larger dataset acquired from different commercial systems have to be considered to study the robustness of the proposed approach.



In this work, we presented a new method for normalizing/standardizing \ac{dce}-\ac{mri} data.
This method aimed at reducing the inter-patient variations occurring during data acquisition.
A graph-based approach was used to correct intensity offset in conjunction with a model-based correction to reduce time offset as well as intensity scaling.
We show the benefit of our normalization method prior to extract quantitative and semi-quantitative features, with a significant improvement of the classification performance.
Nevertheless, we also show that using the whole normalized \ac{dce}-\ac{mri} signal outperforms all quantitative approaches.

As avenues for future research, this normalization has to be part of a \ac{mpmri} \ac{cad} system in which \ac{dce}-\ac{mri} modality needs to be combined with other complementary modalities.




\acresetall
\chapter{Proposed CAD system for CaP}\label{chap:6}

In this chapter, we develop and investigate a \ac{cad} system for the \ac{cap} detection, using all \ac{mri} modalities, namely \ac{t2w}-\ac{mri}, \ac{dce}-\ac{mri}, \ac{dw}-\ac{mri}, and \ac{mrsi}.
Furthermore, we address some of the issues drawn in the conclusion of \acs{chp}\,\ref{chap:3}: (i) the methods investigated in \acs{chp}\,\ref{chap:5} are used in the pre-processing step of the proposed \ac{cad}; (ii) the discriminative power of each individual modality is investigated; (iii) the problem of learning from imbalanced dataset is investigated using state-of-the-art methods as well as (iv) several strategies for feature selection and combination.

Therefore, the organization of this chapter is as follows: the methodology is described in \acs{sec}\,\ref{sec:chp6:method} by presenting the image regularization framework as well as the image classification framework. \Acl{sec}~\ref{sec:chp6:exp-res} provides different experiments to investigate the performance of the proposed \ac{cad} system. This chapter is concluded by a concise discussion in \acs{sec}\,\ref{sec:chp6:discussion}.

\section{Methodology}\label{sec:chp6:method}

\subsection{Materials}

The \ac{mpmri} data are acquired from a cohort of patients with
higher-than-normal level of \ac{psa}.
Acquisition is achieved with a \SI{3}{\tesla} whole body
\ac{mri} scanner (Siemens Magnetom Trio TIM, Erlangen, Germany) using
sequences to obtain \ac{t2w}-\ac{mri}, \ac{dce}-\ac{mri},
\ac{dw}-\ac{mri}, and \ac{mrsi}.
In addition of the \ac{mri} examination, these patients also have undergone
a \ac{trus} guided-biopsy.
The dataset is composed of a total of 19 patients, 17 of which
have biopsies that were positive for \ac{cap} and 2 patients are considered
``healthy'' because they have negative biopsies.
In all 12 patients have a \ac{cap} in the \ac{pz}, 3 patients
have \ac{cap} in the \ac{cg}, 2 patients have invasive \ac{cap} in
both the \ac{pz} and the \ac{cg}, and 2 patients are considered
``healthy''.
An experienced radiologist segmented the prostate organ --- on
\ac{t2w}-\ac{mri}, \ac{dce}-\ac{mri}, and \ac{adc} --- as
well as the prostate zones --- i.e., \ac{pz} and \ac{cg} ---, and
\ac{cap} on the \ac{t2w}-\ac{mri}.
The full description and the data set are available at \acs*{iccvb}
website\footnote{\url{http://i2cvb.github.io/}}~\cite{Lemaitre2016thesis}.

\subsection{\acs*{cad} pipeline for \acs*{cap}}

Our \ac{mpmri} \ac{cad} system consists of 7 different steps:
pre-processing, segmentation, registration, feature detection, feature
balancing, feature selection/extraction, and finally classification.
%% It should be noted that \ac{cad} system designed deals with
%% multiparametric \ac{mri} data.

\subsubsection{Pre-processing}\label{subsec:chp6:method:PP}

Normalization is, a crucial step to reduce the inter-patient
variations which allows to improve the learning during the
classification stage.
However, the \ac{mri} modalities provide specific type of data --- static
\emph{vs.} dynamic information, images \emph{vs.} signals --- that
required a dedicated pre-processing.
Therefore, we pre-process differently the data:
\ac{t2w}-\ac{mri} is normalized using a Rician
a-priori that has been shown to be better than the traditional
$z$-score~\cite{lemaitre2016normalization}.
In contrast to \ac{t2w}-\ac{mri}, in \ac{adc} map the \ac{pdf} within the
prostate does not follow a known distribution and thus one cannot use
a parametric model to normalize these images and a non-parametric
piecewise-linear normalization~\cite{Nyul2000} is the best option for
this case.
\ac{dce}-\ac{mri} is a dynamic sequence and the data are normalized
based on a mean kinetic expression registration as proposed
in~\cite{Lemaitre2016thesis}.
Finally, the \ac{mrsi} modality has been pre-processed to correct the
phase, suppress the baseline, and align the frequencies~\cite{Parfait2012}.

\subsubsection{Segmentation and registration}\label{subsec:chp6:method:Seg-Reg}

For this work, our radiologist has manually segmented the prostate
organs on the different modalities.
However, the segmented prostate needs to be registered before to
extract features.
The \ac{t2w}-\ac{mri} is used as reference and each segmented prostate
in other modalities are registered to this reference.
Indeed, three registrations are used to correct: (i) the patient
motion during the \ac{dce}-\ac{mri} acquisition, (ii) the patient
motion between the \ac{t2w}-\ac{mri} and the \ac{dce}-\ac{mri}
acquisitions, and (iii) the patient motion between the
\ac{t2w}-\ac{mri} and the \ac{adc} map acquisition.
Additionally, volumes from all modalities have been interpolated to the
resolution of \ac{t2w}-\ac{mri}.

\subsubsection{Feature detection}\label{subsec:chp6:method:fea-det}
Similarly to the pre-processing, specific features are extracted
depending of the specificity of each \ac{mri} modality.
\begin{description}[style=unboxed,leftmargin=0cm]
\item[\ac{t2w}-\ac{mri} and \ac{adc} map features]
Additionally to the normalized intensity, edge- and texture-based
features are commonly extracted from \ac{t2w}-\ac{mri} and \ac{adc}
map.
The following set of filters characterizing edges have been used: (i)
Kirsch, (ii) Laplacian, (iii) Prewitt, (iv) Scharr, (v) Sobel, and
(vi) Gabor.
Except for the Kirsch filter, the other filters are applied in 3D,
taking advantage of the volume information instead of slice
information, as it is usually done.
Additionally, features based on phase congruency are
computed~\cite{kovesi1999image}.
To characterize the local texture, both second-order \ac{glcm}-based
features~\cite{Haralick1973} and rotation invariant and uniform
\ac{lbp}~\cite{ojala2002multiresolution} are extracted.
To encode 3D information, the 13 first Haralick features are computed
for the 13 possible directions.
For the same reason, the \ac{lbp} codes are computed for the
three-orthogonal-planes of each \ac{mri} volume.
All these features are extracted at each voxel of the volume.

\item[\ac{dce}-\ac{mri} features]
In brief, the entire enhanced signal, semi-quantitative~\cite{Huisman2001}, and
quantitative-based
models~\cite{brix1991pharmacokinetic,hoffmann1995pharmacokinetic,tofts1995quantitative,giannini2015fully}
are computed.

\item[\ac{mrsi} features]
Three different techniques are used to extract
discriminative features: (i) relative quantification based on
metabolite quantification, (ii) relative
quantification based on bounds integration, and (iii) spectra
extraction from \SIrange{2}{4}{\ppm}~\cite{Lemaitre2016thesis}.

\item[Anatomical features]
Four different metrics are computed based on the relative distance to the
prostate boundary as well as the prostate center, and the relative
position in the Euclidean and cylindrical coordinate
systems~\cite{Chen2002,Litjens2014}.

\end{description}

\subsubsection{Feature balancing}\label{subsec:chp6:method:fea-bal}
Imbalanced dataset is a common problem in medical imaging.
The number of cancerous voxels is much lower than the number of
``healthy'' voxels for a patient.
However, the problem of imbalanced dataset compromises the learning
process.
solving the problem of imbalanced is equivalent to under- or
over-sampling part of the dataset to obtain equal number of samples
in both classes.
We used several methods and selected the most efficient for our
dataset~\cite{imblearn}.

\subsubsection{Feature selection and extraction}\label{subsec:chp6:method:fea-sel}

Feature selection and extraction are used in our experiment.
\ac{mrsi} and \ac{dce}-\ac{mri} are decomposed using three feature
extraction methods: \ac{pca}, sparse-\ac{pca}, and \ac{ica} are used
to decompose signal-based data.
Additionally to feature extraction, two methods of feature selection
are used: (i) the one-way \ac{anova} and (ii) the Gini importance
obtained while learning the \ac{rf} classifiers.

\subsubsection{Classification}\label{subsec:chp6:method:clas}

\ac{rf} has been chosen as our base classifier --- allowing for
feature selection as well --- to perform classification of individual
modality as well as the combination of modalities.
Additionally, we use stacking to create ensemble of base learners
using a meta-classifier~\cite{wolpert1992stacked}, namely \ac{adb} and \ac{gb}.


\section{Experiments and results}\label{sec:chp6:exp-res}

In this section, different experiments are proposed to design and investigate our \ac{mpmri} \ac{cad} for the detection of \ac{cap}.
First, the classification performance of each independent modality is investigated in \acs{sec}\,\ref{subec:chp6:exp-res:Ex1}.
For each modality, the ``quantification'' approaches maximizing the classification performance are selected.
Additionally, we focus on to directly combined \ac{mpmri} modalities, which we referred to as ``coarse'' combination as presented in \acs*{sec}\,\ref{subsec:chp6:exp-res:Ex2}.
Subsequently, \acs{sec}\,\ref{subsec:chp6:exp-res:Ex3} presents the benefit of balancing the dataset on the learning stage and strategies for feature selection and extraction, for each feature modality as well as an aggregation of them.
Consequently, different combination classifier rules are studied using the previous fine-tuned feature space in \acs{sec}\,\ref{subsec:chp6:exp-res:Ex4}.
Finally, we conclude in \acs{sec}\,\ref{subsec:chp6:exp-res:Ex5} by investigating the benefit of fusing the \ac{mrsi} information with the other modality.

All these experiments are conducted on a subset of the public \ac{mpmri} prostate presented in \acs{sec}\,\ref{sec:data3t}.
We used the \SI{3}{\tesla} dataset which is composed of a total of 19 patients of which 17 patients had biopsy proven \ac{cap} and 2 patients are ``healthy'' with negative biopsies. 
In this study, our subset consists of 17 patients with \ac{cap}.

\begin{landscape}
\begin{figure}
  \hspace*{\fill}
  \subfigure[Performance of the quantitative methods on \acs*{dce}-\acs*{mri}.]{\label{fig:inddcemodel}\includegraphics[height=.4\textheight]{5_normalization/figures/DCE-normalization/normalized_methods_0.pdf}}
  \hfill
  \subfigure[Performance of enhanced \acs*{dce}-\acs*{mri} signal.]{\label{fig:inddcesignal}\includegraphics[height=.4\textheight]{5_normalization/figures/DCE-normalization/full_signal_0.pdf}}
  \hspace*{\fill} \\
  \hspace*{\fill}
  \subfigure[Performance of image-based features for \acs*{t2w}-\acs*{mri} and \acs*{adc} map.]{\label{fig:indadct2w}\includegraphics[height=.4\textheight]{6_pipeline/figures/exp-1/t2w_adc.pdf}}
  \hfill
  \subfigure[Performance of different approaches for the \acs*{mrsi} modality.]{\label{fig:indmrsi}\includegraphics[height=.4\textheight]{6_pipeline/figures/exp-1/mrsi_all.pdf}}
  \hspace*{\fill}
  \caption[Analysis of the classification performance for each individual \acs*{mri} modality.]{Analysis of the classification performance for each individual \acs*{mri} modality. Different models have been tested for \acs*{dce}-\acs*{mri} and \acs*{mrsi} modalities.}
  \label{fig:res-Ex1}
\end{figure}
\end{landscape}

\subsection{Assessment of classification performance of individual modality}\label{subec:chp6:exp-res:Ex1}

In this experiment, we attend to assess the classification performance of each individual \ac{mri} modality.

\paragraph{\ac{t2w}-\ac{mri} and \ac{adc} map features} All features presented in \acs{tab}~\ref{tab:featureadct2w} are extracted for both \ac{t2w}-\ac{mri} and \ac{adc} map.
These features are combined per modality and for each of them, a \ac{rf} classifier is trained.

\paragraph{\ac{dce}-\ac{mri} features} This experiment has been presented in \acs{sec}\,\ref{subsec:chp5:DCE-norm:exp-res}.
We aim at finding the most discriminative ``quantification'' method for \ac{dce}-\ac{mri} modality, by assessing the classification performance of the different models.
Therefore, the pharmacokinetic parameters from the Brix, Hoffmann, Tofts, and \ac{pun} models, the semi-quantitative parameters, and the enhanced \ac{dce}-\ac{mri} signal are extracted.
For each set of feature, a \ac{rf} classifier is trained.

\paragraph{\ac{mrsi} features} Similarly to \ac{dce}-\ac{mri}, 4 \ac{rf} classifiers are trained on different features:
(i) the cropped \ac{mrsi} signal,
(ii) the relative concentration of the citrate over the relative concentration of the choline, both computed through fitting as presented in the previous section,
(iii) the ratio of the two previous features, and finally
(iv) the ratio of the relative concentration of the citrate over the relative concentration of the choline, using fix integration bounds.

\paragraph{Results}
Each trained \ac{rf} is evaluated using a \ac{lopo}.
A \ac{roc} analysis is carried out and the \ac{auc} score is computed to report and compare the classification performance of each classifier.
The results are depicted in \acs{fig}\,\ref{fig:res-Ex1}.
As presented is the previous chapter, classification of \ac{dce}-\ac{mri} data using the normalized enhanced \ac{dce}-\ac{mri} signal is the strategy leading the highest \ac{auc} --- i.e., $0.666 \pm 0.154$ ---, outperforming any quantification method.
Similarly to these findings, classification of the cropped \ac{mrsi} signal outperforms other quantification-based methods, with an \ac{auc} of $0.697 \pm 0.165$.
Classification of the extracted features based on \ac{adc} offer a close performance with an identical mean \ac{auc} and a smaller standard deviation of $0.128$.
Finally, the features extracted from \ac{t2w}-\ac{mri} are shown to be the most discriminative with an \ac{auc} reaching $0.720 \pm 0.122$.
As a conclusion, the most efficient features in terms of classification performance for each modality are selected for the remainder of the experiment section.

\subsection{Coarse combination of \acs*{mpmri} modalities} \label{subsec:chp6:exp-res:Ex2}

\begin{figure}
  \centering
  \includegraphics[width=0.7\linewidth]{6_pipeline/figures/exp-2/comb_all.pdf}
  \caption[Comparison of different combination approaches.]{Comparison of different combination approaches: (i) aggregation of the different features in conjunction with a \acs*{rf} classifier, (ii) a stacking approach using 4 \acs*{rf}s and \acs*{adb} as meta-classifier, and (iii) a stacking approach using 4 \acs*{rf}s and \acs*{gb} as meta-classifier.}
  \label{fig:res-Exp2}
\end{figure}

As a first attempt to design a \ac{mpmri} \ac{cad} system, 3 different approaches are used to combine the selected feature from each modality:
(i) feature aggregation,
(ii) stacking using \ac{adb},
(iii) stacking using \ac{gb}.
We refer these combinations as being coarse since no tuning --- i.e., feature balancing/selection/extraction --- aiming at improving the classification performance is involved.
This experiment can be considered as the baseline to obtain a \ac{mpmri} \ac{cad} for the detection of \ac{cap}.

In the first approach, the features from all the different modalities are concatenated together to form a unique matrix.
Additionally, the anatomical features are concatenated within the same matrix.
The second and third approaches are based on the stacking which has been presented in the previous section.
They differ in the choice of the meta-learner since the first stack uses an \ac{adb} classifier while the second stack uses a \ac{gb}.
Each base learner is similar to the \ac{rf} selected in the previous experiment.
The difference lie in the concatenation of the anatomical features with each feature set derived from the \ac{mri} modality presented in the previous experiment.

\paragraph{Results}
The three coarse combinations are tested using a \ac{lopo}.
Furthermore, for the stacking approaches, the training set is split into a smaller training set and a validation set composed of 10 and 6 patients, respectively.
A \ac{roc} analysis is carried out for each combination and the \ac{auc} is computed as reported in \acs{fig}\,\ref{fig:res-Exp2}.

A single learner using aggregated features outperforms the stacking-based classifier with an \ac{auc} of $0.802 \pm 0.130$.
Furthermore, \ac{gb} chosen as a meta-classifier leads to better classification performance than \ac{adb}, with an improved \ac{auc} from $0.761 \pm 0.135$ to $0.769 \pm 0.128$.

\subsection{Benefits of data balancing and feature selection/extraction}\label{subsec:chp6:exp-res:Ex3}

In this section, we focus on optimizing the different feature set used in the previous classification.
Therefore, our contribution is twofold: (i) we compare the different balancing methods to distinguish which method is the best suited and (ii) we use feature selection/extraction methods to identify which feature are the most discriminative among each set.

\begin{figure}
  \centering
  \includegraphics[width=0.7\linewidth]{6_pipeline/figures/exp-3/all.pdf}
  \caption{Analysis of the benefit of balancing the training dataset before the learning process while concatenating all features.}
  \label{fig:allbalance}
\end{figure}

\paragraph{Comparison of balancing strategies}
For this experiment, a \ac{rf} classifier is trained for each feature set selected from the first experiment.
As in the previous experiment, a \ac{lopo} is used as validation model.
During the learning phase, the training sets are balanced using the methods presented in \acs{sec}\,\ref{subsec:chp6:method:fea-bal}.
The possible improvements offered by the balancing methods is analyzed through a \ac{roc} analysis and computing the \ac{auc}.
The results are depicted in \acs{fig}\,\ref{fig:res-Ex3-bal} and give rise to two observations:
(i) there is at least one balancing method which improves the classification performance and
(ii) \ac{iht} and \ac{smote} are the methods performing the best on individual modality features.
On the one hand, \ac{iht} outperforms the other methods while balancing the feature sets based on the \ac{dce}-\ac{mri} and \ac{adc} map.
The \ac{auc} increases of $0.019$ and $0.018$ for the feature sets of the \ac{dce}-\ac{mri} and \ac{adc} map, respectively.
On the other hand, \ac{smote} increases the \ac{auc} of $0.042$ for \ac{t2w}-\ac{mri}.
However, there is no significant improvement for the \ac{mrsi} since only the standard deviation of the \ac{auc} decreases of $0.019$.
Once all features are concatenated together, \ac{nm3} is the method providing the best enhancement of the classification performance with an \ac{auc} of $0.824 \pm 0.076$, as depicted in \acs{fig}\,\ref{fig:allbalance}.
In conclusion, the methods leading to the best performance are applied prior to feature selection/extraction for the remainder of the experiment.

\paragraph{Feature selection and extraction}

Noisy or non-discriminative features included in the learning process might degrade the overall performance of a classifier.
Thus, the feature selection and extraction methods presented in \acs{sec}\,\ref{subsec:chp6:method:fea-bal} are used to obtain a fine-tuned feature space.
The selection approaches --- i.e., \ac{anova} F-value and Gini importance --- are applied on the image-based features extracted from \ac{t2w}-\ac{mri} and \ac{adc} map modalities.
For both methods, a threshold defines the percentage of features to select.
Additionally, several thresholds are defined to find the number of features maximizing the classification performance.

Features computed from \ac{mrsi} and \ac{dce}-\ac{mri} modalities are related to signal and feature extraction seems more appropriate rather than feature selection.
Therefore, the 3 feature extraction methods --- i.e., \ac{pca}, sparse-\ac{pca}, and \ac{ica} --- are applied by varying the number of components or the sparsity level, which allows to find the level which maximizes the classification performance.
Finally, the feature selection methods have been applied on the concatenation of all the features.

\begin{landscape}
\begin{figure}
  \hspace*{\fill}
  \subfigure[\ac{t2w}-\ac{mri}]{\label{fig:ex3:T2W}\includegraphics[height=.4\textheight]{6_pipeline/figures/exp-3/t2w.pdf}}
  \hfill
  \subfigure[\ac{adc}-\ac{mri}]{\label{fig:ex3:ADC}\includegraphics[height=.4\textheight]{6_pipeline/figures/exp-3/adc.pdf}}
  \hspace*{\fill} \\
  \hspace*{\fill}
  \subfigure[\ac{dce}-\ac{mri}]{\label{fig:ex3-DCE}\includegraphics[height=.4\textheight]{6_pipeline/figures/exp-3/dce.pdf}}
  \hfill
  \subfigure[\ac{mrsi}-\ac{mri}]{\label{fig:ex3-MRSI}\includegraphics[height=.4\textheight]{6_pipeline/figures/exp-3/mrsi.pdf}}
  \hspace*{\fill}
  \caption{Analysis of the benefit of balancing the training dataset before learning process for each modality.}
  \label{fig:res-Ex3-bal}
\end{figure}

\begin{table}
  \caption{Results in terms of \acs*{auc} of the feature selection based on \acs*{anova} F-value for \acs*{t2w}-\acs*{mri}.}
  \centering
  \scriptsize
  \begin{tabularx}{\linewidth}{@{}l >{\centering\arraybackslash}X >{\centering\arraybackslash}X >{\centering\arraybackslash}X >{\centering\arraybackslash}X >{\centering\arraybackslash}X >{\centering\arraybackslash}X >{\centering\arraybackslash}X @{}}
    \toprule
    \multirow{2}{*}{\textbf{Methods}} & \multicolumn{7}{c}{\textbf{Percentiles}} \\
    \cmidrule{2-8}
    & 15 & 17.5 & 20 & 22.5 & 25 & 27.5 & 30 \\
    \midrule
    \acs*{anova} F-score & $0.755 \pm 0.049$ & $0.770 \pm 0.058$ & $0.777 \pm 0.064$ & $0.782 \pm 0.066$ & $\mathbf{0.784 \pm 0.067}$ & $0.783 \pm 0.072$ & $0.782 \pm 0.070$ \\
    \bottomrule
  \end{tabularx}
  \label{tab:ginit2w}
\end{table}

\begin{table}
  \caption{Results in terms of \acs*{auc} of the feature selection based on Gini importance for \acs*{t2w}-\acs*{mri}.}
  \centering
  \scriptsize
  \begin{tabularx}{\linewidth}{@{}l >{\centering\arraybackslash}X >{\centering\arraybackslash}X >{\centering\arraybackslash}X >{\centering\arraybackslash}X >{\centering\arraybackslash}X >{\centering\arraybackslash}X >{\centering\arraybackslash}X @{}}
    \toprule
    \multirow{2}{*}{\textbf{Methods}} & \multicolumn{7}{c}{\textbf{Percentiles}} \\
    \cmidrule{2-8}
    & 1 & 2 & 5 & 10 & 15 & 20 & 30 \\
    \midrule
    Gini importance & $0.726 \pm 0.064$ & $0.731 \pm 0.055$ & $0.751 \pm 0.065$ & $0.758 \pm 0.076$ & $0.752 \pm 0.087$ & $0.761 \pm 0.077$ & $\mathbf{0.764 \pm 0.079}$ \\
    \bottomrule
  \end{tabularx}
  \label{tab:anovat2w}
\end{table}

\begin{table}
  \caption{Results in terms of \acs*{auc} of the feature selection based on \acs*{anova} F-value for \acs*{adc}.}
  \centering
  \scriptsize
  \begin{tabularx}{\linewidth}{@{}l >{\centering\arraybackslash}X >{\centering\arraybackslash}X >{\centering\arraybackslash}X >{\centering\arraybackslash}X >{\centering\arraybackslash}X >{\centering\arraybackslash}X >{\centering\arraybackslash}X @{}}
    \toprule
    \multirow{2}{*}{\textbf{Methods}} & \multicolumn{7}{c}{\textbf{Percentiles}} \\
    \cmidrule{2-8}
    & 10 & 12.5 & 15 & 17.5 & 20 & 22.5 & 25 \\
    \midrule
    \acs*{anova} F-score & $0.684 \pm 0.123$ & $0.713 \pm 0.125$ & $0.712 \pm 0.134$ & $0.710 \pm 0.144$ & $\mathbf{0.714 \pm 0.142}$ & $0.708 \pm 0.150$ & $0.708 \pm 0.150$ \\
    \bottomrule
  \end{tabularx}
  \label{tab:giniadc}
\end{table}

\begin{table}
  \caption{Results in terms of \acs*{auc} of the feature selection based on Gini importance for \acs*{adc} map.}
  \centering
  \scriptsize
  \begin{tabularx}{\linewidth}{@{}l >{\centering\arraybackslash}X >{\centering\arraybackslash}X >{\centering\arraybackslash}X >{\centering\arraybackslash}X >{\centering\arraybackslash}X >{\centering\arraybackslash}X >{\centering\arraybackslash}X @{}}
    \toprule
    \multirow{2}{*}{\textbf{Methods}} & \multicolumn{7}{c}{\textbf{Percentiles}} \\
    \cmidrule{2-8}
    & 1 & 2 & 5 & 10 & 15 & 20 & 30 \\
    \midrule
    Gini importance & $0.672 \pm 0.132$ & $0.690 \pm 0.138$ & $\mathbf{0.743 \pm 0.139}$ & $0.730 \pm 0.136$ & $0.730 \pm 0.142$ & $0.724 \pm 0.141$ & $0.722 \pm 0.142$ \\
    \bottomrule
  \end{tabularx}
  \label{tab:anovaadc}
\end{table}

\begin{table}
  \caption{Results in terms of \acs*{auc} of the feature extraction methods for \acs*{dce}-\ac{mri}.}
  \centering
  \scriptsize
  \begin{tabularx}{\linewidth}{@{}l >{\centering\arraybackslash}X >{\centering\arraybackslash}X >{\centering\arraybackslash}X >{\centering\arraybackslash}X >{\centering\arraybackslash}X >{\centering\arraybackslash}X >{\centering\arraybackslash}X @{}}
    \toprule
    \multirow{2}{*}{\textbf{Methods}} & \multicolumn{7}{c}{\textbf{Number of components or sparsity level}} \\
    \cmidrule{2-8}
    & 2 & 4 & 8 & 16 & 24 & 32 & 36 \\
    \midrule
    \acs*{pca} & $0.656 \pm 0.133$ & $0.634 \pm 0.121$ & $0.668 \pm 0.149$ & $0.680 \pm 0.145$ & $0.682 \pm 0.146$ & $0.679 \pm 0.151$ & $0.683 \pm 0.149$ \\
    Sparse-\acs*{pca} & $0.578 \pm 0.117$ & $0.546 \pm 0.121$ & $0.554 \pm 0.097$ & --- & --- & --- & --- \\
    \acs*{ica} & $0.657 \pm 0.132$ & $0.629 \pm 0.117$ & $0.671 \pm 0.157$ & $0.686 \pm 0.158$ & $\mathbf{0.691 \pm 0.158}$ & $0.681 \pm 0.161$ & $0.679 \pm 0.166$ \\
    \bottomrule
  \end{tabularx}
  \label{tab:dcefeatext}
\end{table}


\begin{table}
  \caption{Results in terms of \acs*{auc} of the feature extraction methods for \acs*{mrsi}.}
  \centering
  \scriptsize
  \begin{tabularx}{\linewidth}{@{}l >{\centering\arraybackslash}X >{\centering\arraybackslash}X >{\centering\arraybackslash}X >{\centering\arraybackslash}X >{\centering\arraybackslash}X >{\centering\arraybackslash}X >{\centering\arraybackslash}X @{}}
    \toprule
    \multirow{2}{*}{\textbf{Methods}} & \multicolumn{7}{c}{\textbf{Number of components or sparsity level}} \\
    \cmidrule{2-8}
    & 2 & 4 & 8 & 16 & 24 & 32 & 36 \\
    \midrule
    \acs*{pca} & $0.566 \pm 0.120$ & $0.575 \pm 0.141$ & $0.648 \pm 0.162$ & $0.662 \pm 0.177$ & $0.659 \pm 0.184$ & $0.671 \pm 0.179$ & $0.672 \pm 0.182$ \\
    Sparse-\acs*{pca} & $0.502 \pm 0.050$ & $0.571 \pm 0.158$ & $0.585 \pm 0.111$ & --- & --- & --- & --- \\
    \acs*{ica} & $0.567 \pm 0.119$ & $0.578 \pm 0.140$ & $0.654 \pm 0.145$ & $0.656 \pm 0.167$ & $0.650 \pm 0.187$ & $0.663 \pm 0.174$ & $\mathbf{0.677 \pm 0.171}$ \\
    \bottomrule
  \end{tabularx}
  \label{tab:mrsifeatext}
\end{table}

\begin{table}
  \caption{Results in terms of \acs*{auc} of the feature selection based on \acs*{anova} F-value for the aggregation of feature from all \acs*{mpmri} features.}
  \centering
  \scriptsize
  \begin{tabularx}{\linewidth}{@{}l >{\centering\arraybackslash}X >{\centering\arraybackslash}X >{\centering\arraybackslash}X >{\centering\arraybackslash}X >{\centering\arraybackslash}X >{\centering\arraybackslash}X >{\centering\arraybackslash}X @{}}
    \toprule
    \multirow{2}{*}{\textbf{Methods}} & \multicolumn{7}{c}{\textbf{Percentiles}} \\
    \cmidrule{2-8}
    & 10 & 12.5 & 15 & 17.5 & 20 & 22.5 & 25 \\
    \midrule
    \acs*{anova} F-score & $0.764 \pm 0.095$ & $0.765 \pm 0.079$ & $0.800 \pm 0.083$ & $0.817 \pm 0.089$ & $\mathbf{0.828 \pm 0.084}$ & $0.822 \pm 0.0.084$ & $0.815 \pm 0.086$ \\
    \bottomrule
  \end{tabularx}
  \label{tab:anovacomb}
\end{table}

\begin{table}
  \caption{Results in terms of \acs*{auc} of the feature selection based on Gini importance for the aggregation of feature from all \acs*{mpmri} features.}
  \centering
  \scriptsize
  \begin{tabularx}{\linewidth}{@{}l >{\centering\arraybackslash}X >{\centering\arraybackslash}X >{\centering\arraybackslash}X >{\centering\arraybackslash}X >{\centering\arraybackslash}X >{\centering\arraybackslash}X >{\centering\arraybackslash}X @{}}
    \toprule
    \multirow{2}{*}{\textbf{Methods}} & \multicolumn{7}{c}{\textbf{Percentiles}} \\
    \cmidrule{2-8}
    & 10 & 12.5 & 15 & 17.5 & 20 & 22.5 & 25 \\
    \midrule
    Gini importance & $0.834 \pm 0.085$ & $0.834 \pm 0.088$ & $0.834 \pm 0.084$ & $\mathbf{0.836 \pm 0.083}$ & $0.834 \pm 0.079$ & $0.828 \pm 0.086$ & $0.830 \pm 0.077$ \\
    \bottomrule
  \end{tabularx}
  \label{tab:ginicomb}
\end{table}

\begin{table}
  \caption{Selected feature and number of occurrence for \acs*{t2w}-\acs*{mri}, \acs*{adc} map, and one all the features are concatenated.}
  \centering
  \scriptsize
  \begin{tabular}{llllll}
    \toprule
    \multicolumn{1}{c}{\textbf{\acs*{t2w}-\acs*{mri}}} & \multicolumn{1}{c}{\textbf{\acs*{adc}}} & \multicolumn{1}{c}{\textbf{\acs*{t2w}-\acs*{mri}}} & \multicolumn{1}{c}{\textbf{\acs*{adc}}} & \multicolumn{1}{c}{\textbf{\acs*{dce}-\acs*{mri}}} & \multicolumn{1}{c}{\textbf{\acs*{mrsi}}} \\
    \cmidrule(lr){1-1} \cmidrule(lr){2-2} \cmidrule(lr){3-6}
    8 edges & 1 \acs*{dct} & 113 Gabor filters & 53 Gabor filters & 14 samples  & 78 samples \\
    155 Gabor filters & 32 Gabor filters & 1 phase congruency & 2 phase congruency & & \\ 
    2 Haralick features & 1 phase congruency & 4 edges & & & \\
    1 intensity & & 1 intensity & & & \\
    4 \acs*{lbp} & & & & & \\
    2 phase congruency & & & & & \\
    \cmidrule(lr){1-1} \cmidrule(lr){2-2} \cmidrule(lr){3-6}
    \multicolumn{1}{c}{\textbf{172 features}} & \multicolumn{1}{c}{\textbf{34 features}} & \multicolumn{4}{c}{\textbf{267 features}} \\
    \bottomrule
  \end{tabular}
  \label{tab:selfeatocc}
\end{table}

\end{landscape}

As the previous experiments, the classification is performed using a \ac{rf} with \ac{lopo} model validation.
A \ac{roc} analysis is performed and for each \ac{roc}, the \ac{auc} score is computed.
The results are reported from \acs{tab}~\ref{tab:ginit2w} to \acs{tab}~\ref{tab:anovacomb}, in which the best results are highlighted in \textbf{bold}.

% feature selection
% results
Overall, feature selection or extraction lead to increase the classification performance.
For features extracted from the \ac{t2w}\ac{mri}, \ac{anova}-based selection lead to better performance than Gini importance selection, with a \ac{auc} of $0.784 \pm 0.067$.
The opposite conclusion is drawn for the features extracted from the \ac{adc} map.
The selection using the Gini importance criterion leads to an \ac{auc} of $0.743 \pm 0.139$.
% improvements
Therefore, the feature selection leads to an improve \ac{auc} of $0.022$ and $0.013$ for \ac{t2w}-\ac{mri} and \ac{adc} map, respectively.
% which features are selected
The features which have been selected are reported in the 1\textsuperscript{st} and 2\textsuperscript{nd} columns of \acs{tab}~\ref{tab:selfeatocc}.
To conclude, from the 690 original features, 172 and 43 features are selected from the \ac{t2w}-\ac{mri} and \ac{adc} map, respectively.

% feature extraction
Regarding feature extraction, \ac{ica} outperforms other methods for both \ac{mrsi} and \ac{dce}-\ac{mri} with \ac{auc} scores of $0.677 \pm 0.171$ and $0.691 \pm 0.158$, respectively.
However, only the projection applied to \ac{dce}-\ac{mri} features leads to improved results with a gain of $0.013$ with 24 components selected instead of the original 40 dimensions.

% feature selection for concatenation
% results
Gini importance selection method is also outperforming \ac{anova}-based method while selecting the features from the concatenation of all of them.
The reported \ac{auc} is $0.836 \pm 0.083$ with an increase of $0.034$.
% which features are selected
The features which have been selected are reported from the 3\textsuperscript{rd} to the 6\textsuperscript{th} columns of \acs{tab}~\ref{tab:selfeatocc}.
To conclude, from the 1533 original features, 267 features are selected from the entire set of feature.

\subsection{Fine-tuned combination of \ac{mpmri} modalities}\label{subsec:chp6:exp-res:Ex4}

\begin{figure}
  \centering
  \includegraphics[width=0.7\linewidth]{6_pipeline/figures/exp-5/combine_all.pdf}
  \caption[Analysis of feature combination approaches after fine tuning.]{Analysis of feature combination approaches after fine tuning through balancing and feature selection/extraction.}
  \label{fig:res-Ex4}
\end{figure}

\begin{figure}
  \centering
  \includegraphics[width=0.7\linewidth]{6_pipeline/figures/exp-5/plot_all_patients.pdf}
  \caption{Individual patient \acs*{auc} for the best configuration of the \acs*{mpmri} \acs*{cad}.}
  \label{fig:indauc}
\end{figure}

This experiment aims at providing the most efficient \ac{mpmri} \ac{cad} for \ac{cap} using the fine-tuned feature space from the previous experiment.
Three strategies are applied:
(i) the selected features from each modality --- i.e., 331 features --- are concatenated together and used in a \ac{rf} classifier,
(ii) the selected features from each modality --- i.e., 331 features --- are used to train a stacking classifier with a \ac{gb} as meta-classifier, and
(iii) the selected features from the concatenated set of feature --- i.e., 267 features --- are used to train a single \ac{rf} classifier.

\begin{landscape}
\begin{figure}
  \hspace*{\fill}
  \subfigure[\acs*{auc} = 0.922]{\label{fig:pat634}\includegraphics[width=.45\textwidth]{6_pipeline/figures/examples/patient_634.pdf}}
  \hfill
  \subfigure[\acs*{auc} = 0.942]{\label{fig:pat778}\includegraphics[width=.45\textwidth]{6_pipeline/figures/examples/patient_778.pdf}}
  \hfill
  \subfigure[\acs*{auc} = 0.914]{\label{fig:pat1036}\includegraphics[width=.45\textwidth]{6_pipeline/figures/examples/patient_1036.pdf}}
  \hspace*{\fill}\\
  \hspace*{\fill}
  \subfigure[\acs*{auc} = 0.692]{\label{fig:pat634}\includegraphics[width=.45\textwidth]{6_pipeline/figures/examples/patient_410.pdf}}
  \hfill
  \subfigure[\acs*{auc} = 0.879]{\label{fig:pat778}\includegraphics[width=.45\textwidth]{6_pipeline/figures/examples/patient_784.pdf}}
  \hfill
  \subfigure[\acs*{auc} = 0.735]{\label{fig:pat1036}\includegraphics[width=.45\textwidth]{6_pipeline/figures/examples/patient_1041.pdf}}
  \hspace*{\fill}
  \caption[Illustration the resulting detection of our \acs*{mpmri} \acs*{cad} for \acs*{cap} detection.]{Illustration the resulting detection of our \acs*{mpmri} \acs*{cad} for \acs*{cap} detection. The blue contours corresponds to the \ac{cap} while the \texttt{jet} overlay represents the probability.}
  \label{fig:resultcad}
\end{figure}
\end{landscape}

As previously done, the experiment is performed in a \ac{lopo} fashion and a \ac{roc} analysis is carried out.
The comparative results are shown in \acs{fig}\,\ref{fig:res-Ex4}.
In overall, classification using the fine-tuned features improve the classification performance.
The third classification configuration is, however, the one which outperforms others with an \ac{auc} of $0.836 \pm 0.083$.
The improvement in terms of \ac{auc} is of $0.028$ and $0.050$ compared with the 1\textsuperscript{st} and 2\textsuperscript{nd}, respectively.

In clinical setting, the \ac{auc} score is categorized in 3 levels: (i) ``acceptable'' discrimination for an \ac{auc} ranging from $0.7$ to $0.8$, (ii) ``excellent'' discrimination for an \ac{auc} ranging from $0.8$ to $0.9$, and ``outstanding'' discrimination when the \ac{auc} is over $0.9$~\cite{hosmer2004applied}.
Therefore, the combination of all \ac{mri} modalities in conjunction with fine-tuning allow to upgrade our \ac{cad} system from an ``acceptable'' to an ``excellent'' discrimination level.

Additionally, the individual \ac{roc} analysis for each patient for the best configuration is shown in \acs{fig}\,\ref{fig:indauc}.
It can be noted that 12 patients have an \ac{auc} superior to $0.800$ and 2 patients have a rather low \ac{auc} below $0.700$.
Regarding the 4 patients with an \ac{auc} below $0.800$, 3 patients have a \ac{cap} localized in the \ac{cg}.

To illustrate qualitatively the results of our \ac{mpmri} \ac{cad} system, 6 diverse examples are presented in \acs{fig}\,\ref{fig:resultcad} by overlapping the probability map of having a \ac{cap} with the original \ac{t2w}-\ac{mri} slice.

\subsection{Benefit of the \acs*{mrsi} modality}\label{subsec:chp6:exp-res:Ex5}

\begin{figure}
  \centering
  \includegraphics[width=0.7\linewidth]{6_pipeline/figures/exp-6/stacking_wt_mrsi.pdf}
  \caption{Illustration of the gain of including the \acs*{mrsi} modality in a \acs*{mpmri} \acs*{cad}.}
  \label{fig:resmrsigain}
\end{figure}

We recall that the goal of this thesis is to use all the \ac{mpmri} modalities.
In this regard, \ac{mrsi} has nearly never been used together with the other modalities --- i.e., \ac{t2w}-\ac{mri}, \ac{dce}-\ac{mri}, and \ac{adc} map --- apart of the recent work of \citeauthor{trigui2017automatic}~\cite{trigui2016classification,trigui2017automatic}.
Therefore, we propose in this experiment to compare the classification performance by removing the \ac{mrsi} features.
In this regard, we propose to train 2 stacking classifiers --- with a \ac{gb} as meta-classifier --- while removing the feature related to \ac{mrsi} for one of them.
The same \ac{lopo} validation model is used as before and the results obtained from \ac{roc} analysis are depicted in \acs{fig}\,\ref{fig:resmrsigain}.

Therefore, including \ac{mrsi} into the classification pipeline increases the \ac{auc} from $0.756 \pm 0.092$ to $0.786 \pm 0.098$ for a gain of $0.030$.


\section{Conclusion}

In this paper, we presented one of the the first \ac{cad} system using all
the \ac{mpmri} modalities for prostate cancer detection.
Indeed, \ac{mrsi} has nearly never been used together with the other
modalities.
With an extensive validation approach to select the best
features, the best balancing strategy as well as the best classifier,
we obtained results on a rather complicated dataset of 17 patients
with an average \ac{auc} of $0.836 \pm 0.083$ which put our system in the
state-of-the-art, even so different \ac{cad}s were tested on different
datasets.

\acresetall
\chapter{Conclusion}\label{chap:7}

In this section, we summarize the work presented along this thesis as well the final contributions.
We also a give short discussion regarding the potential avenues for future research.

\section{Summary of the thesis}

This manuscript begins by introducing problematic and societal challenges related to \ac{cap} which give rise to the research motivation of this work.
An overview is given in \acs{chp}\,\ref{chap:2} regarding the \ac{mri} techniques currently used for medical screening.
Therefore, in \acs{chp}\,\ref{chap:3} we focus on reviewing the current state-of-the-art on mono- and \ac{mpmri} \ac{cad} systems for the detection of \ac{cap}.
As a conclusion, the target of this thesis has been fixed to design and investigate a new \ac{mpmri} \ac{cad} system based on all \ac{mri} modalities currently in used in clinical settings.
Additionally, we point out several missing pieces from the given puzzle: (i) no \ac{mpmri} dataset nor \ac{cad} system are currently publicly available, (ii) the knowledge about pre-processing methods in the current \ac{cad} systems is limited, and (iii) the problem of data balancing never has been explored in the past.
Subsequently, we present in \acs{chp}\,\ref{chap:4} the contours of our working materials --- i.e., \ac{mpmri} dataset, source code, communication website --- which we made publicly available along this thesis.
Finally, \acs{chp}\,\ref{chap:5} and \acs{chp}\,\ref{chap:6} present the technical investigations regarding pre-processing and our \ac{mpmri} \ac{cad} system for the detection of \ac{cap}.

\section{Contributions}

The major contributions of this thesis can be summarized as:

\begin{description}
\item[Public \ac{mpmri} dataset] Together with clinicians, we collected, annotated, and publicly made available the first \ac{mpmri} dataset for the detection of \ac{cap}.
\item[Normalization methods] We proposed and extensively investigated two normalization methods for both \ac{t2w}-\ac{mri} and \ac{dce}-\ac{mri} modalities.
\item[\Ac{mpmri} \ac{cad} for \ac{cap} detection] We proposed and extensively investigated a new \ac{mpmri} \ac{cad} for \ac{cap} detection.
This \ac{cad} system uses all current \ac{mri} modalities currently in use in clinical settings.
\end{description}

\section{Avenues for future research}

Although the proposed \ac{mpmri} \ac{cad} system provides satisfactory results, drastic improvements are needed to the system to become in use in clinical environment.
The current \ac{cad} system is taking decision at the voxel level.
Spatial information is, however, an important factor to be included which suggests to move from a voxel-based system to a super-voxel based system.
In \acs{chp}\,\ref{chap:6}, we identified strong image features --- i.e., Gabor filters and phase congruency features --- which should be further investigated.
In the context of the search for large number of quantitative features~\cite{lambin2012radiomics} --- a.k.a. \emph{radiomics} ---, these two types of features have the merit to be further investigated.
Additionally, other strategies to avoid hand-crafted feature detection have to be explored such as deep convolutional neural-networks.
However, the challenge given by the limited number of cases as in many medical applications stresses the need for transfer learning while applying deep-learning~\cite{shin2016deep}.
Subsequently, collecting additional \ac{mpmri} cases would be beneficial to move towards unsupervised learning.


\include{8_appendices/appendices}        % description of lab methods




% --------------------------------------------------------------
%:                  BACK MATTER: appendices, refs,..
% --------------------------------------------------------------

% the back matter: appendix and references close the thesis


%: ----------------------- bibliography ------------------------

% The section below defines how references are listed and formatted
% The default below is 2 columns, small font, complete author names.
% Entries are also linked back to the page number in the text and to external URL if provided in the BibTex file.

% PhDbiblio-url2 = names small caps, title bold & hyperlinked, link to page 
%\begin{multicols}{2} % \begin{multicols}{ # columns}[ header text][ space]
%\begin{tiny} % tiny(5) < scriptsize(7) < footnotesize(8) < small (9)

%\bibliographystyle{Latex/Classes/PhDbiblio-url2} % Title is link if provided
\bibliographystyle{abbrvnat} % calls style file plainnat.bst

\renewcommand{\bibname}{References} % changes the header; default: Bibliography

%\bibliography{9_backmatter/references} % adjust this to fit your BibTex file
\bibliography{9_backmatter/literature_review_2}

%\end{tiny}
%\end{multicols}

% --------------------------------------------------------------
% Various bibliography styles exit. Replace above style as desired.

% in-text refs: (1) (1; 2)
% ref list: alphabetical; author(s) in small caps; initials last name; page(s)
%\bibliographystyle{Latex/Classes/PhDbiblio-case} % title forced lower case
%\bibliographystyle{Latex/Classes/PhDbiblio-bold} % title as in bibtex but bold
%\bibliographystyle{Latex/Classes/PhDbiblio-url} % bold + www link if provided

%\bibliographystyle{Latex/Classes/jmb} % calls style file jmb.bst
% in-text refs: author (year) without brackets
% ref list: alphabetical; author(s) in normal font; last name, initials; page(s)

%\bibliographystyle{plainnat} % calls style file plainnat.bst
% in-text refs: author (year) without brackets
% (this works with package natbib)


% --------------------------------------------------------------

% according to Dresden med fac summary has to be at the end
%
% Thesis Abstract -----------------------------------------------------


%\begin{abstractslong}    %uncommenting this line, gives a different abstract heading
\begin{abstracts}        %this creates the heading for the abstract page
Prostate cancer (CaP) is the second most diagnosed cancer in men all over the world.
CaP growth is characterized by two main types of evolution: (i) the slow-growing tumours progress slowly and usually remain confined to the prostate gland; (ii) the fast-growing tumours metastasize from prostate gland to other organs, which might lead to incurable diseases.
Therefore, early diagnosis and risk assessment play major roles in patient treatment and follow-up.
In the last decades, new imaging techniques based on Magnetic Resonance Imaging (MRI) have been developed improving diagnosis.
In practise, diagnosis can be affected by multiple factors such as observer variability and visibility and complexity of the lesions.
In this regard, computer-aided detection and computer-aided diagnosis systems are being designed to help radiologists in their clinical practice.

Our research extensively analyzes the current state-of-the-art in the development of computer-aided diagnosis and detection systems for prostate cancer detection.
Currently, no computer-aided system using all available MRI modalities has been proposed and tested on a common dataset.
Therefore, we propose a new computer-aided system taking advantage of all MRI modalities (i.e., \acs*{t2w}-\acs*{mri}, \acs*{dce}-\acs*{mri}, DW-\acs*{mri}, \acs*{mrsi}).
Particular attention is paid to the normalization of the \acs*{mri} modalities prior to develop our computer-aided system.
This system has been extensively tested on a dataset which has been made publicly available.
\end{abstracts}
%\end{abstractlongs}
%-------------------------------------------------------------------------

\begin{abstractCatalan}

El c\`ancer de pr\`ostata (CaP) \'es el segon c\`ancer m\'es diagnosticat en homes a tot el m\'on.
El creixement del CaP es caracteritza per dos tipus principals d'evoluci\'o: (i) els tumors de creixement lent que progressen lentament i en general romanen confinats en la gl\`andula de la pr\`ostata; (ii) els tumors de creixement r\`apid que desenvolupen met\`astasi de la pr\`ostata a altres \`organs, el que podria conduir a malalties incurables.
Conseq\"uentment, el diagn\`ostic preco\c{c} i l'avaluaci\'o del risc exerceixen un paper important en el tractament del pacient i el seguiment.
En les \'ultimes d\`ecades s'han desenvolupat noves t\`ecniques d'imatge basades en imatge de resson\`ancia magn\`etica (RM, o MRI de l'angl\`es) per millorar el diagn\`ostic.
A la pr\`actica, el diagn\`ostic pot ser afectat per diversos factors com ara la variabilitat de l'observador i la visibilitat i la complexitat de les lesions.
En aquest sentit, s'estan desenvolupant sistemes per a l'ajuda a la detecci\'o i diagn\`ostic per ordinador per ajudar els radi\`olegs en la seva pr\`actica cl\'inica.

La nostra recerca analitza \`ampliament l'estat de l'art en el desenvolupament de sistemes per a l'ajuda a la detecci\'o i diagn\`ostic per ordinador per a la detecci\'o del c\`ancer de pr\`ostata.
En l'actualitat, no hi ha cap sistema d'ajuda al diagn\`ostic que utilitzi totes les modalitats de MRI disponibles i que hagi estat avaluat en un conjunt de dades com\'u.
Per tant, proposem un nou sistema d'ajuda al diagn\`ostic per ordinador aprofitant totes les modalitats de resson\`ancia magn\`etica (\'es a dir \acs*{t2w}-MRI, DCE-MRI, DW-MRI, MRSI).
Com a etapa pr\`evia al desenvolupament del sistema, es presta especial atenci\'o a la normalitzaci\'o de les modalitats de resson\`ancia magn\`etica.
El sistema desenvolupat ha estat avaluat extensivament en un conjunt de dades que s'han posat a disposici\'o p\'ublica.
 
\end{abstractCatalan}

% ---------------------------------------------------------------------- 
\begin{abstractSpanish}

El c\'ancer de pr\'ostata (CaP) es el segundo c\'ancer m\'as diagn\'osticado en hombres en todo el mundo.
El crecimiento del CaP se caracteriza por dos tipos principales de evoluci\'on: (i) los tumores de crecimiento lento que progresan lentamente y por lo general permanecen confinados en la gl\'andula de la pr\'ostata; (ii) los tumores de crecimiento r\'apido que desarrollan met\'astasis de la pr\'ostata a otros \'organos, lo que podr\'ia conducir a enfermedades incurables.
Consecuentemente, el diagn\'ostico precoz y la evaluaci\'on del riesgo desempe\~nan un papel importante en el tratamiento del paciente y el seguimiento. En las \'ultimas d\'ecadas se han desarrollado  nuevas t\'ecnicas de imagen basadas en imagen de resonancia magn\'etica (RM, o MRI del ingl\'es) para mejorar el diagn\'ostico. En la pr\'actica, el diagn\'ostico puede ser afectado por varios factores tales como la variabilidad del observador y la visibilidad y la complejidad de las lesiones.
En este sentido, se est\'an desarrollando sistemas para la ayuda a la detecci\'on y diagn\'ostico por ordenador para ayudar a los radi\'ologos en su pr\'actica cl\'inica.

Nuestra investigaci\'on analiza ampliamente el estado del arte en el desarrollo de sistemas para la ayuda a la detecci\'on y diagn\'ostico por ordenador para la detecci\'on del c\`ancer de pr\'ostata.
En la actualidad, no existe ning\'un sistema de ayuda al diagn\'ostico que utilice todas las modalidades de MRI disponibles y que haya sido evaluado en un conjunto de datos com\'un.
Por lo tanto, proponemos un nuevo sistema de ayuda al diagn\'ostico por ordenador aprovechando todas las modalidades de resonancia magn\'etica (es decir T2W-MRI, DCE-MRI, DW-MRI, MRSI).
Como etapa previa al desarrollo del sistema, se presta especial atenci\'on a la normalizaci\'on de las modalidades de resonancia magn\'etica.
El sistema desarrollado ha sido evaluado extensivamente en un conjunto de datos que se han puesto a disposici\'on p\'ublica.
 
\end{abstractSpanish}

%-------------------------------------------------------------------------
\begin{abstractFrench}

Le cancer de la prostate est le second type de cancer le plus diagnostiqu\'e au monde.
Il est caract\'eris\'e par deux evolutions distinctes : (i) les tumeurs \`a croissances lentes progressent lentement et restent g\'en\'eralement confin\'ees dans la glande prostatique; (ii) les tumeurs \`a croissances rapides se m\'etastasent de la prostate \`a d'autres organes p\'eriph\'eriques, pouvant causer le d\'evelopement de maladies incurables.
C'est pour cela qu'un diagnostic pr\'ecoce et une \'evaluation du risque jouent des r\^oles majeurs dans le traitement et le suivi du patient.
Durant la derni\`ere d\'ec\'enie, de nouvelles m\'ethodes d'imagerie bas\'ees sur l'Imagerie par R\'esonance Magn\'etique (IRM) ont \'et\'e d\'evelop\'ees.
En pratique, le diagnostic clinique peut \^etre affect\'e par de multiples facteurs comme la variabilit\'e entre observateurs et la complexit\'e des l\'esions lues.
Pour ce faire, des syst\`emes de d\'etection et de diagnostic assist\'e par ordinateur (DAO) ont \'et\'e d\'evelop\'es pour aider les radiologistes durant leurs t\^aches cliniques.

Notre recherche analyse extensivement l'\'etat de l'art actuel concernant le d\'evelopement des syst\`emes de DAO pour la d\'etection du cancer de la prostate.
Actuellement, il n'\'existe aucun syst\`eme de DAO utilisant toutes les modalit\'es IRM disponibles et qui plus est, test\'e sur une base de donn\'ees commune.
Par cons\'equent, nous proposons un nouveau syst\`eme de DAO tirant profit de toutes les modalit\'es IRM (i.e., T2W-MRI, DCE-MRI, DW-MRI, MRSI).
Une attention particuli\`ere est port\'ee sur la normalisation de ces donn\'ees multi-param\'etriques avant la conception du syst\`eme de DAO.
De plus, notre syst\`eme de DAO a \'et\'e test\'e sur une base de donn\'ees que nous rendons publique.

\end{abstractFrench}


%: Declaration of originality
% \include{9_backmatter/declaration}



\end{document}
