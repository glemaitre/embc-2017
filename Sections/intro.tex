\section{Introduction}
Current \ac{cap} screening consists of 3 different stages.
First, \ac{psa} control is performed to distinguish between low- and
high-risk \ac{cap}.
To assert such diagnosis, samples are taken during prostate biopsy and
analyzed to make an accurate prognosis of the \ac{cap}.

Although \ac{psa} screening has been shown to improve early detection
of \ac{cap}~\cite{Chou2011}, its lack of reliability motivates further
investigations using \ac{mri}-based \ac{cad}.
Consequently, current research is focused on identifying new
biological markers to replace \ac{psa}-based
screening~\cite{Brenner2013}.
Until such research comes to fruition, these needs can be met through
active-surveillance strategy using \ac{mpmri}
techniques~\cite{Moore2013}.

%A general \ac{cad} work-flow is presented in \acs{fig}\,\ref{fig:wkfcad}.
%The \ac{cad} work-flow is presented in \acs{fig}~\ref{fig:wkfcad}.
Lemaitre\,\emph{et~al.} recently reviewed
more than 50 research works that focused on \ac{cad} system for
\ac{cap}~\cite{Lemaitre2015}.
These studies are based on \ac{cad} systems that consists of the
following steps:
(i) pre-processing,
(ii) segmentation,
(iii) registration,
(iv) feature detection,
%(v) feature balancing,
(v) feature selection-extraction, and
(vi) finally classification.

The reviewed \ac{mpmri}-based \ac{cad} used 2 to 3
\ac{mri} modalities among \ac{t2w}-\ac{mri}, \ac{dce}-\ac{mri}, and
\ac{dw}-\ac{mri}, discarding the potential discriminative power of
\ac{mrsi}.
Furthermore, only half of these studies tackled the challenging
detection of \ac{cap} in the \ac{cg}.
Additionally, none of the works investigated the issue related to
feature balancing when developing their \ac{cad} systems.
Finally, none of the datasets nor source codes used have been
released, making impossible the possibilities to compare the methods.

%In \iac{cade} framework, \textit{possible lesions are segmented
%automatically} and further used as input of \iac{cadx}.
%Nevertheless, some works also used a fusion of \ac{cade}-\ac{cadx}
%framework in which a voxel-based features are directly used, in which
%the location of the malignant lesions are obtained as results.
%On the other hand, manual lesions segmentation is not considered to
%be part of \ac{cade}.
%The output of the \ac{cade} is used as input of the \ac{cadx}.

%$\Ac{cadx} is composed of the processes allowing to
%\textit{distinguish malignant from non-malignant tumours}.
%Here, \ac{cap} malignancy is defined using the grade of the \ac{gs}
%determined after post biopsy or prostatectomy.
%As presented in \ac{fig}\,\ref{fig:wkfcad}, \ac{cadx} is usually
%composed of the three common steps used in a classification
%framework: (i) features detection, (ii) feature extraction/selection,
%and (iii) feature classification.

In this work, we propose a \ac{cad} system to detect \ac{cap} in
\ac{pz} and \ac{cg}, using the 4 aformentioned \ac{mri}
modalities.
In addition, our framework include a step of feature balancing.
The dataset used and the source code developed are released for future
comparisons and reproducibility.
