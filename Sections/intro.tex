\section{Introduction}
Current \ac{cap} screening consists of three different stages.
First, \ac{psa} control is performed to distinguish between low- and
high-risk \ac{cap}.
To assert such diagnosis, samples are taken during prostate biopsy and
finally analyzed to evaluate the prognosis and the stage of \ac{cap}.

Although \ac{psa} screening has been shown to improve early detection
of \ac{cap}~\cite{Chou2011}, its lack of reliability motivates further
investigations using \ac{mri}-based \ac{cad}.
Current research is focused on identifying new biological markers to
replace \ac{psa}-based
screening~\cite{Bourdoumis2010,Morgan2011,Brenner2013}.
Until such research comes to fruition, these needs can be met through
active-surveillance strategy using \ac{mpmri}
techniques~\cite{Hoeks2011,Moore2013}.

%A general \ac{cad} work-flow is presented in \acs{fig}\,\ref{fig:wkfcad}.
%The \ac{cad} work-flow is presented in \acs{fig}~\ref{fig:wkfcad}.
In a recent paper, Lemaitre et al. ~\cite{Lemaitre2015} reviewed
around fifty papers recently published and focusing on CAD system for
CaP. Most of the work reviewed could fit on the pipeline presented in
~\cite{Lemaitre2016thesis}which consists of  different steps:
pre-processing, segmentation, registration, feature detection, feature
balancing, feature selection/extraction, and finally classification.
However, non of the reviewed work, except thisone used all the
\ac{mri} modalities (\ac{t2w}, \ac{dce},\ac{dw} and \ac{mrsi}) and
only half of the studies were related to the central gland of the
prostate.


%In \iac{cade} framework, \textit{possible lesions are segmented
%automatically} and further used as input of \iac{cadx}.
%Nevertheless, some works also used a fusion of \ac{cade}-\ac{cadx}
%framework in which a voxel-based features are directly used, in which
%the location of the malignant lesions are obtained as results.
%On the other hand, manual lesions segmentation is not considered to be part of \ac{cade}.
%The output of the \ac{cade} is used as input of the \ac{cadx}.

%$\Ac{cadx} is composed of the processes allowing to
%\textit{distinguish malignant from non-malignant tumours}.
%Here, \ac{cap} malignancy is defined using the grade of the \ac{gs}
%determined after post biopsy or prostatectomy.
%As presented in \ac{fig}\,\ref{fig:wkfcad}, \ac{cadx} is usually
%composed of the three common steps used in a classification
%framework: (i) features detection, (ii) feature extraction/selection,
%and (iii) feature classification.
