%\section{\acs*{mri} imaging techniques}\label{sec:chp2:imaging}

\Ac{mri} provides promising imaging techniques to overcome the drawbacks of current clinical screening techniques mentioned in \acs{sec}\,\ref{chap:1}.
Unlike \ac{trus} biopsy, \ac{mri} examination is a non-invasive protocol and has been shown to be the most accurate and harmless technique currently available~\cite{Turkbey2012}.
In this section, we review different \ac{mri} imaging techniques developed for \ac{cap} detection and diagnosis.
Features strengthening each modality will receive particular attention together with their drawbacks.
Commonly, these features form the basis for developing analytic tools and automatic algorithms.
However, we refer the reader to \acs{sec}\,\ref{subsec:chp3:img-clas:CADX-fea-dec} for more details on automatic feature detection methods since they are part and parcel of the \acs{cad} framework.

%% % We are using enumerate with a small margin and some indent to organize our thoughts by paragraphs.
%% \setenumerate{listparindent=\parindent,itemsep=10px}
%% \setlist{noitemsep}

\begin{figure}
\centering
	\hspace*{\fill}
	\subfigure[\acs*{t2w}-\acs*{mri} slice of a healthy prostate acquire with a \SI{1.5}{\tesla} \acs*{mri} with an endorectal coil. The blue contour represents the \acs*{cg} while the \acs*{pz} corresponds to the green contour.]{\label{subfig:t2whealthy}\includegraphics[width=0.3\linewidth]{2_modality/figures/t2w/t2w_healthy.eps}} \hfill
	\subfigure[\acs*{t2w}-\acs*{mri} slice of a prostate with a \acs*{cap} highlighted in the \acs*{pz} using a \SI{3}{\tesla} \acs*{mri} scanner without an endorectal coil.]{\label{subfig:t2wcancerpz}\includegraphics[width=0.3\linewidth]{2_modality/figures/t2w/t2w_cancer_pz.eps}} \hfill
	\subfigure[\acs*{t2w}-\acs*{mri} slice of a prostate with a \acs*{cap} highlighted in the \acs*{cg} using a \SI{1.5}{\tesla} \acs*{mri} scanner with an endorectal coil.]{\label{subfig:t2wcancercg}\includegraphics[width=0.3\linewidth]{2_modality/figures/t2w/t2w_cancer_cg.eps}}
	\hspace*{\fill}
	\caption[Rendering of \acs*{t2w}-\acs*{mri} prostate images.]{Rendering of \acs*{t2w}-\acs*{mri} prostate image with both \SI{1.5}{\tesla} and \SI{3}{\tesla} \acs*{mri} scanner.}
	\label{fig:t2w}
\end{figure}

%T2W \ac{mri}
\section{\acs*{t2w}-\acs*{mri}}\label{subsec:chp2:imaging:t2w} 
\ac{t2w}-\ac{mri} has been the first \ac{mri}-modality used to perform \ac{cap} diagnosis using \ac{mri}~\cite{Hricak1983}.
Nowadays, radiologists make use of it for \ac{cap} detection, localization, and staging purposes.
This imaging technique is well suited to render zonal anatomy of the prostate~\cite{Barentsz2012}. 

This modality relies on a sequence based on setting a long \ac{tr}, reducing the T$_{1}$ effect in \ac{nmr} signal measured, and fixing the \ac{te} to sufficiently large values in order to enhance the T$_{2}$ effect of tissues.
Thus, \ac{pz} and \ac{cg} tissues are well perceptible in these images.
The former is characterized by an intermediate/high-\ac{si} while the latter is depicted by a low-\ac{si}~\cite{Hricak1987}.
An example of a healthy prostate is shown in \acs{fig}\,\ref{subfig:t2whealthy}.

In \ac{pz}, round or ill-defined low-SI masses are synonymous with \acp{cap}~\cite{Hricak1983} as shown in \acs{fig}\,\ref{subfig:t2wcancerpz}.
Detecting \ac{cap} in \ac{cg} is more challenging.
In fact both normal \ac{cg} tissue and malignant tissue, have a low-\ac{si} in \ac{t2w}-\ac{mri}, reinforcing difficulties to distinguish one among them.
However, \acp{cap} in \ac{cg} appear often as homogeneous mass possessing ill-defined edges with lenticular or ``water-drop'' shapes~\cite{Akin2006,Barentsz2012} as depicted in \acs{fig}\,\ref{subfig:t2wcancercg}. 

\ac{cap} aggressiveness has been shown to be inversely correlated with \ac{si}.
Indeed, \acp{cap} assessed with a \ac{gs} of 4-5 implied lower \ac{si} than the one with a \ac{gs} of 2-3~\cite{Wang2008}.

In spite of the availability of these useful and encouraging features, the \ac{t2w} modality lacks reliability~\cite{Kirkham2006,Hoeks2011}.
Sensitivity is affected by the difficulties in detecting cancers in \ac{cg}~\cite{Kirkham2006} while specificity rate is highly affected by outliers~\cite{Barentsz2012}.
In fact, various conditions emulate patterns of \ac{cap} such as \ac{bph}, post-biopsy hemorrhage, atrophy, scars, and post-treatment~\cite{Hricak1987,Quint1991,Scheidler1999,Cruz2002,Barentsz2012}.
These issues are partly addressed using more innovative and advanced modalities.

%T2 Map
\section{T$_2$ map} \label{subsec:chp2:imaging:t2}
As previously mentioned, \ac{t2w}-\ac{mri} modality shows low sensitivity.
Moreover, \ac{t2w}-\ac{mri} images are a composite of multiple effects~\cite{Hegde2013}.
However, T$_2$ values alone have been shown to be more discriminative~\cite{Liu2011} and highly correlated with citrate concentration, a biological marker in \ac{cap}~\cite{Liney1996,Liney1997}.

T$_2$ values are computed using the characteristics of transverse relaxation which is formalized as in \acs{eq}\,\eqref{eq:tramag}.

\begin{equation}
	M_{xy}(t) = M_{xy}(0) \exp \left( - \frac{t}{\text{T}_2} \right) \ ,
	\label{eq:tramag}
\end{equation}

\noindent where $M_{xy}(0)$ is the initial value of $M_{xy}(t)$ and T$_2$ is the relaxation time.

By rearranging \acs{eq}\,\eqref{eq:tramag}, T$_2$ map is computed by performing a linear fitting on the model presented in \acs{eq}\,\eqref{eq:t2map} using several TE, $t=\{ \text{TE}_1,\text{TE}_2, \dotsc ,\text{TE}_m \}$.

\begin{equation}
	\ln \left[ \frac{M_{xy}(t)}{M_{xy}(0)} \right] = - \frac{t}{\text{T}_2} \ .
	\label{eq:t2map}
\end{equation}

The \Ac{fse} sequence has been shown to be particularly well suited in order to build a T$_2$ map and obtain accurate T$_2$ values~\cite{Liney1996a}.
Similar to \ac{t2w}-\ac{mri}, T$_2$ values associated with \ac{cap} are significantly lower than those of healthy tissues~\cite{Liney1996,Gibbs2001}.

\begin{figure}
\centering
	\hspace*{\fill}
	\subfigure[\acs*{t1w}-\acs*{mri} image where the cancer is delimited by the red contour. The green area was still not invaded by the \acs*{cap}]{\label{subfig:t1w}\includegraphics[width=0.4\linewidth]{2_modality/figures/dce/slice.eps}} \hfill
	\subfigure[Enhancement curve computed during the \acs*{dce}-\acs*{mri} analysis. The red curve is typical from \acs*{cap} cancer while the green curve is characteristic of healthy tissue.]{\label{subfig:dce}\includegraphics[width=0.45\linewidth]{2_modality/figures/dce/dce_cancer_healthy.eps}}
	\hspace*{\fill}
	\caption[Enhancement of \acs*{dce}-\acs*{mri} signal.]{Illustration of typical enhancement signal observed in \acs*{dce}-\acs*{mri} analysis collected with a \SI{3}{\tesla} \acs*{mri} scanner.}
	\label{fig:dceana}
\end{figure}

%DCE \ac{mri}
\section{\acs*{dce}-\acs*{mri}}\label{subsec:chp2:imaging:dce}
\ac{dce}-\ac{mri} is an imaging technique which exploits the vascularity characteristic of tissues.
Contrast media, usually gadolinium-based, is injected intravenously into the patient.
The media extravasates from vessels to \ac{ees} and is released back into the vasculature before being eliminated by the kidneys~\cite{Gribbestad2005}.
Furthermore, the diffusion speed of the contrast agent may vary due to several parameters: (i) the permeability of the micro-vessels, (ii) their surface area, and (iii) the blood flow~\cite{Padhani2002}.

Healthy \ac{pz} is mainly made up of glandular tissue, around \SI{70}{\percent}~\cite{Choi2007}, which implies a reduced interstitial space restricting exchanges between vessels and \ac{ees}~\cite{Buckley2004,Niekerk2009}.
Normal \ac{cg} has a more disorganized structure, composed of mainly fibrous tissue~\cite{Choi2007,Hoeks2011}, which facilitates the arrival of the contrast agent in \ac{ees}~\cite{Niekerk2013}.
To understand the difference between contrast media kinetic in malignant tumours and the two previous behaviours mentioned, one has to focus on the process known as angiogenesis~\cite{Carmeliet2000}.
In order to ensure growth, malignant tumours produce and release angiogenic promoter substances~\cite{Carmeliet2000}.
These molecules stimulate the creation of new vessels towards the tumour~\cite{Carmeliet2000}.
However, the new vessel networks in tumours differ from those present in healthy tissue~\cite{Gribbestad2005}.
They are more porous due to the fact that their capillary walls have a large number of ``openings''~\cite{Gribbestad2005,Choi2007}.
In contrast to healthy cases, this increased vascular permeability results in increased contrast agent exchanges between vessels and \ac{ees}~\cite{Verma2012}.

By making use of the previous aspects, \ac{dce}-\ac{mri} is based on an acquisition of a set of \ac{t1w}-\ac{mri} images over time.
The gadolinium-based contrast agent shortens T$_1$ relaxation time enhancing contrast in \ac{t1w}-\ac{mri} images.
The aim is to post-analyze the pharmacokinetic behaviour of the contrast media concentration in prostate tissues~\cite{Verma2012}.
The image analysis is carried out in two dimensions: (i) in the spatial domain on a pixel-by-pixel basis and (ii) in the time domain corresponding to the consecutive images acquired with the \ac{mri}.
Thus, for each spatial location, a signal linked to contrast media concentration is measured as shown in \acs{fig}\,\ref{subfig:dce}~\cite{Tofts2010}. 

By taking the above remarks into account, \acp{cap} is characterized by a signal having an earlier and faster enhancement and an earlier wash-out --- i.e, the rate of the contrast agent flowing out of the tissue --- as shown in \acs{fig}\,\ref{subfig:dce}~\cite{Verma2012}.
Three different approaches exist to analyze these signals with the aim of labelling them as corresponding to either normal or malignant tissues.

Qualitative analysis is based on a qualitative assessment of the signal shape~\cite{Hoeks2011}.
Quantitative approaches consist of inferring pharmocokinetic parameter values~\cite{Tofts2010}.
Those parameters are part of mathematical-pharmacokinetic models which are directly based on physiological exchanges between vessels and \ac{ees}.
Several pharmacokinetic models have been proposed such as the Kety model~\cite{Kety1951}, the Tofts model~\cite{Tofts1997}, and mixed models~\cite{Larsson1996,StLawrence1998}.
The last family of methods mixed both approaches and are grouped together under the heading of semi-quantitative methods.
They rely on shape characterization using mathematical modelling to extract a set of parameters such as wash-in gradient, wash-out, integral under the curve, maximum signal intensity, time-to-peak enhancement, and start of enhancement~\cite{Hoeks2011,Verma2012}.
These parameters are depicted in \acs{fig}\,\ref{fig:dceparam}.
It has been shown that semi-quantitative and quantitative methods improve localization of \ac{cap} when compared with qualitative methods~\cite{Rosenkrantz2013}.
\Ac{sec}~\ref{subsubsec:chp3:img-clas:CADX-fea-dec:DCE-fea} provides a full description of quantitative and semi-quantitative approaches.

\ac{dce}-\ac{mri} combined with \ac{t2w}-\ac{mri} has shown to enhance sensitivity compared to \ac{t2w}-\ac{mri} alone~\cite{Jager1997,Kim2005,Schlemmer2004,Zelhof2009}.
Despite this fact, \ac{dce}-\ac{mri} possesses some drawbacks.
Due to its ``dynamic'' nature, patient motions during the image acquisition lead to spatial mis-registration of the image set~\cite{Verma2012}.
Furthermore, it has been suggested that malignant tumours are difficult to distinguish from prostatitis located in \ac{pz} and \ac{bph} located in \ac{cg}~\cite{Hoeks2011,Verma2012}.
These pairs of tissues tend to have similar appearances.
Later studies have shown that \acp{cap} in \ac{cg} do not always manifest in homogeneous fashion.
Indeed, tumours in this zone can present both hypo-vascularization and hyper-vascularization which illustrates the challenge of \ac{cap} detection in \ac{cg}~\cite{Niekerk2013}.

\begin{figure}
\centering
	\hspace*{\fill}
	\subfigure[\acs*{dw}-\acs*{mri} image acquired with a \SI{1.5}{\tesla} \acs*{mri} scanner. The cancer corresponds to the high \acs*{si} region highlighted in red.]{\label{subfig:dwi}\includegraphics[width=0.25\linewidth]{2_modality/figures/dwi/dwi_cancer.eps}} \hfill
	\subfigure[\acs*{adc} map computer after acquisition of \acs*{dw}-\acs*{mri} images with \SI{1.5}{\tesla} \acs*{mri} scanner. The cancer corresponds to the low \acs*{si} region highlighted in red.]{\label{subfig:adc}\includegraphics[width=0.25\linewidth]{2_modality/figures/dwi/adc_cancer.eps}}
	\hspace*{\fill}
	\caption[Example of \acs*{dw}-\acs*{mri} and \acs*{dce} map.]{Illustration of of \acs*{dw}-\acs*{mri} and \acs*{adc} map. The signal intensity corresponding to cancer are inversely correlated on these modalities.}
	\label{fig:dwi}
\end{figure}

%DWI \ac{mri}
\section{\acs*{dw}-\acs*{mri}}\label{subsec:chp2:imaging:dw}
As previously mentioned in the introduction, \ac{dw}-\ac{mri} is the most recent \ac{mri} imaging technique aiming at \ac{cap} detection and diagnosis~\cite{Scheidler1999}.
This modality exploits the variations in the motion of water molecules in different tissues~\cite{LeBihan1988,Koh2007}.

The distinction between healthy and \ac{cap} in \ac{dw}-\ac{mri} is based on the following physiological bases.
On the one hand, \ac{pz}, as previously mentioned, is mainly a glandular and tubular structure allowing water molecules to move freely~\cite{Choi2007,Hoeks2011}.
On the other hand, \ac{cg} is made up of muscular or fibrous tissue causing the motion of the water molecules to be more constrained and heterogeneous than in \ac{pz}~\cite{Hoeks2011}.
Then, \ac{cap} growth leads to the destruction of normal glandular structure and is associated with an increase in cellular density~\cite{Hoeks2011,Koh2007,Somford2008}.
Furthermore, these factors both have been shown to be inversely correlated with water diffusion~\cite{Koh2007,Somford2008}: higher cellular density implies a restricted water diffusion.
Thus, water diffusion in \ac{cap} will be more restricted than both healthy \ac{pz} and \ac{cg}~\cite{Koh2007,Hoeks2011}.

From the \ac{nmr} principle side, \ac{dw}-\ac{mri} sequence produces contrasted images due to variation of water molecules motion.
The method is based on the fact that the signal in \ac{dw}-\ac{mri} images is inversely correlated to the degree of random motion of water molecules~\cite{Huisman2003}.
In fact, gradients are used in \ac{dw}~\ac{mri} modality to encode spatial location of nuclei temporarily.
Simplifying the problem in only one direction, a gradient is applied in that direction, dephasing the spins of water nuclei.
Hence, the spin phases vary along the gradient direction depending of the gradient intensity at those locations.
Then, a second gradient is applied aiming at cancelling the spin dephasing.
Thus, the immobile water molecules will be subject to the same gradient intensity as the initial one while moving water molecules will be subject to a different gradient intensity.
Thus, spins of moving water molecules will stay dephased whereas spins of immobile water molecules will come back in phase.
As a consequence, a higher degree of random motion results in a more significant signal loss whereas a lower degree of random motion is synonymous with lower signal loss~\cite{Huisman2003}.
Under these conditions, the \ac{mri} signal is measured as:

\begin{align}
  M_{x,y}\left(t,b\right) & = M_{x,y}(0) \exp \left( - \frac{t}{\text{T}_2} \right) S_{\text{ADC}}(b) \ , \label{eq:t2dif} \\
  S_{\text{ADC}}(b) & = \exp \left( -b \times \text{ADC} \right) \ , \label{eq:dif}
\end{align}

\noindent where $S_{\text{ADC}}$ refers to signal drop due to diffusion effect, $\text{ADC}$ is the \acl{adc}, and $b$ is the attenuation coefficient depending only on the gradient pulses parameters: (i) gradient intensity and (ii) gradient duration~\cite{LeBihan1986}.

By using this formulation, image acquisition with a parameter $b$ equal to \SI{0}{\second\per\milli\metre\squared} corresponds to a \ac{t2w}-\ac{mri} acquisition.
Then, increasing the attenuation coefficient $b$ --- i.e., increase gradient intensity and duration --- enhances the contrast in \ac{dw}-\ac{mri} images.

To summarize, in \ac{dw}-\ac{mri} images, \acp{cap} are characterized by high-\ac{si} compared to normal tissues in \ac{pz} and \ac{cg} as shown in \acs{fig}\,\ref{subfig:dwi}~\cite{Barentsz2012}.
However, some tissues in \ac{cg} can look similar to \ac{cap} with higher \ac{si}~\cite{Barentsz2012}.

Diagnosis using \ac{dw}-\ac{mri} combined with \ac{t2w}-\ac{mri} has shown a significant improvement compared with \ac{t2w}-\ac{mri} alone and provides highly contrasted images~\cite{Shimofusa2005,Padhani2011,Choi2007}.
As drawbacks, this modality suffers from poor spatial resolution and specificity due to false positive detection~\cite{Choi2007}.
With a view to eliminate these drawbacks, radiologists use quantitative maps extracted from \ac{dw}-\ac{mri}, which is presented in the next section.

%ADC map
\section{\acs*{adc} map}\label{subsec:chp2:imaging:adc} 
The \ac{nmr} signal measured for \ac{dw}-\ac{mri} images is not only affected by diffusion as shown in \acs{eq}\,\eqref{eq:t2dif}.
However, the signal drop --- \acs{eq}\,\eqref{eq:dif} --- is formulated such that the only variable is the acquisition parameter $b$~\cite{LeBihan1986}.
The \ac{adc} is considered as a ``pure'' diffusion coefficient and is extracted to build a quantitative map known as the \acs{adc} map.
From \acs{eq}\,\eqref{eq:t2dif}, it is clear that performing multiple acquisitions only varying $b$ will not have any effect on the term  $M_{x,y}(0) \exp \left( - \frac{t}{\text{T}_2} \right)$.
Thus, \acs{eq}\,\eqref{eq:t2dif} can be rewritten as:
\begin{equation}
	S(b) = S_0 \exp \left( -b \times \text{ADC} \right) \ .
	\label{eq:t2adcrew}
\end{equation}

To compute the \ac{adc} map, a minimum of two acquisitions are necessary: (i) for $b$ equal to \SI{0}{\second\per\milli\metre\squared} where the measured signal is equal to $S_0$, and (ii) $b_1$ greater than \SI{0}{\second\per\milli\metre\squared}, typically \SI{1000}{\second\per\milli\metre\squared}.
Then, the \ac{adc} map can be computed as:

\begin{equation}
	\text{ADC} = - \frac{\ln \left( \cfrac{S(b_1)}{S_0} \right) }{b_1} \ .
	\label{eq:adcres1}
\end{equation}

More accurate \ac{adc} maps are computed by acquiring a set of images with different values for the parameter $b$ and fitting linearly a semi-logarithm function using the model presented in \acs{eq}\,\eqref{eq:t2adcrew}.

Regarding the appearance of the \ac{adc} maps, it has been previously stated that by increasing the value of $b$, the signal of \ac{cap} tissue increases significantly.
Considering \acs{eq}\,\eqref{eq:adcres1}, the tissue appearance in the \ac{adc} map is the inverse of \ac{dw}-\ac{mri} images.
Then, \ac{cap} tissue is associated with low-\ac{si} whereas healthy tissue appears brighter as depicted in \acs{fig}\,\ref{subfig:adc}~\cite{Barentsz2012}.

Similar to the gain achieved by \ac{dw}-\ac{mri}, diagnosis using \ac{adc} map combined with \ac{t2w}-\ac{mri} significantly outperforms \ac{t2w}-\ac{mri} alone~\cite{Doo2012,Choi2007}.
Moreover, it has been shown that \ac{adc} coefficient is correlated with \ac{gs}~\cite{Hambrock2011,Itou2011,Peng2013}.

However, some tissues of the \ac{cg} mimic \ac{cap} with low-\ac{si}~\cite{Kirkham2006} and image distortion can arise due to hemorrhage~\cite{Choi2007}.
It has also been noted that a high variability of the \ac{adc} occurs between different patients making it difficult to define a static threshold to distinguish \ac{cap} from non-malignant tumours~\cite{Choi2007}. 

\begin{figure}
	\centering
	\hspace*{\fill}
	\subfigure[Illustration of an \acs*{mrsi} spectrum of a healthy voxel acquired with a \SI{3}{\tesla} \acs*{mri}.]{\label{subfig:mrsihea}\includegraphics[width=0.45\linewidth]{2_modality/figures/mrsi/mrsi_healthy.eps}} \hfill
	\subfigure[Illustration of an \acs*{mrsi} spectrum of a cancerous voxel acquired with a \SI{3}{\tesla} \acs*{mri}.]{\label{subfig:mrsican}\includegraphics[width=0.45\linewidth]{2_modality/figures/mrsi/mrsi_cancer.eps}}
	\hspace*{\fill}
	\caption[Illustration of healthy and cancerous \acs*{mrsi} spectrum.]{Illustration of an \acs*{mrsi} spectrum for both healthy and cancerous voxels with a \SI{3}{\tesla} \acs*{mri}. The highlighted areas correspond to the related concentration of the metabolites which is computed by integrating the area under each peak. Acronyms: choline (Cho), spermine (Spe), creatine (Cr) and citrate (Cit).}
	\label{fig:mrsi}
\end{figure}


%MRSI
\section{\acs*{mrsi}}\label{subsec:chp2:imaging:mrsi}
\ac{cap} induces metabolic changes in the prostate compared with healthy tissue.
Thus, \ac{cap} detection can be carried out by tracking changes of metabolite concentration in prostate tissue.
\ac{mrsi} is an \ac{nmr}-based technique which generates spectra of relative metabolite concentration in \iac{roi}.

In order to track changes of metabolite concentration, it is important to know which metabolites are associated with \ac{cap}.
To address this question, clinical studies identified three biological markers: (i) citrate, (ii) choline, and (iii) polyamines composed mainly of spermine, and in less abundance of spermidine and putrescine~\cite{Awwad2012,Costello2006,Giskeodegard2013}. 

Citrate is involved in the production and secretion of the prostatic fluid, and the glandular prostate cells are associated with a high production of citrate enabled by zinc accumulation by these same cells~\cite{Costello2006}.
However, the metabolism allowing the accumulation of citrate requires a large amount of energy~\cite{Costello2006}.
In contrast, malignant cells do not have high zinc levels leading to lower citrate levels due to citrate oxidization~\cite{Costello2006}.
Furthermore, this change results in a more energy-efficient metabolism enabling malignant cells to grow and spread~\cite{Costello2006}.

An increased concentration of choline is related to \ac{cap}~\cite{Awwad2012}.
Malignant cell development requires epigenetic mechanisms resulting in metabolic changes and relies on two mechanisms: \ac{dna} methylation and phospholid metabolism which both result in choline uptake, explaining its increased level in \ac{cap} tissue~\cite{Awwad2012}.
Spermine is also considered as a biological marker in \ac{cap}~\cite{Graaf2000,Giskeodegard2013}.
In \ac{cap}, reduction of the ductal volume due to shifts in polyamine homeostasis might lead to a reduced spermine concentration~\cite{Graaf2000}.

To determine the concentration of these biological markers, one has to focus on the \ac{mrsi} modality.
In theory, in presence of a homogeneous magnetic field, identical nuclei precesses at the same operating frequency known as the Lamor frequency~\cite{Haacke1999}.
However, \ac{mrsi} is based on the fact that identical nuclei will slightly precess at different frequencies depending on the chemical environment in which they are immersed~\cite{Haacke1999}, a phenomenon known as the \ac{cse}~\cite{Parfait2010}.
Given this property, metabolites are identified and their concentrations are determined.
In this regard, the Fourier transform is used to obtain the frequency spectrum of the \ac{nmr} signal~\cite{Haacke1999,Parfait2010}.
In this spectrum, each peak is associated with a particular metabolite and the area under each peak corresponds to the relative concentration of this metabolite, as illustrated in \acs{fig}\,\ref{fig:mrsi}~\cite{Parfait2010}.

Two different quantitative approaches are used to decide whether or not the spectra of \iac{roi} is associated with \ac{cap}: (i) relative quantification or (ii) absolute quantification~\cite{Lemaitre2011}.
In relative quantification, the ratio of choline-polyamines-creatine to citrate is computed.
The integral of the signal is computed from choline to creatine --- i.e., from \SIrange{3.21}{3.02}{\ppm} --- because the peaks in this region can be merged at clinical magnetic field strengths~\cite{Hoeks2011,Graaf2000}, as depicted in \acs{fig}\,\ref{fig:mrsi}).
Considering the previous assumptions that choline concentration rises and citrate concentration decreases in the presence of \ac{cap}, the ratio computed should be higher in malignant tissue than in healthy tissue. 

In contrast with relative quantification, absolute quantification measures molar concentrations by normalizing relative concentrations using water as reference~\cite{Lemaitre2011}.
In this case, ``true'' concentrations are directly used to differentiate malignant from healthy tissue.
However, this method is not commonly used as it requires an additional step of acquiring water signals, inducing time and cost acquisition constraints.

\ac{mrsi} allows examination with high specificity and sensitivity compared to other \ac{mri} modalities~\cite{Choi2007}.
Furthermore, it has been shown that combining \ac{mrsi} with \ac{mri} improves detection and diagnosis performance~\cite{Scheidler1999a,Kaji1998,Vilanova2009}.
Citrate and spermine concentrations are inversely correlated with the \ac{gs} allowing us to distinguish low- from high- grade \acp{cap}~\cite{Giskeodegard2013}.
However, choline concentration does not provide the same properties~\cite{Giskeodegard2013}.

Unfortunately, \ac{mrsi} also presents several drawbacks.
First, \ac{mrsi} acquisition is time consuming which prevents this modality from being used in daily clinical practise~\cite{Barentsz2012}.
In addition, \ac{mrsi} suffers from low spatial resolution due to the fact that \ac{snr} is linked to the voxel size.
However, this issue is addressed by developing new scanners with higher magnetic field strengths such as \SI{7.5}{\tesla}~\cite{Giskeodegard2013}.
Finally, a high variability of the relative concentrations between patients has been observed~\cite{Choi2007}.
The same observation has been made depending on the zones studied (ie., \ac{pz}, \ac{cg}, base, mid-gland, apex)~\cite{Walker2010,Lemaitre2011}.
Due to this variability, it is difficult to use a fixed threshold in order to differentiate \ac{cap} from healthy tissue.

\section{Summary and conclusions}

\acs{tab}~\ref{tab:modmri} provides an overview of the different modalities presented in the previous section.
Indeed, each \ac{mri} modality alone provides a different discriminative level to distinguish \ac{cap} from healthy tissue.
A recurrent statement in the literature is, however, the ability to combine these \ac{mri} modalities would lead to the best diagnosis performance.
In this regard, we will present in the next chapter automatic tools which have been developed to design \ac{mpmri} \ac{cad} systems for the detection of \ac{cap}.

\begin{landscape}
\begin{table}
  \scriptsize
    \caption[Overview of the features associated with each \acs*{mri} modality used for medical diagnosis by radiologists.]{Overview of the features associated with each \acs*{mri} modality used for medical diagnosis by radiologists. Acronyms: \acf{cap} - \acf{si} - \acf{gs}.}\label{tab:modmri}
    \begin{threeparttable}
      \centering
      \noindent
      \begin{tabularx}{\linewidth}{@{} l X X X l @{}}
        \toprule
        \textbf{Modality} & \textbf{Significant features} & \textbf{\acs*{cap}} & \textbf{Healthy tissue} & \textbf{\acs*{gs} correlation} \\
        \midrule
        \acs*{t2w}-\acs*{mri} & \acs*{si} & low-\acs*{si} in \acs*{pz}~\cite{Hricak1987} & intermediate to high-\acs*{si} in \acs*{pz}~\cite{Hricak1987} & +~\cite{Wang2008} \\ 
        & Shape & round or ill-defined mass in \acs*{pz}~\cite{Hricak1983} &  & 0 \\
        & \acs*{si} & low-\acs*{si} in \acs*{cg}~\cite{Akin2006,Barentsz2012} & low-\acs*{si} in \acs*{cg}~\cite{Akin2006,Barentsz2012} & 0 \\
        & Shape & homogeneous mass with ill-defined edges in \acs*{cg}~\cite{Akin2006, Barentsz2012} &  & 0 \\ \\
        T$_2$ map & \acs*{si} & low-\acs*{si}~\cite{Liney1996,Gibbs2001} & intermediate to high-\acs*{si}~\cite{Liney1996,Gibbs2001} & +~\cite{Liu2011,Liney1996,Liney1997}  \\ \\
        \acs*{dce} \acs*{mri} & Semi-quantitative features~\cite{Verma2012}: & & & \\
        & $\bullet$ wash-in & faster & slower & 0 \\
        & $\bullet$ wash-out & faster & slower & 0 \\
        & $\bullet$ integral under the curve & higher & lower & 0 \\
        & $\bullet$ maximum signal intensity & higher & lower & 0 \\
        & $\bullet$ time-to-peak enhancement & faster & slower & 0 \\ \\
        & Quantitative features (Tofts' parameters~\cite{Tofts2010}): & & & \\
        & $\bullet$ $\text{k}_{\text{ep}}$ & higher & lower & 0 \\
        & $\bullet$ $\text{K}^{\text{trans}}$ & higher & lower & 0 \\ \\
        \acs*{dw} \acs*{mri} & \acs*{si} & higher-\acs*{si}~\cite{Huisman2003,Barentsz2012} & lower-\acs*{si}~\cite{Huisman2003,Barentsz2012} & + \\ \\ 
        \acs*{adc} map & \acs*{si} & low-\acs*{si}~\cite{Barentsz2012} & high-\acs*{si}~\cite{Barentsz2012} & +~\cite{Hambrock2011, Itou2011, Peng2013} \\ \\
        \acs*{mrsi}& Metabolites: & & & \\
        & $\bullet$ citrate (2.64 ppm)~\cite{Verma2010} & lower concentration~\cite{Awwad2012,Costello2006,Graaf2000} & higher concentration~\cite{Awwad2012,Costello2006,Graaf2000} & +~\cite{Giskeodegard2013} \\
        & $\bullet$ choline (3.21 ppm)~\cite{Verma2010} & higher concentration~\cite{Awwad2012,Costello2006,Graaf2000} & lower concentration~\cite{Awwad2012,Costello2006,Graaf2000} & 0~\cite{Giskeodegard2013} \\
        & $\bullet$ spermine (3.11 ppm)~\cite{Verma2010} & lower concentration~\cite{Awwad2012,Costello2006,Graaf2000} & higher concentration~\cite{Awwad2012,Costello2006,Graaf2000} & +~\cite{Giskeodegard2013} \\
        \bottomrule
      \end{tabularx}
      \begin{tablenotes}
      \item Notes:
      \item + = significantly correlated;
      \item 0 = no correlation.
      \end{tablenotes}
    \end{threeparttable}
  \label{tab:modmri}
\end{table}
\end{landscape}
