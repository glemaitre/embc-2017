
\subsection{\acs*{cade}: \acsp*{roi} detection/selection} \label{subsec:chp3:img-clas:roiSel}

\begin{table}
  \caption{Overview of the \acs*{cade} strategies employed in \acs*{cad} systems.}
  \scriptsize
  \centering
  \begin{tabularx}{\textwidth}{l >{\raggedleft\arraybackslash}X}
    \toprule
    \textbf{\ac{cade}: \acp{roi} selection strategy} & \textbf{References} \\
    \midrule
    All voxels-based approach & \cite{Artan2009,Artan2010,Giannini2013,Kelm2007,Liu2009,Lopes2011,Matulewicz2013,Mazzetti2011,Ozer2009,Ozer2010,Parfait2012,Sung2011,Tiwari2007,Tiwari2008,Tiwari2009,Tiwari2009a,Tiwari2010,Tiwari2012,Tiwari2013,Viswanath2008,Viswanath2008a,Viswanath2009,Viswanath2011,Viswanath2012,trigui2016classification,trigui2017automatic,lehaire2014computer,khalvati2015automated,rampun2015classifying,rampun2015computer,rampun2016computer,rampun2016computerb,rampun2016quantitative} \\
    Lesions candidate detection & \cite{Litjens2011,Litjens2012,Litjens2014,Vos2012,cameron2014multiparametric,cameron2016maps} \\
    \bottomrule
  \end{tabularx}
\label{tab:cade}
\end{table}

As discussed in the introduction and shown in \acs{fig}\,\ref{fig:wkfcad}, the image classification framework is often composed of a \ac{cade} and a \ac{cadx}.
In this section, we focus on studies which embed a \ac{cade} in their framework.
Two approaches are considered to define a \ac{cade}: (i) voxel-based delineation and (ii) lesion segmentation.
These methods are summarized in \acs{tab}~\ref{tab:cade}.
The first strategy is in fact linked to the nature of the classification framework and concerns the majority of the studies reviewed~\cite{Artan2009,Artan2010,Giannini2013,Kelm2007,Liu2009,Lopes2011,Matulewicz2013,Mazzetti2011,Ozer2009,Ozer2010,Parfait2012,Sung2011,Tiwari2007,Tiwari2008,Tiwari2009,Tiwari2009a,Tiwari2010,Tiwari2012,Tiwari2013,Viswanath2008,Viswanath2008a,Viswanath2009,Viswanath2011,Viswanath2012,trigui2016classification,trigui2017automatic,lehaire2014computer,khalvati2015automated,rampun2015classifying,rampun2015computer,rampun2016computer,rampun2016computerb,rampun2016quantitative}.
Each voxel is a possible candidate and will be classified as cancer or healthy.
The second group of methods is composed of method implementing a lesion segmentation algorithm to delineate potential candidates to further obtain a diagnosis through the \ac{cadx}.
This approach is borrowed from other application areas such as breast cancer.
These methods are in fact very similar to the classification framework used in \ac{cadx} later.

Regarding lesion candidate detection, \citeauthor{Vos2012} highlighted lesion candidates by detecting blobs in the \ac{adc} map~\cite{Vos2012}.
These candidates are filtered using some \textit{a priori} criteria such as \ac{si} or diameter.
As mentioned in \acs{sec}\,\ref{subsec:chp2:imaging:mrsi} and \acs{tab}~\ref{tab:modmri}, low \ac{si} in \ac{adc} map can be linked to potential \ac{cap}.
Hence, blob detectors are suitable to highlight these regions. 
Blobs are detected in a multi-resolution scheme, by computing the three main eigenvalues $\{ \lambda_{\sigma,1},\lambda_{\sigma,2},\lambda_{\sigma,3} \}$ of the Hessian matrix, for each voxel location of the \ac{adc} map at a specific scale $\sigma$~\cite{Li2003}.
The probability $p$ of a voxel $\mathbf{x}$ being a part of a blob at the scale $\sigma$ is given by:

\begin{equation}
P(\mathbf{x},\sigma) = \begin{cases}
	\frac{\| \lambda_{\sigma,3}(\mathbf{x}) \|^{2}}{\| \lambda_{\sigma,1} (\mathbf{x}) \|} \ , & \text{if } \lambda_{\sigma,k}(\mathbf{x}) > 0 \text{ with } k = \{1,2,3\} \  , \\
	0 \ , & \text{otherwise} \ .
\end{cases}
\label{eq:blobdet}
\end{equation}

\noindent The fusion of the different scales is computed as:

\begin{equation}
	L(\mathbf{x}) = \max P(\mathbf{x},\sigma) , \forall \sigma \ .
	\label{eq:fusionBlob}
\end{equation}

The candidate blobs detected are then filtered depending on their appearances --- i.e., maximum of the likelihood of the region, diameter of the lesion --- and their \ac{si} in \ac{adc} and \ac{t2w}-\ac{mri} images.
The detected regions are then used as inputs for the \ac{cadx}.
\citeauthor{cameron2016maps} used a similar approach by automatically selecting low \ac{si} connected regions in the \ac{adc} map with a size larger than \SI{1}{\milli\metre\squared}~\cite{cameron2014multiparametric,cameron2016maps}.

\citeauthor{Litjens2011} used a pattern recognition approach in order to delineate the \acp{roi}~\cite{Litjens2011}.
A blobness map is computed in the same manner as in~\cite{Vos2010} using the multi-resolution Hessian blob detector on the \ac{adc} map, \ac{t2w}, and pharmacokinetic parameters maps (see \acs{sec}\,\ref{subsec:chp3:img-clas:CADX-fea-dec} for details about those parameters).
Additionally, the position of the voxel $\mathbf{x}=\{x,y,z\}$ is used as a feature as well as the Euclidean distance of the voxel to the prostate center.
Hence, each feature vector is composed of 8 features and a \ac{svm} classifier is trained using a \ac{rbf} kernel (see \acs{sec}\,\ref{subsec:chp3:img-clas:CADX-clas} for more details).

Subsequently, \citeauthor{Litjens2012} modified this approach by including only features related to the blob detection on the different maps as well as the original \acp{si} of the parametric images~\cite{Litjens2012}.
Two new maps are introduced based on texture and a \ac{knn} classifier is used instead of a \ac{svm} classifier.
The candidate regions are then extracted by performing a local maxima detection followed by post-processing region-growing and morphological operations. 
